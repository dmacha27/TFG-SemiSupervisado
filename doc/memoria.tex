\documentclass[a4paper,12pt,twoside]{memoir}

% Castellano
\usepackage[spanish,es-tabla]{babel}
\selectlanguage{spanish}
\usepackage[utf8]{inputenc}
\usepackage[T1]{fontenc}
\usepackage{lmodern} % Scalable font
\usepackage{microtype}
\usepackage{placeins}
\usepackage{dirtytalk}

\RequirePackage{booktabs}
\RequirePackage[table]{xcolor}
\RequirePackage{xtab}
\RequirePackage{multirow}

% Links
\PassOptionsToPackage{hyphens}{url}\usepackage[colorlinks]{hyperref}
\hypersetup{
	allcolors = {red}
}

% Ecuaciones
\usepackage{amsmath}
\usepackage[ruled,noline,linesnumbered,spanish]{algorithm2e}
\SetKwIF{If}{ElseIf}{Else}{if}{}{else if}{else}{end if}
\SetKwFor{For}{for}{}{endfor}
\SetKwFor{While}{while}{}{endwhile}

\newcommand{\bfit}[1]{\ensuremath{\textbf{\textit{#1}}}}

% Rutas de fichero / paquete
\newcommand{\ruta}[1]{{\sffamily #1}}

% Párrafos
\nonzeroparskip
\usepackage{tcolorbox}
\usepackage{minted}

% Huérfanas y viudas
\widowpenalty100000
\clubpenalty100000

% Imágenes

% Comando para insertar una imagen en un lugar concreto.
% Los parámetros son:
% 1 --> Ruta absoluta/relativa de la figura
% 2 --> Texto a pie de figura
% 3 --> Tamaño en tanto por uno relativo al ancho de página
\usepackage{graphicx}
\newcommand{\imagen}[3]{
	\begin{figure}[!ht]
		\centering
		\includegraphics[width=#3\textwidth]{#1}
		\caption{#2}\label{fig:#1}
	\end{figure}
	\FloatBarrier
}

\newcommand{\imagenconurl}[4]{
	\begin{figure}[!ht]
		\centering
		\includegraphics[width=#4\textwidth]{#1}
		\caption[#2]{#3}\label{fig:#1}
	\end{figure}
	\FloatBarrier
}

% Comando para insertar una imagen sin posición.
% Los parámetros son:
% 1 --> Ruta absoluta/relativa de la figura
% 2 --> Texto a pie de figura
% 3 --> Tamaño en tanto por uno relativo al ancho de página
\newcommand{\imagenflotante}[3]{
	\begin{figure}
		\centering
		\includegraphics[width=#3\textwidth]{#1}
		\caption{#2}\label{fig:#1}
	\end{figure}
}

% El comando \figura nos permite insertar figuras comodamente, y utilizando
% siempre el mismo formato. Los parametros son:
% 1 --> Porcentaje del ancho de página que ocupará la figura (de 0 a 1)
% 2 --> Fichero de la imagen
% 3 --> Texto a pie de imagen
% 4 --> Etiqueta (label) para referencias
% 5 --> Opciones que queramos pasarle al \includegraphics
% 6 --> Opciones de posicionamiento a pasarle a \begin{figure}
\newcommand{\figuraConPosicion}[6]{%
  \setlength{\anchoFloat}{#1\textwidth}%
  \addtolength{\anchoFloat}{-4\fboxsep}%
  \setlength{\anchoFigura}{\anchoFloat}%
  \begin{figure}[#6]
    \begin{center}%
      \Ovalbox{%
        \begin{minipage}{\anchoFloat}%
          \begin{center}%
            \includegraphics[width=\anchoFigura,#5]{#2}%
            \caption{#3}%
            \label{#4}%
          \end{center}%
        \end{minipage}
      }%
    \end{center}%
  \end{figure}%
}

%
% Comando para incluir imágenes en formato apaisado (sin marco).
\newcommand{\figuraApaisadaSinMarco}[5]{%
  \begin{figure}%
    \begin{center}%
    \includegraphics[angle=90,height=#1\textheight,#5]{#2}%
    \caption{#3}%
    \label{#4}%
    \end{center}%
  \end{figure}%
}
% Para las tablas
\newcommand{\otoprule}{\midrule [\heavyrulewidth]}
%
% Nuevo comando para tablas pequeñas (menos de una página).
\newcommand{\tablaSmall}[5]{%
 \begin{table}
  \begin{center}
   \rowcolors {2}{gray!35}{}
   \begin{tabular}{#2}
    \toprule
    #4
    \otoprule
    #5
    \bottomrule
   \end{tabular}
   \caption{#1}
   \label{tabla:#3}
  \end{center}
 \end{table}
}

%
% Nuevo comando para tablas pequeñas (menos de una página).
\newcommand{\tablaSmallSinColores}[5]{%
 \begin{table}[H]
  \begin{center}
   \begin{tabular}{#2}
    \toprule
    #4
    \otoprule
    #5
    \bottomrule
   \end{tabular}
   \caption{#1}
   \label{tabla:#3}
  \end{center}
 \end{table}
}

\newcommand{\tablaApaisadaSmall}[5]{%
\begin{landscape}
  \begin{table}
   \begin{center}
    \rowcolors {2}{gray!35}{}
    \begin{tabular}{#2}
     \toprule
     #4
     \otoprule
     #5
     \bottomrule
    \end{tabular}
    \caption{#1}
    \label{tabla:#3}
   \end{center}
  \end{table}
\end{landscape}
}

%
% Nuevo comando para tablas grandes con cabecera y filas alternas coloreadas en gris.
\newcommand{\tabla}[6]{%
  \begin{center}
    \tablefirsthead{
      \toprule
      #5
      \otoprule
    }
    \tablehead{
      \multicolumn{#3}{l}{\small\sl continúa desde la página anterior}\\
      \toprule
      #5
      \otoprule
    }
    \tabletail{
      \hline
      \multicolumn{#3}{r}{\small\sl continúa en la página siguiente}\\
    }
    \tablelasttail{
      \hline
    }
    \bottomcaption{#1}
    \rowcolors {2}{gray!35}{}
    \begin{xtabular}{#2}
      #6
      \bottomrule
    \end{xtabular}
    \label{tabla:#4}
  \end{center}
}

%
% Nuevo comando para tablas grandes con cabecera.
\newcommand{\tablaSinColores}[6]{%
  \begin{center}
    \tablefirsthead{
      \toprule
      #5
      \otoprule
    }
    \tablehead{
      \multicolumn{#3}{l}{\small\sl continúa desde la página anterior}\\
      \toprule
      #5
      \otoprule
    }
    \tabletail{
      \hline
      \multicolumn{#3}{r}{\small\sl continúa en la página siguiente}\\
    }
    \tablelasttail{
      \hline
    }
    \bottomcaption{#1}
    \begin{xtabular}{#2}
      #6
      \bottomrule
    \end{xtabular}
    \label{tabla:#4}
  \end{center}
}

%
% Nuevo comando para tablas grandes sin cabecera.
\newcommand{\tablaSinCabecera}[5]{%
  \begin{center}
    \tablefirsthead{
      \toprule
    }
    \tablehead{
      \multicolumn{#3}{l}{\small\sl continúa desde la página anterior}\\
      \hline
    }
    \tabletail{
      \hline
      \multicolumn{#3}{r}{\small\sl continúa en la página siguiente}\\
    }
    \tablelasttail{
      \hline
    }
    \bottomcaption{#1}
  \begin{xtabular}{#2}
    #5
   \bottomrule
  \end{xtabular}
  \label{tabla:#4}
  \end{center}
}



\definecolor{cgoLight}{HTML}{EEEEEE}
\definecolor{cgoExtralight}{HTML}{FFFFFF}

%
% Nuevo comando para tablas grandes sin cabecera.
\newcommand{\tablaSinCabeceraConBandas}[5]{%
  \begin{center}
    \tablefirsthead{
      \toprule
    }
    \tablehead{
      \multicolumn{#3}{l}{\small\sl continúa desde la página anterior}\\
      \hline
    }
    \tabletail{
      \hline
      \multicolumn{#3}{r}{\small\sl continúa en la página siguiente}\\
    }
    \tablelasttail{
      \hline
    }
    \bottomcaption{#1}
    \rowcolors[]{1}{cgoExtralight}{cgoLight}

  \begin{xtabular}{#2}
    #5
   \bottomrule
  \end{xtabular}
  \label{tabla:#4}
  \end{center}
}



\graphicspath{ {./img/} }

% Capítulos
\chapterstyle{bianchi}
\newcommand{\capitulo}[2]{
	\setcounter{chapter}{#1}
	\setcounter{section}{0}
	\setcounter{figure}{0}
	\setcounter{table}{0}
	\chapter*{#2}
	\addcontentsline{toc}{chapter}{#2}
	\markboth{#2}{#2}
}

% Apéndices
\renewcommand{\appendixname}{Apéndice}
\renewcommand*\cftappendixname{\appendixname}

\newcommand{\apendice}[1]{
	%\renewcommand{\thechapter}{A}
	\chapter{#1}
}

\renewcommand*\cftappendixname{\appendixname\ }

% Formato de portada
\makeatletter
\usepackage{xcolor}
\newcommand{\tutor}[1]{\def\@tutor{#1}}
\newcommand{\cotutor}[1]{\def\@cotutor{#1}}
\newcommand{\course}[1]{\def\@course{#1}}
\definecolor{cpardoBox}{HTML}{E6E6FF}
\def\maketitle{
  \null
  \thispagestyle{empty}
  % Cabecera ----------------
\noindent\includegraphics[width=\textwidth]{cabecera}\vspace{1cm}%
  \vfill
  % Título proyecto y escudo informática ----------------
  \colorbox{cpardoBox}{%
    \begin{minipage}{.8\textwidth}
      \vspace{.5cm}\Large
      \begin{center}
      \textbf{TFG del Grado en Ingeniería Informática}\vspace{.6cm}\\
      \textbf{\LARGE\@title{}}
      \end{center}
      \vspace{.2cm}
    \end{minipage}

  }%
  \hfill\begin{minipage}{.20\textwidth}
    \includegraphics[width=\textwidth]{escudoInfor}
  \end{minipage}
  \vfill
  % Datos de alumno, curso y tutores ------------------
  \begin{center}%
  {%
    \noindent\LARGE
    Presentado por \@author{}\\ 
    en Universidad de Burgos \\
    el \@date{}\\
    \begin{tabbing}
    Tutores: \= \kill
    Tutores: \> \@tutor{}\\
          \> \@cotutor{}\\
    \end{tabbing}
  }%
  \end{center}%
  \null
  \cleardoublepage
  }
\makeatother

\newcommand{\titulo}{Herramienta docente para la visualización en Web de algoritmos de aprendizaje Semi-Supervisado}
\newcommand{\nombre}{David Martínez Acha}
\newcommand{\tut}{Dr. Álvar Arnaiz González}
\newcommand{\cotut}{Dr. César Ignacio García Osorio}

% Datos de portada
\title{\titulo}
\author{\nombre}
\tutor{\tut}
\cotutor{\cotut}
\date{\today}

\begin{document}

\maketitle


\newpage\null\thispagestyle{empty}\newpage


%%%%%%%%%%%%%%%%%%%%%%%%%%%%%%%%%%%%%%%%%%%%%%%%%%%%%%%%%%%%%%%%%%%%%%%%%%%%%%%%%%%%%%%%
\thispagestyle{empty}


\noindent\includegraphics[width=\textwidth]{cabecera}\vspace{1cm}

\noindent El \tut, junto a el \cotut, profesores del departamento de Ingeniería
Informática, área de Lenguajes y Sistemas informáticos.

\noindent Exponen:

\noindent Que el alumno D. \nombre, con DNI 71310644H, ha realizado el Trabajo
final de Grado en Ingeniería Informática titulado \titulo.


\noindent Y que dicho trabajo ha sido realizado por el alumno bajo la dirección
del que suscribe, en virtud de lo cual se autoriza su presentación y defensa.

\begin{center} %\large
En Burgos, {\large \today}
\end{center}

\vfill\vfill\vfill

% Author and supervisor
\begin{minipage}{0.45\textwidth}
\begin{flushleft} %\large
Vº. Bº. del Tutor:\\[2cm]
\tut
\end{flushleft}
\end{minipage}
\hfill
\begin{minipage}{0.45\textwidth}
\begin{flushleft} %\large
Vº. Bº. del co-tutor:\\[2cm]
\cotut
\end{flushleft}
\end{minipage}
\hfill

\vfill

% para casos con solo un tutor comentar lo anterior
% y descomentar lo siguiente
%Vº. Bº. del Tutor:\\[2cm]
%D. nombre tutor


\newpage\null\thispagestyle{empty}\newpage




\frontmatter

% Abstract en castellano
\renewcommand*\abstractname{Resumen}
\begin{abstract}
El aprendizaje semi-supervisado es realmente útil en un contexto real, debido a
la clara dificultad y coste de obtener datos etiquetados de calidad. Sin
embargo, estos algoritmos no suelen tenerse muy en cuenta, e incluso en los
contenidos docentes, suelen obviarse (centrándose en el aprendizaje supervisado
y/o no supervisado).

En este proyecto, se ha desarrollado una biblioteca con cuatro algoritmos
semi-supervisados: \emph{Self-Training}, \emph{Co-Training}, \emph{Democratic
Co-Learning} y \emph{Tri-Training}. El aspecto añadido de todos ellos es poder
registrar el proceso de entrenamiento en cuanto a cómo se infiere el
conocimiento sobre los datos no etiquetados y la extracción de estadísticas.

Como objetivo final, se ha desarrollado una aplicación web dedicada a la
visualización de este proceso de entrenamiento para ayudar a aprender el
funcionamiento de estos algoritmos.

Como objetivo secundario, la aplicación también incluye una gestión de usuarios
que les permite almacenar sus conjuntos de datos y sus ejecuciones previas
(además de un panel de administración). Aun así, la aplicación está
completamente pensada para un acceso gratuito sin necesidad de un
registro.

La aplicación se encuentra disponible en \mbox{\url{https://vass.dmacha.dev}}.
\end{abstract}

\renewcommand*\abstractname{Descriptores}
\begin{abstract}
aprendizaje automático, aprendizaje semi-supervisado, visualización de algoritmos, web
\end{abstract}

\clearpage

% Abstract en inglés
\renewcommand*\abstractname{Abstract}
\begin{abstract}
Semi-supervised learning is really useful in a real context, due to the clear
difficulty and cost to obtain quality labelled data. However, these algorithms
are not taken into account, and even in teaching contents, tend to be
ignored (focusing on supervised learning and/or unsupervised learning).

In this project, a four semi-supervised algorithm library has been developed
that includes: \emph{Self-Training}, \emph{Co-Training}, \emph{Democratic
Co-Learning} and \emph{Tri-Training}. The additional aspect of all of them is to
register the training process on how knowledge is inferred to the unlabelled
data and the extraction of statistics.

As final objective, a web application dedicated to the visualization of this
training process has been developped to help to learn how these algorithms work.

As a secondary objective, the application also includes a user management system
that allows them to store their datasets and previous runs (plus an admin
dashboard). Even so, the application has been conceived to be free access
without registration.

The application is available at \mbox{\url{https://vass.dmacha.dev}}.
\end{abstract}

\renewcommand*\abstractname{Keywords}
\begin{abstract}
machine learning, semi-supervised learning, algorithm visualization, web
\end{abstract}

\clearpage

% Indices
\tableofcontents

\clearpage

\listoffigures

\clearpage

\listoftables
\clearpage

\mainmatter
\capitulo{1}{Introducción}

El ámbito del aprendizaje automático es un campo muy interesante y del que cada
vez hay más atención. La realidad es que la mayor parte del conocimiento está
muy centrado en ciertos tipos de aprendizaje automático: el supervisado y el no
supervisado. En cuanto se indaga un poco en <<machine learning>> estos dos
conceptos empiezan a rondar. Pero del que no se oye tanto, y puede ser muy
beneficioso, es el aprendizaje semi-supervisado. 

El aprendizaje supervisado, en pocas palabras, permite aprovechar situaciones en
las que se sabe qué representa un dato (por ejemplo, dada una animal, se sabe si
el animal es un perro o un pato), el no supervisado no tiene esta <<suerte>>, se
utiliza en casos en los que no se tiene ese conocimiento, sino que es él mismo
el que intenta extraer las representaciones (por ejemplo, para un conjunto de
animales, podría distinguir entre los que tiene pico y alas y los que tienen
cuatro patas sin necesidad de saber qué animal concreto es). En la realidad
(obviando los ejemplos tan sencillos comentados), el <<etiquetado>> de los datos
suele ser muy costoso y se tienen muchos más datos de los que no se sabe qué
representan, el aprendizaje semi-supervisado es el ideal en estos casos y es que
permite inferir conocimiento para estos últimos y determinar a qué
corresponden.

Centrando más el objetivo final de este trabajo, no existen muchas aplicaciones
que permitan compaginar la teoría de estos conceptos con visualizaciones
interesantes que permitan comprender su funcionamiento y mucho menos para los
algoritmos semi-supervisados que incluso en muchos casos suelen obviarse. Vista la
carencia en este ámbito, este trabajo pretende resultar en la creación de una
aplicación amigable y atractiva que permita, mediante visualizaciones,
comprender realmente cómo funcionan los principales algoritmos
semi-supervisados.

Las herramientas de fácil acceso, como las páginas Web que no requieren de
instalación por parte del usuario, van de la mano con la globalización de
internet. Es por ello que esta herramienta, categorizada ya como
\textit{docente}, será accesible desde internet. La idea de esto es permitir a
los usuarios la rapidez y facilidad de acceder simplemente a una <<URL>> sin
prácticamente tener que realizar otro trabajo.

Además, los datos de las visualizaciones será proporcionados por una biblioteca
propia donde se implementen estos algoritmos. Estarán adaptados para la
obtención de la información del entrenamiento y estadísticas relevantes, pero su
caso están pensados para que puedan ser utilizados de forma general.



\capitulo{2}{Objetivos del proyecto}

Este apartado explica de forma precisa y concisa cuáles son los objetivos que se
persiguen con la realización del proyecto.


Los \textbf{objetivos generales} del proyecto son los que siguen:
\begin{enumerate}
    \item Implementación de una biblioteca con cuatro algoritmos de aprendizaje
    semi-supervisado más comunes: \emph{Self-Training}, \emph{Co-Training},
    \emph{Democratic Co-Learning} y \emph{Tri-Training}.
    \item Diseño y creación de una aplicación Web desde la accesibilidad y
    enfoque docente (ayudas contextuales, explicaciones, pseudocódigos,
    instrucciones...).
    \item Integración de los cuatro algoritmos en la aplicación Web para su
    visualización.
    \item Crear un sistema de usuarios que les permita controlar sus ficheros y ejecuciones.
    \item Internacionalización de la página web de la aplicación.
\end{enumerate}

Como \textbf{requisitos técnicos} y más particulares del proyecto se encuentran
los siguientes:

\begin{enumerate}
    \item Implementación de los algoritmos en Python 3.10.
    \item Utilización del \emph{framework} Flask para el desarrollo de la
    aplicación web.
    \item Diseño de la web basado en Bootstrap 5.
    \item Creación de las visualizaciones mediante JavaScript y la biblioteca
    \texttt{D3.js}.
    \item Optimizar al máximo posible el procesamiento de datos y entrenamiento.
    \item Internacionalización mediante Babel (Flask-Babel).
    \item Creación y manejo de una base de datos para usuarios (y relativos) con
    independencia de la tecnología utilizada (SQLAlchemy).
    \item Realización de pruebas sobre el software desarrollado
    (comparativa/validación con otras librerías y test unitarios mediante
    \texttt{pytest}).
    \item Crear una documentación de usuario y programador precisa y completa.
\end{enumerate}
\capitulo{3}{Conceptos teóricos}


\section{Aprendizaje automático}

Según \cite{intelligent:ml}, el aprendizaje automático (\textit{machine
learning}) es una rama de la Inteligencia artificial como una técnica de
análisis de datos que enseña a las computadoras a aprender de la
\textbf{experiencia} (es decir, lo que realizan los humanos). Para ello, el
aprendizaje automático se nutre de gran cantidad de datos (o los suficientes
para el problema concreto) que son procesados por ciertos algoritmos. Estos
datos son ejemplos (también llamados instancias o prototipos), \cite{pascual:ml}
mediante los cuales, los algoritmos son capaces de generalizar comportamientos
que se encuentran ocultos. 

La característica principal de estos algoritmos es que son capaces de mejorar su
rendimiento de forma automática basándose en procesos de entrenamiento y también
en las fases posteriores de explotación. Debido a sus propiedades, el
aprendizaje automático se ha convertido en un campo de alta importancia,
aplicándose a multitud de campos como medicina, automoción, visión
artificial\ldots Los tipos de aprendizaje automático se suelen clasificar en los
siguientes: aprendizaje supervisado, aprendizaje no supervisado y aprendizaje
por refuerzo. Sin embargo, aparece una nueva disciplina que se encuentra a
caballo entre el supervisado y no supervisado (utiliza tanto datos etiquetados
como no etiquetados para el entrenamiento) \cite{vanEngelen2020}.

En la figura \ref{fig:taxonomia} se puede ver una
clasificación de aprendizaje automático.

\imagen{taxonomia}{Clasificación de aprendizaje automático \cite{neova:taxonomy}.}{1}


\subsection{Aprendizaje supervisado}

El aprendizaje supervisado es una de las aproximaciones del aprendizaje
automático. Los algoritmos de aprendizaje supervisado son entrenados con datos
que han sido etiquetados para una salida concreta \cite{david:sl}. Por ejemplo,
dadas unas biopsias de pacientes, una posible etiqueta es si padecen de cáncer o
no. Estos datos tienen una serie de características (e.g. en el caso de una
biopsia se tendría la edad, tamaño tumoral, si ha tenido lugar mitosis o no...)
y todas ellas pueden ser binarias, categóricas o continuas \cite{salim:sl}.

Es común que antes del entrenamiento, estos datos son particionados en: conjunto de
entrenamiento, conjunto de test o conjunto de validación. De forma resumida, el
conjunto de entrenamiento serán los datos que utilice el propio algoritmo para
aprender y generalizar los comportamientos ocultos de los mismos. El conjunto de
validación se utilizará para tener un control de que el modelo está
generalizando y no sobreajustando (memorizando los datos) y por último, el
conjunto de test sirve para estimar el rendimiento real que podrá tener el
modelo en explotación \cite{enwiki:conjuntos}. En la figura
\ref{fig:aprendizajesupervisado} puede visualizarse el funcionamiento general.

\imagen{aprendizajesupervisado}{Funcionamiento general del aprendizaje supervisado \cite{salim:sl}.}{1}

El aprendizaje supervisado está altamente influenciado por esto. Por un lado, si
las etiquetas son categóricas o binarias el modelo será de
\textbf{clasificación} y por otro, si las etiquetas son continuas el modelo será
de \textbf{regresión}.

\begin{itemize}
    \item \textbf{Clasificación}: Los algoritmos de clasificación, a veces
    denominados simplemente como clasificadores, tratan de predecir la clase de
    una nueva entrada a partir del entrenamiento previo realizado. Estas clases
    son discretas y en clasificación pueden referirse a clases (o etiquetas)
    binarias o clases múltiples.
    
    \item \textbf{Regresión}: En este caso, el algoritmo asigna un valor
    continuo a una entrada. Es decir, trata de encontrar una función continua
    basándose en las variables de entrada. Se denomina también ajuste de
    funciones.
\end{itemize}

\clearpage

\subsection{Aprendizaje no supervisado}

A diferencia del aprendizaje supervisado, en el no supervisado, los algoritmos
no se nutren de datos etiquetados. En otras palabras, los usuarios no "<supervisan"> el modelo
\cite{salim:usl}. Esto quiere decir que no aprenderán de etiquetas, sino de la
propia estructura que se encuentre en los datos (patrones). Por ejemplo, dadas
unas imágenes de animales, sin especificar cuál es cuál, el aprendizaje no
supervisado identificará las similitudes entre imágenes y como resultado podría
dar la separación de las especies (o separaciones entre colores, pelaje,
raza...).

Como principales usos del aprendizaje no supervisado, suele aplicarse a:
\vspace{-4px}
\begin{enumerate}
    \item \textbf{Agrupamiento (Clustering)}: Este modelo de aprendizaje no
    supervisado trata de dividir los datos en grupos. Para ello, estudia las
    similitudes entre ellos y también en las disimilitudes con otros. Estos
    modelos pueden tanto descubrir por ellos mismos los "<clústeres"> o grupos
    que se encuentran o indicarle cuántos debe identificar \cite{salim:usl}.
    \item \textbf{Reducción de la dimensionalidad}: Para empezar, el término
    "<dimensionalidad"> hace referencia al número de variables de entrada que
    tienen los datos. En la realidad, los conjuntos de datos sobre los que se
    trabaja suelen tener una dimensionalidad grande. Según
    \cite{javatpoint:reduccionsdims} la reducción de dimensionalidad se denomina
    como "<Una forma de convertir conjuntos de datos de alta dimensionalidad en
    conjunto de datos de menor dimensionalidad, pero garantizando que proporciona
    información similar">. Es decir, simplificar el problema pero sin perder
    toda esa estructura interesante de los datos. Algunos ejemplos pueden ser:
    \begin{itemize}
        \item Análisis de Componentes Principales (PCA)
        \item Cuantificación vectorial
        \item Autoencoders
    \end{itemize}
\end{enumerate}

\imagenconurl{clustering}{Clusters}{\footnotesize{Clusters. Ejemplo de
clustering, a la izquierda los datos no etiquetados y a la derecha los datos
coloreados según las clases identificadas por el algoritmo de clustering. By
hellisp - Own work, Public Domain,
\url{https://commons.wikimedia.org/w/index.php?curid=36929773}. }}{0.5} 
\subsection{Aprendizaje semi-supervisado}

Según \cite{vanEngelen2020}, el aprendizaje semi-supervisado es la rama del
aprendizaje automático referido al uso de datos tanto etiquetados como no
etiquetados simultáneamente para realizar tareas de aprendizaje. Se encuentra a
caballo  entre el aprendizaje supervisado y no supervisado. Concretamente, los
problemas donde más se aplica, y donde más investigación se realiza es en
clasificación. Los métodos semi-supervisados resultan especialmente útiles
cuando se tienen escasos datos etiquetados, que, aparte de ser una situación
común en problemas reales, el proceso de etiquetado es una labor compleja, que
consume tiempo y es costosa.

\subsubsection{Suposiciones}
El objetivo de usar datos no etiquetados es construir un clasificador que sea
mejor que el aprendizaje supervisado, en el que solo se tienen datos
etiquetados. Pero para que el aprendizaje semi-supervisado mejore a lo ya
existente, tiene una serie de suposiciones que han de cumplirse.

En primera instancia se dice que la condición necesaria es que la distribución
\textit{p(x)} del espacio de entrada contiene información sobre la distribución
posterior \textit{p(y|x)} \cite{vanEngelen2020}.

Pero la forma en el que interactúan los datos de una distribución y la posterior,
no siempre es la misma:

\begin{tcolorbox}[colback=cyan!5!white,colframe=cyan!75!black,title=\textit{Smoothness assumption}]
    Esta suposición indica que si dos ejemplos (o instancias) de la entrada
    están cerca en ese espacio de entrada, entonces, probablemente, sus
    etiquetas sean las mismas.
\end{tcolorbox}

\begin{tcolorbox}[colback=cyan!5!white,colframe=cyan!75!black,title=\textit{Low-density assumption}]
    Esta suposición indica que en clasificación, los límites de decisión deben
    encontrarse en zonas en las que haya pocos de estos ejemplos (o instancias).
\end{tcolorbox}

\begin{tcolorbox}[colback=cyan!5!white,colframe=cyan!75!black,title=\textit{Manifold assumption}]
    Los datos pueden tener una dimensionalidad alta (muchas características)
    pero generalmente no todas las características son completamente útiles. Los
    datos a menudo se encuentran en unas estructuras de más baja
    dimensionalidad. Estas estructuras se conocen como "<manifolds">. Esta
    suposición indica que si los datos del espacio de entrada se encuentran en
    estas "<manifolds"> entonces aquellos puntos que se encuentren en el mismo
    "<manifolds"> tendrán la misma etiqueta. \cite{towardsdatascience:semi,vanEngelen2020}
\end{tcolorbox}

\begin{tcolorbox}[colback=cyan!5!white,colframe=cyan!75!black,title=\textit{Cluster assumption}]
    Como generalización de las anteriores, aquellos datos que se encuentren en
    un mismo clúster tendrán la misma etiqueta.
\end{tcolorbox}


De estas suposiciones se extrae el concepto de "<similitud"> en el que en todas
ellas se encuentra presente. Y en realidad, todas son versiones de
\textit{Cluster assumption} en la que los puntos similares tienden a pertenecer
al mismo grupo. 

Además, la suposición de clúster resulta necesaria para que el aprendizaje
semi-supervisado mejore al supervisado. Si los datos no pueden ser agrupados,
entonces no mejorará ningún método supervisado~\cite{vanEngelen2020}.


Para tener un punto de vista general, en la figura \ref{fig:aprendizajesemisupervisado} se presenta la
taxonomía de los métodos de aprendizaje semi-supervisado.

\imagen{aprendizajesemisupervisado}{Taxonomía de métodos semi-supervisados
\cite{vanEngelen2020}.}{1}

El núcleo de este proyecto está basado en los métodos inductivos. Su idea es muy
sencilla y está altamente relacionada con el objetivo del aprendizaje
supervisado, trata de crear un clasificador que prediga etiquetas para datos
nuevos. Por lo tanto, los algoritmos construidos tendrán este objetivo, aunque
con un punto más de concreción: los métodos "<wrapper"> o de envoltura.

Los conocidos métodos "<wrapper"> se basan en el \textit{pseudo etiquetado}
("<pseudo-labelling">), es el proceso en el que los clasificadores entrenados
con datos etiquetados generan etiquetas para los no etiquetados. Una vez
completado este proceso, el clasificador se vuelve a entrenar pero añadiendo
estos nuevos datos.

La gran ventaja que suponen estos métodos es que pueden utilizarse con casi
todos los clasificadores (supervisados) existentes \cite{vanEngelen2020}.

\paragraph{Self-Training}
Se trata del método de aprendizaje semi-supervisado más sencillo y "<directo">.
Este método envuelve un único clasificador base, que entrena con los datos
etiquetados iniciales y aprovecha el proceso de pseudo etiquetado comentado para
continuar su entrenamiento.

El método comienza por entrenar ese clasificador con los datos etiquetados que
se tienen. A partir de este aprendizaje, se etiqueta el resto de datos. De todas
las nuevas predicciones se seleccionan aquellas que parecen haber acertado con
mayor probabilidad. Una vez seleccionados, el clasificador es reentrenado con la
unión los ya etiquetados y estos nuevos. El proceso continúa hasta un criterio
de parada (generalmente hasta etiquetar todos los datos o un número máximo de
iteraciones).

En este proceso, el paso más importante es la incorporación de nuevos datos al
de conjunto de etiquetados, porque, \textbf{probablemente}, la predicción sea la
correcta. Es importante entonces que el cálculo de la probabilidad se realice
correctamente para asegurar que los nuevos datos son de interés. En caso
contrario, no es posible aprovechar los beneficios que ofrece self-training~
\cite{vanEngelen2020}. Todo este proceso queda descrito en el algoritmo
\ref{pseudo:self-training}.

\begin{algorithm}
    \DontPrintSemicolon
    \KwIn{Conjunto de datos etiquetados \textbf{\textit{L}}, 
    no etiquetados \textbf{\textit{U}} y clasificador \textbf{\textit{H}}}
    \KwOut{Clasificador}
     \While{$|U| \neq 0$}{
        Entrenar \textbf{\textit{H}} con \textbf{\textit{L}}\;
        Predecir etiquetas de \textbf{\textit{U}}\;
        Seleccionar un conjunto \textbf{\textit{T}} con aquellos datos que tenga la mayor probabilidad\;
        $\textbf{\textit{L}} = \textbf{\textit{L}} \cup \textbf{\textit{T}}$\;
        $\textbf{\textit{U}} = \textbf{\textit{U}} - \textbf{\textit{T}}$\;
     }
     Entrenar \textbf{\textit{H}} con \textbf{\textit{L}}\;
     \textbf{return} \textbf{\textit{H}}
     \caption{Self-Training}\label{pseudo:self-training}
\end{algorithm}

Sobre esta base, el algoritmo tiene muchas formas de diseñarse. En algunos casos
la condición de parada suele tomarse como un número máximo de iteraciones.
También, la cantidad de datos que se incorporan al conjunto \textbf{L} (con
mayor confianza) puede ser fija, o mediante un límite mínimo de
confianza/probabilidad (todas las instancias con mayor probabilidad se
añadirían).

\paragraph{Co-Training}
Basado fuertemente en Self-Training, en este caso \textbf{varios} clasificadores
(normalmente dos) se encargan del proceso e <<interactúan>> entre sí. Del mismo
modo, una vez entrenados predicen las etiquetas de los no clasificados y todos
los clasificadores añaden las mejores predicciones (mayor
confianza/probabilidad).

En \cite{blum1998combining}, Blum y Mitchel propusieron el funcionamiento básico
de Co-Training con dos vistas sobre los datos (<<multi-view>>). Estas vistas
corresponden no con subconjuntos de las instancias sino de subconjuntos de las
características de las mismas. Es decir, cada clasificador va a entrenarse
teniendo en cuenta características distintas. Idealmente estas vistas son
independientes y pueden predecir por sí solas la etiqueta (aunque no siempre se
cumplirá). Cuando los clasificadores predicen etiquetas sobre los datos se
seleccionan de ambos los de mayor confianza y construyen el nuevo conjunto de
entrenamiento para la siguiente iteración.


\begin{algorithm}
    \DontPrintSemicolon
    \KwIn{Conjunto de datos etiquetados \textbf{\textit{L}}, 
    no etiquetados \textbf{\textit{U}}, clasificadores \textbf{\textit{H\textsubscript{1}}}
    y \textbf{\textit{H\textsubscript{2}}}, \textit{p} (positivos), 
    \textit{n} (negativos), \textit{u} (datos iniciales), \textit{k} (iteraciones)}
    \KwOut{Clasificadores entrenados}
    Crear subconjunto \textbf{\textit{U'}} seleccionando \textit{u} instancias aleatorias de \textbf{\textit{U}}\;
    \For{k iteraciones}{
        Entrenar \textbf{\textit{H\textsubscript{1}}} con \textbf{\textit{L}}
        solo considerando un subconjunto (\textit{x\textsubscript{1}}) de las características de cada instancia (\textit{x})\;
        Entrenar \textbf{\textit{H\textsubscript{2}}} con \textbf{\textit{L}}
        solo considerando el otro subconjunto (\textit{x\textsubscript{2}}) de las características de cada instancia (\textit{x})\;

        Hacer que \textbf{\textit{H\textsubscript{1}}} prediga \textit{p} instancias positivas y \textit{n} negativas de \textbf{\textit{U'}} que tengan la mejor confianza\;
        Hacer que \textbf{\textit{H\textsubscript{2}}} prediga \textit{p} instancias positivas y \textit{n} negativas de \textbf{\textit{U'}} que tengan la mejor confianza\;
        Añadir estas instancias seleccionadas a \textbf{\textit{L}}\;
        Reponer \textbf{\textit{U'}} añadiendo 2p + 2n instancias de \textbf{\textit{U}}\;
     }
     \textbf{return} \textbf{\textit{H\textsubscript{1}}},\textbf{\textit{H\textsubscript{2}}}
     \caption{Co-Training}\label{pseudo:co-training}
\end{algorithm}

\clearpage
\paragraph{Democratic Co-Learning}

Yan Zhou y Sally Goldman presentan en \cite{zhou2004democratic} un algoritmo de
aprendizaje semi-supervisado en la línea de Co-Training (varios clasificadores)
pero que en el fondo poco tienen en común. La diferencia sustancial es que no
trabajan con dos (o más) conjunto de atributos (para que cada clasificador
utilice uno de ellos), en este caso solo se tiene un único conjunto de atributos
(<<single-view>>). Este algoritmo, y comentado por sus autores, tiene como
objetivo mejorar el <<accuracy>> de los algoritmos supervisados cuando se tiene
un pequeño conjunto de datos etiquetados y gran cantidad de no etiquetados.

Partiendo de los datos etiquetados, varios clasificadores realizan votación
ponderada sobre los no etiquetados. Es decir, la nueva etiqueta para la
instancia será la que vote la mayoría. Además, para aquellos clasificadores que
no votan como la mayoría se les impone esa etiqueta en su conjunto de
entrenamiento (se añade la instancia con esa etiqueta). Todo el proceso se
repite hasta que no se añadan más instancias a ningún conjunto de entrenamiento,
esto se alcanza cuando añadiendo nuevas etiquetas no se mejora la precisión.

\begin{algorithm}
    \DontPrintSemicolon
    \KwIn{Conjunto de datos etiquetados \textbf{\textit{L}}, 
    no etiquetados \textbf{\textit{U}} y algoritmos de aprendizaje \textbf{\textit{A\textsubscript{1}}},..., \textbf{\textit{A\textsubscript{n}}}
    }
    \For{i = 1,...,n}{
        $L_i = L$\;
        $e_i = L$\;
     }
     \Repeat{L\textsubscript{1},..., L\textsubscript{n} no cambien}{
        \For{i = 1,...,n}{
            Computar \textbf{\textit{H\textsubscript{i}}} entrenando \textbf{\textit{A\textsubscript{i}}} con \textbf{\textit{L\textsubscript{i}}}\;
        }

        \For{cada instancia no etiquetada $x \in U$}{
            \For{cada posible etiqueta j = 1,...,n}{
                $c_j = |\{H_i|H_i(x) = j\}|$
            }
            $k = arg~max_j\{c_j\}$
        }
        /* Instancias propuestas para etiquetar*/\;
        \For{i = 1,...,n}{
            Utilizar \textbf{\textit{L}} el intervalo de confianza al 95\%, [\textit{l\textsubscript{i}}, \textit{h\textsubscript{i}}] para \textbf{\textit{H\textsubscript{i}}}\;
            $w_i = (l_i + h_i)/2$
        }

        \For{i = 1,...,n}{
            $L'_i = 0 $\;
        }

        \If{$\sum_{H_j (x)=c_k} w\textsubscript{j} > max_{c'_k \neq c_k} \sum_{H_j (x)=c'_k w_j}$}{
            $L'_i = L'_i  ~\cup~ \{(x,c_k)\}, ~\forall i$ tal que $H_i(x) \neq c_k$
        }

        /* Estimar si añadir $L'_i$ a $L_i$ mejora la exactitud*/\;

        \For{i = 1,...,n}{
            Utilizar \textbf{\textit{L}} el intervalo de confianza al 95\%, [\textit{l\textsubscript{i}}, \textit{h\textsubscript{i}}] para \textbf{\textit{H\textsubscript{i}}}\;
            $q_i = |L_i|(1-2(\frac{e_i}{|L_i|})^2)$   /*Tasa de error*/\;
            $e'_i=(1-\frac{\sum_{i=1}^{d}l_i}{d})|L'_i|$  /*Nueva tasa de error*/\;
            $q'_i = |L_i ~\cup~ L'_i|(1-\frac{2(e_i~+~e'_i)}{|L_i ~\cup~ L'_i|})^2$\;

            \If{$q'_i > q_i$}{
                $L_i = L_i ~\cup~ L'_i$\;
                $e_i = e_i~+~e'_i$\;
            }
        }
        
     }
     \caption{Democratic Co-Learning}\label{pseudo:democraticco-learning}
\end{algorithm}

\begin{algorithm}
    \DontPrintSemicolon
    \KwIn{$\pmb{H}_1,\pmb{H}_2,...,\pmb{H}_n$ y espacio de instancias}
    \KwOut{Hipótesis combinadas (predicción)}
    \For{i = 1,...,n}{
            Utilizar \textbf{\textit{L}} el intervalo de confianza al 95\%, [\textit{l\textsubscript{i}}, \textit{h\textsubscript{i}}] para \textbf{\textit{H\textsubscript{i}}}\;
            $w_i = (l_i + h_i)/2$\;
    }
    \For{cada instancia $x$ en el espacio de instancias}{
        \For{i = 1,...,n}{
            \If{$H_i(x)$ predice $c_j$ y $w_i > 0.5$}{
                Añadir $H_i$ al grupo $G_j$ /* j es etiqueta */\;
            }
        }
        
        \For{j = 1,...,r}{
            $\bar{C}_{G_j} = \frac{|G_j|+0.5}{|G_j|+1} * \frac{\sum_{H_i \in G_j} w_i}{|G_j|}$
        }
    }
    H predice con el grupo $G_k$ con $k = arg~max_j(\bar{C}_{G_j})$\;
    \textbf{return} H
    \caption{Combinar}\label{pseudo:combinar}
\end{algorithm}

Aclaración sobre la predicción final (en combinación): una vez calculadas las confianzas de cada
instancia a predecir (x) y por cada posible etiqueta, la idea de la combinación
es obtener la etiqueta (k, posición en el grupo) con mayor confianza.
\capitulo{4}{Técnicas y herramientas}

Esta parte de la memoria tiene como objetivo presentar las técnicas metodológicas y las herramientas de desarrollo que se han utilizado para llevar a cabo el proyecto. Si se han estudiado diferentes alternativas de metodologías, herramientas, bibliotecas se puede hacer un resumen de los aspectos más destacados de cada alternativa, incluyendo comparativas entre las distintas opciones y una justificación de las elecciones realizadas. 
No se pretende que este apartado se convierta en un capítulo de un libro dedicado a cada una de las alternativas, sino comentar los aspectos más destacados de cada opción, con un repaso somero a los fundamentos esenciales y referencias bibliográficas para que el lector pueda ampliar su conocimiento sobre el tema.


Las secciones se incluyen con el comando section.

\section{Secciones}

\subsection{Subsecciones}

Además de secciones tenemos subsecciones.

\subsubsection{Subsubsecciones}

Y subsecciones. 


\section{Referencias}

Las referencias se incluyen en el texto usando cite \cite{wiki:latex}. Para citar webs, artículos o libros \cite{koza92}.


\section{Imágenes}

Se pueden incluir imágenes con los comandos standard de \LaTeX, pero esta plantilla dispone de comandos propios como por ejemplo el siguiente:

\imagen{escudoInfor}{Autómata para una expresión vacía}{.5}



\section{Listas de items}

Existen tres posibilidades:

\begin{itemize}
	\item primer item.
	\item segundo item.
\end{itemize}

\begin{enumerate}
	\item primer item.
	\item segundo item.
\end{enumerate}

\begin{description}
	\item[Primer item] más información sobre el primer item.
	\item[Segundo item] más información sobre el segundo item.
\end{description}
	
\begin{itemize}
\item 
\end{itemize}

\section{Tablas}

Igualmente se pueden usar los comandos específicos de \LaTeX o bien usar alguno de los comandos de la plantilla.

\tablaSmall{Herramientas y tecnologías utilizadas en cada parte del proyecto}{l c c c c}{herramientasportipodeuso}
{ \multicolumn{1}{l}{Herramientas} & App AngularJS & API REST & BD & Memoria \\}{ 
HTML5 & X & & &\\
CSS3 & X & & &\\
BOOTSTRAP & X & & &\\
JavaScript & X & & &\\
AngularJS & X & & &\\
Bower & X & & &\\
PHP & & X & &\\
Karma + Jasmine & X & & &\\
Slim framework & & X & &\\
Idiorm & & X & &\\
Composer & & X & &\\
JSON & X & X & &\\
PhpStorm & X & X & &\\
MySQL & & & X &\\
PhpMyAdmin & & & X &\\
Git + BitBucket & X & X & X & X\\
Mik\TeX{} & & & & X\\
\TeX{}Maker & & & & X\\
Astah & & & & X\\
Balsamiq Mockups & X & & &\\
VersionOne & X & X & X & X\\
} 
\capitulo{5}{Aspectos relevantes del desarrollo del proyecto}

En este apartado se van a comentar los aspectos más interesantes o que han
influido en el desarrollo. Por lo general, estos aspectos vendrán acompañados de
tomas de decisiones que se tuvieron que hacer, se argumentará y explicará el
desarrollo final de estos aspectos.

\section{Elección del proyecto}

La idea de este proyecto no fue propia, ambos tutores tenían en mente realizar
una aplicación enfocada a la docencia de algoritmos de aprendizaje
semi-supervisado. Durante los meses anteriores aparecieron diversos proyectos de
muy diferente índole que podía realizar. La realidad es que la inteligencia
artificial es un campo que me atrae mucho y que incluso me abre la mente a
proyectos personales. Dentro de la oferta de proyectos, el que iba en la línea
de lo que quería hacer era este. Al principio era un poco reacio a la idea de
realizar una aplicación tanto web como de escritorio. Demasiado enfocado en
ello, quería aplicar los algoritmos a algo concreto. Sin embargo, pensándolo
bien, el hecho de poder aprender algoritmos (e incluso la metodología que su
desarrollo conlleva de investigación y entendimiento) pero además, poder
desarrollar habilidades en web (o escritorio, aunque finalmente no ha sido así),
fue la clave para decantarme por el proyecto. Además, dado que disfruto mucho
<<cacharreando>>, los tutores también me comentaron que este proyecto podría
tenerme entretenido muchas horas para desarrollar ambas partes. Con todo ello,
parecía muy interesante y muy provechoso para mi aprendizaje elegir este
proyecto.

\section{Versión de Python}

Al comienzo del proyecto, se valoró la versión de Python en la que realizarlo.
En un principio parecía importante abarcar el mayor número de equipos en los que
el proyecto podía ser instalado y se pensó en alguna versión desde la 3.7. Sin
embargo, pese a que en la biblioteca de algoritmos que se iba a programar sí que
podía tener sentido aumentar los posibles usuarios, la realidad es que el
objetivo del final proyecto es una aplicación web completamente nueva. Se supone
además que, simulando un entorno completamente real y/o empresarial, el equipo
podría tener a su disposición servidor/es propio (o por lo menos,
configurables). En este sentido, dado que el usuario solo necesita acceder a la
Web, no tiene por qué conocer ni tener instalada una versión u otra. Por todo
esto, se pensó en utilizar versiones más recientes. 

En el momento de inicio del proyecto, la más reciente era la 3.10, esto también
es una ventaja en el medio plazo debido a que el periodo de actualizaciones y de
soporte para esta versión termina a finales de 2026 (mientras que algunas
anteriores finalizan en 2024 o incluso 2023).

\section{Utilidades para los algoritmos}

Antes de realizar la aplicación web, pero previendo lo que podía encontrarse.
Surgieron dos grandes problemas, el tratamiento de los datos que utilizan los
algoritmos (etiquetados, no etiquetados, particiones...) y por otro lado, cómo
ajustar estos datos y comunicarlos a la Web cuando los requiera.

No resultaba correcto vincular todos estos pasos en el código de los algoritmos,
ni tampoco en el de la propia aplicación. Tenía mucho más sentido crear unas
utilidades que actuasen de intermediarias en ciertos pasos del procedimiento.

Para el primer problema se crearon tres utilidades principales: 

Un <<particionador>> de datos que se encargara de dividir los datos en conjunto
de entrenamiento y test. Pero que además, si el conjunto estaba pensado para
aprendizaje supervisado, generase aleatoriamente datos no etiquetados.

Un codificador de etiquetas para transformar las etiquetas nominales en
numéricas (necesarias para los algoritmos). De base, el codificador de etiquetas
que propone SKlearn podría ser suficiente. Sin embargo, era necesario por un
lado no tratar los datos no etiquetados (internamente tratados como -1s) y por
el otro devolver de alguna forma las transformaciones realizadas Es decir, a qué
clase nominal corresponde cada número codificado.

Y por último, un cargador de conjuntos de datos (<<datasetloader>>) que
automatizara toda la lectura de un fichero ARFF o CSV, conversión a DataFrame y
las transformaciones de etiquetas (utilizando la utilidad anterior). Y que
además devuelva los datos de los atributos ($\mathbf{x}$) y por separado las
etiquetas ($y$). Es decir, un cargador cuyo resultado pueda ser introducido en los
algoritmos. 

Sobre el segundo problema (datos válidos para la aplicación web) se creó una
utilidad encargada de transformar el conjunto de datos a dos dimensiones. Esto
es, transformar los atributos (que pueden ser más de dos) a exactamente dos,
para poder representarlo en dos dimensiones. Sin embargo, cuando se avanzó un
poco en el desarrollo, era obvio que no interesaba modificar el conjunto de
datos, sino el resultado de la ejecución de la aplicación web. Fue cuando se
implementó y visualizó Self-Training cuando se pudo ver esta casuística.
Finalmente, lo que se hace es: transformar la estructura de datos que retorna la
ejecución de los algoritmos (consultar anexo D) que incluye todos los datos (y
todos sus atributos), a dos dimensiones (solo los atributos, no el <<target>> ni
el resto de columnas). El usuario puede seleccionar dos posibilidades, o
realizar PCA o seleccionar él mismo dos atributos del conjunto.

En ambos casos apareció la decisión de la estandarización. Al principio se
realizaba en ambos casos directamente, pero finalmente se permitía al usuario
elegir si estandarizar o no.

Como se puede ver, no se está normalizando sino estandarizando. Esta decisión es
debida, principalmente, al efecto de los \emph{outliers} (valores atípicos).
Un valor extremo en normalización puede hacer que la visualización sea muy
pobre, juntando la mayoría de los puntos a una zona. Con la estandarización este
efecto es mucho menor, pues se tiene en cuenta la media de los valores.

\clearpage
\section{Desarrollo de los algoritmos}

Junto con la aplicación web, este desarrollo es el más importante realizado. Ha
supuesto un verdadero reto documentarse con los artículos científicos de estos
algoritmos. Al fin y al cabo, no se tienen absolutamente todos los conocimientos
del aprendizaje y en muchas ocasiones se <<perdía>> más tiempo leyendo y
entendiendo que programando. Sí que es verdad que los algoritmos que se han
desarrollado tienen ideas bastante lógicas y directas.

Para desarrollar un algoritmo lo primero que se realizaba era leer y entender la
teoría que se presentaba en los artículos, consultando conceptos que no se
entendían en internet. En segundo lugar, en todos ellos siempre aparecía un
pseudocódigo que describía formalmente todo el proceso del algoritmo y que luego
se pasaba a la implementación. Exceptuando Self-Training, no era posible
entender su funcionamiento sin entender la teoría y la dificultad se encontraba
en que no se había trabajado con este tipo de notaciones particulares y era muy
fácil perderse.

Entrando más en cómo se desarrollaban, la idea siempre era crear un objeto con
ese algoritmo. De hecho, al principio del proyecto (con Self-Training) se
realizaba como una función única. Sin embargo, la idea de la orientación a
objetos permitía un manejo mucho más fino para crear funciones separadas
(entrenamiento, estadísticas, predicciones...) pero del mismo contexto y
facilitaban su uso en la parte de la aplicación Web (Flask).

Por lo general, se creaban unas versiones básicas en las que simplemente se
quería tener una implementación funcional. En esas primeras fases no se
consideraban <<refactorizaciones>> en cuanto a estructuras de datos más óptimas
o exceso de complejidad, sino que era una toma de contacto para ajustar el
pensamiento a cada uno de ellos en particular. Para probar estos algoritmos de
forma rudimentaria, se utilizaban los conjuntos de datos de <<juguete>> de
\texttt{Scikit-Learn}.

Cuando se pensaba que la implementación correspondía (en principio) a los
artículos, se comenzaban esas tareas de depuración. Por ejemplo, tanto en
Democratic Co-Learning como en Tri-Training se detectó que los tiempos de
entrenamiento eran demasiado grandes. En el primer caso, resultó ser porque la
condición de parada estaba mal planteada y ejecutaba muchas más iteraciones de
las necesarias. En el segundo caso, resultó ser precisamente el uso de
estructuras de datos más lentas (se subsanó utilizando \texttt{numpy}).

\subsection{Validación}

Aparte de intentar depurar el código generado, no era suficiente con la
intuición de que estos algoritmos eran correctos. Para comprobarlo, se ha
comparado con la biblioteca \texttt{sslearn}
~\cite{jose_luis_garrido_labrador_2023_7781117}. Esta biblioteca cubre todos los
algoritmos desarrollados en este proyecto y que además está mantenida
periódicamente.

Antes de presentar los resultados de esta validación, se definen las
estadísticas que se han utilizado para comprobar que la implementación propia es
comparable con \texttt{sslearn} y concluir que es correcta.

\begin{itemize}
    \item \textit{\textbf{Accuracy}}: Esta métrica será el núcleo principal de
    la comparación. Representa la proporción de aciertos entre todos los datos.

    \item \textit{\textbf{F1 score}}: Esta métrica se calcula como la media
    armónica de la precision\footnote{Precision: proporción de clasificados
    positivos (verdaderos positivos + falsos positivos) que son clasificados
    positivos (verdaderos positivos)} y \texttt{recall} (es igual que la
    sensibilidad).

    \begin{center}
        $ F_1 = 2 \cdot \frac{precision \cdot recall}{precision + recall} $
    \end{center}

    \item \textit{\textbf{Geometric mean (Gmean)}}: En este contexto, la media
    geométrica trata maximizar cada una de las medidas de \emph{accuracy} de
    cada clase~\cite{imbalanced_learn}.

    Para realizar su cálculo, se tiene en cuenta si se trata de clasificación
    binaria o multiclase:
    \begin{itemize}
        \item Binaria: la media geométrica se calcula como la raíz cuadrada del
        producto de la sensibilidad\footnote{Sensibilidad: proporción de
        positivos correctamente identificados (clasificados como positivos
        respecto a todos los positivos)~\cite{eswiki:145396343}} y la
        especificidad\footnote{Especificidad: proporción de negativos
        correctamente identificados (clasificados como negativos respecto a
        todos los negativos)~\cite{eswiki:145396343}}.

        \begin{center}
            $ Gmean = \sqrt{\textit{sensibilidad} \cdot \textit{especificidad}} $
        \end{center}
        

        \item Multiclase: la media geométrica se calcula como la raíz $n$-ésima
        del producto de las sensibilidades de cada clase.

        \begin{center}
            $Gmean = \sqrt[\leftroot{5} \uproot{5} n]{\textit{sensibilidad}_1 \cdot
            \textit{sensibilidad}_2 ~...~..~\cdot \textit{sensibilidad}_n }$
        \end{center}
    \end{itemize}

    El mejor valor es 1 y el peor 0.
\end{itemize}

La comparación consistirá en una validación cruzada con 10 \textit{folds} con
tres conjuntos de datos diferentes. Por cada iteración de la validación cruzada
se tendrá el 90\% de datos para entrenar y 10\% de test.

Sobre ese 90\% de entrenamiento, solo el 20\% va a estar etiquetados, el resto
serán no etiquetados. En conjunto, esos serán los datos para entrenar el
algoritmo.

Por cada uno de los conjuntos de datos se presenta un gráfico comparando cada
algoritmo propio con el de \texttt{sslearn} mediante gráficos de caja. En el eje
$X$ se presentan los algoritmos que se están comparando y en el eje $Y$ cada una de
las estadísticas antes comentadas.

Contenido de cada gráfico de cajas:
\vspace{-0.4cm}
\begin{itemize}
    \item \textbf{Línea continua azul}: representa la media.
    \item \textbf{Línea discontinua negra}: representa la mediana (segundo
    cuartil).
    \item \textbf{Borde superior de la caja coloreada}: representa el tercer cuartil.
    \item \textbf{Borde inferior de la caja coloreada}: representa el primer cuartil.
    \item \textbf{Límite del bigote superior}: representa el máximo de los valores
    encontrados.
    \item \textbf{Límite del bigote inferior}: representa el mínimo de los valores
    encontrados.
\end{itemize}

En la tabla~\ref{tabla:clasificadores} se pueden ver los clasificadores base
seleccionados para cada algoritmo. La selección es completamente arbitraria (con
base en los clasificadores que se han usado personalmente en el proyecto y fuera
de él).

\begin{table}[H]
    \centering
    \resizebox{\textwidth}{!}{%
    \begin{tabular}{l|l|l|l}
    \textbf{Self-Training}         & \textbf{Co-Training}                             & \textbf{Democratic Co-Learning}                                              & \textbf{Tri-Training}                      \\ \hline
                                   & Árbol de decisión & Vectores de Soporte C (SVC)  & Árbol de decisión \\
    Naive Bayes Gaussiano          & Naive Bayes Gaussiano & Naive Bayes Gaussiano & kNN \\
                                   &  & Árbol de decisión & Naive Bayes Gaussiano \\
    \end{tabular}%
    }
\caption{Clasificadores base}
\label{tabla:clasificadores}
\end{table}


\imagen{memoria/Breast}{Comparación de métodos con el conjunto de datos \textit{Breast Cancer}.}{1}

\imagen{memoria/Iris}{Comparación  de métodos con el conjunto de datos \textit{Iris}.}{1}

\imagen{memoria/Wine}{Comparación  de métodos con el conjunto de datos \textit{Wine}.}{1}


Por cada una de las figuras anteriores, lo interesante para comprobar si las
implementaciones pueden considerarse como correctas es fijarse en
\texttt{sslearn\footnote{Biblioteca de algoritmos semi-supervisados creada por
José Luis Garrido-Labrador~\cite{jose_luis_garrido_labrador_2023_7781117}}}.

Como se puede ver, en todos los casos los valores de la implementación propia
coinciden en gran medida con las de \texttt{sslearn}. Es de destacar que en los
casos de Self-Training, Democratic Co-Learning y Tri-Training se comportan de
manera muy similar. En el caso de Co-Training en todas hay más variabilidad.
Esto se debe a las diferencias de implementación.

Aunque haya pequeñas variaciones, los valores obtenidos son muy parecidos. Se
concluye por tanto, que los algoritmos implementados se consideran correctos.


\section{Desarrollo web}

Pese a que los algoritmos han sido una buena parte de todo el proyecto, el
desarrollo de la aplicación web es el que ha ocupado la mayor parte del tiempo
(de forma aproximada, se calcula que un 70\%). El por qué esto ha sido así
reside en dos cuestiones. 

Por un lado, el tamaño de la aplicación, que pese a no ser una aplicación
extremadamente grande, todo se ha realizado desde cero. No se adquirió ninguna
plantilla o estructura existente y la única ayuda utilizada fue
\texttt{Bootstrap} (que sí agiliza el estilado de la web). Y en segundo lugar,
por el desconocimiento del desarrollo web. Durante los estudios no se han dado
asignaturas de esta temática sea el desarrollo web. No ha sido hasta este mismo
curso cuando se nos ha introducido el mundo web (Diseño y Mantenimiento del
Software).

Durante todo el desarrollo, y hasta el final, se tomaron decisiones
constantemente. El \emph{framework} Flask fue el que se manipuló en esa
asignatura, era interesante seguir en esa línea y desarrollar el proyecto con
él. 

De hecho, al comenzar el proyecto, se tomó las estructura básica del proyecto
final de la asignatura dejando solo lo básico para arrancar una aplicación Web.

\subsection{JavaScript}
La toma de contacto con JavaScript fue compleja, no se había trabajado con este
lenguaje y solo se tenían conceptos básicos de HTML y CSS (junto con
\emph{backend}, que al ser Python, ya se tenía cierta soltura). Para intentar
avanzar lo máximo posible, se optó por ver tutoriales y
cursos\footnote{\url{https://www.w3schools.com/}} al mismo tiempo que ya se
iniciaban las visualizaciones de los algoritmos (completamente rudimentarias).
En este sentido, se iba a utilizar la librería \texttt{D3.js} que, comparándola
con otras, era algo más difícil de manejar con soltura.

Esta fue la línea general de desarrollo en JavaScript, en el momento que se
presentaba un problema o el uso de una nueva librería, ir a fondo con ello
directamente sobre la Web, hasta conseguir avances. Todo ello guiado por foros o
páginas como \textit{Stack Overflow}. Además, se optó por desarrollar la Web en
JavaScript <<Vanilla>>, sin librerías como jQuery. Se pensó que aprender las
bases de JavaScript era más provechoso.

Es destacable la cantidad de veces que hubo que refactorizar el código. Durante
el manejo y aprendizaje, aparecían mejores soluciones a lo que ya se tenía. Por
ejemplo, el uso de \emph{imports} permitía centralizar y reutilizar código (que
al principio se desconocía).

Aparte de todo esto, exceptuando las visualizaciones (que son bastante
particulares) y considerando todos los problemas que aparecían, el resto del
código de JavaScript no era extremadamente complejo y la mayoría de las veces
existían soluciones parciales que simplemente podían adaptarse. 

El avance, aunque algunas veces lento, era constante.

\subsection{Rediseño completo}

Uno de los puntos más destacables del desarrollo fue durante los sprints 13 y
14. La aplicación tuvo un rediseño completo (tantoen  estructura como
apariencia) partiendo desde el <<backend>> hasta el <<frontend>>.

Tal y como estaba la aplicación estructurada, todas las rutas estaban en un
único fichero, que a su vez era el mismo que instanciaba la aplicación. Esto,
para demos o aplicaciones muy pequeñas, es una solución rápida y buena. Sin
embargo, lo que se vio de otros proyectos es que cuando el tamaño es grande, es
necesario compartimentar rutas, modelos de bases de datos u otros elementos que
crecen en número. Tener absolutamente todo en un único fichero es engorroso.

Es por eso que se utilizó la idea de \textit{Application Factory} y
\textit{Blueprints}. Por un lado, \textit{Application Factory} ha permitido
hacer un único punto de creación de la aplicación donde se configura y se
desvincula del resto del código.

Los \textit{Blueprints} son como pequeñas aplicaciones independientes de Flask
(que luego son unificadas en una, en esa \textit{Application Factory}
comentada). En cada \textit{Blueprint} se han definido las rutas que intervienen
en cada contexto. Por ejemplo, para la configuración de los algoritmos, se han
definido sus rutas exclusivas, como si fuera una aplicación por sí sola. Esto
permite aplicaciones \textbf{mantenibles y escalables}.

Tal y como está al final, con la cantidad de rutas y utilidades codificadas,
hubiera sido imposible mantener un único fichero para toda la aplicación.

\subsection{Babel}

Un aspecto a destacar de la aplicación Web del proyecto es la
internacionalización. En principio, dado que no era necesario abarcar un abanico
grande de idiomas, la aplicación está pensada para Inglés y Español, con la
ventaja de que \texttt{Babel} hace muy fácil el manejo de las traducciones y es
perfectamente escalable. Desde las primeras fases de desarrollo se incluyó esta
forma de mantener unas traducciones de forma automática, y junto con el
mantenimiento de las traducciones actualizadas cada poco tiempo, la tarea no ha
resultado difícil, aunque sí laboriosa.

\section{Despliegue básico}

Añadido a los objetivos del proyecto, se quiso aprender cómo se despliega una
aplicación web cuando se dispone un servidor <<propio>> (comprado o alquilado).
En esta sección se va a comentar a grandes rasgos cómo se ha desplegado.

El servidor donde está establecida la aplicación se trata de un servicio
\textit{Cloud} exactamente igual a lo que sería tener un ordenador, pero en este
caso no tiene acceso físico, sino remoto.

La aplicación en sí (lanzada mediante la ejecución normal de Flask) no puede ser
accedida desde el exterior pues Flask se ejecuta en local (al igual que se ha
hecho en el desarrollo accediendo a \url{127.0.0.1}).

Para permitir el acceso desde el exterior, se ha establecido un \emph{proxy
inverso}. Este tipo de \textit{proxy} tiene dos ideas importantes, en primer
lugar, permite no conocer exactamente qué es lo que el servidor tiene
funcionando por dentro, y por otro, permite redirigir las peticiones que recibe
el servidor a las aplicaciones o puertos que se deseen (actúa como
\emph{servidor web}).

Esto se ha realizado con \texttt{Nginx}, que es fácil de configurar y es muy
usado. Como las peticiones que se realizan para acceder a una página web se
hacen sobre el puerto 80, en la configuración de \texttt{Nginx} se ha incluido
una entrada que redirecciona las peticiones de ese puerto (dirección IP del
servidor) a \url{localhost:5000} (que es donde se ha considerado ejecutar la
aplicación Flask).

Otro aspecto a destacar es que se ha adquirido el dominio \texttt{dmacha.dev}, y
se tiene también un \emph{certificado SSL} proporcionado por \textit{Let's
Encrypt}. Con el certificado <<instalado>> e indicándole a \texttt{Nginx} el
dominio (utilizar el dominio como si fuera IP del servidor), la aplicación es
completamente accesible desde internet.

Con todo lo anterior, existe la posibilidad de ejecutar el servidor propio de
Flask (que suele usarse en un entorno de desarrollo), y un servidor WSGI (Web
Server Gateway Interface\footnote{Indicará cómo comunicar \texttt{Nginx} (en
este caso) con la aplicación Flask.}) en entornos de producción. Para este
servidor WSGI se ha considerado \texttt{gunicorn} (\textit{green unicorn}), una
utilidad de Python que permite levantar aplicaciones como Flask. 

En una única ocasión con \texttt{gunicorn}, el tiempo de respuesta alcanzaba los
2 minutos o incluso no respondía (los procesos que manejaban las peticiones se
reiniciaban constantemente debido a la falta de recursos). De este modo, y como
no se obtuvieron más datos de lo ocurrido, se considera la posibilidad de
ejecutar directamente el servidor Flask para mantener el servicio en caso de
fallos, pero siempre con preferencia por WSGI.
\capitulo{6}{Trabajos relacionados}

Con el aprendizaje automático y el uso de gran cantidad de datos, es
completamente necesaria la visualización de los datos, procesos de entrenamiento
y ciertas estadísticas. Al fin y al cabo, cuando se construye un modelo, se debe
tener una realimentación de cómo de bien está funcionando para poder extraer
conclusiones sobre el mismo.

De forma general, sin centrarse directamente en el aprendizaje automático,
resulta interesante y conveniente que los visualizadores de algoritmos sean
accesibles y que, como están apareciendo, se creen aplicaciones Web que resultan
mucho más directas. Desde el punto de vista de la docencia y aprendizaje los
visualizadores permiten culminar la comprensión los aspectos teóricos
subyacentes.

\section{Visualizadores}

\subsubsection{Seshat} 
Es una herramienta web que trata de facilitar el aprendizaje de la teoría de
lenguajes y autómatas (análisis léxico) que utilizan los
compiladores~\cite{arnaiz2018seshat}, puede accederse desde
\url{http://cgosorio.es/Seshat}. La herramienta propone en primera instancia
unas explicaciones teóricas de los algoritmos con sus conceptos generales, el
concepto concreto de las expresiones regulares y qué es un autómata finito. A
partir de la teoría, se encuentran implementados varios algoritmos que se
visualizan paso a paso. En el momento de la ejecución también se tienen
elementos de interés como la propia descripción o explicación del algoritmo. Los
algoritmos implementados por la herramienta son:
\begin{enumerate}
    \item Construcción de un autómata finito no determinista (AFND) a partir de una expresión regular.
    \item Conversión de un autómata finito no determinista (AFND) a un autómata finito determinista (AFD).
    \item Construcción de un autómata finito determinista (AFD) a partir de una expresión regular.
    \item Minimización de un autómata finito.
\end{enumerate}
Para su construcción se ha usado el \emph{framework} Flask en Python que actúa
como servidor. La interfaz de usuario está construida con HTML, SVGs y
Javascript para proporcionar el contenido dinámico.

En este proyecto se realizó un estudio de la opinión de 42 estudiantes después
de su uso. El estudio consistía en unas afirmaciones (5) respecto a la buena
utilidad, buen diseño o el interés que puede producir al visualizar los
algoritmos de esa forma. De forma general, en una escala del 1 al 5 (donde el 5
representa que están profundamente de acuerdo con la afirmación) el 95\% de los
resultados se concentran entre el 4 y el 5 en la valoración. Esto indicó que la
herramienta sí que resultó útil e incluso ellos mismos solicitaban este recurso
para facilitar su aprendizaje.

\subsubsection{Herramienta de apoyo a la docencia de algoritmos de selección de instancias}
En~\cite{arnaiz2012herramienta} se presenta una herramienta de escritorio para
la visualización de la ejecución y resultados de los algoritmos de selección de
instancias, debido a la carencia de este tipo de aplicaciones para dichos
algoritmos. La herramienta es altamente personalizable pudiendo subir el
conjunto de datos o seleccionar qué característica visualizar en los ejes. En la
visualización se pueden ver todos los pasos de los algoritmos junto, por
ejemplo, a las regiones de Voronoi o el pseudocódigo. Esto hace que el alumno
pueda conocer el progreso y los conceptos particulares (influencia,
vecindad...). Los algoritmos implementados por la herramienta son:
\begin{enumerate}
    \item Algoritmo Condensado de Hart (CNN)~\cite{CNNHart1968}.
    \item Algoritmo Condensado Reducido (RNN)~\cite{RNNGates1972}.
    \item Algoritmo Subconjunto Selectivo Modificado (MSS)~\cite{MSSBarandela2005}.
    \item Algoritmos Decremental Reduction Optimization Preocedure (DROP)~\cite{DROPWilson2000}.
    \item Algoritmo Iterative Case Filtering (ICF)~\cite{ICFBrighton2002}.
    \item Algoritmo Democratic Instance Selection (DIS)~\cite{DemoISGarcia2010}.
\end{enumerate}
Está desarrollado completamente en Java lo que lo hace portable a cualquier
plataforma y sin instalación.

Como punto característico a tener en cuenta, la herramienta estaba muy orientada
a ofrecer bastante granularidad en cada paso, de tal forma que mediante, por
ejemplo, la visualización del pseudocódigo (al mismo tiempo que la visualización
gráfica), permite para analizar el comportamiento subyacente.

Los estudiantes, de forma general, valoraron muy positivamente la herramienta
pues les ayudó a comprender los algoritmos.

\subsubsection{Towards Developing an Effective Algorithm Visualization Tool for
Online Learning} 

En~\cite{8560314} se presenta una <<guía>> especialmente interesante con las
consideraciones a tener en cuenta para el desarrollo de una correcta herramienta
de visualización de algoritmos.


La realidad actual es que resulta muy difícil hacer que los alumnos entiendan
todos los conceptos que involucran los algoritmos y más aún en la modalidad "<a
distancia">. Para solventar este problema se han desarrollado múltiples
herramientas de visualización de algoritmos (AV) que pretenden facilitar la
compresión de los mismos, pero la realidad de estas herramientas es que se
desarrollan con una fuerte componente "<elegante"> y no enriquecedora o
pedagógica. Este artículo trata de abordar este último problema, el cómo
construir herramientas de visualización de algoritmos que sean divertidas y
atractivas, pero siendo fieles a los conceptos, y también cómo estas pueden
ayudar a los docentes a asegurar el aprendizaje de sus alumnos.

Para lograr el objetivo de desarrollar un visualizador efectivo, se analiza
desde el punto de vista de la pedagogía, usabilidad y accesibilidad. En la Tabla
~\ref{tabla:objetivospedagogicos} se pueden ver algunas de las características
que se deben tener en cuenta a la hora de desarrollar una herramienta de este
estilo.
\begin{table}[]
    \resizebox{\textwidth}{!}{%
    \begin{tabular}{ll}
    \hline
    \rowcolor[HTML]{C0C0C0} 
    \textbf{Característica}     & \textbf{Cómo lograrlo}   \\ \hline
    \rowcolor[HTML]{FFFFFF} 
    \textbf{\begin{tabular}[c]{@{}l@{}}Incrementar la\\comprensión\end{tabular}} & \begin{tabular}[c]{@{}l@{}}Explicaciones textuales\\ Realimentación con las acciones del usuario\\ Ayuda integrada\\ Comodidad en la entrada de datos\end{tabular}                                                              \\ \hline
    \textbf{\begin{tabular}[c]{@{}l@{}}Promocionar\\ interés\end{tabular}}     & \begin{tabular}[c]{@{}l@{}}Permitir introducir datos al algoritmo\\ Manipular la visualización\\ Capacidad de volver <<rebobinar>>\\ Variación de la velocidad\\ Avanzar en la ejecución\end{tabular}         \\ \hline
    \textbf{Facilidad de aprendizaje}                                          & \begin{tabular}[c]{@{}l@{}}Controles familiares\\ Generalidad con otros sistemas\\ Predictibilidad de las acciones\\ Interfaz simple\end{tabular}                                                                                  \\ \hline
    \textbf{Facilidad de memorización}                                         & Interfaz común que soporte múltiples animaciones                                                                                                                                                                                   \\ \hline
    \textbf{Facilidad de uso}                                                  & \begin{tabular}[c]{@{}l@{}}Interfaz común que soporte múltiples animaciones\\ Claridad de los modelos y conceptos\end{tabular}                                                                                                     \\ \hline
    \textbf{Robustez}                                                          & \begin{tabular}[c]{@{}l@{}}Explicaciones de la visualización\\ Integración de ajustes predefinidos\\ Recuperación de errores\end{tabular}                                                                                       \\ \hline
    \textbf{Eficacia}                                                          & \begin{tabular}[c]{@{}l@{}}Visibilidad del estado del sistema\\ Control y libertad del usuario\\ Prevención de errores\\ Flexibilidad de uso\\ Ayuda al usuario para reconocer, diagnosticar y recuperarse de errores\end{tabular} \\ \hline
    \rowcolor[HTML]{FFFFFF} 
    \textbf{Accesibilidad}                                                     & \begin{tabular}[c]{@{}l@{}}Globalmente disponible\\ Plataforma y dispositivo independiente\end{tabular}                                                                                                                            \\ \hline
    \end{tabular}%
    }
    \caption{Características que influyen en los objetivos pedagógicos.}
\label{tabla:objetivospedagogicos}
    \end{table}


\section{Otras herramientas/bibliotecas}

Algorithm-Visualizer, \url{https://algorithm-visualizer.org/}. Se trata de una
Web que incluye una cantidad enorme de algoritmos, todos están programados
directamente en Javascript que genera una visualización de los mismos y también
incluye una explicación corta de cada uno. Algunos de los algoritmos son:
\texttt{Backtracking}, \texttt{Divide and Conquer}, \texttt{Greedy} o más
concretamente \texttt{K-Means Clustering}, entre muchos otros.

\imagen{memoria/algorithm-visualizer}{K-Means Clustering en algorithm-visualizer.}{1}

VISUALGO, \url{https://visualgo.net/en}. Es una página web creada por la
Universidad Nacional de Singapur. Esta Web permite visualizar algoritmos básicos
(como de ordenación o camino más corto) pero también estructuras de datos (como
\emph{hash tables} o conjuntos disjuntos).

\imagen{memoria/visualgo}{Viajante de comercio en VISUALGO.}{1}

Visualizer, por Zhenbang Feng, \url{https://jasonfenggit.github.io/Visualizer/}
\footnote{ Github: \url{https://github.com/JasonFengGit/Visualizer}}. Su autor
lo define como una página web para proporcionar visualizaciones intuitivas e
innovadoras de algoritmos generales y de inteligencia artificial. Las
visualizaciones son muy atractivas, con las que se capta muy bien el
funcionamiento de los algoritmos. Implementa algoritmos de búsqueda de caminos,
ordenamiento e inteligencia artificial (como minimax).

\imagen{memoria/visualizer}{Mergesort en Visualizer (Zhenbang Feng).}{1}


Clutering-Visualizer, \url{https://clustering-visualizer.web.app/}. Es una
página Web para visualizar algoritmos de clustering. Realmente, no tiene muchos
implementados, pero las visualizaciones son muy descriptivas. En cada paso
muestra cuál es la evolución/posición de los distintos \textit{clusters} y mediante línea
es posible ver las distancias en ellos y los datos.

\imagen{memoria/Clutering-Visualizer}{K-Means en Clutering-Visualizer.}{1}


MLDemos, puede visitarse desde \url{https://basilio.dev/}. Es una herramienta de
visualización de código abierto para algoritmos de aprendizaje automático creada
específicamente para ayudar en el estudio y compresión del funcionamiento de una
gran cantidad de algoritmos. Entre ellos: clasificación, regresión, reducción de
dimensionalidad y clustering.

\imagen{memoria/MLDemos}{Clasificación mediante Support Vector Machine en MLDemos.}{1}

\capitulo{7}{Conclusiones y Líneas de trabajo futuras}

En esta sección final se presentan las conclusiones a las que se ha llegado con
la finalización del proyecto. La aplicación desarrollada tiene muchos puntos por
los que puede ser expandida, algunos sencillos y otros más grandes que ofrecen
posibilidades interesantes para los usuarios. Por esto último, también se
incluye una serie de líneas de trabajo para la mejora/expansión del proyecto.

\section{Conclusiones}

Una vez finalizado el proyecto, esta son las conclusiones que pueden extraerse:

\begin{itemize}
    \item Se ha desarrollado la biblioteca de algoritmos semi-supervisados
    personalizados, siendo validados mediante la comparación contra
    \texttt{sslearn} y obteniendo resultados satisfactorios suficientes. Con
    ella se ha podido extraer el proceso de entrenamiento de estos algoritmos
    para poder visualizarlos.
    \item El proyecto se ha culminado con la creación de una aplicación web
    completamente funcional que permite visualizar estos algoritmos. Además,
    también ofrece a los usuarios la posibilidad de registrarse y poder
    gestionar sus conjuntos de datos y ejecuciones.
    \item En líneas generales, la aplicación se ha desarrollado con la intención
    de ser completamente extensible. En este proyecto se han incluido cuatro
    algoritmos semi-supervisados, pero existen muchos más y se ha dejado puntos
    de extensión para ellos.
    \item Con la terminación del proyecto, se ha creado una documentación
    completa de todos los aspectos técnicos considerados, así como una manual
    para el programador y para el usuario. En estos manuales se ha incluido todo
    lo que se ha creído imprescindible para entender perfectamente la
    aplicación.
    \item Durante la realización del proyecto (algoritmos y web) se han
    utilizado muchos conceptos aprendidos en asignaturas este curso y durante
    cursos previos. Entre ellas es de destacar \emph{Bases de datos},
    \emph{Metodología de la programación} (orientación a objetos),
    \emph{Estructuras de datos}, \emph{Sistemas inteligentes},
    \emph{Algoritmia}, \emph{Diseño y Mantenimiento del software} y
    \emph{Minería de datos}.
    \item Se han tenido que aprender otros muchos conceptos nunca antes vistos.
    Principalmente, el aprendizaje se ha centrado en el desarrollo web, con
    aspectos como las peticiones, respuestas, códigos HTTP, seguridad... y
    también sobre minería de datos, particularizando sobre clasificación.
    \item En programación, se ha tenido que aprender el lenguaje de marcas HTML junto
    con JavaScript para generar todo el contenido dinámico. Aunque ha supuesto
    un verdadero reto, ha sido un proceso interesante y satisfactorio.
    \item La documentación se ha realizado en pasos constantes (aumentando el
    ritmo hasta el final). Y pese a que no se ha apurado a realizarlo al final,
    sí que ha conllevado mucho tiempo del proyecto, incluso más que, por
    ejemplo, el desarrollo de los algoritmos.
\end{itemize}




\bibliographystyle{plain}
\bibliography{bibliografia}

\end{document}
