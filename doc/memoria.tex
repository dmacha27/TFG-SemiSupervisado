\documentclass[a4paper,12pt,twoside]{memoir}

% Castellano
\usepackage[spanish,es-tabla]{babel}
\selectlanguage{spanish}
\usepackage[utf8]{inputenc}
\usepackage[T1]{fontenc}
\usepackage{lmodern} % Scalable font
\usepackage{microtype}
\usepackage{placeins}
\usepackage{dirtytalk}

\RequirePackage{booktabs}
\RequirePackage[table]{xcolor}
\RequirePackage{xtab}
\RequirePackage{multirow}

% Links
\PassOptionsToPackage{hyphens}{url}\usepackage[colorlinks]{hyperref}
\hypersetup{
	allcolors = {red}
}

% Ecuaciones
\usepackage{amsmath}
\usepackage[ruled,noline,linesnumbered,spanish]{algorithm2e}
\SetKwIF{If}{ElseIf}{Else}{if}{}{else if}{else}{end if}
\SetKwFor{For}{for}{}{endfor}
\SetKwFor{While}{while}{}{endwhile}

\newcommand{\bfit}[1]{\ensuremath{\textbf{\textit{#1}}}}

% Rutas de fichero / paquete
\newcommand{\ruta}[1]{{\sffamily #1}}

% Párrafos
\nonzeroparskip
\usepackage{tcolorbox}
\usepackage{minted}

% Huérfanas y viudas
\widowpenalty100000
\clubpenalty100000

% Imágenes

% Comando para insertar una imagen en un lugar concreto.
% Los parámetros son:
% 1 --> Ruta absoluta/relativa de la figura
% 2 --> Texto a pie de figura
% 3 --> Tamaño en tanto por uno relativo al ancho de página
\usepackage{graphicx}
\newcommand{\imagen}[3]{
	\begin{figure}[!ht]
		\centering
		\includegraphics[width=#3\textwidth]{#1}
		\caption{#2}\label{fig:#1}
	\end{figure}
	\FloatBarrier
}

\newcommand{\imagenconurl}[4]{
	\begin{figure}[!ht]
		\centering
		\includegraphics[width=#4\textwidth]{#1}
		\caption[#2]{#3}\label{fig:#1}
	\end{figure}
	\FloatBarrier
}

% Comando para insertar una imagen sin posición.
% Los parámetros son:
% 1 --> Ruta absoluta/relativa de la figura
% 2 --> Texto a pie de figura
% 3 --> Tamaño en tanto por uno relativo al ancho de página
\newcommand{\imagenflotante}[3]{
	\begin{figure}
		\centering
		\includegraphics[width=#3\textwidth]{#1}
		\caption{#2}\label{fig:#1}
	\end{figure}
}

% El comando \figura nos permite insertar figuras comodamente, y utilizando
% siempre el mismo formato. Los parametros son:
% 1 --> Porcentaje del ancho de página que ocupará la figura (de 0 a 1)
% 2 --> Fichero de la imagen
% 3 --> Texto a pie de imagen
% 4 --> Etiqueta (label) para referencias
% 5 --> Opciones que queramos pasarle al \includegraphics
% 6 --> Opciones de posicionamiento a pasarle a \begin{figure}
\newcommand{\figuraConPosicion}[6]{%
  \setlength{\anchoFloat}{#1\textwidth}%
  \addtolength{\anchoFloat}{-4\fboxsep}%
  \setlength{\anchoFigura}{\anchoFloat}%
  \begin{figure}[#6]
    \begin{center}%
      \Ovalbox{%
        \begin{minipage}{\anchoFloat}%
          \begin{center}%
            \includegraphics[width=\anchoFigura,#5]{#2}%
            \caption{#3}%
            \label{#4}%
          \end{center}%
        \end{minipage}
      }%
    \end{center}%
  \end{figure}%
}

%
% Comando para incluir imágenes en formato apaisado (sin marco).
\newcommand{\figuraApaisadaSinMarco}[5]{%
  \begin{figure}%
    \begin{center}%
    \includegraphics[angle=90,height=#1\textheight,#5]{#2}%
    \caption{#3}%
    \label{#4}%
    \end{center}%
  \end{figure}%
}
% Para las tablas
\newcommand{\otoprule}{\midrule [\heavyrulewidth]}
%
% Nuevo comando para tablas pequeñas (menos de una página).
\newcommand{\tablaSmall}[5]{%
 \begin{table}
  \begin{center}
   \rowcolors {2}{gray!35}{}
   \begin{tabular}{#2}
    \toprule
    #4
    \otoprule
    #5
    \bottomrule
   \end{tabular}
   \caption{#1}
   \label{tabla:#3}
  \end{center}
 \end{table}
}

%
% Nuevo comando para tablas pequeñas (menos de una página).
\newcommand{\tablaSmallSinColores}[5]{%
 \begin{table}[H]
  \begin{center}
   \begin{tabular}{#2}
    \toprule
    #4
    \otoprule
    #5
    \bottomrule
   \end{tabular}
   \caption{#1}
   \label{tabla:#3}
  \end{center}
 \end{table}
}

\newcommand{\tablaApaisadaSmall}[5]{%
\begin{landscape}
  \begin{table}
   \begin{center}
    \rowcolors {2}{gray!35}{}
    \begin{tabular}{#2}
     \toprule
     #4
     \otoprule
     #5
     \bottomrule
    \end{tabular}
    \caption{#1}
    \label{tabla:#3}
   \end{center}
  \end{table}
\end{landscape}
}

%
% Nuevo comando para tablas grandes con cabecera y filas alternas coloreadas en gris.
\newcommand{\tabla}[6]{%
  \begin{center}
    \tablefirsthead{
      \toprule
      #5
      \otoprule
    }
    \tablehead{
      \multicolumn{#3}{l}{\small\sl continúa desde la página anterior}\\
      \toprule
      #5
      \otoprule
    }
    \tabletail{
      \hline
      \multicolumn{#3}{r}{\small\sl continúa en la página siguiente}\\
    }
    \tablelasttail{
      \hline
    }
    \bottomcaption{#1}
    \rowcolors {2}{gray!35}{}
    \begin{xtabular}{#2}
      #6
      \bottomrule
    \end{xtabular}
    \label{tabla:#4}
  \end{center}
}

%
% Nuevo comando para tablas grandes con cabecera.
\newcommand{\tablaSinColores}[6]{%
  \begin{center}
    \tablefirsthead{
      \toprule
      #5
      \otoprule
    }
    \tablehead{
      \multicolumn{#3}{l}{\small\sl continúa desde la página anterior}\\
      \toprule
      #5
      \otoprule
    }
    \tabletail{
      \hline
      \multicolumn{#3}{r}{\small\sl continúa en la página siguiente}\\
    }
    \tablelasttail{
      \hline
    }
    \bottomcaption{#1}
    \begin{xtabular}{#2}
      #6
      \bottomrule
    \end{xtabular}
    \label{tabla:#4}
  \end{center}
}

%
% Nuevo comando para tablas grandes sin cabecera.
\newcommand{\tablaSinCabecera}[5]{%
  \begin{center}
    \tablefirsthead{
      \toprule
    }
    \tablehead{
      \multicolumn{#3}{l}{\small\sl continúa desde la página anterior}\\
      \hline
    }
    \tabletail{
      \hline
      \multicolumn{#3}{r}{\small\sl continúa en la página siguiente}\\
    }
    \tablelasttail{
      \hline
    }
    \bottomcaption{#1}
  \begin{xtabular}{#2}
    #5
   \bottomrule
  \end{xtabular}
  \label{tabla:#4}
  \end{center}
}



\definecolor{cgoLight}{HTML}{EEEEEE}
\definecolor{cgoExtralight}{HTML}{FFFFFF}

%
% Nuevo comando para tablas grandes sin cabecera.
\newcommand{\tablaSinCabeceraConBandas}[5]{%
  \begin{center}
    \tablefirsthead{
      \toprule
    }
    \tablehead{
      \multicolumn{#3}{l}{\small\sl continúa desde la página anterior}\\
      \hline
    }
    \tabletail{
      \hline
      \multicolumn{#3}{r}{\small\sl continúa en la página siguiente}\\
    }
    \tablelasttail{
      \hline
    }
    \bottomcaption{#1}
    \rowcolors[]{1}{cgoExtralight}{cgoLight}

  \begin{xtabular}{#2}
    #5
   \bottomrule
  \end{xtabular}
  \label{tabla:#4}
  \end{center}
}



\graphicspath{ {./img/} }

% Capítulos
\chapterstyle{bianchi}
\newcommand{\capitulo}[2]{
	\setcounter{chapter}{#1}
	\setcounter{section}{0}
	\setcounter{figure}{0}
	\setcounter{table}{0}
	\chapter*{#2}
	\addcontentsline{toc}{chapter}{#2}
	\markboth{#2}{#2}
}

% Apéndices
\renewcommand{\appendixname}{Apéndice}
\renewcommand*\cftappendixname{\appendixname}

\newcommand{\apendice}[1]{
	%\renewcommand{\thechapter}{A}
	\chapter{#1}
}

\renewcommand*\cftappendixname{\appendixname\ }

% Formato de portada
\makeatletter
\usepackage{xcolor}
\newcommand{\tutor}[1]{\def\@tutor{#1}}
\newcommand{\cotutor}[1]{\def\@cotutor{#1}}
\newcommand{\course}[1]{\def\@course{#1}}
\definecolor{cpardoBox}{HTML}{E6E6FF}
\def\maketitle{
  \null
  \thispagestyle{empty}
  % Cabecera ----------------
\noindent\includegraphics[width=\textwidth]{cabecera}\vspace{1cm}%
  \vfill
  % Título proyecto y escudo informática ----------------
  \colorbox{cpardoBox}{%
    \begin{minipage}{.8\textwidth}
      \vspace{.5cm}\Large
      \begin{center}
      \textbf{TFG del Grado en Ingeniería Informática}\vspace{.6cm}\\
      \textbf{\LARGE\@title{}}
      \end{center}
      \vspace{.2cm}
    \end{minipage}

  }%
  \hfill\begin{minipage}{.20\textwidth}
    \includegraphics[width=\textwidth]{escudoInfor}
  \end{minipage}
  \vfill
  % Datos de alumno, curso y tutores ------------------
  \begin{center}%
  {%
    \noindent\LARGE
    Presentado por \@author{}\\ 
    en Universidad de Burgos \\
    el \@date{}\\
    \begin{tabbing}
    Tutores: \= \kill
    Tutores: \> \@tutor{}\\
          \> \@cotutor{}\\
    \end{tabbing}
  }%
  \end{center}%
  \null
  \cleardoublepage
  }
\makeatother

\newcommand{\titulo}{Herramienta docente para la visualización en Web de algoritmos de aprendizaje Semi-Supervisado}
\newcommand{\nombre}{David Martínez Acha}
\newcommand{\tut}{Dr. Álvar Arnaiz González}
\newcommand{\cotut}{Dr. César Ignacio García Osorio}

% Datos de portada
\title{\titulo}
\author{\nombre}
\tutor{\tut}
\cotutor{\cotut}
\date{\today}

\begin{document}

\maketitle


\newpage\null\thispagestyle{empty}\newpage


%%%%%%%%%%%%%%%%%%%%%%%%%%%%%%%%%%%%%%%%%%%%%%%%%%%%%%%%%%%%%%%%%%%%%%%%%%%%%%%%%%%%%%%%
\thispagestyle{empty}


\noindent\includegraphics[width=\textwidth]{cabecera}\vspace{1cm}

\noindent El \tut, junto a el \cotut, profesores del departamento de Ingeniería
Informática, área de Lenguajes y Sistemas informáticos.

\noindent Exponen:

\noindent Que el alumno D. \nombre, con DNI 71310644H, ha realizado el Trabajo
final de Grado en Ingeniería Informática titulado \titulo.


\noindent Y que dicho trabajo ha sido realizado por el alumno bajo la dirección
del que suscribe, en virtud de lo cual se autoriza su presentación y defensa.

\begin{center} %\large
En Burgos, {\large \today}
\end{center}

\vfill\vfill\vfill

% Author and supervisor
\begin{minipage}{0.45\textwidth}
\begin{flushleft} %\large
Vº. Bº. del Tutor:\\[2cm]
\tut
\end{flushleft}
\end{minipage}
\hfill
\begin{minipage}{0.45\textwidth}
\begin{flushleft} %\large
Vº. Bº. del co-tutor:\\[2cm]
\cotut
\end{flushleft}
\end{minipage}
\hfill

\vfill

% para casos con solo un tutor comentar lo anterior
% y descomentar lo siguiente
%Vº. Bº. del Tutor:\\[2cm]
%D. nombre tutor


\newpage\null\thispagestyle{empty}\newpage




\frontmatter

% Abstract en castellano
\renewcommand*\abstractname{Resumen}
\begin{abstract}
El aprendizaje semi-supervisado es realmente útil en un contexto real, debido a
la clara dificultad y coste de obtener datos etiquetados de calidad. Sin
embargo, estos algoritmos no suelen tenerse muy en cuenta, e incluso en los
contenidos docentes, suelen obviarse (centrándose en el aprendizaje supervisado
y/o no supervisado).

En este proyecto, se ha desarrollado una biblioteca con cuatro algoritmos
semi-supervisados: \emph{Self-Training}, \emph{Co-Training}, \emph{Democratic
Co-Learning} y \emph{Tri-Training}. El aspecto añadido de todos ellos es poder
registrar el proceso de entrenamiento en cuanto a cómo se infiere el
conocimiento sobre los datos no etiquetados y la extracción de estadísticas.

Como objetivo final, se ha desarrollado una aplicación web dedicada a la
visualización de este proceso de entrenamiento para ayudar a aprender el
funcionamiento de estos algoritmos.

Como objetivo secundario, la aplicación también incluye una gestión de usuarios
que les permite almacenar sus conjuntos de datos y sus ejecuciones previas
(además de un panel de administración). Aun así, la aplicación está
completamente pensada para un acceso gratuito sin necesidad de un
registro.

La aplicación se encuentra disponible en \mbox{\url{https://vass.dmacha.dev}}.
\end{abstract}

\renewcommand*\abstractname{Descriptores}
\begin{abstract}
aprendizaje automático, aprendizaje semi-supervisado, visualización de algoritmos, web
\end{abstract}

\clearpage

% Abstract en inglés
\renewcommand*\abstractname{Abstract}
\begin{abstract}
Semi-supervised learning is really useful in a real context, due to the clear
difficulty and cost to obtain quality labelled data. However, these algorithms
are not taken into account, and even in teaching contents, tend to be
ignored (focusing on supervised learning and/or unsupervised learning).

In this project, a four semi-supervised algorithm library has been developed
that includes: \emph{Self-Training}, \emph{Co-Training}, \emph{Democratic
Co-Learning} and \emph{Tri-Training}. The additional aspect of all of them is to
register the training process on how knowledge is inferred to the unlabelled
data and the extraction of statistics.

As final objective, a web application dedicated to the visualization of this
training process has been developped to help to learn how these algorithms work.

As a secondary objective, the application also includes a user management system
that allows them to store their datasets and previous runs (plus an admin
dashboard). Even so, the application has been conceived to be free access
without registration.

The application is available at \mbox{\url{https://vass.dmacha.dev}}.
\end{abstract}

\renewcommand*\abstractname{Keywords}
\begin{abstract}
machine learning, semi-supervised learning, algorithm visualization, web
\end{abstract}

\clearpage

% Indices
\tableofcontents

\clearpage

\listoffigures

\clearpage

\listoftables
\clearpage

\mainmatter
\capitulo{1}{Introducción}

El ámbito del aprendizaje automático <<machine learning>> es un campo muy
interesante y que cada vez recibe más atención. La realidad es que la mayor
parte del conocimiento está muy centrado en dos tipos de aprendizaje automático:
el supervisado y el no supervisado. En cuanto se indaga un poco en <<machine
learning>> aparecen estos dos conceptos. Pero del que no se oye tanto, y puede
ser muy beneficioso, es el aprendizaje semi-supervisado. 

El aprendizaje supervisado, en pocas palabras, permite aprovechar situaciones en
las que se sabe qué representa un dato (por ejemplo, dado una animal, se sabe si
el animal es un perro o un pato), el no supervisado no tiene esta <<suerte>>, se
utiliza en casos en los que no se tiene ese conocimiento, sino que es él mismo
el que intenta extraer las representaciones (por ejemplo, para un conjunto de
animales, podría distinguir entre los que tiene pico y alas y los que tienen
cuatro patas sin necesidad de saber qué animal concreto es). En la realidad
(obviando los ejemplos tan sencillos comentados), el <<etiquetado>> de los datos
suele ser muy costoso y se tienen muchos más datos que no se sabe qué
representan, el aprendizaje semi-supervisado es el ideal en estos casos dado que
permite inferir conocimiento para estos últimos y determinar a qué corresponden.

Centrando más el objetivo final de este trabajo, no nos consta que existan
aplicaciones que permitan compaginar la teoría de estos conceptos con
visualizaciones interesantes que ayuden a comprender su funcionamiento y mucho
menos, para los algoritmos semi-supervisados que incluso en muchos casos suelen
obviarse. Vista la carencia en este ámbito, este trabajo pretende crear una
aplicación amigable y atractiva que permita, mediante visualizaciones, facilitar
la comprensión de cómo funcionan realmente los principales algoritmos
semi-supervisados cuando se compagina con los conceptos teóricos.

Las herramientas de fácil acceso, como las páginas Web que no requieren de
instalación por parte del usuario, van de la mano de la globalización de
internet. Es por ello que esta herramienta, categorizada ya como
\textit{docente}, será accesible desde internet. La idea de esto es permitir a
los usuarios la rapidez y facilidad de acceder simplemente a una <<URL>> sin
necesidad de herramientas auxiliares.

Además, los datos de las visualizaciones serán proporcionados por una biblioteca
propia donde se implementen estos algoritmos. Estarán adaptados para la
obtención de la información del entrenamiento y estadísticas relevantes, pero en
su caso también están pensados para que puedan ser utilizados de forma general
independientemente de que no haya una visualización posterior a su utilización.



\capitulo{2}{Objetivos del proyecto}

Este apartado explica de forma precisa y concisa cuáles son los objetivos que se
persiguen con la realización del proyecto.


Los \textbf{objetivos generales} del proyecto son los que siguen:
\begin{enumerate}
    \item Implementación de una librería con 4 algoritmos de aprendizaje semi-supervisado más comunes (en el momento del inicio no estaban elegidos)
    \item Diseño y creación de una aplicación Web desde la accesibilidad y enfoque docente (ayudas contextuales, explicaciones, pseudocódigos, instrucciones...)
    \item Integración de los 4 algoritmos en la aplicación Web para su visualización.
    \item Crear un sistema de usuarios que les permita controlar sus ficheros y ejecuciones.
    \item Internacionalización de la Web.
\end{enumerate}

Como \textbf{requisitos técnicos} y más particulares del proyecto se encuentran los
siguientes:

\begin{enumerate}
    \item Objetivos técnicos
    \item Implementación de los algoritmos en Python 3.10.
    \item Utilización del framewrok Flask para el desarrollo de la aplicación Web.
    \item Diseño de la Web basado en Bootstrap 5.
    \item Creación de las visualizaciones mediante JavaScript y la librería D3.js.
    \item Optimizar al máximo posible el procesamiento de datos y entrenamiento.
    \item Internacionalización mediante Babel (Flask-Babel).
    \item Instanciación y manejo de una base de datos para usuarios (y relativos) con independencia de la tecnología utilizada (SQLAlchemy).
    \item Realización de pruebas sobre el software desarrollado (comparativa/validación con otras librerías y
    test unitarios mediante pytest).
    \item Crear una documentación de usuario y programador precisa y completa.
\end{enumerate}
\capitulo{3}{Conceptos teóricos}

En esta sección se presentarán los principales conceptos teóricos del proyecto.
Estos incluyen el aprendizaje automático y más específicamente el aprendizaje
semi-supervisado.

\section{Aprendizaje automático}

Según \cite{intelligent:ml}, el aprendizaje automático (\textit{machine
learning}) es una rama de la Inteligencia Artificial como una técnica de
análisis de datos que enseña a las computadoras a aprender de la
\textbf{experiencia} (es decir, lo que realizan los humanos). Para ello, el
aprendizaje automático se nutre de gran cantidad de datos (o los suficientes
para el problema concreto) que son procesados por ciertos algoritmos. Estos
datos son ejemplos (también llamados instancias o prototipos), \cite{pascual:ml}
mediante los cuales, los algoritmos son capaces de generalizar comportamientos
que se encuentran ocultos. 

La característica principal de estos algoritmos es que son capaces de mejorar su
rendimiento de forma automática basándose en procesos de entrenamiento y también
en las fases posteriores de explotación. Debido a sus propiedades, el
aprendizaje automático se ha convertido en un campo de alta importancia,
aplicándose a multitud de campos como medicina, automoción, visión
artificial... Los tipos de aprendizaje automático se suelen clasificar en los
siguientes: aprendizaje supervisado, aprendizaje no supervisado y aprendizaje
por refuerzo. Sin embargo, aparece una nueva disciplina que se encuentra a
caballo entre el supervisado y no supervisado (utiliza tanto datos etiquetados
como no etiquetados para el entrenamiento) \cite{vanEngelen2020}.

En la figura \ref{fig:memoria/taxonomia} se puede ver una
clasificación de aprendizaje automático.

\imagen{memoria/taxonomia}{Clasificación de aprendizaje automático \cite{neova:taxonomy}.}{1}


\subsection{Aprendizaje supervisado}

El aprendizaje supervisado es una de las aproximaciones del aprendizaje
automático. Los algoritmos de aprendizaje supervisado son entrenados con datos
que han sido etiquetados para una salida concreta \cite{david:sl}. Por ejemplo,
dadas unas biopsias de pacientes, una posible etiqueta es si padecen de cáncer o
no. Estos datos tienen una serie de características (e.g. en el caso de una
biopsia se tendría la edad, tamaño tumoral, si ha tenido lugar mitosis o no...)
y todas ellas pueden ser binarias, categóricas o continuas \cite{salim:sl}.

Es común que antes del entrenamiento, estos datos sean particionados en:
conjunto de entrenamiento, conjunto de test y conjunto de validación. De forma
resumida, el conjunto de entrenamiento serán los datos que utilice el propio
algoritmo para aprender y generalizar los comportamientos ocultos de los mismos.
El conjunto de validación se utilizará para tener un control de que el modelo
está generalizando y no sobreajustando (memorizando los datos) y también para
decidir cuando finalizar el entrenamiento. Por último, el conjunto de test sirve
para estimar el rendimiento real que podrá tener el modelo en explotación
\cite{enwiki:conjuntos}. En la figura \ref{fig:memoria/aprendizajesupervisado}
puede visualizarse el funcionamiento general.

\imagen{memoria/aprendizajesupervisado}{Funcionamiento general del aprendizaje supervisado \cite{salim:sl}.}{1}

El aprendizaje supervisado está altamente influenciado por esto. Por un lado, si
el valor a predecir es uno entre un conjunto finito, el modelo será de
\textbf{clasificación} y por otro, si el valor a predecir es un valor continuo,
el modelo será de \textbf{regresión}.

\begin{itemize}
    \item \textbf{Clasificación}: Los modelos de clasificación (generados a
    partir de algoritmos de aprendizaje), a veces denominados simplemente como
    clasificadores, tratan de predecir la clase de una nueva entrada a partir
    del entrenamiento previo realizado. Estas clases son discretas y en
    clasificación pueden referirse a clases (o etiquetas) binarias o clases
    múltiples.
    
    \item \textbf{Regresión}: En este caso, el modelo asigna un valor continuo a
    una entrada. Es decir, trata de encontrar una función continua basándose en
    las variables de entrada. Se denomina también ajuste de funciones.
\end{itemize}

\clearpage

\subsection{Aprendizaje no supervisado}

A diferencia del aprendizaje supervisado, en el no supervisado, los algoritmos
de aprendizaje no se nutren de datos etiquetados. En otras palabras, los
usuarios no "<supervisan"> el algoritmo \cite{salim:usl}. Esto quiere decir que
no aprenderán de etiquetas, sino de la propia estructura que se encuentre en los
datos (patrones). Por ejemplo, dadas unas imágenes de animales, sin especificar
cuál es cuál, el aprendizaje no supervisado identificará las similitudes entre
imágenes y como resultado podría dar la separación de las especies (o
separaciones entre colores, pelaje, raza...).

Como principales usos del aprendizaje no supervisado, suele aplicarse a:
\vspace{-4px}
\begin{enumerate}
    \item \textbf{Agrupamiento (Clustering)}: Este tipo de algoritmo de
    aprendizaje no supervisado trata de dividir los datos en grupos. Para ello,
    estudia las similitudes entre ellos y también en las disimilitudes con
    otros. Estos algoritmos pueden tanto descubrir por ellos mismos los
    <<clústeres>> o grupos que se encuentran o indicarle cuántos debe
    identificar \cite{salim:usl}.
    \item \textbf{Reducción de la dimensionalidad}: Para empezar, el término
    "<dimensionalidad"> hace referencia al número de variables de entrada que
    tienen los datos. En la realidad, los conjuntos de datos sobre los que se
    trabaja suelen tener una dimensionalidad grande. Según
    \cite{javatpoint:reduccionsdims} la reducción de dimensionalidad se denomina
    como: \begin{quote}<<Una forma de convertir conjuntos de datos de alta dimensionalidad en
    conjunto de datos de menor dimensionalidad, pero garantizando que proporciona
    información similar.>>\end{quote} Es decir, simplificar el problema pero sin perder
    toda esa estructura interesante de los datos. Algunos ejemplos pueden ser:
    \begin{itemize}
        \item Análisis de Componentes Principales (PCA).
        \item Cuantificación vectorial.
        \item Autoencoders.
    \end{itemize}
\end{enumerate}

\imagenconurl{memoria/clustering}{Clusters}{\footnotesize{\emph{Clusters}. Ejemplo de
agrupamiento, a la izquierda los datos no etiquetados y a la derecha los datos
coloreados según las clases identificadas por el algoritmo de clustering. By
hellisp - Own work, Public Domain,
\url{https://commons.wikimedia.org/w/index.php?curid=36929773}. }}{0.5} 
\subsection{Aprendizaje semi-supervisado}

Según \cite{vanEngelen2020}, el aprendizaje semi-supervisado es la rama del
aprendizaje automático referido al uso simultáneo de datos tanto etiquetados
como no etiquetados para realizar tareas de aprendizaje. Se encuentra a caballo
entre el aprendizaje supervisado y el no supervisado. Concretamente, los
problemas donde más se aplica, y donde más investigación se realiza es en
clasificación. Los métodos semi-supervisados resultan especialmente útiles
cuando se tienen escasos datos etiquetados, que, aparte de ser una situación
común en problemas reales, hacen que el proceso de etiquetado sea una labor
compleja, que consume tiempo y es costosa.

\subsubsection{Suposiciones}
El objetivo de usar datos no etiquetados es construir un clasificador que sea
mejor que el que se obtendría utilizando aprendizaje supervisado, donde solo se
tienen datos etiquetados. Pero para que el aprendizaje semi-supervisado mejore a
lo ya existente, tiene una serie de suposiciones que han de cumplirse.

En primera instancia se dice que la condición necesaria es que la distribución
$p(x)$ del espacio de entrada contiene información sobre la distribución
posterior $p(y|x)$ \cite{vanEngelen2020}.

Pero la forma en el que interactúan los datos de una distribución y la posterior,
no siempre es la misma:

\begin{tcolorbox}[colback=cyan!5!white,colframe=cyan!75!black,title=\textit{Smoothness assumption}]
    Esta suposición indica que si dos ejemplos (o instancias) de la entrada
    están cerca en ese espacio de entrada, entonces, probablemente, sus
    etiquetas sean las mismas.
\end{tcolorbox}

\begin{tcolorbox}[colback=cyan!5!white,colframe=cyan!75!black,title=\textit{Low-density assumption}]
    Esta suposición indica que en clasificación, los límites de decisión deben
    encontrarse en zonas en las que haya pocos de estos ejemplos (o instancias).
\end{tcolorbox}

\begin{tcolorbox}[colback=cyan!5!white,colframe=cyan!75!black,title=\textit{Manifold assumption}]
    Los datos pueden tener una dimensionalidad alta (muchas características)
    pero generalmente no todas las características son completamente útiles. Los
    datos a menudo se encuentran en unas estructuras de más baja
    dimensionalidad. Estas estructuras se conocen como <<\emph{manifolds}>>.
    Esta suposición indica que si los datos del espacio de entrada se encuentran
    en estas <<\emph{manifolds}>> entonces aquellos puntos que se encuentren en
    el mismo <<\emph{manifolds}>> tendrán la misma etiqueta.
    \cite{towardsdatascience:semi,vanEngelen2020}
\end{tcolorbox}

\begin{tcolorbox}[colback=cyan!5!white,colframe=cyan!75!black,title=\textit{Cluster assumption}]
    Como generalización de las anteriores, aquellos datos que se encuentren en
    un mismo clúster tendrán la misma etiqueta.
\end{tcolorbox}


De estas suposiciones se extrae el concepto de "<similitud"> que está presente
en todas ellas. Y en realidad, todas son versiones de la \textit{Cluster
assumption}, que dice que los puntos similares tienden a pertenecer al mismo
grupo. 

Además, la suposición de clúster resulta necesaria para que el aprendizaje
semi-supervisado mejore al supervisado. Si los datos no pueden ser agrupados,
entonces no mejorará ningún método supervisado~\cite{vanEngelen2020}.


Para tener un punto de vista general, en la figura \ref{fig:memoria/aprendizajesemisupervisado} se presenta la
taxonomía de los métodos de aprendizaje semi-supervisado.

\imagen{memoria/aprendizajesemisupervisado}{Taxonomía de métodos semi-supervisados
\cite{vanEngelen2020}.}{1}

El núcleo de este proyecto está basado en los métodos inductivos. Su idea es muy
sencilla y está altamente relacionada con el objetivo del aprendizaje
supervisado, trata de crear un clasificador que prediga etiquetas para datos
nuevos. Por lo tanto, los algoritmos construidos tendrán este objetivo, aunque
con un punto más de concreción: el proyecto se centrará en los métodos
<<\emph{wrapper}>> o de envoltura.

Los conocidos métodos <<\emph{wrapper}>> se basan en el \textit{pseudo
etiquetado} ("<pseudo-labelling">), es el proceso en el que los clasificadores
entrenados con datos etiquetados generan etiquetas para los no etiquetados. Una
vez completado este proceso, el clasificador se vuelve a entrenar pero añadiendo
estos nuevos datos. La gran ventaja que suponen estos métodos es que pueden
utilizarse con casi todos los clasificadores (supervisados) existentes
\cite{vanEngelen2020}.

\section{Self-Training}
Se trata del método de aprendizaje semi-supervisado más sencillo y <<directo>>.
Este método <<envuelve>> un único clasificador base, que entrena con los datos
etiquetados iniciales y aprovecha el proceso de pseudo etiquetado comentado para
continuar su entrenamiento.

El método comienza por entrenar ese clasificador con los datos etiquetados que
se tienen. A partir de este aprendizaje inicial, se etiqueta el resto de datos.
De todas las nuevas predicciones se seleccionan aquellas en las que el
clasificador más confianza tiene (de haber acertado). Una vez seleccionados, el
clasificador es reentrenado con la unión de los ya etiquetados y estos recién
etiquetados. El proceso continúa hasta que se verifica un criterio de parada
(generalmente hasta etiquetar todos los datos o un número máximo de
iteraciones).

En este proceso, el paso más importante es la incorporación de nuevos datos al
conjunto de etiquetados porque, \textbf{probablemente}, la predicción sea la
correcta. Es importante entonces que el cálculo de la probabilidad se realice
correctamente para asegurar que los nuevos datos son de interés. En caso
contrario, no es posible aprovechar los beneficios que ofrece self-training~
\cite{vanEngelen2020}. Todo este proceso queda descrito en el algoritmo
\ref{pseudo:self-training}.

\begin{algorithm}
    \DontPrintSemicolon
    \KwIn{Conjunto de datos etiquetados \bfit{L}, 
    no etiquetados \bfit{U} y clasificador \bfit{H}}
    \KwOut{Clasificador entrenado}
     \While{$|U| \neq 0$}{
        Entrenar \bfit{H} con \bfit{L}\;
        Predecir etiquetas de \bfit{U}\;
        Seleccionar un conjunto \bfit{T} con aquellos datos que tenga la mayor probabilidad\;
        $\bfit{L} = \bfit{L} \cup \bfit{T}$\;
        $\bfit{U} = \bfit{U} - \bfit{T}$\;
     }
     Entrenar \bfit{H} con \bfit{L}\;
     \textbf{return} \bfit{H}
     \caption{Self-Training}\label{pseudo:self-training}
\end{algorithm}

Sobre esta base, el algoritmo tiene muchas formas de diseñarse. En algunos casos
la condición de parada suele tomarse como un número máximo de iteraciones.
También, la cantidad de datos que se incorporan al conjunto \bfit{L}
(con mayor confianza) puede ser fija, o mediante un límite mínimo de
confianza/probabilidad (todas las instancias con mayor probabilidad se
añadirían).

\section{Co-Training}
Basado fuertemente en Self-Training, en este caso \textbf{varios} clasificadores
(normalmente dos) se encargan del proceso e <<interactúan>> entre sí. Del mismo
modo, una vez entrenados predicen las etiquetas de los no clasificados y todos
los clasificadores añaden las mejores predicciones.

En \cite{blum1998combining}, Blum y Mitchel propusieron el funcionamiento básico
de Co-Training con dos vistas sobre los datos (<<\emph{multi-view}>>). Estas
vistas corresponden no con subconjuntos de las instancias, sino con subconjuntos
de las características de las mismas. Es decir, cada clasificador va a
entrenarse teniendo en cuenta características distintas. Idealmente estas vistas
tendrían que ser independientes y servir por sí solas para predecir la etiqueta
(aunque no siempre se cumplirá). Cuando los clasificadores predicen etiquetas
sobre los datos, se seleccionan de ambos los de mayor confianza y se construye
el nuevo conjunto de entrenamiento para la siguiente iteración.


\begin{algorithm}
    \DontPrintSemicolon
    \KwIn{Conjunto de datos etiquetados $\pmb{L}$, 
    no etiquetados $\pmb{U}$, clasificadores $\pmb{H}_1$
    y $\pmb{H}_2$, $p$ (positivos), 
    $n$ (negativos), $u$ (número de datos iniciales), $k$ (iteraciones)}
    \KwOut{Clasificadores entrenados}
    Crear un subconjunto $\pmb{U'}$ seleccionando $u$ instancias aleatorias de $\pmb{U}$\;
    \For{k iteraciones}{
        Entrenar $\pmb{H}_1$ con $\pmb{L}$
        solo considerando un subconjunto ($\pmb{x}_1$) de las características de cada instancia ($x$)\;
        Entrenar $\pmb{H}_2$ con $\pmb{L}$
        solo considerando el otro subconjunto ($\pmb{x}_2$) de las características de cada instancia ($x$)\;

        Hacer que $\pmb{H}_1$ prediga $p$ instancias positivas y $n$ negativas de $\pmb{U'}$ que tengan la mayor confianza\;
        Hacer que $\pmb{H}_2$ prediga $p$ instancias positivas y $n$ negativas de $\pmb{U'}$ que tengan la mayor confianza\;
        Añadir estas instancias seleccionadas a $\pmb{L}$\;
        Reponer $\pmb{U'}$ añadiendo $2p + 2n$ instancias de $\pmb{U}$\;
     }
     \Return{$\pmb{H}_1,\pmb{H}_2$}
     \caption{Co-Training}\label{pseudo:co-training}
\end{algorithm}

\clearpage
\section{Democratic Co-Learning}

Yan Zhou y Sally Goldman presentaron en~\cite{zhou2004democratic} un algoritmo
de aprendizaje semi-supervisado en la línea del Co-Training (varios
clasificadores). La diferencia sustancial es que los clasificadores base no
trabajan con dos (o más) conjunto de atributos (para que cada clasificador
utilice uno de ellos), en este caso solo se tiene un único conjunto de atributos
(<<\emph{single-view}>>).

Partiendo de los datos etiquetados, varios clasificadores realizan votación
ponderada sobre los no etiquetados. Lo que quiere decir que, para una instancia,
su nueva etiqueta será la que vote la mayoría. Además, para aquellos
clasificadores que no votan como la mayoría, la instancia se añade a su conjunto
de entrenamiento junto a la etiqueta mayoritaria, de esta forma se <<obliga>> en
la siguiente iteración a <<aprenderla>>. Todo el proceso se repite hasta que no
se añadan más instancias a ningún conjunto de entrenamiento, esto ocurrirá
cuando no se mejora la precisión pese a la adición de nuevas instancias
pseudo-etiquetadas.

\begin{algorithm}
    \DontPrintSemicolon
    \KwIn{Conjunto de datos etiquetados $\pmb{L}$, 
    no etiquetados $\pmb{U}$ y algoritmos de aprendizaje $\pmb{A}_1$,..., $\pmb{A}_n$
    }
    \For{i = 1,...,n}{
        $L_i = L$\;
        $e_i = 0$\;
     }
     \Repeat{$L_1$,..., $L_n$ no cambien}{
        \For{i = 1,...,n}{
            Calcular $\pmb{H}_i$ entrenando $\pmb{A}_i$ con $\pmb{L}_i$\;
        }

        \For{cada instancia no etiquetada $x \in U$}{
            \For{cada posible etiqueta j = 1,...,n}{
                $c_j = |\{H_i|H_i(x) = j\}|$
            }
            $k = arg~max_j\{c_j\}$
        }
        /* Instancias propuestas para etiquetar */\;
        \For{i = 1,...,n}{
            Utilizar $\pmb{L}$ para calcular el intervalo de confianza al 95\%, [$l_i$,~$h_i$] de $\pmb{H}_i$\;
            $w_i = (l_i + h_i)/2$
        }

        \For{i = 1,...,n}{
            $L'_i = \emptyset$\;
        }

        \If{$\sum_{H_j (x)=c_k} w\textsubscript{j} > max_{c'_k \neq c_k} \sum_{H_j (x)=c'_k w_j}$}{
            $L'_i = L'_i  ~\cup~ \{(x,c_k)\}, ~\forall i$ tal que $H_i(x) \neq c_k$
        }

        /* Estimar si añadir $L'_i$ a $L_i$ mejora la exactitud */\;

        \For{i = 1,...,n}{
            Utilizar $\pmb{L}$ para calcular el intervalo de confianza al 95\%, [$l_i$,~$h_i$] de $\pmb{H}_i$\;
            $q_i = |L_i|(1-2(\frac{e_i}{|L_i|})^2)$   /* Tasa de error */\;
            $e'_i=(1-\frac{\sum_{i=1}^{d}l_i}{d})|L'_i|$  /* Nueva tasa de error */\;
            $q'_i = |L_i ~\cup~ L'_i|(1-\frac{2(e_i~+~e'_i)}{|L_i ~\cup~ L'_i|})^2$\;

            \If{$q'_i > q_i$}{
                $L_i = L_i ~\cup~ L'_i$\;
                $e_i = e_i~+~e'_i$\;
            }
        }
        
     }
     \Return{Combinar($\pmb{H}_1,\pmb{H}_2,...,\pmb{H}_n$)}
     \caption{Democratic Co-Learning --- entrenamiento}\label{pseudo:democraticco-learning}
\end{algorithm}

\begin{algorithm}
    \DontPrintSemicolon
    \KwIn{$\pmb{H}_1,\pmb{H}_2,...,\pmb{H}_n$ y $\mathbf{x}$ (instancia)}
    \KwOut{Hipótesis combinadas (predicción)}
    \For{i = 1,...,n}{
        Utilizar $\pmb{L}$ para calcular el intervalo de confianza al 95\%, [$l_i$,~$h_i$] de $\pmb{H}_i$\;
            $w_i = (l_i + h_i)/2$\;
    }
    \For{i = 1,...,n}{
        \If{$H_i(\mathbf{x})$ predice $c_j$ ~y ~$w_i >$ 0.5}{
            Añadir $H_i$ al grupo $G_j$ /* $j$ es etiqueta */\;
        }
    }
    
    \For{j = 1,...,r}{
        $\bar{C}_{G_j} = \frac{|G_j|+0.5}{|G_j|+1} * \frac{\sum_{H_i \in G_j} w_i}{|G_j|}$
    }
    $H$ predice con el grupo $G_k$ con $k = \arg \max_j(\bar{C}_{G_j})$\;
    \textbf{return} $H$
    \caption{Democratic Co-Learning --- predicción (combinación)}\label{pseudo:combinar}
\end{algorithm}

Aclaración sobre la predicción final (en combinación): una vez calculadas las
confianzas de la instancia a predecir ($\mathbf{x}$) y por cada posible etiqueta, la idea
de la combinación es obtener la etiqueta ($k$, posición en el grupo) con mayor
confianza.

\clearpage
\clearpage

\section{Tri-Training}

En Co-Training, los algoritmos son entrenados con el conjunto de datos
etiquetados para luego ser reentrenados con los datos no etiquetados que han
pasado a estarlo.

En \cite{1512038} (Tri-Training), Zhi-Hua Zhou y Ming Li, comentan la idea de
que para determinar qué ejemplo no etiquetado debe seleccionarse (para
etiquetar) y qué clasificador debe tener más influencia, se debe calcular la
\textit{confianza de etiquetado} de cada uno de los clasificadores. Sin embargo, esto
puede ser muy costoso de calcular.

En este caso, se tendrán tres clasificadores base. La idea es que, en línea de
lo anterior, para que un clasificador etiquete un ejemplo sin etiquetar, los
otros dos clasificadores deben coincidir en la etiqueta de ese ejemplo. Aunque
no es necesario calcular esa \textit{confianza de etiquetado} de cada
clasificador en particular.

Una consecuencia de lo anterior es que, si los otros dos clasificadores fallan
en su predicción (y coinciden), se añadirá esa etiqueta y se estaría
introduciendo un ejemplo mal etiquetado. Sin embargo, en el peor caso, el ruido
introducido puede ser compensado si hay suficientes datos nuevos etiquetados
(bajo ciertas condiciones \cite{1512038}).

Una particularidad de Tri-Training es que tanto la cantidad como los ejemplos
concretos no etiquetados que son seleccionados para ser etiquetados no será
siempre el mismo en cada iteración. Para comprender esto, cada uno de los
clasificadores tiene, aparte del conjunto de entrada etiquetado $L$, un conjunto
con los datos recién etiquetados en cada iteración $L_i$\footnote{Constituido
por esos ejemplos en los que los otros dos clasificadores determinan la misma
etiqueta}. Y de hecho, este conjunto es <<vaciado>> entre una iteración y otra.
Es utilizado para reentrenar al clasificador al final de la iteración uniendo
$L$ y $L_i$.

\paragraph{Proceso general}

El primer paso es entrenar a cada uno de los tres clasificadores ($h_i$, $h_j$ y
$h_k$) mediante una muestra aleatoria del conjunto de entrada $L$.

A continuación, el algoritmo entra en un bucle cuya condición de parada es que
ningún clasificador obtenga nuevos ejemplos para entrenar (ejemplos que eran no
etiquetados). 

Dentro del bucle y suponiendo la perspectiva del clasificador $i$ lo primero que
se hace es <<vaciar>> el conjunto de datos seleccionados para él, $L_i$. En
segundo lugar, se realiza una estimación del error de la combinación de las
hipótesis $h_j$ y $h_k$\footnote{El error de clasificación se aproxima
dividiendo el número de instancias (de entrenamiento) en las que $h_j$ y $h_k$
se equivocan entre el número de instancias en las que predicen la misma etiqueta
\cite{1512038}.}. Si el error no es menor que el de la iteración anterior $e_i$,
se vuelve al inicio del bucle.

Si el error sí era menor, el conjunto $L_i$ es rellenado con aquellos ejemplos
del conjunto de no etiquetados $U$ en los que $h_j$ y $h_k$ predicen la misma
etiqueta.

A partir de aquí, se realizan comprobaciones sobre ecuaciones del ratio de ruido
y errores \cite{1512038} que determinarán si los ejemplos realmente serán
utilizados para reentrenar. De hecho, es posible también que el conjunto $L_i$
sea reducido.

Al final del bucle si las comprobaciones realizadas determinaron que $L_i$
contiene ejemplos de interés para $h_i$, dicho clasificador es entrenado con la
unión de $L$ y $L_i$.

\paragraph{Predicción}

La predicción de una nueva instancia es un proceso sencillo. Cada uno de los
clasificadores $h_i$, $h_j$ y $h_k$ predicen la etiqueta de esta nueva instancia
y se retorna la mayoritaria.

\begin{algorithm}
    \DontPrintSemicolon
    \KwIn{Conjunto de datos etiquetados $\pmb{L}$, 
    no etiquetados $\pmb{U}$ y algoritmo de aprendizaje \textit{Learn}
    }
    \For{$i \in \{1..3\}$}{
        $S_i \leftarrow BootstrapSample(L)$\;
        $h_i \leftarrow Learn(S_i)$\;
        $h_i \leftarrow .5$; $l'_i \leftarrow 0$\;
     }
     \Repeat{ningún $h_i~(i \in \{1..3\})$ cambie}{
        \For{$i \in \{1..3\}$}{
            $L_i \leftarrow \emptyset$\;
            $update_i \leftarrow False$\;
            $e_i \leftarrow MeasureError(h_j \& h_k)~(j, k \neq i)$\;
            \If{$e_i < e'_i$}{
                \For{every $x \in U$}{
                    \If{$h_j~(x) = h_k~(x)~(j,k \neq i)$}{
                        $L_i \leftarrow L_i \cup \{(x, h_j~(x))\}$
                    }
                }
                \If{$l'_i = 0$  /* $h_i$ no ha sido actualizado antes */}{
                    $l'_i \leftarrow \lfloor\frac{e_i}{e'_i - e_i} + 1 \rfloor$
                }
                \If{$l'_i < |L_i|$}{
                    \If{$e_i|L_i|< e'_i l'_i$}{
                        $update_i \leftarrow True$
                    }\ElseIf{$l'_i > \frac{e_i}{e'_i - e_i}$}{
                        $L_i \leftarrow Subsample(L_i, \lceil\frac{e'_il'_i}{e_i} - 1\rceil)$\;
                        $update_i \leftarrow True$
                    }
                }
            }
        }

        \For{$i \in \{1..3\}$}{
            \If{$update_i = True$}{
                $h_i \leftarrow Learn(L \cup L_i); e'_i \leftarrow e_i; l'_i \leftarrow |L_i|$
            }
        }
     }
     \Return{$h(x) \leftarrow arg~max_{y \in label}  \sum_{i:h_i(x)=y} 1$}
     \caption{Tri-Training}\label{pseudo:tri-training}
\end{algorithm}
\clearpage

\section{Técnicas de tratamiento de datos}

En el proceso de minería de datos y aprendizaje automático una de las primeras
etapas, y principales, es el preprocesamiento de datos. Este concepto hace
referencia a la manipulación,  eliminación y/o transformación de datos antes de
ser usados para mejorar el rendimiento del proceso \cite{enwiki:1138293751}.

El tratamiento de datos también es particularmente útil en el área de la
interpretación y la visualización de datos. En ocasiones, el conocimiento
obtenido es difícil de interpretar con reglas o puras matemáticas.

En esta sección se van a comentar algunos conceptos que se han aplicado en el
área de tratamiento de datos (tanto en la parte de preprocesamiento como
visualización).

\subsection{Codificación de variables categóricas}
Este es el único tratamiento que el proyecto tiene incluido dentro del
preprocesamiento de los datos (el usuario es el encargado de tratar el resto de
cuestiones). Además, solo se aplica al atributo de la clase.

Muchos algoritmos de aprendizaje automático no son capaces de trabajar con
variables categóricas (no numéricas). Además, la categorización está presente en
muchos conjuntos de datos, ya que son muy útiles para aportar significado (los
números serían más difíciles de entender). Debido a las limitaciones de los
algoritmos, estos valores categóricos deben ser codificados como unos valores
numéricos.

Una de las ideas más directas (y la aplicada en el proyecto) es la
<<Codificación de etiquetas>> (\textit{Label encoding}). Esta técnica
simplemente trata cada valor único de etiqueta\footnote{Entendiendo etiqueta
como un valor del atributo de clase, no de una característica.} como un valor
numérico también único.

\begin{table}[H]
    \centering
\begin{tabular}{l|c}
Etiquetas originales & Etiquetas codificadas \\ \hline
Amarillo             & 0                     \\
Rojo                 & 1                     \\
Verde                & 2                     \\
Amarillo             & 0                     \\
Azul                 & 3                    
\end{tabular}
\caption{Codificación de etiquetas}
\end{table}

Desde el punto de vista del algoritmo, no necesita tener esa información que
aportaba la categoría (y que a nosotros sí nos ayuda). Simplemente, con mantener
la misma codificación para cada etiqueta, es suficiente.

\subsection{PCA (\textit{Principal Component Analysis})} 

Desde el punto de vista de la interpretación de los datos, PCA puede ser muy
útil (también se utiliza en el preprocesamiento en algún proceso de minería).

El análisis de componentes principales es un algoritmo matemático que reduce la
dimensionalidad de los datos manteniendo la mayoría de las variaciones en los
datos. Esto lo realiza identificando componentes principales
\cite{ringner2008principal}. Una componente principal es una nueva variable que
es combinación lineal de las variables originales con la particularidad de que
esa combinación de variables mantiene la mayor cantidad de varianza del conjunto
de datos original. La idea es mantener los patrones escondidos en los datos, en
las componentes principales, para que, si se aplica a un proceso de minería de
datos, se pueda seguir extrayendo conocimiento.

Aplicar PCA permite utilizar solo unas pocas componentes que pueden ser
graficadas y hacer posible la visualización de diferencias y similitudes.

Esta idea es la que se ha usado para las visualizaciones del proyecto. En muchos
casos se tendrá una dimensionalidad alta y por lo tanto no se puede visualizar
en dos dimensiones.

\subsection{Estandarización}

La estandarización en el campo de tratamiento de datos hace referencia a la
transformación de los valores para centrarlos en una media y con una cierta
desviación estándar (típicamente media cero y desviación 1).

\begin{equation}
  z = (x - u) / s
\end{equation}

\noindent $z$ es el nuevo valor, $x$ el valor original, $u$ la media y $s$ la
desviación típica.

La estandarización es útil cuando no se quiere establecer un rango fijo  de
valores (como en la normalización), y además, los \textit{outliers} (en
castellano valores atípicos) no afectan demasiado, ya que se considera la media
de todos ellos.

En el proyecto, es una opción que se le da al usuario para mostrar los datos
finales, ya que permite mantener las relaciones entre los datos pero
compactándolos lo suficiente para una buena visualización.

\subsection{HTTP (Descripción general)}

Al ser una aplicación Web, este proyecto tiene alta relación con este protocolo.
En el resto del apartado se van a documentar los conceptos que se han utilizado
en el desarrollo.

\emph{Hypertext Transfer Protocol (HTTP)} o Protocolo de Transferencia de
Hipertexto es un protocolo de la capa de aplicación\footnote{Permite a las
aplicaciones trabajar con los servicios de las capas inferiores y define
protocolos para el intercambio de datos~\cite{eswiki:149372346}} para transmitir
documentos. Se trata de un modelo cliente-servidor clásico, el cliente realiza
una petición al servidor, y espera a que este le devuelva una respuesta
\cite{http:mdn}.

\imagen{memoria/HTTP}{Diagrama cliente-servidor.}{0.8}


Características de HTTP \cite{http:features}:
\begin{itemize}
	\item Protocolo sin conexión: esto significa que la comunicación antes
	comentada ocurre sin realizar un acuerdo previo entre el cliente y el
	servidor.
	\item Independiente al tipo de contenido: quiere decir que HTTP no limita
	los datos que se pueden enviar, siempre y cuando, tanto el cliente como el
	servidor sepan manejar esos datos.
	\item Protocolo sin estado: cliente y servidor solo saben de su existencia
	durante la comunicación, no retienen la información de peticiones
	anteriores.
\end{itemize}


\subsubsection{Estructura de las peticiones y respuestas}

La realidad es que, peticiones y respuestas, tienen una estructura muy similar. Para las peticiones:
\begin{itemize}
	\item \textbf{Línea de inicio}, formada por:
  \subitem El método HTTP (se describirán a continuación), indica la operación a realizar.
  \subitem El objetivo, generalmente una URL. Para algunos métodos, aquí ser
  incorporan también los parámetros de la petición (\texttt{?param=X}).
  \subitem Versión del protocolo.
	\item \textbf{Cabeceras}: Son <<metadatos>> que proporcionan información sobre la
  petición. Cada una de ellas (puede haber varias) es un par nombre-valor.	Por
  ejemplo, especificar el tipo de contenido de la petición se haría con
  <<\texttt{Content-Type: application/json}>>.
  \item \textbf{Cuerpo}: Contiene información adicional para el servidor, pudiendo enviar
  cualquier dato. No todas las peticiones llevan cuerpo, una petición GET no lo
  necesita, pero sí un POST.
\end{itemize}

Para las respuestas:
\begin{itemize}
	\item \textbf{Línea de estado}, formada por:
  \subitem Versión del protocolo.
  \subitem Código de estado que indican el estado de la petición (se describirán a continuación).
  \subitem Descripción textual del código de estado (para poder interpretarlo).
	\item \textbf{Cabeceras}: Del mismo modo que las peticiones, son <<metadatos>> que
  proporcionan información sobre la petición. Cada una de ellas (puede haber
  varias) es un par nombre-valor.
  \item \textbf{Cuerpo}: Para aquellas respuestas que tengan cuerpo (algunas no lo tienen),
  contiene los datos solicitados o generados por el servidor.
\end{itemize}

\imagen{memoria/curl}{Ejemplo de respuesta con su estructura.}{0.85}

Como se ha visto en las operaciones HTTP, todas tienen un indicativo, llamado
método, que permite distinguir qué tipo de operación se quiere realizar. El
protocolo HTTP soporta una enorme cantidad de métodos, algunos de ellos son para
contextos muy específicos. Los más utilizados (constantemente en entornos
reales) son \cite{enwiki:1151700575}:
\begin{itemize}
	\item \textbf{GET}: Solicitud de un recurso. Solo para recuperar datos, sin
  otras operaciones. Generalmente a la URL se le añaden parámetros que
  configuran esta solicitud para que el servidor pueda personalizar las
  respuestas.
  \item \textbf{POST}: Solicitud que envía datos en el cuerpo de la petición (es
  el método de los formularios HTML, por ejemplo). Tiene diversas funciones,
  como crear un recurso o actualizarlo. Sin embargo, pese a que no está pensado
  para ello, en muchas ocasiones, también se utiliza para solicitar información.
  
  Entre otras cosas, solicitar información con POST, permite reducir la longitud
  de las URLs para operaciones GET demasiado extensas (incluso que puedan
  sobrepasar el límite de caracteres) y también permite ocultar información
  comprometida \cite{http:postnotget}. 
  
  Las peticiones GET son parametrizadas mediante la propia URL, mientras que en
  POST, los parámetros pueden incorporarse al cuerpo de la petición,
  ocultándolos (interesante, por ejemplo, si no se quiere compartir
  identificadores).
  \item \textbf{PUT}: Método similar a POST, también envía datos al servidor,
  pero en este caso está orientado específicamente a la actualización de los
  recursos.
  \item \textbf{DELETE}: Método para eliminar recursos.
\end{itemize}

Los códigos de estado antes mencionados son parte de la respuesta del servidor a
los clientes. Indicarán, por lo general, el éxito o fracaso de las operaciones.

Hay aproximadamente 60 códigos distintos, cada uno brinda la posibilidad de
personalizar las respuestas dependiendo del contexto. Todos estos códigos pueden
ir acompañados de la estructura antes comentada. Por ejemplo, para una petición
exitosa, la respuesta de un POST podría tener un \texttt{cuerpo} con más
información junto con el código \textit{200}. 

Los códigos más utilizados son:

\begin{itemize}
	\item \textbf{200}: \emph{OK}, respuesta a peticiones HTTP exitosas.
  \item \textbf{400}: \emph{Bad Request}, el servidor no puede procesar la petición
  debido a una petición mal construida.
  \item \textbf{401}: \emph{Unauthorized}, no se tiene las credenciales de
  autenticación válidas para acceder al recurso solicitado.
  \item \textbf{403}: \emph{Forbidden}, el servidor entiende la petición pero la
  rechaza.
  \item \textbf{404}: \emph{Not Found}, el recurso no puedo ser encontrado.
  \item \textbf{405}: \emph{Method Not Allowed}, el método no es soportado para acceder
  al recurso. Por ejemplo, utilizar un GET cuando está pensado para utilizarse
  un POST.
  \item \textbf{500}: \emph{Internal Server Error}, respuesta genérica para errores
  internos, sin mensaje específico de los mismos.
\end{itemize}

\paragraph{HTTP Cookies}

El concepto de las \textit{cookies} es sencillo, son pequeños bloques de datos
que el servidor manda al navegador del usuario mientras realiza peticiones HTTP
a él. Una vez que el cliente ha almacenado esta información, puede usarla para
agilizar ciertos pasos en las siguientes peticiones. Cuando vuelva a acceder,
junto con la petición <<normal>> enviará esta porción de datos, que el servidor
puede interpretar y reconocer. Por ejemplo, pueden guardarse los inicios de
sesión, preferencias o, en entornos comerciales, el comportamiento del usuario.

La idea interesante de las \textit{cookies} es que permite almacenar
\textbf{información de estado} para el protocolo HTTP, que es \textbf{sin
estado} (por sí solo no puede guardar esta información).

\section{CSRF (Cross-Site Request Forgery)}
\emph{Cross-Site Request Forgery} (falsificación de petición en sitios cruzados)
es un ataque sobre aplicaciones web vulnerables en las que se utiliza a un
usuario autenticado en la misma (o en el que la web confía) para enviar una
petición maliciosa. Los ataques CSRF explotan la confianza que una web tiene con
un usuario, al no poder diferenciar entre una petición generada por el usuario
(la persona) y la petición generada por el usuario, pero sin su consentimiento
\cite{csrf} (generalmente de forma cruzada, proveniente de otro contexto).

Se presenta un ejemplo explicativo:

\begin{tcolorbox}[colback=cyan!5!white,colframe=cyan!75!black,title=Situación]
Web vulnerable: \url{https://ejemplo.es}

Usuario: David (con sesión iniciada)

URL de cambio de contraseña: \url{https://ejemplo.es/cambiarcontrasena} (permite
un parámetro <<nueva>> mediante POST para especificar la nueva contraseña)

El servidor ha <<securizado>> parcialmente la URL anterior comprobando que el
que accede a esa URL es el propio usuario con sesión iniciada.
\end{tcolorbox}


\begin{tcolorbox}[colback=red!5!white,colframe=red!75!black,title=Atacante]
    \begin{enumerate}
        \item Ha hecho ingeniería social para saber el Email de David:
        \mbox{\texttt{david@ejemplo.es}}.
        \item Envía un correo electrónico con un enlace que parece de Google.
        \item En realidad, todos los botones de esa página envían un formulario (POST)
        hacia \url{https://ejemplo.es/cambiarcontrasena}.
        \item El formulario tiene un campo oculto:
        \mint{html}|<input type="hidden" name="nueva" value="hackeado"/>|
    \end{enumerate}
\end{tcolorbox}

\begin{tcolorbox}[colback=green!5!white,colframe=green!75!black,title=David]
    \begin{enumerate}
        \item Recibe el correo en su bandeja de entrada.
        \item No se percata de que no es la página original de Google y entra en ella.
        \item Busca información en Google y pincha en un botón.
        \item Se realiza la petición POST a la aplicación web vulnerable en la que ha iniciado sesión.
        \item El servidor procesa la petición pues cree que es el propio usuario el que ha
        construido la petición conscientemente.
        \item La contraseña ha cambiado a <<hackeado>>.
    \end{enumerate}
\end{tcolorbox}

\begin{tcolorbox}[colback=red!5!white,colframe=red!75!black,title=Atacante]

Con el Email <<david@ejemplo.es>> y la contraseña <<hackeado>>, accede a la Web
<<ejemplo.es>> como si fuera David.

\end{tcolorbox}

\paragraph{Solución} Dada la vulnerabilidad comentada anteriormente, la solución
pasa por idear una forma de saber que esa petición se ha creado desde la propia
aplicación (no por terceros). Por lo tanto, la idea más efectiva es la creación
de un <<\emph{CSRF token}>> único para los usuarios, que el servidor
(aplicación) genera y <<guarda>> en la sesión de los mismos. Este <<token>> es
incrustado como un parámetro oculto en las peticiones (en los formularios) y al
realizar el envío del formulario, el servidor comprobará que el <<token>>
recibido era el correcto para el usuario.


\capitulo{4}{Técnicas y herramientas}

En este apartado se presentarán las técnicas (procedimientos) y herramientas que
se han utilizado para el desarrollo del presente proyecto. En algunos de los
casos la experiencia u otros aspectos han hecho decantarse por una u otra o
simplemente se seleccionaron directamente.

\section{Técnicas}

Una de las herramientas utilizadas (ver siguiente sección) es Bootstrap, un
conjunto de estilos ya predefinidos que pueden utilizarse sin necesidad de tener
que codificar CSS. La idea de la aplicación Web desarrollada es que tenga la
máxima accesibilidad. Esto se consigue partiendo de un diseño que sea apto para
los dispositivos móviles, lo que se denomina estilo adaptable (<<responsive>>).
Se trata de una técnica de diseño web para adaptar la visualización de la página
al dispositivo desde el que se accede, haciendo que sea atractiva una mayor
cantidad de usuarios. El diseño <<responsive>> se consolida como una de las
mejores prácticas en el diseño web \cite{40defiebre}. En toda la aplicación,
cuando el tamaño de la pantalla se reduce lo suficiente, los contenidos se
adaptan y se posicionan de tal forma que puedan seguir utilizándose.

\section{Herramientas}

\paragraph{PyCharm}
Se trata de un \texttt{entorno de desarrollo integrado} ("<IDE">) desarrollado por
JetBrains. Está creado para la programación en lenguaje Python y aunque no se ha
usado en este proyecto también Java. Este "<IDE"> se ha convertido junto con Visual
Studio Code y Jupyter en el más utilizado por los desarrolladores.

Ofrece multitud de funcionalidades (\url{https://www.jetbrains.com/es-es/pycharm/features/}):
\begin{itemize}
	\item Inspección de código.
	\item Indicación de errores (compilación).
	\item Refactorización de código automático (rápidas y seguras).
	\item Depuración.
	\item Pruebas.
	\item Herramientas para bases de datos.
	\item Integración con Git.
\end{itemize}

Estas son solo algunas de las muchas funcionalidades que permite y que han hecho
que se haya seleccionado para este proyecto. Además, dado que el proyecto tiene
una fuerte componente de desarrollo Web así como frameworks, PyCharm tiene una
integración completa con estos ámbitos, pudiendo desarrollar también de forma
nativa con JavaScript o lenguajes de marcas (HTML o CSS).

Otra ventaja del uso de PyCharm y concretamente gracias a la Universidad de
Burgos es que el alumnado tiene acceso a la edición "<Professional"> que
habilita ese desarrollo Web, por ejemplo.

Por último, esta aplicación ya había sido usada con anterioridad y por tanto su
aprendizaje básico no era necesario.

\paragraph{Visual Studio Code}
Se trata de un editor de código fuente desarrollado por Microsoft. Es el editor
por excelencia en todo el ámbito de la programación pues permite la programación
en casi cualquier lenguaje de programación haciéndole muy versátil y un "<todo
en uno">.

Ofrece todas las funcionalidades que cabe esperar (incluyendo a las que ofrece
PyCharm): sintaxis, depuración, personalización, integración git...

Además del núcleo propio del editor y su constante actualización, uno de sus
puntos fuertes son las extensiones que empresas o incluso la comunidad
desarrollan y que permiten agilizar en lo posible las tareas del programador.

A pesar de todas las ventajas, el manejo de PyCharm ha resultado más sencillo de
utilizar (durante la experiencia previa a este proyecto) en el desarrollo del
software. Pero debido a su versatilidad y al desarrollo de esta
documentación en Latex, VS Code es el editor adecuado para ello. Gracias a las
extensiones y a la instalación local de \LaTeX, permite tener un control completo
para la creación de este tipo de documentos. 


\paragraph{GitHub Desktop}
Es una aplicación que permite interactuar con GitHub utilizando una interfaz
gráfica respecto al manejo tradicional mediante la línea de comandos. Permite
realizar las operaciones más comunes y básicas de Git.

Pese a que no tiene toda la funcionalidad que sí ofrece la línea de comandos no
se previó un uso muy exhaustivo de Git y, por tanto, es suficiente
("<Commit">,"<Push">,"<Pull">,"<Merge">...).

Además, puede visualizarse cada cambio respecto a la última versión de un
vistazo y seleccionar dinámicamente los cambios que se deseen aplicar.

Aún con todo esto, no es más que una aplicación para agilizar el ya por lo
general rápido proceso de control de versiones.

\paragraph{Librerías Python}
Para el proyecto se están utilizando múltiples librerías que facilitan en gran
medida el desarrollo del mismo. Implementan funcionalidades que pueden ser
utilizadas directamente. En el proyecto, las librerías usadas están
especificadas en los requisitos (\textit{requirements}).

\begin{itemize}
	\item \textbf{pandas}: Librería para el manejo de estructuras de datos que implementa
	muchas operaciones útiles (guardado/lectura CSV, reordenaciones,
	divisiones...) y de manera eficiente.
	\item \textbf{Babel}: Colección de utilidades para la internacionalización y
	localización de aplicaciones en Python.
	\item \textbf{Flask}: Es un micro Framework escrito en Python para el desarrollo de
	aplicaciones Web. Impone además su formato de plantillas (Jinja2).
	\item \textbf{flask-babel}: Es una extensión de Flask que permite la
	internacionalización concreta de aplicacones Flask (con la ayuda de Babel).
	\item \textbf{scikit-learn}: Librería más extendida de aprendizaje automático
	(<<machine learning>>) para Python. Junto con Flask es el núcleo del
	proyecto ya que utiliza muchos de sus paquetes implementados.
	\item \textbf{numpy}: Es una librería especializada en el cálculo numérico y análisis
	de datos. Se caracteriza por su eficiente manejo de grandes volúmenes de
	datos gracias a su parcial implementación en lenguaje C (mucho más
	eficiente).
	\item \textbf{scipy}: Librería que incluye algoritmos matemáticos.
	\item \textbf{setuptools}: Librería que permite crear e instalar paquetes de
	software de Python.
	\item \textbf{pytest}: Es un Framework de pruebas unitarias para Python.
	\item \textbf{sslearn}: Librería para aprendizaje automático en conjuntos de datos
	Semi-Supervisados como extensión de scikit-learn. Creada por José Luis Garrido-Labrador.
	\item \textbf{matplotlib}: Librería para la creación de gráficos en 2D.
\end{itemize}

\paragraph{Bibliotecas Web} Para el desarrollo Web conviene comentar la
utilización de dos bibliotecas que han resultado fundamentales.

\textbf{Bootstrap}: No es considerado una biblioteca al uso, sino que está
categorizado como un <<framework>> de CSS. Fue lanzado en 2011 con una gran
cantidad de actualizaciones y de lanzamientos de nuevas versiones. La versión
más reciente y la utilizada en el proyecto, es Bootstrap 5, que mejora el código
fuente de la versión 4 y añade propiedades nuevas. La intención del equipo de
Bootstrap es promover el desarrollo web con novedades de CSS y menores
dependencias. Su principal objetivo es la creación de Webs con diseño responsivo
(adaptados a móviles) de la forma más sencilla posible. Contiene infinidad de
plantillas creadas con HTML, CSS y JavaScript para componentes de la interfaz
que el usuario simplemente puede hacer uso de ellas. En HTML, Bootstrap se
incorpora en las clases de los componentes y en su página oficial se tiene toda
la documentación\footnote{Bootstrap: \url{https://getbootstrap.com/}}.

\textbf{D3.js}: su nombre se debe a Data-Drive-Documents, es una librería de
JavaScript que permite representar datos en distintas visualizaciones mediante
HTML, SVG y CSS. D3 trabaja sobre el Modelo de Objetos del Documento (<<Document
Object Model>>) para aplicar transformaciones dirigidas por datos. Todas las
visualizaciones presentadas en la Web se han realizado con D3. D3 permite un
amplio abanico de posibilidades y creatividad, pueden crearse visualizaciones
más bien estáticas y añadir estilos, anotaciones o leyendas entre otra cosas,
pero también permite la creación de gráficos altamente interactivos incluyendo
<<zoom>>, aprovechar los eventos en el DOM o transiciones completamente
personalizadas.

Para este caso se cree necesario comentar y justificar el uso de D3 ya que en el
ámbito de las visualizaciones, existe una gran cantidad de librerías disponibles
como JSAV (JavaScript Algorithm Visualization Library) o Algorithm-Visualizer
(contexto concreto de este proyecto) o incluso otras grandes librerías como
Anime.js, Chart.js o FusionCharts. Sin embargo, durante el periodo de estudio de
las distintas herramientas disponibles, D3 es superior a la demás en su
documentación. Tanto porque existen páginas dedicadas a la ejemplificación con
D3, como la gran cantidad de foros que se han creado solucionando dudas y
problemas. Sobre las últimas librerías mencionadas (y otras grandes existentes),
D3 está particularmente pensado para la completa personalización y aunque estas
también, no al nivel de D3.

La desventaja de D3 es que tiene una curva de aprendizaje bastante grande, pero
gracias a la documentación en internet, esto puede ser solventado en mayor
medida.
\capitulo{5}{Aspectos relevantes del desarrollo del proyecto}

En este apartado se van a comentar los aspectos más interesantes o que han
influido en el desarrollo. Por lo general, estos aspectos vendrán acompañados de
tomas de decisiones que se tuvieron que hacer, se argumentará y explicará el
desarrollo final de estos aspectos.

\section{Elección del proyecto y la realidad}

La idea de este proyecto no fue propia, ambos tutores tenían en mente realizar
una aplicación de este estilo aplicado a estos algoritmos. Durante los meses
anteriores aparecieron diversos proyectos de muy diferente índole que podía
realizar. La realidad es que la inteligencia artificial es un campo que me atrae
mucho y que incluso me abre la mente a proyectos personales. Dentro de la baraja
de proyectos, el que iba en la línea de lo que quería hacer, era este. Al
principio, era un poco reacio a la idea de realizar una aplicación tanto Web
como de escritorio. Yo, demasiado enfocado en ello, quería aplicar los
algoritmos a algo concreto. Sin embargo, pensándolo bien, el hecho de poder
aprender algoritmos (e incluso la metodología que su desarrollo conlleva de
investigación y entendimiento) pero además, poder desarrollar habilidades en Web
(o escritorio en su caso aunque finalmente no ha sido así), fue la clave para
decantarme por el proyecto. Además, como soy una persona que le gusta mucho
<<cacharrear>>, los tutores también me comentaron que este proyecto podría
tenerme entretenido muchas horas para desarrollar ambas partes. Con todo ello,
parecía muy interesante y sobre todo, muy nutritivo para mi aprendizaje, elegir
este proyecto.

\section{Versión de Python}

Al comienzo del proyecto, se valoró la versión de Python en la que realizarlo.
En un principio parecía importante abarcar el mayor número de equipos en los que
el proyecto podía ser instalado y se pensó en alguna versión desde la 3.7. Sin
embargo, pese a que en la biblioteca de algoritmos que se iba a programar sí que
podía tener sentido aumentar los posibles usuarios, la realidad es que el
objetivo del final proyecto es una aplicación Web completamente nueva. Se supone
además que, simulando un entorno completamente real y/o empresarial, el equipo
podría tener a su disposición servidor/es propio (o por lo menos,
configurables). En este sentido, dado que el usuario solo necesita acceder a la
Web, no tiene por qué conocer ni tener instalada una versión u otra. Por todo
esto, se pensó en utilizar versiones más recientes. En el momento de inicio del
proyecto, la más reciente era la 3.10, esto también es una ventaja en el medio
plazo debido a que el periodo de actualizaciones y de soporte para esta versión
termina a finales de 2026 (mientras que algunas anteriores finalizan en 2024 o
incluso 2023).

\section{Utilidades para los algoritmos}

Antes de realizar la aplicación Web, pero previendo lo que podía encontrarse.
Surgieron dos grandes problemas, el tratamiento de los datos que utilizan los
algoritmos (etiquetados, no etiquetados, particiones...) y por otro lado, cómo
ajustar estos datos y comunicarlos a la Web cuando los requiera.

No resultaba correcto vincular todos estos pasos ni en el código de los
algoritmos ni en el de la propia aplicación. Tenía mucho más sentido crear unas
utilidades que actúen de intermediarias en ciertos pasos del procedimiento.

Para el primer problema se crearon tres utilidades principales: Un <<particionador>>
de datos que se encargara de dividir los datos en conjunto de entrenamiento
y test. Pero que además, si el conjunto estaba pensado para aprendizaje
Supervisado, generase aleatoriamente datos no etiquetados.

Un codificador de etiquetas para transformar las etiquetas nominales en
numéricas (necesarias para los algoritmos). De base, el codificador de etiquetas
que propone SKlearn podría ser suficiente. Sin embargo, era necesario por un
lado no tratar los datos no etiquetados (internamente tratados como -1s) y por
el otro devolver de alguna forma las transformaciones realizadas Es decir, a qué
clase nominal corresponde cada número codificado.

Y por último, un cargador de conjuntos de datos (datasetloader) que automatizara
toda la lectura de un fichero ARFF o CSV, conversión a DataFrame y las
transformaciones de etiquetas (utilizando la utilidad anterior). Y que además
devuelva los datos de los atributos (x) y por separado las etiquetas (y). Es
decir, un cargador cuyo resultado pueda ser introducido en los algoritmos. 

Sobre el segundo problema (datos válidos para la aplicación Web) se creó una
utilidad encargada de transformar el conjunto de datos a dos dimensiones. Esto
es, transformar los atributos (que pueden ser más de dos) a exactamente dos,
para poder representarlo en dos dimensiones. Sin embargo, cuando se avanzó un
poco en el desarrollo, era obvio que no interesaba modificar el conjunto de
datos, sino el resultado de la ejecución de la aplicación Web. Fue cuando se
implementó y visualizó Self-Training cuando se pudo ver esta casuística.
Finalmente, lo que se hace es, transformar la estructura de datos que retorna la
ejecución de los algoritmos (Anexo D) que incluye todos los datos (y todos sus
atributos), a dos dimensiones (solo los atributos, no el <<target>> ni el resto
de columnas). El usuario puede seleccionar dos posibilidades, o realizar PCA o
seleccionar él dos atributos del conjunto.

En ambos casos apareció la decisión de la normalización. Al principio se
realizaba en ambos casos directamente, pero finalmente se permitía al usuario
elegir si normalizar o no.
\capitulo{6}{Trabajos relacionados}

Con el aprendizaje automático y el uso de gran cantidad de datos, es
completamente necesaria la visualización de los datos, procesos de entrenamiento
y ciertas estadísticas. Al fin y al cabo, cuando se construyen modelos, se debe
tener una realimentación de cómo de bien está funcionando para poder extraer
conclusiones sobre el mismo.

De forma general, sin centrarse directamente en el aprendizaje automático,
resulta interesante y conveniente que los visualizadores de algoritmos sean
accesibles y que, como están apareciendo, se creen aplicaciones Web que resultan
mucho más directas. Desde el punto de vista de la docencia y aprendizaje los
visualizadores permiten culminar la comprensión los aspectos teóricos
subyacentes.

\section{Visualizadores}

\subsubsection{Seshat} 
Es una herramienta web que trata de facilitar el aprendizaje de
la teoría de lenguajes y autómatas (análisis léxico) que utilizan los
compiladores~\cite{arnaiz2018seshat} , puede accederse desde
\url{http://cgosorio.es/Seshat}. La herramienta propone en primera instancia
unas explicaciones teóricas de los algoritmos con sus conceptos generales, el
concepto concreto de las expresiones regulares y qué es un autómata finito. A
partir de la teoría, se encuentran implementados varios algoritmos que se
visualizan paso a paso. En el momento de la ejecución también se tienen
elementos de interés como la propia descripción o explicación del algoritmo. Los
algoritmos implementados por la herramienta son:
\begin{enumerate}
    \item Construcción de un autómata finito no determinista (AFND) a partir de una expresión regular.
    \item Conversión de un autómata finito no determinista (AFND) a un autómata finito determinista (AFD).
    \item Construcción de un autómata finito determinista (AFD) a partir de una expresión regular.
    \item Minimización de un autómata finito.
\end{enumerate}
Para su construcción se ha usado el framework Flask en Python que actúa como
servidor. La interfaz de usuario está construida con HTML, SVGs y Javascript
para proporcionar el contenido dinámico.

En este proyecto se realizó un estudio de la opinión de 42 estudiantes después
de su uso. El estudio consistía en unas afirmaciones (5) respecto a la buena
utilidad, buen diseño o el interés que puede producir al visualizar los
algoritmos de esa forma. De forma general, en una escala del 1 al 5 (donde el 5
representa que están profundamente de acuerdo con la afirmación) el 95\% de los
resultados se concentran entre el 4 y el 5 en la valoración. Esto indicó que la
herramienta sí que resultó útil e incluso ellos mismos solicitaban este recurso
para facilitar su aprendizaje.

\subsubsection{Herramienta de apoyo a la docencia de algoritmos de selección de instancias}
En \cite{arnaiz2012herramienta} se presenta una herramienta de escritorio para
la visualización de la ejecución y resultados de los algoritmos de selección de
instancias, debido a la carencia de este tipo de aplicaciones para dichos
algoritmos. La herramienta es altamente personalizable pudiendo subir el
conjunto de datos o seleccionar qué característica visualizar en los ejes. En la
visualización se pueden ver todos los pasos de los algoritmos junto, por
ejemplo, a las regiones de Boronoi o el pseudocódigo. Esto hace que el alumno
pueda conocer el progreso y los conceptos particulares (influencia,
vecindad...). Los algoritmos implementados por la herramienta son:
\begin{enumerate}
    \item Algoritmo Condensado de Hart (CNN) \cite{CNNHart1968}.
    \item Algoritmo Condensado Reducido (RNN) \cite{RNNGates1972}.
    \item Algoritmo Subconjunto Selectivo Modificado (MSS) \cite{MSSBarandela2005}.
    \item Algoritmos Decremental Reduction Optimization Preocedure (DROP) \cite{DROPWilson2000}.
    \item Algoritmo Iterative Case Filtering (ICF) \cite{ICFBrighton2002}.
    \item Algoritmo Democratic Instance Selection (DIS) \cite{DemoISGarcia2010}.
\end{enumerate}
Está desarrollado completamente en Java lo que lo hace portable a cualquier
plataforma y sin instalación.

Como punto característico a tener en cuenta, la herramienta estaba muy orientada
a ofrecer bastante granularidad en cada paso, de tal forma que mediante, por
ejemplo, la visualización del pseudocódigo (al mismo tiempo que la visualización
gráfica), permite para analizar el comportamiento subyacente.

Los estudiantes, de forma general, valoraron muy positivamente la herramienta
pues les ayudó a comprender los algoritmos.

\subsubsection{Towards Developing an Effective Algorithm Visualization Tool for
Online Learning} 

En \cite{8560314} se presenta una <<guía>> especialmente interesante con las
consideraciones a tener en cuenta para el desarrollo de una correcta herramienta
de visualización de algoritmos.


La realidad actual es que resulta muy difícil hacer que los alumnos entiendan
todos los conceptos que involucran los algoritmos y más aún en la modalidad "<a
distancia">. Para solventar este problema se han desarrollado múltiples
herramientas de visualización de algoritmos (AV) que pretenden facilitar la
compresión de los mismos, pero la realidad de estas herramientas es que se
desarrollan con una fuerte componente "<elegante"> y no enriquecedora o
pedagógica. Este artículo trata de abordar este último problema, el cómo
construir herramientas de visualización de algoritmos que sean divertidas y
atractivas, pero siendo fieles a los conceptos, y también cómo estas pueden
ayudar a los docentes a asegurar el aprendizaje de sus alumnos.

Para lograr el objetivo de desarrollar un visualizador efectivo, se analiza
desde el punto de vista de la pedagogía, usabilidad y accesibilidad. En la Tabla
\ref{tabla:objetivospedagogicos} se pueden ver algunas de las características
que se deben tener en cuenta a la hora de desarrollar una herramienta de este
estilo.
\begin{table}[]
    \resizebox{\textwidth}{!}{%
    \begin{tabular}{ll}
    \hline
    \rowcolor[HTML]{C0C0C0} 
    \textbf{Característica}     & \textbf{Cómo lograrlo}   \\ \hline
    \rowcolor[HTML]{FFFFFF} 
    \textbf{\begin{tabular}[c]{@{}l@{}}Incrementar la\\comprensión\end{tabular}} & \begin{tabular}[c]{@{}l@{}}Explicaciones textuales\\ Realimentación con las acciones del usuario\\ Ayuda integrada\\ Comodidad en la entrada de datos\end{tabular}                                                              \\ \hline
    \textbf{\begin{tabular}[c]{@{}l@{}}Promocionar\\ interés\end{tabular}}     & \begin{tabular}[c]{@{}l@{}}Permitir introducir datos al algoritmo\\ Manipular la visualización\\ Capacidad de volver <<rebobinar>>\\ Variación de la velocidad\\ Avanzar en la ejecución\end{tabular}         \\ \hline
    \textbf{Aprendibilidad}                                                    & \begin{tabular}[c]{@{}l@{}}Controles familiares\\ Generalidad con otros sistemas\\ Predictibilidad de las acciones\\ Interfaz simple\end{tabular}                                                                                  \\ \hline
    \textbf{Memorabilidad}                                                     & Interfaz común que soporte múltiples animaciones                                                                                                                                                                                   \\ \hline
    \textbf{Facilidad de uso}                                                  & \begin{tabular}[c]{@{}l@{}}Interfaz común que soporte múltiples animaciones\\ Claridad de los modelos y conceptos\end{tabular}                                                                                                     \\ \hline
    \textbf{Robustez}                                                          & \begin{tabular}[c]{@{}l@{}}Explicaciones de la visualización\\ Integración de ajustes predefinidos\\ Recuperación de errores\end{tabular}                                                                                       \\ \hline
    \textbf{Eficacia}                                                          & \begin{tabular}[c]{@{}l@{}}Visibilidad del estado del sistema\\ Control y libertad del usuario\\ Prevención de errores\\ Flexibilidad de uso\\ Ayuda al usuario para reconocer, diagnosticar y recuperarse de errores\end{tabular} \\ \hline
    \rowcolor[HTML]{FFFFFF} 
    \textbf{Accesibilidad}                                                     & \begin{tabular}[c]{@{}l@{}}Globalmente disponible\\ Plataforma y dispositivo independiente\end{tabular}                                                                                                                            \\ \hline
    \end{tabular}%
    }
    \caption{Características que influyen en los objetivos pedagógicos.}
\label{tabla:objetivospedagogicos}
    \end{table}


\section{Otras herramientas/bibliotecas}

Algorithm-Visualizer, \url{https://algorithm-visualizer.org/}. Se trata de una
Web que incluye una cantidad enorme de algoritmos, todos están programados
directamente en Javascript que genera una visualización de los mismos y también
incluye una explicación corta de cada uno. Algunos de los algoritmos son:
\texttt{Backtracking}, \texttt{Divide and Conquer}, \texttt{Greedy} o más
concretamente \texttt{K-Means Clustering}, entre muchos otros.

\imagen{memoria/algorithm-visualizer}{K-Means Clustering en algorithm-visualizer.}{1}

VISUALGO, \url{https://visualgo.net/en}. Es una página web creada por la
Universidad Nacional de Singapur. Esta Web permite visualizar algoritmos básicos
(como de ordenación o camino más corto) pero también estructuras de datos (como
"<hash tables"> o conjuntos disjuntos).

\imagen{memoria/visualgo}{Viajante de comercio en VISUALGO.}{1}

Visualizer, por Zhenbang Feng, \url{https://jasonfenggit.github.io/Visualizer/}
\footnote{ Github: \url{https://github.com/JasonFengGit/Visualizer}}. Su autor
lo define como una página web para proporcionar visualizaciones intuitivas e
innovadoras de algoritmos generales y de inteligencia artificial. Las
visualizaciones son muy visuales con las que se capta muy bien el funcionamiento
de los algoritmos. Implementa algoritmos de búsqueda de caminos, ordenamiento e
inteligencia artificial (como minimax).

\imagen{memoria/visualizer}{Mergesort en Visualizer (Zhenbang Feng).}{1}


Clutering-Visualizer, \url{https://clustering-visualizer.web.app/}. Es una
página Web para visualizar algoritmos de clustering. Realmente, no tiene muchos
implementados, pero las visualizaciones son muy descriptivas. En cada paso
muestra cuál es la evolución/posición de los distintos \textit{clusters} y mediante línea
es posible ver las distancias en ellos y los datos.

\imagen{memoria/Clutering-Visualizer}{K-Means en Clutering-Visualizer.}{1}


MLDemos, puede visitarse desde \url{https://basilio.dev/}. Es una herramienta de
visualización de código abierto para algoritmos de aprendizaje automático creada
específicamente para ayudar en el estudio y compresión del funcionamiento de una
gran cantidad de algoritmos. Entre ellos: clasificación, regresión, reducción de
dimensionalidad y clustering.

\imagen{memoria/MLDemos}{Clasificación mediante Support Vector Machine en MLDemos.}{1}

\capitulo{7}{Conclusiones y Líneas de trabajo futuras}

En esta sección final se presentan las conclusiones a las que se ha llegado con
la finalización del proyecto. La aplicación desarrollada tiene muchos puntos por
los que puede ser expandida, algunos sencillos y otros más grandes que ofrecen
posibilidades interesantes para los usuarios. Por esto último, también se
incluye una serie de líneas de trabajo para la mejora/expansión del proyecto.

\section{Conclusiones}

Una vez finalizado el proyecto, esta son las conclusiones que pueden extraerse:

\begin{itemize}
    \item Se ha desarrollado la biblioteca de algoritmos semi-supervisados
    personalizados, siendo validados mediante la comparación contra
    \texttt{sslearn} y obteniendo resultados satisfactorios suficientes. Con
    ella se ha podido extraer el proceso de entrenamiento de estos algoritmos
    para poder visualizarlos.
    \item El proyecto se ha culminado con la creación de una aplicación web
    completamente funcional que permite visualizar estos algoritmos. Además,
    también ofrece a los usuarios la posibilidad de registrarse y poder
    gestionar sus conjuntos de datos y ejecuciones.
    \item En líneas generales, la aplicación se ha desarrollado con la intención
    de ser completamente extensible. En este proyecto se han incluido cuatro
    algoritmos semi-supervisados, pero existen muchos más y se han dejado puntos
    de extensión para ellos.
    \item Con la terminación del proyecto, se ha creado una documentación
    completa de todos los aspectos técnicos considerados, así como un manual
    para el programador y para el usuario. En estos manuales se ha incluido todo
    lo que se ha creído imprescindible para entender perfectamente la
    aplicación.
    \item Durante la realización del proyecto (algoritmos y web) se han
    utilizado muchos conceptos aprendidos en asignaturas este curso y durante
    cursos previos. Entre ellas es de destacar \emph{Bases de datos},
    \emph{Metodología de la programación} (orientación a objetos),
    \emph{Estructuras de datos}, \emph{Sistemas inteligentes},
    \emph{Algoritmia}, \emph{Diseño y Mantenimiento del software} y
    \emph{Minería de datos}.
    \item Se han tenido que aprender otros muchos conceptos nunca antes vistos.
    Principalmente, el aprendizaje se ha centrado en el desarrollo web, con
    aspectos como las peticiones, respuestas, códigos HTTP, seguridad... y
    también sobre minería de datos, particularizando sobre clasificación.
    \item En lo relativo a programación, se ha tenido que aprender el lenguaje
    de marcas HTML junto con JavaScript para generar todo el contenido dinámico.
    Aunque ha supuesto un verdadero reto, ha sido un proceso interesante y
    satisfactorio.
    \item La documentación se ha realizado en pasos constantes (aumentando el
    ritmo hasta el final). Y pese a que no se ha apurado a realizarlo al final,
    sí que ha conllevado mucho tiempo del proyecto, incluso más que, por
    ejemplo, el desarrollo de los algoritmos.
\end{itemize}




\bibliographystyle{plain}
\bibliography{bibliografia}

\end{document}
