\capitulo{6}{Trabajos relacionados}

Con el aprendizaje automático y el uso de gran cantidad de datos, es
completamente necesaria la visualización de los datos, procesos de entrenamiento
y ciertas estadísticas. Al fin y al cabo, cuando se construye un modelo, se debe
tener una realimentación de cómo de bien está funcionando para poder extraer
conclusiones sobre el mismo.

De forma general, sin centrarse directamente en el aprendizaje automático,
resulta interesante y conveniente que los visualizadores de algoritmos sean
accesibles y que, como están apareciendo, se creen aplicaciones Web que resultan
mucho más directas. Desde el punto de vista de la docencia y aprendizaje los
visualizadores permiten culminar la comprensión los aspectos teóricos
subyacentes.

\section{Visualizadores}

\subsubsection{Seshat} 
Es una herramienta web que trata de facilitar el aprendizaje de la teoría de
lenguajes y autómatas (análisis léxico) que utilizan los
compiladores~\cite{arnaiz2018seshat}, puede accederse desde
\url{http://cgosorio.es/Seshat}. La herramienta propone en primera instancia
unas explicaciones teóricas de los algoritmos con sus conceptos generales, el
concepto concreto de las expresiones regulares y qué es un autómata finito. A
partir de la teoría, se encuentran implementados varios algoritmos que se
visualizan paso a paso. En el momento de la ejecución también se tienen
elementos de interés como la propia descripción o explicación del algoritmo. Los
algoritmos implementados por la herramienta son:
\begin{enumerate}
    \item Construcción de un autómata finito no determinista (AFND) a partir de una expresión regular.
    \item Conversión de un autómata finito no determinista (AFND) a un autómata finito determinista (AFD).
    \item Construcción de un autómata finito determinista (AFD) a partir de una expresión regular.
    \item Minimización de un autómata finito.
\end{enumerate}
Para su construcción se ha usado el \emph{framework} Flask en Python que actúa
como servidor. La interfaz de usuario está construida con HTML, SVGs y
Javascript para proporcionar el contenido dinámico.

En este proyecto se realizó un estudio de la opinión de 42 estudiantes después
de su uso. El estudio consistía en unas afirmaciones (5) respecto a la buena
utilidad, buen diseño o el interés que puede producir al visualizar los
algoritmos de esa forma. De forma general, en una escala del 1 al 5 (donde el 5
representa que están profundamente de acuerdo con la afirmación) el 95\% de los
resultados se concentran entre el 4 y el 5 en la valoración. Esto indicó que la
herramienta sí que resultó útil e incluso ellos mismos solicitaban este recurso
para facilitar su aprendizaje.

\subsubsection{Herramienta de apoyo a la docencia de algoritmos de selección de instancias}
En~\cite{arnaiz2012herramienta} se presenta una herramienta de escritorio para
la visualización de la ejecución y resultados de los algoritmos de selección de
instancias, debido a la carencia de este tipo de aplicaciones para dichos
algoritmos. La herramienta es altamente personalizable pudiendo subir el
conjunto de datos o seleccionar qué característica visualizar en los ejes. En la
visualización se pueden ver todos los pasos de los algoritmos junto, por
ejemplo, a las regiones de Voronoi o el pseudocódigo. Esto hace que el alumno
pueda conocer el progreso y los conceptos particulares (influencia,
vecindad...). Los algoritmos implementados por la herramienta son:
\begin{enumerate}
    \item Algoritmo Condensado de Hart (CNN)~\cite{CNNHart1968}.
    \item Algoritmo Condensado Reducido (RNN)~\cite{RNNGates1972}.
    \item Algoritmo Subconjunto Selectivo Modificado (MSS)~\cite{MSSBarandela2005}.
    \item Algoritmos Decremental Reduction Optimization Preocedure (DROP)~\cite{DROPWilson2000}.
    \item Algoritmo Iterative Case Filtering (ICF)~\cite{ICFBrighton2002}.
    \item Algoritmo Democratic Instance Selection (DIS)~\cite{DemoISGarcia2010}.
\end{enumerate}
Está desarrollado completamente en Java lo que lo hace portable a cualquier
plataforma y sin instalación.

Como punto característico a tener en cuenta, la herramienta estaba muy orientada
a ofrecer bastante granularidad en cada paso, de tal forma que mediante, por
ejemplo, la visualización del pseudocódigo (al mismo tiempo que la visualización
gráfica), permite para analizar el comportamiento subyacente.

Los estudiantes, de forma general, valoraron muy positivamente la herramienta
pues les ayudó a comprender los algoritmos.

\subsubsection{Towards Developing an Effective Algorithm Visualization Tool for
Online Learning} 

En~\cite{8560314} se presenta una <<guía>> especialmente interesante con las
consideraciones a tener en cuenta para el desarrollo de una correcta herramienta
de visualización de algoritmos.


La realidad actual es que resulta muy difícil hacer que los alumnos entiendan
todos los conceptos que involucran los algoritmos y más aún en la modalidad "<a
distancia">. Para solventar este problema se han desarrollado múltiples
herramientas de visualización de algoritmos (AV) que pretenden facilitar la
compresión de los mismos, pero la realidad de estas herramientas es que se
desarrollan con una fuerte componente "<elegante"> y no enriquecedora o
pedagógica. Este artículo trata de abordar este último problema, el cómo
construir herramientas de visualización de algoritmos que sean divertidas y
atractivas, pero siendo fieles a los conceptos, y también cómo estas pueden
ayudar a los docentes a asegurar el aprendizaje de sus alumnos.

Para lograr el objetivo de desarrollar un visualizador efectivo, se analiza
desde el punto de vista de la pedagogía, usabilidad y accesibilidad. En la Tabla
~\ref{tabla:objetivospedagogicos} se pueden ver algunas de las características
que se deben tener en cuenta a la hora de desarrollar una herramienta de este
estilo.
\begin{table}[]
    \resizebox{\textwidth}{!}{%
    \begin{tabular}{ll}
    \hline
    \rowcolor[HTML]{C0C0C0} 
    \textbf{Característica}     & \textbf{Cómo lograrlo}   \\ \hline
    \rowcolor[HTML]{FFFFFF} 
    \textbf{\begin{tabular}[c]{@{}l@{}}Incrementar la\\comprensión\end{tabular}} & \begin{tabular}[c]{@{}l@{}}Explicaciones textuales\\ Realimentación con las acciones del usuario\\ Ayuda integrada\\ Comodidad en la entrada de datos\end{tabular}                                                              \\ \hline
    \textbf{\begin{tabular}[c]{@{}l@{}}Promocionar\\ interés\end{tabular}}     & \begin{tabular}[c]{@{}l@{}}Permitir introducir datos al algoritmo\\ Manipular la visualización\\ Capacidad de volver <<rebobinar>>\\ Variación de la velocidad\\ Avanzar en la ejecución\end{tabular}         \\ \hline
    \textbf{Facilidad de aprendizaje}                                          & \begin{tabular}[c]{@{}l@{}}Controles familiares\\ Generalidad con otros sistemas\\ Predictibilidad de las acciones\\ Interfaz simple\end{tabular}                                                                                  \\ \hline
    \textbf{Facilidad de memorización}                                         & Interfaz común que soporte múltiples animaciones                                                                                                                                                                                   \\ \hline
    \textbf{Facilidad de uso}                                                  & \begin{tabular}[c]{@{}l@{}}Interfaz común que soporte múltiples animaciones\\ Claridad de los modelos y conceptos\end{tabular}                                                                                                     \\ \hline
    \textbf{Robustez}                                                          & \begin{tabular}[c]{@{}l@{}}Explicaciones de la visualización\\ Integración de ajustes predefinidos\\ Recuperación de errores\end{tabular}                                                                                       \\ \hline
    \textbf{Eficacia}                                                          & \begin{tabular}[c]{@{}l@{}}Visibilidad del estado del sistema\\ Control y libertad del usuario\\ Prevención de errores\\ Flexibilidad de uso\\ Ayuda al usuario para reconocer, diagnosticar y recuperarse de errores\end{tabular} \\ \hline
    \rowcolor[HTML]{FFFFFF} 
    \textbf{Accesibilidad}                                                     & \begin{tabular}[c]{@{}l@{}}Globalmente disponible\\ Plataforma y dispositivo independiente\end{tabular}                                                                                                                            \\ \hline
    \end{tabular}%
    }
    \caption{Características que influyen en los objetivos pedagógicos.}
\label{tabla:objetivospedagogicos}
    \end{table}


\section{Otras herramientas/bibliotecas}

Algorithm-Visualizer, \url{https://algorithm-visualizer.org/}. Se trata de una
Web que incluye una cantidad enorme de algoritmos, todos están programados
directamente en Javascript que genera una visualización de los mismos y también
incluye una explicación corta de cada uno. Algunos de los algoritmos son:
\texttt{Backtracking}, \texttt{Divide and Conquer}, \texttt{Greedy} o más
concretamente \texttt{K-Means Clustering}, entre muchos otros.

\imagen{memoria/algorithm-visualizer}{K-Means Clustering en algorithm-visualizer.}{1}

VISUALGO, \url{https://visualgo.net/en}. Es una página web creada por la
Universidad Nacional de Singapur. Esta Web permite visualizar algoritmos básicos
(como de ordenación o camino más corto) pero también estructuras de datos (como
\emph{hash tables} o conjuntos disjuntos).

\imagen{memoria/visualgo}{Viajante de comercio en VISUALGO.}{1}

Visualizer, por Zhenbang Feng, \url{https://jasonfenggit.github.io/Visualizer/}
\footnote{ Github: \url{https://github.com/JasonFengGit/Visualizer}}. Su autor
lo define como una página web para proporcionar visualizaciones intuitivas e
innovadoras de algoritmos generales y de inteligencia artificial. Las
visualizaciones son muy atractivas, con las que se capta muy bien el
funcionamiento de los algoritmos. Implementa algoritmos de búsqueda de caminos,
ordenamiento e inteligencia artificial (como minimax).

\imagen{memoria/visualizer}{Mergesort en Visualizer (Zhenbang Feng).}{1}


Clutering-Visualizer, \url{https://clustering-visualizer.web.app/}. Es una
página Web para visualizar algoritmos de clustering. Realmente, no tiene muchos
implementados, pero las visualizaciones son muy descriptivas. En cada paso
muestra cuál es la evolución/posición de los distintos \textit{clusters} y mediante línea
es posible ver las distancias en ellos y los datos.

\imagen{memoria/Clutering-Visualizer}{K-Means en Clutering-Visualizer.}{1}


MLDemos, puede visitarse desde \url{https://basilio.dev/}. Es una herramienta de
visualización de código abierto para algoritmos de aprendizaje automático creada
específicamente para ayudar en el estudio y compresión del funcionamiento de una
gran cantidad de algoritmos. Entre ellos: clasificación, regresión, reducción de
dimensionalidad y clustering.

\imagen{memoria/MLDemos}{Clasificación mediante Support Vector Machine en MLDemos.}{1}
