\capitulo{6}{Trabajos relacionados}

Con el aprendizaje automático y el uso de gran cantidad de datos, es
completamente necesaria la visualización de los datos, procesos de entrenamiento
y ciertas estadísticas. Al fin y al cabo, cuando se construyen modelos, se debe
tener una realimentación de cómo de bien está funcionando para poder extraer
conclusiones sobre el mismo.

De forma general, sin centrarse directamente en el aprendizaje automático,
resulta interesante y conveniente que los visualizadores de algoritmos sean
accesibles y que, como están apareciendo, se creen aplicaciones Web que resultan
mucho más directas. Desde el punto de vista de la docencia y aprendizaje los
visualizadores permiten culminar la comprensión los aspectos teóricos
subyacentes.

\section{Visualizadores}

\textbf{Seshat} es una herramienta web que trata de facilitar el aprendizaje de
la teoría de lenguajes y autómatas (análisis léxico) que utilizan los
compiladores \cite{arnaiz2018seshat} , puede accederse desde
\url{http://cgosorio.es/Seshat}. La herramienta propone en primera instancia
unas explicaciones teóricas de los algoritmos con sus conceptos generales, el
concepto concreto de las expresiones regulares y qué es un autómata finito. A
partir de la teoría, se encuentran implementados varios algoritmos que se
visualizan paso a paso. En el momento de la ejecución también se tienen
elementos de interés como la propia descripción o explicación del algoritmo. Los
algoritmos implementados por la herramienta son:
\begin{enumerate}
    \item Construcción de un autómata finito no determinista (AFND) a partir de una expresión regular.
    \item Conversión de un autómata finito no determinista (AFND) a un autómata finito determinista (AFD).
    \item Construcción de un autómata finito determinista (AFD) a partir de una expresión regular.
    \item Minimización de un autómata finito.
\end{enumerate}
Para su construcción se ha usado el framework Flask en Python que actúa como
servidor. La interfaz de usuario está construida con HTML, SVGs y Javascript
para proporcionar el contenido dinámico.


\textbf{Herramienta de apoyo a la docencia de algoritmos de selección de instancias}
\cite{arnaiz2012herramienta}. Es una herramienta de escritorio para la
visualización de la ejecución y resultados de los algoritmos de selección de
instancias debido a la carencia de este tipo de aplicaciones para dichos
algoritmos. La herramienta es altamente personalizable pudiendo subir el
conjunto de datos o seleccionar qué característica visualizar en los ejes. En la
visualización se pueden ver todos los pasos de los algoritmos junto, por
ejemplo, a las regiones de Boronoi o el pseudocódigo. Esto hace que el alumno
pueda conocer el progreso y los conceptos particulares (influencia, vecindad...). Los
algoritmos implementados por la herramienta son:
\begin{enumerate}
    \item Algoritmo Condensado de Hart (CNN)
    \item Algoritmo Condensado Reducido (RNN).
    \item Algoritmo Subconjunto Selectivo Modificado (MSS)
    \item Algoritmos Decremental Reduction Optimization Preocedure (DROP)
    \item Algoritmo Iterative Case Filtering (ICF)
    \item Algoritmo Democratic Instance Selection (DIS)
\end{enumerate}
Está desarrollado completamente en Java lo que lo hace portable a cualquier
plataforma y sin instalación.

\section{Otras herramientas/librerías}
Existen algunas herramientas bastante interesantes y visuales pero más pequeñas.


