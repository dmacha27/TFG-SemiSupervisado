\capitulo{2}{Objetivos del proyecto}

Este apartado explica de forma precisa y concisa cuáles son los objetivos que se
persiguen con la realización del proyecto.


Los \textbf{objetivos generales} del proyecto son los que siguen:
\begin{enumerate}
    \item Implementación de una biblioteca con cuatro algoritmos de aprendizaje
    semi-supervisado más comunes: \emph{Self-Training}, \emph{Co-Training},
    \emph{Democratic Co-Learning} y \emph{Tri-Training}.
    \item Diseño y creación de una aplicación Web desde la accesibilidad y
    enfoque docente (ayudas contextuales, explicaciones, pseudocódigos,
    instrucciones...).
    \item Integración de los cuatro algoritmos en la aplicación Web para su
    visualización.
    \item Crear un sistema de usuarios que les permita controlar sus ficheros y ejecuciones.
    \item Internacionalización de la página web de la aplicación.
\end{enumerate}

Como \textbf{requisitos técnicos} y más particulares del proyecto se encuentran
los siguientes:

\begin{enumerate}
    \item Implementación de los algoritmos en Python 3.10.
    \item Utilización del \emph{framewrok} Flask para el desarrollo de la
    aplicación web.
    \item Diseño de la web basado en Bootstrap 5.
    \item Creación de las visualizaciones mediante JavaScript y la biblioteca
    \texttt{D3.js}.
    \item Optimizar al máximo posible el procesamiento de datos y entrenamiento.
    \item Internacionalización mediante Babel (Flask-Babel).
    \item Creación y manejo de una base de datos para usuarios (y relativos) con
    independencia de la tecnología utilizada (SQLAlchemy).
    \item Realización de pruebas sobre el software desarrollado
    (comparativa/validación con otras librerías y test unitarios mediante
    \texttt{pytest}).
    \item Crear una documentación de usuario y programador precisa y completa.
\end{enumerate}