\apendice{Plan de Proyecto Software}

\section{Introducción}

\section{Planificación temporal}
La planificación temporal se llevará a cabo mediante Sprint de 2 semanas.
En la presente sección se comentará el desarrollo realiza en cada uno de ellos.

\subsection{Sprint 0}

Desde el punto de vista temporal corresponde desde el inicio del curso del
primer cuatrimestre académico (septiembre) hasta el Sprint 1. El día 15 de
septiembre se tuvo la primera reunión con los tutores sobre el trabajo presente
donde se establecieron las líneas generales y temática sobre el mismo.

Se creó el repositorio del TFG en Github: \url{https://github.com/dma1004/TFG-SemiSupervisado}
y se añadió la plantilla de la documentación.

\subsection{Sprint 1}

Corresponde con el periodo temporal del 5 al 19 de octubre de 2022. 

El mismo día 5 tuvo lugar una reunión de seguimiento del trabajo. Durante el sprint se
realizaron unos arreglos de la plantilla y una lectura de conceptos teóricos
para posteriormente añadirlos a la documentación. Concretamente se crearon las
tareas ``Añadir conceptos teóricos aprendizaje'' y ``Trabajos relacionados'' a
día 9 de octubre.

\subsection{Sprint 2}

Corresponde con el periodo temporal del 19 de octubre al 2 de noviembre de 2022. 

Durante el sprint se implementó un prototipo del algoritmo Self-Training en el
que posteriormente se hicieron unas correcciones en el código. También se
comenzó con la redacción de conceptos teóricos (tarea ``In progess''), concretamente, sobre el
aprendizaje automático.

\subsection{Sprint 3}

Corresponde con el periodo temporal del 2 al 16 de noviembre de 2022.

\subsection{Sprint 4}

Corresponde con el periodo temporal del 16 al 30 de noviembre de 2022.

Durante el sprint se aumentaron los conceptos teóricos sobre el aprendizaje
supervisado, no supervisado y semi-supervisado. Se refactorizó el prototipo para
su documentación (PEP), evitar datos duplicados y modularizando el código.

La memoria fue parcialmente modificada basándose en las correcciones propuestas
de los tutores.

\subsection{Sprint 5}

Corresponde con el periodo temporal del 25 al 8 de febrero de 2023.

Durante el sprint se retomaron las tareas y el desarrollo general del proyecto.
Se mejoró el algoritmo de SelfTraining que estaba como prototipo y se avanzó en
la tarea de primera aproximación en Web mediante Flask. Sobre esto último, se creó
una visualización del proceso de entrenamiento muy básica por cada iteración.

Se creó un prototipo del algoritmo Co-Training sin cumplir con todas sus
condiciones que posteriormente se completaron a falta de revisión. 

Sobre estos dos algoritmos se propuso la versión 1.0. 


\section{Estudio de viabilidad}

\subsection{Viabilidad económica}

\subsection{Viabilidad legal}


