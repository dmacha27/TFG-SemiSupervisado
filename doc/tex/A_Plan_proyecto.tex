\apendice{Plan de Proyecto Software}

\section{Introducción}

En el presente apartado de los anexos se analizará la gestión del proyecto
software desarrollado. Este proyecto será organizado mediante la metodología
Scrum en la que el trabajo estará dividido en Sprints. Por cada Sprint se
realiza una reunión para la revisión del avance y los objetivos para el
siguiente.  Con esta metodología se mantendrá en todo momento lo que se conoce
como \texttt{Product Backlog} que es una lista de las tareas a realizar. Esta
lista será actualizada, en principio, en cada reunión para así mantener un
desarrollo constante. Las reuniones en un principio se realizan cada dos
semanas, intensificando a cada semana en el momento del inicio del periodo
temporal del segundo cuatrimestre.

El objetivo de este plan es servir como herramienta para registrar el avance del
proyecto y también para poder cumplir con el objetivo final del desarrollo.


\section{Planificación temporal}
La planificación temporal se llevará a cabo mediante Sprint de 2 semanas.
En la presente sección se comentará el desarrollo realiza en cada uno de ellos.

\subsection{Sprint 0}

Desde el punto de vista temporal corresponde desde el inicio del curso del
primer cuatrimestre académico (septiembre) hasta el Sprint 1. El día 15 de
septiembre se tuvo la primera reunión con los tutores sobre el trabajo presente
donde se establecieron las líneas generales y temática sobre el mismo.

Se creó el repositorio del TFG en Github: \url{https://github.com/dma1004/TFG-SemiSupervisado}
y se añadió la plantilla de la documentación.

\subsection{Sprint 1}

Corresponde con el periodo temporal del 5 al 19 de octubre de 2022. 

El mismo día 5 tuvo lugar una reunión de seguimiento del trabajo. Durante el sprint se
realizaron unos arreglos de la plantilla y una lectura de conceptos teóricos
para posteriormente añadirlos a la documentación. Concretamente se crearon las
tareas "<Añadir conceptos teóricos aprendizaje"> y "<Trabajos relacionados"> a
día 9 de octubre.

\subsection{Sprint 2}

Corresponde con el periodo temporal del 19 de octubre al 2 de noviembre de 2022. 

Durante el sprint se implementó un prototipo del algoritmo Self-Training en el
que posteriormente se hicieron unas correcciones en el código. También se
comenzó con la redacción de conceptos teóricos (tarea "<In progess">), concretamente, sobre el
aprendizaje automático.


\subsection{Sprint 3}

Corresponde con el periodo temporal del 16 al 30 de noviembre de 2022.

Durante el sprint se aumentaron los conceptos teóricos sobre el aprendizaje
supervisado, no supervisado y semi-supervisado. Se refactorizó el prototipo para
su documentación (PEP), evitar datos duplicados y modularizando el código.

La memoria fue parcialmente modificada basándose en las correcciones propuestas
de los tutores.

\subsection{Sprint 4}

Corresponde con el periodo temporal del 25 de enero al 1 de febrero de 2023. En este
momento las duraciones de los Sprints cambiaron a una semana, iniciando así el
periodo temporal real del desarrollo del proyecto (segundo cuatrimestre)

Durante el sprint se retomaron las tareas y el desarrollo general del proyecto.
Se mejoró el algoritmo de \texttt{SelfTraining} que estaba como prototipo y se avanzó en
la tarea de primera aproximación en Web mediante Flask. Sobre esto último, se creó
una visualización del proceso de entrenamiento muy básica por cada iteración.

Se creó un prototipo del algoritmo \texttt{CoTraining} sin cumplir con todas sus
condiciones que posteriormente se completaron a falta de revisión. 

Sobre estos dos algoritmos se propuso la versión 1.0. 

Continuando con la Web, se realizó la interfaz general funcional. Incluye:
\begin{itemize}
    \item Página de Inicio donde seleccionar el algoritmo.
    \item Página de subida de archivos en formatos ARFF y CSV de los conjuntos de datos
    \item Páginas correspondientes para SelfTraining y CoTraining: Cada una
    tiene sus parámetros específicos con la posibilidad de seleccionar si
    utilizar PCA (Principal Component Analysis) o dos componentes que elija el usuario.
    \item Página de visualización del algoritmo (su entrenamiento): Se tiene la
    vista principal que será común a todos los algoritmos (con algunas
    variaciones en caso necesario) con la posibilidad de avanzar en la
    visualización (con controles) y barra de progreso. Desde el punto de vista
    del gráfico los colores están automatizados dependiendo del número de
    clases, leyenda y etiqueta de ejes. 
\end{itemize}
En el servidor (Flask) a nivel de programación se añadieron los "<endpoints">
correspondientes (subida, configuración, visualización...) y un control de
acceso a las páginas muy básico (por ejemplo, si no se configuró el algoritmo,
no se puede visualizar y le redirecciona a la configuración con un mensaje de error)

\subsection{Sprint 5}
Corresponde con el periodo temporal del 1 al 8 de febrero de 2023.

En la reunión del 1 de febrero se revisó lo realizado en el anterior y se fijó
una serie de mejoras/modificaciones y nuevas tareas:
\begin{enumerate}
    \item Modificación de los algoritmos para trabajar con la convención de
    "<-1s"> en el conjunto de datos para los datos no etiquetados. Así el
    usuario podrá subir un archivo ya \texttt{Semi-Supervisado}.
    \item Permitir al usuario seleccionar los porcentajes de no etiquetados y de
    test (para las futuras estadísticas)
    \item De la página (no gráfico) visualización de los algoritmos: volver a la configuración, el
    "<feedback"> de la iteración actual y el nombre del conjunto de datos utilizado.
    \item Del gráfico de la visualización: Diferenciar en el algoritmo
    CoTraining cuál de los dos clasificadores han etiquetado cada punto y los
    puntos "<etiquetados"> en la iteración 0 deben mostrarse de forma diferente.
    \item Avanzar con los trabajos relacionados.
    \item Avanzar con la documentación teórica y anexos.
\end{enumerate}

El punto 1 ha llevado unas 12 horas de compresión y desarrollo. Esto es debido a
que los dos algoritmos implementados hasta ahora debían ser modificados para
trabajar con la nueva convención. Además, el problema principal fue (aunque no
implementado en este Sprint) dejar preparado una forma de carga del conjunto de
datos que permita tratar datos no etiquetados ("<?"> por ejemplo en el caso de
ARFF) pues además los algoritmos (su correcto funcionamiento) se han probado con
ficheros. También conllevó la creación de un codificador de etiquetas propio
para ignorar los no etiquetados en clases categóricas (y no realizar la
conversión en esos casos)

El punto 2 volvió a causar bastantes problemas tanto en la ejecución de los
algoritmos como en la Web. Hasta el momento, el usuario no seleccionaba los
porcentajes de las divisiones. Al incluir esto, los algoritmos ya no se encargan
de esta tarea y había que modificar tanto los algoritmos como aquellas rutas de
la Web que debían encargarse de esto. Aproximadamente 4 horas.

El punto 3 no resultó demasiado difícil más allá de seguir habituándose a
Javascript/HTML. Unas 3 horas.

El punto 4 requirió unas 10 horas, en un principio se perdió mucho tiempo
intentando solucionarlo de una forma que resultó inútil, pero finalmente ahora en
el algoritmo se diferencian los datos clasificados por cada uno.

Los trabajos relacionados (no terminados) se realizaron en varios días con un
tiempo aproximado de 6 horas.

\subsection{Sprint 6}
Corresponde con el periodo temporal del 8 al 15 de febrero de 2023.

En la reunión del 8 de febrero se revisó lo realizado en el anterior y se
comentaron algunas tareas a realizar:
\begin{enumerate}
    \item En la línea del anterior, los algoritmos deben poder ejecutarse
    directamente con conjuntos de datos semi-supervisados.
    \item Permitir al usuario introducir ese tipo de conjuntos de datos.
    \item Realizar alguna visualización de estadísticas.
    \item Valor por defecto en las configuraciones.
    \item Sobre el gráfico: mejorar la diferenciación de los puntos, información
    útil en los "<tooltips"> y colocación leyenda.
    \item Avanzar con los trabajos relacionados.
    \item Avanzar con la documentación teórica y anexos.
\end{enumerate}


\section{Estudio de viabilidad}

\subsection{Viabilidad económica}

\subsection{Viabilidad legal}


