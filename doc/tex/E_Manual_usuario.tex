\apendice{Documentación de usuario}

\section{Introducción}

Es esta sección se presenta los requisitos que el usuario debe satisfacer para
utilizar la aplicación junto con la instalación y manual de la misma.

\section{Requisitos de usuarios}

Los requisitos que debe cumplir el usuario son:
\begin{itemize}
    \item Periféricos básicos: pantalla, teclado y ratón.
    \item Navegador Web (Firefox, Chrome, Edge...).
    \item Conexión a internet.
    \item JavaScript habilitado en el navegador.
\end{itemize}

\section{Instalación}

El usuario no necesita instalar el software para utilizarlo. Puede acceder a él
directamente desde su navegador.

\section{Manual del usuario}

Con la ayuda de imágenes capturadas directamente de la Web, esta sección
describe cómo se realizan todas las acciones de la aplicación.

Dado que la documentación presente se encuentra en español, todas las interfaces
se mostrarán en español. Aun así, la aplicación está preparada para el idioma
inglés de igual forma.

Este manual está pensado para los usuarios anónimos y los usuarios con cuenta
registrada.

\subsection{Visualizar un algoritmo}

El flujo para visualizar un algoritmo en el siguiente:

\imagen{anexos/manual-usuario/FlujoVisualizarAlgoritmo}{Flujo de la visualización de un algoritmo}

\subsubsection{Seleccionar algoritmo}
Para seleccionar un algoritmo, se puede hacer de dos formas. La primera es
haciendo clic a los enlaces que aparecen en la barra de navegación (que además
está siempre presente en todas las pestañas) y también haciendo clic en las
tarjetas de presentación de cada algoritmo en la página principal.

Una vez que se haya seleccionado, será redirigido a la página de subida del
fichero. 

\subsubsection{Carga del conjunto de datos}

El fichero contendrá el conjunto de datos. Para subir un fichero, simplemente se
puede arrastrar desde el propio sistema hasta la zona marcada con rayas o
abriendo el explorador de archivos con el botón de <<Selecciona fichero>>. Podrá
ver durante la carga el progreso de la misma.

Si el usuario no dispone de un fichero, la aplicación incluye un enlace para
descargar uno de prueba, pulsando en el botón <<Descargar fichero de prueba>>. A
partir de aquí, es el mismo procedimiento comentado anteriormente.

\imagen{anexos/manual-usuario/Carga del conjunto de datos - VASS}{Carga del conjunto de datos}

\paragraph{Consideraciones del conjunto de datos} En primer lugar, los ficheros
subidos solo podrán tener extensiones ARFF o CSV, en caso contrario, al intentar
pasar al siguiente paso, el usuario será devuelto a esta misma página con un
mensaje de error.

El contenido del fichero de datos tiene que cumplir un requisito fundamental
derivado de la ausencia de un pre-procesado completo: 

\begin{tcolorbox}[colback=red!5!white,colframe=red!75!black,fontupper=\footnotesize,title=Requisito fundamental]
Todos los atributos del conjunto de datos deben ser numéricos (internamente los
algoritmos requieren de este tipo de datos), esto \textbf{no} incluye al
atributo de la clase, que sí puede ser categórico/nominal (esa parte del
pre-procesado sí que es realizada).
\end{tcolorbox}

Las condiciones de entrada vienen especificadas al pulsar en el botón de la
esquina superior derecha del recuadro de selección de fichero <<Condiciones>>.
Al pulsar en él aparecerá una ventana emergente con dichas condiciones.

\imagencontamano{anexos/manual-usuario/Condiciones - VASS}{Condiciones de entrada}{0.6}

Es importante remarcar que salvo la condición de la extensión, la aplicación no
puede controlar en este punto el resto de condiciones (ocurrirán mensajes de
error en durante el manejo posterior).

Una vez que el fichero ha sido cargado (porcentaje completado), se habrá
habilitado el botón de configuración. Pulsando en él, se redirigirá a la
siguiente página del flujo (configuración).

\subsubsection{Configuración del algoritmo}
\label{mu:configuracion}
Se encontrará en la página de configuración del algoritmo, donde podrá observar
un apartado teórico con sus conceptos generales y su pseudocódigo. Por otro
lado, tendrá un formulario con todos los parámetros que se pueden configurar
para ese algoritmo.

\imagen{anexos/manual-usuario/Configuración del algoritmo - VASS}{Configuración del algoritmo}

Todos los parámetros tienen establecido un valor por defecto (con la
configuración <<estable>>), pero se tiene libertad completa para modificar cada
uno de ellos. En principio, si solamente se quiere ejecutar sin modificar ningún
parámetro, es necesario seleccionar correctamente el atributo de la clase.
Este atributo no puede ser establecido automáticamente por la aplicación, ya que
depende del conjunto de datos subido. 

Una vez configurado, se puede visualizar el algoritmo pulsando en el botón de
<<Ejecutar>>.

\subsubsection{Visualización}
\label{mu:visualizacion}
Ya en la página de visualización, se mostrará durante unos momentos una
animación de carga. Cuando el sistema haya finalizado la ejecución, se podrán
ver los distintos gráficos.

En principio, los errores que ocurran en el sistema serán mostrados. Sin
embargo, en el caso de que la animación dure un periodo de tiempo excesivo, se
recomienda reintentar la configuración (podrían ser problema de red
simplemente).

\imagen{anexos/manual-usuario/Visualización - VASS}{Vista general de una visualización}

\paragraph{Visualización principal} En el gráfico principal se mostrará el
conjunto de datos en dos dimensiones. Aquí podrá verse qué es lo que ocurre
durante el proceso de entrenamiento del algoritmo.

\imagen{anexos/manual-usuario/Visualización principal}{Visualización principal}

Este gráfico es interactivo:
\begin{itemize}
    \item Permite realizar \texttt{zoom} sobre una zona deseada mediante el
    doble clic o moviendo la ruleta del ratón.
    \item Al pasar el ratón por uno de los puntos, se mostrará toda la
    información relativa a esa posición en un recuadro informativo (aparecerá en
    las proximidades del ratón). Esto se está ejemplificando en la imagen
    anterior.
\end{itemize}

Para controlar la evolución, en la parte inferior se encuentra un panel de
control que permite:
\begin{itemize}
    \item Reiniciar \texttt{zoom}. Pese a que es posible reducir/aumentar el
    \texttt{zoom} manualmente, sirve para volver a la posición original.
    \item Visualizar la iteración actual: mediante el número y una barra de
    progreso.
    \item Reproducir automáticamente pulsando en el botón con el símbolo
    \textit{play}.
    \item Avanzar iteración manualmente.
    \item Reducir iteración.
\end{itemize}

Todas estas acciones modificarán el estado de los gráficos.

\paragraph{Tooltip} El \textit{tooltip} que se muestra al pasar el ratón por
encima de un punto tiene varios formatos dependiendo del algoritmo mostrado y de
los datos introducidos.

\paragraph{Tooltip: Casos comunes}

Existe un formato común a todos los algoritmos para los casos de los datos
iniciales. Este tipo de puntos se representa mediante un círculo. Como dato
inicial, tendrá la etiqueta correspondiente. Además, cada punto tiene en la
parte superior la posición que ocupa en el gráfico. Esta posición coincide con
los atributos seleccionados para representar los datos. En la siguiente imagen
se presenta un ejemplo de todo lo anterior donde se seleccionó PCA y por eso
aparece <<C1>> y <<C2>>.

\imagencontamano{anexos/manual-usuario/tooltips/Inicial}{Tooltip con dato inicial}{0.25}

Otro formato común es en el caso de puntos solapados (derivados de ejemplos
duplicados en el conjunto de datos o por PCA). En este caso, donde en el anterior
aparecía la información del punto, ahora aparecerá un listado con todos los
puntos solapados. Ejemplo:

\imagencontamano{anexos/manual-usuario/tooltips/Solapados}{Tooltip con datos solapados}{0.25}

Como se puede ver en este ejemplo anterior, los puntos no han sido clasificados
(captura tomada en iteración cero), esto no es importante, la diferencia con una
iteración posterior es que aparecerá la etiqueta asignada (se verá en cada
algoritmo). Además, aparece un indicativo de <<Clasificador>>, se ha querido
incluir en este ejemplo porque en todos los algoritmos excepto Self-Training se
indica el clasificador que se ha encargado de etiquetar cada punto.

Es interesante comentar que puede darse el caso de puntos solapados en el que
alguno o todos sean datos iniciales. Por ejemplo:

\imagencontamano{anexos/manual-usuario/tooltips/Solapados+Inicial}{Tooltip con datos solapados con uno inicial}{0.25}

\paragraph{Tooltip: Self-Training}

El formato de tooltip para Self-Training no tiene grandes complicaciones y
comprendiendo los anteriores ejemplos resulta sencillo de interpretar.

Existen dos formatos, el primero es en el que el dato no ha sido etiquetado
(porque se etiqueta en una iteración posterior o simplemente porque nunca es
etiquetado). Por ejemplo, una captura tomada en la iteración dos de una dato no
etiquetado es\footnote{En el ejemplo puede ser extraño que no aparezca un título
indicando el punto (algo como <<Punto 1>>), ese formato solo se <<activa>>
cuando existen datos solapados.}:

\imagencontamano{anexos/manual-usuario/tooltips/STNoClasificado}{Tooltip con un dato no clasificado}{0.25}

El otro formato es en el que el dato ya ha sido clasificado (en la iteración
actual o en una previa). Por ejemplo, una captura tomada en la iteración ocho de
una dato etiquetado en la iteración seis es:

\imagencontamano{anexos/manual-usuario/tooltips/STClasificado}{Tooltip con un dato clasificado}{0.3}

En este formato anterior se muestra la etiqueta asignada así como la iteración
en la que se clasificó entre paréntesis.

\paragraph{Tooltip: Co-Training}

Realmente, el tooltip para Co-Training es muy similar a Self-Training. No existe
ningún caso extraño o adicional.

Vuelven a existir dos formatos, cuando el dato no ha sido etiquetado todavía (o
nunca) y cuando sí está etiquetado. 

\imagencontamano{anexos/manual-usuario/tooltips/CTNoClasificado}{Tooltip con un dato no clasificado}{0.3}

Para el caso del dato clasificado sí que es necesario puntualizar alguna
cuestión (se verá en el ejemplo siguiente). Co-Training considera dos
clasificadores base y cada uno de ellos puede clasificar a un punto. Esto se ha
representado de dos maneras. La primera señal es que el símbolo del punto será
uno concreto para cada clasificador. La segunda señal es que en el tooltip
aparecerá una línea de <<Clasificador:>> que estará seguida de dicho símbolo y
el nombre del clasificador. Se mantiene también el número de la iteración en la
que se ha clasificado.

\imagencontamano{anexos/manual-usuario/tooltips/CTClasificado}{Tooltip con un dato clasificado}{0.45}

Esta idea también es la misma si existen datos solapados. En ese caso, en cada
uno de los puntos que se indiquen en el listado de solapados aparecerá el
clasificador (el símbolo y el nombre) junto con la etiqueta.

\imagencontamano{anexos/manual-usuario/tooltips/CTSolapadosClasificados}{Tooltip con datos solapados y clasificados}{0.45}

\paragraph{Tooltip: Democratic Co-Learning y Tri-Training}

En el caso de que un dato no haya sido clasificado es exactamente lo mismo que
para Co-Training, simplemente mostrará la posición y que no ha sido clasificado
(<<No clasificado>>).

La particularidad que tiene Democratic Co-Learning y Tri-Training es que cada
dato puede ser etiquetado por varios clasificadores. Para ejemplificar esto
simplemente se realiza un listado de los clasificadores que lo han etiquetado.

\imagencontamano{anexos/manual-usuario/tooltips/DCLClasificados1}{Tooltip con un dato clasificado por dos clasificadores}{0.45}

Es de destacar que solo aparecen aquellos que en la iteración actual o una
previa han etiquetado el dato.

El formato de ambos algoritmos es el mismo. Sin embargo, se cree conveniente
explicar el funcionamiento básico.

En \textbf{Democratic Co-Learning} cada punto solo puede ser clasificado
\textbf{una vez} por cada clasificador durante toda la ejecución. Esto, a
efectos de visualización, significa que el listado de clasificadores del tooltip
solo puede aumentar (si es que el punto es clasificado alguna vez).

En \textbf{Democratic Co-Learning} cada punto puede ser clasificado
\textbf{varias veces} por cada clasificador. Esto es así porque cada
clasificador mantiene su propio conjunto de entrenamiento y este es vaciado al
comienzo de cada iteración (y rellenado durante la misma). Todo esto significa
que durante una visualización, un ejemplo puede ser clasificado por una lista de
clasificadores y en la siguiente iteración por otra (igual o distinta).

Esto no afecta al formato comentado, pero puede llegar a ser extraño sin
especificarlo.

Por último, como en todos los anteriores formatos, pueden existir datos
solapados. El ejemplo mostrado a continuación es de Democratic Co-Learning, pero
como se ha comentado es el mismo formato que el de Tri-Training.

\imagencontamano{anexos/manual-usuario/tooltips/DCLSolapadosClasificados}{Tooltip con datos solapados y clasificados}{0.45}

En este ejemplo solo un clasificador ha clasificado cada dato solapado. Se ha
elegido este ejemplo porque, como es de esperar, si los tres clasificadores
etiquetan, el tooltip crecerá de igual manera.

\paragraph{Gráficos estadísticos} Pasando a otra parte de la ventana completa de
visualización, en la zona de la derecha se incluirá el resto de gráficos
adicionales, que serán principalmente estadísticas.

\imagen{anexos/manual-usuario/ZonaEstadisticas}{Gráficos estadísticos}

En la parte superior de esta zona se tiene un desplegable (contraído por
defecto) que contiene el pseudocódigo (el mismo que en la fase de
configuración). Si se desea consultar, simplemente se pulsa en cualquier parte
del desplegable:

\imagen{anexos/manual-usuario/Pseudocodigo}{Desplegable pseudocódigo}

El gráfico de estadísticas generales simplemente será a interpretación del
usuario (no puede realizar ninguna opción). 

En el caso de las estadísticas específicas puede seleccionar qué estadística
mostrar mediante el selector superior, y de qué clasificadores mostrarla
mediante las casillas en la parte inferior.

Ambos gráficos anteriores contienen información similar, en el eje X se indican
las iteraciones y en el eje Y el valor de la estadística(s).

\subsection{Ayuda y cambio de idioma}



Aunque la propia aplicación detecta el idioma más adecuado (entre español e
inglés) que mostrar, se puede seleccionar el idioma de forma manual.

En la barra de navegación hay un símbolo de traducción característico que al
pulsar aparece un desplegable.

También hay un icono con una interrogación, este es un enlace directo al manual
de usuario de la aplicación (esta documentación). Al pulsar en él se abre una
nueva pestaña con dicho manual (sin modificar la actual y sin descargar).

\imagencontamano{anexos/manual-usuario/Cambiar de idioma}{Cambiar de idioma}{0.4}

Pulsando en el idioma se refrescará la página acorde al idioma seleccionado.
Esta característica \textbf{no} se incluye en la página de visualización.

\subsection{Registrarse}

Para aquellos usuarios anónimos que deseen crear una cuenta en la aplicación,
será necesario realizar el proceso de registro.

Para acceder a él, en la barra de navegación se dispone de un enlace (en la
parte derecha) que redirecciona al formulario de creación. Una vez accedido, se
deben rellenar los siguientes campos (todos obligatorios):
\begin{itemize}
    \item Nombre: entre 2 y 10 caracteres.
    \item Correo electrónico: será el identificador del usuario en el sistema y
    por lo tanto solo podrá haber uno.
    \item Contraseña: con al menos ocho caracteres.
    \item Confirmar contraseña: misma contraseña que el campo anterior.
\end{itemize}

\imagen{anexos/manual-usuario/Registrarse - VASS}{Formulario de registro}

Una vez enviado el formulario (y después de la comprobación de todos los
campos), la cuenta quedará registrada y se habrá iniciado sesión
automáticamente.

\subsection{Iniciar sesión}

Al igual que para el registro, para iniciar sesión existe un enlace en la barra
de navegación (en la parte derecha) que redirecciona al formulario de inicio de
sesión.

\imagen{anexos/manual-usuario/Iniciar sesión - VASS}{Formulario de inicio de sesión}

Este formulario es más sencillo y solo requiere el correo electrónico y
contraseña introducidos en el registro, o los nuevos si se han modificado (la
modificación de un perfil se verá más adelante).

\subsection{Cerrar sesión}

Si ya tiene sesión iniciada, debe hacer clic en su nombre en la barra de
navegación (parte derecha). Esto abrirá un desplegable en el que, aparte de
otras opciones, podrá cerrar la sesión.

\imagencontamano{anexos/manual-usuario/Cerrar sesión}{Cierre de sesión}{0.4}

\subsection{Personalizar perfil}

Es posible ver el perfil propio y modificar los datos con los que se creó la
cuenta.

En primer lugar, y similar a otros casos, en el desplegable de la barra de
navegación del usuario se pulsa en <<Perfil>>.

\imagencontamano{anexos/manual-usuario/DesplegablePerfil}{Acceso al perfil}{0.4}

Una vez dentro, en el lateral izquierdo aparece la información general del
perfil (ficheros subidos, ejecuciones, correo electrónico...).

La parte de edición (zona derecha) contiene un formulario similar al del
registro. Se pueden modificar todos los datos mostrados, pero para que las
modificaciones puedan realizarse, se debe introducir la contraseña actual. Si
fuera errónea o no se introduce, el formulario no se enviará.

\imagen{anexos/manual-usuario/Mi Perfil - VASS}{Perfil personal y edición}
\label{mu:perfil}

\subsection{Espacio personal}

Todos los usuarios poseen de un espacio personal en el que visualizar y
controlar sus ficheros subidos y las ejecuciones realizadas hasta el momento.

En primer lugar, y similar a otros casos, en el desplegable de la barra de
navegación del usuario se pulsa en <<Mi Espacio>>.

\imagencontamano{anexos/manual-usuario/DesplegableMiEspacio}{Acceso al espacio personal}{0.4}

Una vez dentro, en el lateral izquierdo aparece la información general del
perfil (ficheros subidos, ejecuciones, correo electrónico...).

En la parte derecha se encontrarán dos tablas en las que se reflejan los
ficheros subidos y las ejecuciones.

\imagen{anexos/manual-usuario/Mi Espacio - VASS}{Espacio personal}

Ambas tablas tienen un buscador donde se puede filtrar por cualquier palabra, en
todas las columnas de todas las filas. Además, puede elegir cuantas entradas
mostrar (selector en la esquina superior izquierda) y en su caso, pasar las
páginas para seguir mostrando más entradas (paginado en la esquina inferior
derecha).

\paragraph{Control de los conjuntos de datos subidos} Particularmente, los
conjuntos de datos (ficheros) pueden ser ejecutados o eliminados.

En el caso de querer utilizar el fichero para \textbf{ejecutar un algoritmo},
simplemente se ha de pulsar en el botón con el símbolo \texttt{play}. Al
hacerlo, se mostrará una ventana emergente (\texttt{modal}) para seleccionar el
algoritmo deseado.

\imagen{anexos/manual-usuario/EjecuciónDataset - VASS}{Selección de algoritmo}

Cuando se pulse en uno de los botones se redirige a la pestaña de configuración
(ver explicación de configuración en \ref{mu:configuracion}).

\label{mu:eliminardataset}
Por otro lado, si se quiere \textbf{eliminar un fichero} de la cuenta (y del
sistema), se pulsa en el botón con el símbolo de la papelera. De igual manera,
se mostrará una ventana emergente de confirmación.

\imagen{anexos/manual-usuario/EliminarDataset - VASS}{Eliminar fichero}

Si todo ha ido correctamente, habrá desaparecido la fila correspondiente del
fichero. En caso contrario se mostrará otra ventana emergente con el error.

\paragraph{Control de las ejecuciones} En el historial de ejecuciones también
pueden realizarse varias acciones.

\label{mu:parametrosrun}
En primer lugar, los \textbf{parámetros de configuración} que se introdujeron en
una ejecución se pueden consultar pulsando en el botón de la columna
<<Parámetros>>. Esto mostrará una ventana emergente con un JSON legible y
formateado.

\imagen{anexos/manual-usuario/ParametrosRun - VASS}{Parámetros de una ejecución}

De forma similar a los conjuntos de datos, las ejecuciones pueden ser
\textbf{re-ejecutadas}, repitiendo exactamente lo mismo que ocurrió en su
momento. Para ello simplemente se debe pulsar en el botón amarillo con el
símbolo típico de recargar página.

Este caso no se mostrará ninguna ventana emergente, redirigirá directamente a la
visualización del algoritmo (ver explicación de la visualización en
\ref{mu:visualizacion}).

\label{mu:eliminarrun}
Exactamente igual a los conjuntos de datos, las ejecuciones pueden
\textbf{eliminarse} pulsando en el botón con el símbolo de papelera mostrando
una ventana emergente de confirmación similar a la anterior.

\imagen{anexos/manual-usuario/EliminarRun - VASS}{Eliminar ejecución}


\section{Manual del administrador}

Conviene dividir el manual general para usuarios anónimos y registrados de los
administradores. Aun con ello, un usuario administrador puede realizar las
mismas acciones que el resto de roles.

Para acceder a esta sección se ha creado un administrador público con las
siguientes credenciales:
\begin{itemize}
    \item Email: admin@admin.es
    \item Contraseña: 12345678
\end{itemize}

\subsection{Panel de administración}

El administrador puede controlar a los usuarios, todos los ficheros subidos y
todas las ejecuciones. Para ello, posee de un panel de administración en el que
visualizar tablas con toda esa información.

En primer lugar, para acceder al panel, similar a otros casos, en el desplegable
de la barra de navegación del usuario (con rol administrador) se pulsa en
<<Panel de administración>>.

\imagencontamano{anexos/manual-usuario/DesplegablePanel}{Acceso al panel de administración}{0.4}

Una vez dentro, se tiene un menú organizado en pestañas donde ver las tres tablas.

\imagen{anexos/manual-usuario/PanelUsuarios - VASS}{Administración de usuarios}

\imagen{anexos/manual-usuario/PanelDatasets - VASS}{Conjunto de datos subidos}

\imagen{anexos/manual-usuario/PanelRuns - VASS}{Historial de ejecuciones}

Para el caso de <<Conjunto de datos subidos>> e <<Historial de ejecuciones>>, el
proceso es el mismo que en el manual del usuario. La diferencia es que en las
acciones solo puede eliminar (no ejecutar o re-ejecutar respectivamente).
Consultar las explicaciones  en \ref{mu:eliminardataset} (eliminar fichero),
\ref{mu:parametrosrun} (mostrar parámetros de ejecución) y \ref{mu:eliminarrun}
(eliminar ejecución).

En el caso de los usuarios, el administrador tiene dos posibilidades, editar sus
datos o eliminar el usuario.

Si se quiere \textbf{editar} un usuario, se debe pulsar el botón con el símbolo
del lápiz. Esto redirigirá a una pestaña similar a la de la figura
\ref{mu:perfil} (perfil del usuario) y de hecho, será como adoptar la vista del
usuario que se está editando salvo por la inclusión de un indicativo como
recordatorio al administrador.

\imagen{anexos/manual-usuario/PerfilAjeno - VASS}{Edición de un perfil ajeno}

Es de destacar también que en este caso, el formulario no incluye la contraseña
actual de ese usuario como confirmación. Esto es porque el administrador tiene
todos los privilegios para realizar esta acción. Cuando el administrador
modifique algún dato y envíe el formulario, los datos serán actualizados en el
sistema. 

Obviamente, los campos tienen ciertas limitaciones (similares a las del
registro):
\begin{itemize}
    \item Nombre: entre 2 y 10 caracteres.
    \item Correo electrónico: será el identificador del usuario en el sistema y
    por lo tanto solo podrá haber uno en el sistema.
    \item Nueva contraseña: con al menos ocho caracteres.
\end{itemize}

Si lo que se quiere es \textbf{eliminar} un usuario, el proceso es el mismo que
se ha visto para el resto de eliminaciones. Se debe pulsar en el botón que
incluye el símbolo de la papelera y se pedirá confirmación del proceso.

\imagen{anexos/manual-usuario/EliminarUsuario - VASS}{Eliminación de usuario}