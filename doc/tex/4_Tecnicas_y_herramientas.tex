\capitulo{4}{Técnicas y herramientas}

En este apartado se presentarán las técnicas y herramientas que se han utilizado
para el desarrollo del presente proyecto. En algunos de los casos la experiencia
u otros aspectos han hecho decantarse por una u otra o simplemente se
seleccionaron directamente.

\section{Técnicas}


\paragraph{Scrum} Es un marco de trabajo de gestión de proyectos ágiles. Aunque aplicado
también a otros trabajos, se utiliza principalmente para el desarrollo de
\textit{software} de calidad y de la forma más rápida posible.

Scrum se define como un marco de trabajo ligero, fácil de entender y difícil de
dominar \cite{scrum}. Scrum permite la entrega de valor de forma incremental y
colaborativa.

Además, Scrum define una serie de reuniones, herramientas y roles que permiten
la gestión del trabajo de forma eficiente.

Las \textbf{reuniones} (también llamadas ceremonias) son las siguientes:
\begin{itemize}
    \item \textbf{\textit{Sprint}}: periodo de tiempo en el que se realizan las tareas para crear
    un incremento del producto.
    \item \textbf{\textit{Sprint planning}} (planificación del sprint): es una reunión
    que inicia un Sprint con una duración máxima de 8 horas, en ella se
    establece el objetivo del sprint.
    \item \textbf{\textit{Daily Scrum}}: es una reunión realizada cada día para
    comprobar el progreso del trabajo marcado para el Sprint, su duración varía
    entre 10 y 15 minutos.
    \item \textbf{\textit{Sprint review}}: es una demostración del trabajo realizado
    durante el Sprint para comprobar que el progreso es el adecuado para
    alcanzar el objetivo.
    \item \textbf{\textit{Sprint retrospective}}: es la última reunión del Sprint para
    determinar que es lo que ha ido como se planeaba y qué es lo que no.
\end{itemize}

Los \textbf{artefactos} (o documentos) más relevantes son los siguientes:
\begin{itemize}
	\item \textbf{\textit{Product Backlog}}: por su nombre, hace referencia al
	producto completo y contiene todas las funcionalidades y requisitos que debe
	tener. Este documento se actualiza periódicamente.
	\item \textbf{\textit{Sprint Backlog}}: contiene las funcionalidades y
	requisitos del \textit{product backlog} que se planean cumplimentar en el
	\textit{Sprint}.
	\item \textbf{\textit{Incremento}}: es el resultado de todos los elementos
	completados durante el \textit{Sprint}, que añade valor y es potencialmente
	entregable.
\end{itemize}

Y por último, los \textbf{roles} que intervienen en todo este proceso ágil son
\cite{scrum:team}:
\begin{itemize}
	\item \textbf{\textit{Scrum Master}}: es el encargado de establecer Scrum,
	de que todos lo entiendan y lo apliquen correctamente.
	\item \textbf{\textit{Product Owner}}: se encarga de maximizar el valor del
	producto. Esto lo realiza principalmente manejando el \textit{Product
	Backlog} (ordenando) y asegurando que es visible y entendido.
	\item \textbf{\textit{Desarrolladores}}: son los encargados de llevar a cabo
	el incremente de cada \textit{Sprint}.
\end{itemize}


\paragraph{PEP8} Las siglas de PEP corresponden con \textit{Python Enhacement
Proposal}. Un PEP es un documento dedicado a la comunidad Python para describir
características de Python.

PEP8 se trata de uno de estos documentos, describe una guía sobre cómo escribir
código en Python junto con las mejores prácticas \cite{pep8:desc}. En
definitiva, es una guía de estilos. Fue escrita por Guido van Rossum, Barry
Warsaw, and Nick Coghlan en 2001. Es la guía de estilo que se ha usado en todo
el desarrollo Python del proyecto.

La necesidad de utilizar PEP8 recae concretamente en la legibilidad. La idea
subyacente de esto es que el <<código es leído mucho más a menudo de lo que es
escrito>> (Guido van Rossum). Cuando se ha escrito un código, lo lógico sería
que al volver a él, el desarrollador sepa qué es lo que esa porción de código
hace. Sin seguir unas guías de estilo esta tarea es mucho más difícil (nombres
de variables sin contexto, código amontonado, comentarios en línea y de
funciones escasos e inútiles...).

A grandes rasgos, PEP8 considera los siguientes aspectos \cite{pep8}:

\begin{itemize}
	\item Diseño de código: como las indentaciones de 4 espacios o longitud
	máxima de líneas de 79 caracteres.
	\item Cadenas de caracteres: con libertad si utilizar comillas dobles o
	simple.
	\item Espacios en blanco: recomendaciones sobre cuándo dejar o no espacios.
	\item Comentarios: mantener los comentarios actualizados, claros y
	fácilmente entendibles. Los comentarios en línea deberían ser útiles (sin
	obviedades) y los comentarios de documentación siempre deben aparecer en
	todas las funciones (no privadas) después del <<def>>.
	\item Convenciones de nombres: con nombres a evitar (como 'O') o cómo poner
	nombres correctos de paquetes, módulos, clases, funciones, constantes...
\end{itemize}

Algo que ha facilitado mucho la aplicación de esta guía de estilo es la
capacidad de la propia herramienta utilizada (Pycharm) de detectar
automáticamente la violación de algunas de estas recomendaciones.

\paragraph{Diseño adaptable} Una de las herramientas utilizadas (ver siguiente
sección) es Bootstrap, un conjunto de estilos ya predefinidos que pueden
utilizarse sin necesidad de tener que codificar CSS. La idea de la aplicación
Web desarrollada es que tenga la máxima accesibilidad. Esto se consigue
partiendo de un diseño que sea apto para los dispositivos móviles, lo que se
denomina estilo adaptable (<<responsive>>). Se trata de una técnica de diseño
web para adaptar la visualización de la página al dispositivo desde el que se
accede, haciendo que sea atractiva una mayor cantidad de usuarios. El diseño
<<responsive>> se consolida como una de las mejores prácticas en el diseño web
\cite{40defiebre}. En toda la aplicación (excepto las visualizaciones que
requieren de mucho más espacio), cuando el tamaño de la pantalla se reduce lo
suficiente, los contenidos se adaptan y se posicionan de tal forma que puedan
seguir utilizándose.

\paragraph{Internacionalización (i18n)}\footnote{Denominado así por las 18 letras
entre la <<i>> y la <<n>> en Internationalization}: Se trata de una
técnica/procedimiento por el que se desarrolla el software para poder ser
adaptado y localizado posteriormente a otras culturas y
lenguajes~\cite{lokalise}. Concretamente, este proyecto se está centrando en la
parte lingüística, con Inglés y Español como los idiomas principales. La idea de
esta técnica es conseguir esto, pero haciendo un código completamente neutral,
es decir, permitir que todo lo que se ha hecho para un idioma, pueda hacerse a
posteriori con la misma facilidad. En este entorno de Python y Flask existe la
librería (Babel) que permite capturar el texto (previamente anotado) y
sustituirlo en cada momento por las palabras adecuadas a la localización del
usuario (o bien con un cambio manual mediante botones). Por así decirlo, en vez
de codificar a un solo idioma, las palabras del idioma original se convierten en
claves, que luego son sustituidas cuando el usuario accede.


\section{Herramientas}

\paragraph{PyCharm}
Se trata de un \texttt{entorno de desarrollo integrado} ("<IDE">) desarrollado
por JetBrains. Está creado para la programación en lenguaje Python y aunque no
se ha usado en este proyecto también Java. Este "<IDE"> se ha convertido junto
con Visual Studio Code y Jupyter en el más utilizado por los desarrolladores.

Ofrece multitud de funcionalidades
(\url{https://www.jetbrains.com/es-es/pycharm/features/}):
\begin{itemize}
	\item Inspección de código.
	\item Indicación de errores (compilación).
	\item Refactorización de código automático (rápidas y seguras).
	\item Depuración.
	\item Pruebas.
	\item Herramientas para bases de datos.
	\item Integración con Git.
\end{itemize}

Estas son solo algunas de las muchas funcionalidades que permite y que han hecho
que se haya seleccionado para este proyecto. Además, dado que el proyecto tiene
una fuerte componente de desarrollo Web, PyCharm tiene una integración completa
con estos ámbitos, pudiendo desarrollar también de forma nativa con JavaScript o
lenguajes de marcas (HTML o CSS).

También se ha de destacar otra funcionalidad que ha sido de gran utilidad.
JetBrains tiene una herramienta específica para la conexión con bases de datos
\texttt{DataGrip}. Sin embargo Pycharm, de forma nativa, tiene ciertas de estas
utilidades. En concreto, en este proyecto se ha usado la visualización del
contenido de la base de datos junto con el diagrama de tablas/entidades (que
muestra relaciones, claves primarias, foráneas e incluso los tipos de datos).

Finalmente, el factor decisivo para usar PyCharm y concretamente gracias a la
Universidad de Burgos, es que el alumnado tiene acceso a la edición
"<Professional">, que habilita ese desarrollo Web, por ejemplo.

Esta aplicación ya había sido usada con anterioridad y por tanto su aprendizaje
básico no era necesario.

\paragraph{Visual Studio Code}
Se trata de un editor de código fuente desarrollado por Microsoft. Es el editor
por excelencia en todo el ámbito de la programación pues permite la programación
en casi cualquier lenguaje de programación, haciéndole muy versátil y un "<todo
en uno">.

Ofrece todas las funcionalidades que cabe esperar (incluyendo a las que ofrece
PyCharm): sintaxis, depuración, personalización, integración git...

Además del núcleo propio del editor y su constante actualización, uno de sus
puntos fuertes son las extensiones, que empresas o incluso la comunidad,
desarrollan y que permiten agilizar en lo posible las tareas del programador.

A pesar de todas las ventajas, el manejo de PyCharm ha resultado más sencillo de
utilizar (durante la experiencia previa a este proyecto) en el desarrollo del
software. Pero debido a su versatilidad y al desarrollo de esta documentación en
Latex, VS Code es el editor adecuado para ello. Gracias a las extensiones y a la
instalación local de \LaTeX, permite tener un control completo para la creación
de este tipo de documentos.

\paragraph{GitHub Desktop}
Es una aplicación que permite interactuar con GitHub utilizando una interfaz
gráfica respecto al manejo tradicional mediante la línea de comandos. Permite
realizar las operaciones más comunes y básicas de Git.

Pese a que no tiene toda la funcionalidad que sí ofrece la línea de comandos no
se previó un uso muy exhaustivo de Git y, por tanto, es suficiente
(<<Commit>>>,<<Push>>,<<Pull>>,<<Merge>>...).

Además, puede visualizarse cada cambio respecto a la última versión de un
vistazo y seleccionar dinámicamente los cambios que se deseen aplicar.

Aún con todo esto, no es más que una aplicación para agilizar el proceso de
control de versiones.

\paragraph{Diagrams.net (Draw.io)}
Es una herramienta online y de escritorio para la creación de diagramas. Con
ella se pueden crear diagramas de flujo, de UML o de red entre otras muchas
posibilidades. Para crear estos diagramas, posee un listado de elementos
(organizados) de cada tipo de diagrama (actores, cajas de clases, rombos de
relaciones, bloques...). Pero aunque esté pensado específicamente para
diagramas, existen muchos más elementos como flechas, figuras e iconos que
permiten no solo crear diagramas, sino todo tipo de cosas.

Con esta herramienta se han creado prácticamente todos los diagramas de la
documentación.

\paragraph{PuTTy}
La aplicación Web del proyecto se encuentra en funcionamiento en internet a
través de \url{https://vass.dmacha.dev}. Para gestionar el servidor en el que se
tiene lanzada se ha utilizado PuTTy.

Se trata de una aplicación de escritorio de código abierto y gratuita que
permite emular una terminal de comandos. Soporta múltiples protocolos como SSH
(\textit{Secure Shell}), que es el que se ha usado mediante la generación de
clave pública y privada. Una vez que se establece la conexión con el equipo
remoto, se convierte en la misma terminal que podría visualizarse en la pantalla
de este.

Además de esta funcionalidad, PuTTy alberga muchas otras utilidades como la de
generación de claves o incluso la transferencia de ficheros con SFTP
(\textit{Secure File Transfer Protocol}).

\paragraph{Librerías Python}
Para el proyecto se están utilizando múltiples librerías que facilitan en gran
medida el desarrollo del mismo. Implementan funcionalidades que pueden ser
utilizadas directamente.

\begin{itemize}
	\item \textbf{pandas}: Librería para el manejo de estructuras de datos que
	implementa muchas operaciones útiles (guardado/lectura CSV, reordenaciones,
	divisiones...) y de manera eficiente.
	\item \textbf{Babel}: Colección de utilidades para la internacionalización y
	localización de aplicaciones en Python.
	\item \textbf{Flask}: Es un micro Framework escrito en Python para el
	desarrollo de aplicaciones Web. Impone además su formato de plantillas
	(Jinja2).
	\item \textbf{flask-babel}: Es una extensión de Flask que permite la
	internacionalización concreta de aplicaciones Flask (con la ayuda de Babel).
	\item \textbf{Flask-SQLAlchemy}: Librería que actúa de extensión de Flask y
	añade el soporte para SQLAlchemy. No modifica el comportamiento de este
	último.
	\item \textbf{Flask-Login}: Librería complementaria a Flask para el manejo
	de sesiones de usuarios, automatizando los inicios y cierres de sesión entre
	otras cosas.
	\item \textbf{scikit-learn}: Librería más extendida de aprendizaje
	automático (<<machine learning>>) para Python. Junto con Flask es el núcleo
	del proyecto, ya que utiliza muchos de sus paquetes implementados.
	\item \textbf{numpy}: Es una librería especializada en el cálculo numérico y
	análisis de datos. Se caracteriza por su eficiente manejo de grandes
	volúmenes de datos gracias a su parcial implementación en lenguaje C (mucho
	más eficiente).
	\item \textbf{scipy}: Librería que incluye algoritmos matemáticos.
	\item \textbf{setuptools}: Librería que permite crear e instalar paquetes de
	software de Python.
	\item \textbf{pytest}: Es un Framework de pruebas unitarias para Python.
	\item \textbf{sslearn}: Librería para aprendizaje automático en conjuntos de
	datos Semi-Supervisados como extensión de scikit-learn. Creada por José Luis
	Garrido-Labrador.
	\item \textbf{Werkzeug}: Librería para aplicaciones WSGI (Web Server Gateway
	Inteface). Contiene una colección de utilidades para este tipo de
	aplicaciones.
	\item \textbf{matplotlib}: Librería para la creación de gráficos en 2D.
	\item \textbf{Flask-WTF}: Librería que actúa de extensión de Flask y añade
	el soporte para WTForms.
	\item \textbf{WTForms}: Librería para la creación de formularios seguros.
	Permite añadir múltiples validaciones en los campos de los formularios.
	Además, son útiles para crear <<plantillas>> base existen muchos formularios
	comunes.
	\item \textbf{SQLAlchemy}: Librería que incorpora una serie de herramientas
	SQL junto con el <<mapeo>> de los objetos relacionales. Ha resultado
	especialmente importante para desvincular la instancia de la base de datos
	(SQLite, MySQL...) de las consultas y operaciones.
\end{itemize}

\paragraph{Bibliotecas Web} Para el desarrollo Web conviene comentar la
utilización de dos bibliotecas que han resultado fundamentales.

\textbf{Bootstrap}: No es considerado una biblioteca al uso, sino que está
categorizado como un <<framework>> de CSS. Fue lanzado en 2011, con gran
cantidad de actualizaciones y de lanzamientos de nuevas versiones. La versión
más reciente y la utilizada en el proyecto es Bootstrap 5, que mejora el código
fuente de la versión 4 y añade propiedades nuevas. 

La intención del equipo de Bootstrap es promover el desarrollo web con novedades
de CSS y menores dependencias. Su principal objetivo es la creación de Webs con
diseño responsivo (adaptados a móviles) de la forma más sencilla posible.
Contiene infinidad de plantillas creadas con HTML, CSS y JavaScript para
componentes de la interfaz que el usuario simplemente puede hacer uso de ellas.
En HTML, Bootstrap se incorpora en las clases de los componentes y en su página
oficial se tiene toda la documentación\footnote{Bootstrap:
\url{https://getbootstrap.com/}}.

\textbf{Bootstrap Icons}: es una librería aparte de Bootstrap con iconos SVG.
Los iconos son tratados como clases que se incorporan a los elementos HTML.
Desde \url{https://icons.getbootstrap.com/} se pueden encontrar todos los iconos
que incluye la librería junto con el código a utilizar para copiar y pegar
directamente.

\textbf{D3.js}: su nombre se debe a Data-Drive-Documents, es una librería de
JavaScript que permite representar datos en distintas visualizaciones mediante
HTML, SVG y CSS. D3 trabaja sobre el Modelo de Objetos del Documento (<<Document
Object Model>>) para aplicar transformaciones dirigidas por datos. Todas las
visualizaciones presentadas en la Web se han realizado con D3. D3 permite un
amplio abanico de posibilidades y creatividad, pueden crearse visualizaciones
más bien estáticas y añadir estilos, anotaciones o leyendas entre otra cosas,
pero también permite la creación de gráficos altamente interactivos incluyendo
<<zoom>>, aprovechar los eventos en el DOM o transiciones completamente
personalizadas.

Para este caso se cree necesario comentar y justificar el uso de D3 ya que en el
ámbito de las visualizaciones, existe una gran cantidad de librerías disponibles
como JSAV (JavaScript Algorithm Visualization Library) o Algorithm-Visualizer
(contexto concreto de este proyecto) o incluso otras grandes librerías como
Anime.js, Chart.js o FusionCharts. Sin embargo, durante el periodo de estudio de
las distintas herramientas disponibles, D3 es superior a la demás en su
documentación. Tanto porque existen páginas dedicadas a la ejemplificación con
D3, como la gran cantidad de foros que se han creado solucionando dudas y
problemas. Sobre las últimas librerías mencionadas (y otras grandes existentes),
D3 está particularmente pensado para la completa personalización y aunque estas
también, no al nivel de D3.

La desventaja de D3 es que tiene una curva de aprendizaje bastante grande, pero
gracias a la documentación en internet, esto puede ser solventado en mayor
medida.

\textbf{DataTables}: Es un plug-in para la librería JQuery. Su objetivo es la
creación de tablas dinámicas e interactivas de la forma más sencilla posible.
Este plug-in se ha utilizado en todos los apartados de los usuarios, tanto el
espacio personal como el panel de administración. La elección de esta librería
es por recomendación de los tutores. Pese a que las tareas relativas a los
usuarios fue compleja y se tardaron dos Sprints en terminar todo lo relativo a
las tablas, DataTables tiene una muy buena documentación. Además, en el momento
de consultar una duda, en su propio foro otros usuarios ya habían tenido los
mismos problemas.

A la hora de crear las tablas, tiene multitud de opciones, tanto para la tabla
general, como para celdas, columnas o filas concretas. Como más importante,
existen opciones para <<renderizar>> código HTML en unas columnas (típica
columna de Acciones, por ejemplo).

También es de destacar que se ha incluido la extensión <<responsive>> de
DataTables para habilitar el estilo adaptativo de las tablas.