\capitulo{4}{Técnicas y herramientas}

En este apartado se presentarán las técnicas (procedimientos) y herramientas que
se han utilizado para el desarrollo del presente proyecto. En algunos de los
casos la experiencia u otros aspectos han hecho decantarse por una u otra o
simplemente se seleccionaron directamente.


\section{Herramientas}

\paragraph{PyCharm}
Se trata de un \texttt{entorno de desarrollo integrado} ("<IDE">) desarrollado por
JetBrains. Está creado para la programación en lenguaje Python y aunque no se ha
usado en este proyecto también Java. Este "<IDE"> se ha convertido junto con Visual
Studio Code y Jupyter en el más utilizado por los desarrolladores.

Ofrece multitud de funcionalidades (\url{https://www.jetbrains.com/es-es/pycharm/features/}):
\begin{itemize}
	\item Inspección de código.
	\item Indicación de errores (compilación).
	\item Refactorización de código automática (rápidas y seguras).
	\item Depuración.
	\item Pruebas.
	\item Herramientas para bases de datos.
	\item Integración con Git.
\end{itemize}

Estas son solo algunas de las muchas funcionalidades que permite y que han hecho
que se haya seleccionado para este proyecto. Además, dado que el proyecto tiene
una fuerte componente de desarrollo Web así como frameworks, PyCharm tiene una
integración completa con estos ámbitos, pudiendo desarrollar también de forma
nativa con JavaScript o lenguajes de marcas (HTML o CSS).

Otra ventaja del uso de PyCharm y concretamente gracias a la Universidad de
Burgos es que el alumnado tiene acceso a la edición "<Professional"> que
habilita ese desarrollo Web, por ejemplo.

Por último, esta aplicación ya había sido usada con anterioridad y por tanto su
aprendizaje básico no era necesario.

\paragraph{Visual Studio Code}
Se trata de un editor de código fuente desarrollado por Microsoft. Es el editor
por excelencia en todo el ámbito de la programación pues permite la programación
en casi cualquier lenguaje de programación haciéndole muy versátil y un "<todo
en uno">.

Ofrece todas las funcionalidades que cabe esperar (incluyendo a las que ofrece
PyCharm): sintaxis, depuración, personalización, integración git...

Además del núcleo propio del editor y su constante actualización, uno de sus
puntos fuertes son las extensiones que empresas o incluso la comunidad
desarrollan y que permiten agilizar en lo posible las tareas del programador.

A pesar de todas las ventajas, el manejo de PyCharm ha resultado más sencillo de
utilizar (durante la experiencia previa a este proyecto) en el desarrollo del
software. Pero debido a su versatilidad y al desarrollo de esta
documentación en Latex, VS Code es el editor adecuado para ello. Gracias a las
extensiones y a la instalación local de \LaTeX, permite tener un control completo
para la creación de este tipo de documentos. 


\paragraph{GitHub Desktop}
Es una aplicación que permite interactuar con GitHub utilizando una interfaz
gráfica respecto al manejo tradicional mediante la línea de comandos. Permite
realizar las operaciones más comunes y básicas de Git.

Pese a que no tiene toda la funcionalidad que sí ofrece la línea de comandos no
se previó un uso muy exhaustivo de Git y, por tanto, es suficiente
("<Commit">,"<Push">,"<Pull">,"<Merge">...).

Además, puede visualizarse cada cambio respecto a la última versión de un
vistazo y seleccionar dinámicamente los cambios que se deseen aplicar.

Aún con todo esto, no es más que una aplicación para agilizar el ya por lo
general rápido proceso de control de versiones.

\paragraph{Librerías Python}
Para el proyecto se están utilizando múltiples librerías que facilitan en gran
medida el desarrollo del mismo. Implementan funcionalidades que pueden ser
utilizadas directamente. En el proyecto, las librerías usadas están
especificadas en los requisitos (\textit{requirements}).

\begin{itemize}
	\item \textbf{pandas}: Librería para el manejo de estructuras de datos que implementa
	muchas operaciones útiles (guardado/lectura CSV, reordenaciones,
	divisiones...) y de manera eficiente.
	\item \textbf{Babel}: Colección de utilidades para la internacionalización y
	localización de aplicaciones en Python.
	\item \textbf{Flask}: Es un micro Framework escrito en Python para el desarrollo de
	aplicaciones Web. Impone además su formato de plantillas (Jinja2).
	\item \textbf{flask-babel}: Es una extensión de Flask que permite la
	internacionalización concreta de aplicacones Flask (con la ayuda de Babel).
	\item \textbf{scikit-learn}: Librería más extendida de aprendizaje automático
	(<<machine learning>>) para Python. Junto con Flask es el núcleo del
	proyecto ya que utiliza muchos de sus paquetes implementados.
	\item \textbf{numpy}: Es una librería especializada en el cálculo numérico y análisis
	de datos. Se caracteriza por su eficiente manejo de grandes volúmenes de
	datos gracias a su parcial implementación en lenguaje C (mucho más
	eficiente).
	\item \textbf{scipy}: Librería que incluye algoritmos matemáticos.
	\item \textbf{setuptools}: Librería que permite crear e instalar paquetes de
	software de Python.
	\item \textbf{pytest}: Es un Framework de pruebas unitarias para Python.
	\item \textbf{sslearn}: Librería para aprendizaje automático en conjuntos de datos
	Semi-Supervisados como extensión de scikit-learn.
\end{itemize}