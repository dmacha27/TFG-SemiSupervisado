\capitulo{4}{Técnicas y herramientas}

En este apartado se presentarán las técnicas (procedimientos) y herramientas que
se han utilizado para el desarrollo del presente proyecto. En algunos de los
casos la experiencia u otros aspectos han hecho decantarse por una u otra o
simplemente se seleccionaron directamente.

\section{Técnicas}

\textbf{Diseño adpatable}: Una de las herramientas utilizadas (ver siguiente sección) es Bootstrap, un
conjunto de estilos ya predefinidos que pueden utilizarse sin necesidad de tener
que codificar CSS. La idea de la aplicación Web desarrollada es que tenga la
máxima accesibilidad. Esto se consigue partiendo de un diseño que sea apto para
los dispositivos móviles, lo que se denomina estilo adaptable (<<responsive>>).
Se trata de una técnica de diseño web para adaptar la visualización de la página
al dispositivo desde el que se accede, haciendo que sea atractiva una mayor
cantidad de usuarios. El diseño <<responsive>> se consolida como una de las
mejores prácticas en el diseño web \cite{40defiebre}. En toda la aplicación,
cuando el tamaño de la pantalla se reduce lo suficiente, los contenidos se
adaptan y se posicionan de tal forma que puedan seguir utilizándose.

\textbf{Internacionalización (i18n)\footnote{Denominado así por las 18 letras
entre la <<i>> y la <<n>> en Internationalization}}: Se trata de una
técnica/procedimiento por el que se desarrolla el software para poder ser
adaptado y localizado posteriormente a otras culturas y
lenguajes~\cite{lokalise}. Concretamente, este proyecto se está centrando en la
parte lingüística, con Inglés y Español como los idiomas principales. La idea de
esta técnica es conseguir esto, pero haciendo un código completamente neutral,
es decir, permitir que todo lo que se ha hecho para un idioma, pueda hacerse a
posteriori con la misma facilidad. En este entorno de Python y Flask existe la
librería (Babel) que permite capturar el texto (previamente anotado) y
sustituirlo en cada momento por las palabras adecuadas a la localización del
usuario (o bien con un cambio manual mediante botones). Por así decirlo, en vez
de codificar a un solo idioma, las palabras del idioma original se convierten en
claves, que luego son sustituidas cuando el usuario accede.


\section{Herramientas}

\paragraph{PyCharm}
Se trata de un \texttt{entorno de desarrollo integrado} ("<IDE">) desarrollado por
JetBrains. Está creado para la programación en lenguaje Python y aunque no se ha
usado en este proyecto también Java. Este "<IDE"> se ha convertido junto con Visual
Studio Code y Jupyter en el más utilizado por los desarrolladores.

Ofrece multitud de funcionalidades (\url{https://www.jetbrains.com/es-es/pycharm/features/}):
\begin{itemize}
	\item Inspección de código.
	\item Indicación de errores (compilación).
	\item Refactorización de código automático (rápidas y seguras).
	\item Depuración.
	\item Pruebas.
	\item Herramientas para bases de datos.
	\item Integración con Git.
\end{itemize}

Estas son solo algunas de las muchas funcionalidades que permite y que han hecho
que se haya seleccionado para este proyecto. Además, dado que el proyecto tiene
una fuerte componente de desarrollo Web así como frameworks, PyCharm tiene una
integración completa con estos ámbitos, pudiendo desarrollar también de forma
nativa con JavaScript o lenguajes de marcas (HTML o CSS).

Otra ventaja del uso de PyCharm y concretamente gracias a la Universidad de
Burgos es que el alumnado tiene acceso a la edición "<Professional"> que
habilita ese desarrollo Web, por ejemplo.

Por último, esta aplicación ya había sido usada con anterioridad y por tanto su
aprendizaje básico no era necesario.

\paragraph{Visual Studio Code}
Se trata de un editor de código fuente desarrollado por Microsoft. Es el editor
por excelencia en todo el ámbito de la programación pues permite la programación
en casi cualquier lenguaje de programación haciéndole muy versátil y un "<todo
en uno">.

Ofrece todas las funcionalidades que cabe esperar (incluyendo a las que ofrece
PyCharm): sintaxis, depuración, personalización, integración git...

Además del núcleo propio del editor y su constante actualización, uno de sus
puntos fuertes son las extensiones que empresas o incluso la comunidad
desarrollan y que permiten agilizar en lo posible las tareas del programador.

A pesar de todas las ventajas, el manejo de PyCharm ha resultado más sencillo de
utilizar (durante la experiencia previa a este proyecto) en el desarrollo del
software. Pero debido a su versatilidad y al desarrollo de esta
documentación en Latex, VS Code es el editor adecuado para ello. Gracias a las
extensiones y a la instalación local de \LaTeX, permite tener un control completo
para la creación de este tipo de documentos. 


\paragraph{GitHub Desktop}
Es una aplicación que permite interactuar con GitHub utilizando una interfaz
gráfica respecto al manejo tradicional mediante la línea de comandos. Permite
realizar las operaciones más comunes y básicas de Git.

Pese a que no tiene toda la funcionalidad que sí ofrece la línea de comandos no
se previó un uso muy exhaustivo de Git y, por tanto, es suficiente
("<Commit">,"<Push">,"<Pull">,"<Merge">...).

Además, puede visualizarse cada cambio respecto a la última versión de un
vistazo y seleccionar dinámicamente los cambios que se deseen aplicar.

Aún con todo esto, no es más que una aplicación para agilizar el ya por lo
general rápido proceso de control de versiones.

\paragraph{Librerías Python}
Para el proyecto se están utilizando múltiples librerías que facilitan en gran
medida el desarrollo del mismo. Implementan funcionalidades que pueden ser
utilizadas directamente. En el proyecto, las librerías usadas están
especificadas en los requisitos (\textit{requirements}).

\begin{itemize}
	\item \textbf{pandas}: Librería para el manejo de estructuras de datos que
	implementa muchas operaciones útiles (guardado/lectura CSV, reordenaciones,
	divisiones...) y de manera eficiente.
	\item \textbf{Babel}: Colección de utilidades para la internacionalización y
	localización de aplicaciones en Python.
	\item \textbf{Flask}: Es un micro Framework escrito en Python para el
	desarrollo de aplicaciones Web. Impone además su formato de plantillas
	(Jinja2).
	\item \textbf{flask-babel}: Es una extensión de Flask que permite la
	internacionalización concreta de aplicacones Flask (con la ayuda de Babel).
	\item \textbf{Flask-SQLAlchemy}: Librería que actúa de extensión de Flask y
	añade el soporte para SQLAlchemy. No modifica el comportamiento de este
	último.
	\item \textbf{Flask-Login}: Librería complementaria a Flask para el manejo
	de sesiones de usuarios, automatizando los inicios y cierres de sesión entre
	otras cosas.
	\item \textbf{scikit-learn}: Librería más extendida de aprendizaje
	automático (<<machine learning>>) para Python. Junto con Flask es el núcleo
	del proyecto, ya que utiliza muchos de sus paquetes implementados.
	\item \textbf{numpy}: Es una librería especializada en el cálculo numérico y
	análisis de datos. Se caracteriza por su eficiente manejo de grandes
	volúmenes de datos gracias a su parcial implementación en lenguaje C (mucho
	más eficiente).
	\item \textbf{scipy}: Librería que incluye algoritmos matemáticos.
	\item \textbf{setuptools}: Librería que permite crear e instalar paquetes de
	software de Python.
	\item \textbf{pytest}: Es un Framework de pruebas unitarias para Python.
	\item \textbf{sslearn}: Librería para aprendizaje automático en conjuntos de
	datos Semi-Supervisados como extensión de scikit-learn. Creada por José Luis
	Garrido-Labrador.
	\item \textbf{Werkzeug}: Librería para aplicaciones WSGI (Web Server Gateway
	Inteface). Contiene un colección de utilidades para este tipo de
	aplicaciones.
	\item \textbf{matplotlib}: Librería para la creación de gráficos en 2D.
	\item \textbf{Flask-WTF}: Librería que actúa de extensión de Flask y añade
	el soporte para WTForms.
	\item \textbf{WTForms}: Librería para la creación de formularios seguros.
	Permite añadir múltiples validaciones en los campos de los formularios.
	Además, son útiles para crear <<plantillas>> base existen muchos formularios
	comunes.
	\item \textbf{SQLAlchemy}: Librería que incorpora una serie de herramientas
	SQL junto con el <<mapeo>> de los objetos relacionales. Ha resultado
	especialmente importante para desvincular la instancia de la base de datos
	(SQLite, MySQL...) de las consultas y operaciones.
\end{itemize}

\paragraph{Bibliotecas Web} Para el desarrollo Web conviene comentar la
utilización de dos bibliotecas que han resultado fundamentales.

\textbf{Bootstrap}: No es considerado una biblioteca al uso, sino que está
categorizado como un <<framework>> de CSS. Fue lanzado en 2011 con una gran
cantidad de actualizaciones y de lanzamientos de nuevas versiones. La versión
más reciente y la utilizada en el proyecto, es Bootstrap 5, que mejora el código
fuente de la versión 4 y añade propiedades nuevas. La intención del equipo de
Bootstrap es promover el desarrollo web con novedades de CSS y menores
dependencias. Su principal objetivo es la creación de Webs con diseño responsivo
(adaptados a móviles) de la forma más sencilla posible. Contiene infinidad de
plantillas creadas con HTML, CSS y JavaScript para componentes de la interfaz
que el usuario simplemente puede hacer uso de ellas. En HTML, Bootstrap se
incorpora en las clases de los componentes y en su página oficial se tiene toda
la documentación\footnote{Bootstrap: \url{https://getbootstrap.com/}}.

\textbf{D3.js}: su nombre se debe a Data-Drive-Documents, es una librería de
JavaScript que permite representar datos en distintas visualizaciones mediante
HTML, SVG y CSS. D3 trabaja sobre el Modelo de Objetos del Documento (<<Document
Object Model>>) para aplicar transformaciones dirigidas por datos. Todas las
visualizaciones presentadas en la Web se han realizado con D3. D3 permite un
amplio abanico de posibilidades y creatividad, pueden crearse visualizaciones
más bien estáticas y añadir estilos, anotaciones o leyendas entre otra cosas,
pero también permite la creación de gráficos altamente interactivos incluyendo
<<zoom>>, aprovechar los eventos en el DOM o transiciones completamente
personalizadas.

Para este caso se cree necesario comentar y justificar el uso de D3 ya que en el
ámbito de las visualizaciones, existe una gran cantidad de librerías disponibles
como JSAV (JavaScript Algorithm Visualization Library) o Algorithm-Visualizer
(contexto concreto de este proyecto) o incluso otras grandes librerías como
Anime.js, Chart.js o FusionCharts. Sin embargo, durante el periodo de estudio de
las distintas herramientas disponibles, D3 es superior a la demás en su
documentación. Tanto porque existen páginas dedicadas a la ejemplificación con
D3, como la gran cantidad de foros que se han creado solucionando dudas y
problemas. Sobre las últimas librerías mencionadas (y otras grandes existentes),
D3 está particularmente pensado para la completa personalización y aunque estas
también, no al nivel de D3.

La desventaja de D3 es que tiene una curva de aprendizaje bastante grande, pero
gracias a la documentación en internet, esto puede ser solventado en mayor
medida.

\textbf{DataTables}: Es un plug-in para la librería JQuery. Su objetivo es la
creación de tablas dinámicas e interactivas de la forma más sencilla posible.
Este plug-in se ha utilizado en todos los apartados de los usuarios, tanto el
espacio personal como el panel de administración. La elección de esta librería
es por recomendación de los tutores. Pese a que las tareas relativas a los
usuarios fue compleja y se tardaron dos Sprints en terminar todo lo relativo a
las tablas, DataTables tiene una muy buena documentación. Además, en el momento
de consultar una duda, en su propio foro otros usuarios ya habían tenido los
mismos problemas. A la hora de crear las tablas, tiene multitud de opciones,
tanto para la tabla general, como para celdas, columnas o filas concretas. Como
más importante, existen opciones para <<renderizar>> código HTML en unas
columnas (típica columna de Acciones, por ejemplo).
