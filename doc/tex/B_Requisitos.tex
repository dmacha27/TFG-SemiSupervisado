\apendice{Especificación de Requisitos}

\section{Introducción}

En esta sección se detalla la Especificación de requisitos. Aquí quedará
registrado qué es lo que la aplicación debe hacer (y cómo).

\section{Objetivos generales}
Los objetivos del proyecto pueden resumirse en los siguientes:
\begin{enumerate}
    \item Implementación de Self-Training, Co-Training, Democratic Co-Learning y
    Tri-Training.
    \item Creación de una aplicación Web e integración de los cuatro algoritmos
    en la aplicación Web para su visualización.
    \item Sistema de usuarios que les permita controlar sus ficheros y
    ejecuciones.
    \item Administración de los usuarios y todos los ficheros y ejecuciones
    (mediante un rol administrador).
\end{enumerate}

El núcleo del proyecto, pese a la importancia de las implementaciones los
algoritmos, es la aplicación Web, pues representa la parte interactiva del
proyecto. Por este motivo, el presente apartado analizará con precisión que es
lo que la aplicación debe o no hacer (con sus limitaciones).

\section{Usuarios de la aplicación}

Como se ha comentado, existirá un sistema de usuarios en el que podrán
diferenciarse: Usuario, Administrador y Anónimo (este último así lo denomina el
sistema de gestión de cuentas utilizado, \texttt{Flask-Login}). 

\begin{itemize}
	\item Anónimo: Usuario que no posee una cuenta registrada en el sistema, es
	el usuario base. Puede utilizar toda la aplicación básica: seleccionar
	algoritmo, subir un conjunto de datos, configurar el algoritmo y
	visualizarlo.
	\item Usuario: Usuario que posee una cuenta registrada sin privilegios. Al
	igual que el anónimo base, puede utilizar la aplicación, pero además, los
	ficheros (conjuntos de datos) y ejecuciones son almacenados en el sistema.
	\item Administrador: Usuario que posee una cuenta registrada con
	privilegios. Es exactamente lo mismo que un Usuario, con la particularidad
	de que podrá, además, gestionar estos Usuarios, todos los ficheros y todas
	las ejecuciones (de todos los usuarios). 
\end{itemize}

\section{Catálogo de requisitos}


A continuación se detallan los requisitos funcionales así como los no
funcionales.

\subsection{Requisitos funcionales}

\begin{itemize}
	\item \textbf{RF-1 Visualización de algoritmos semi-supervisados}: la
	aplicación debe permitir visualizar el proceso de entrenamiento de los
	algoritmos implementados (con sus estadísticas pertinentes).
    \begin{itemize}
        \item \textbf{RF-1.1 Selección de algoritmo}: la aplicación debe
        permitir seleccionar uno de los algoritmos implementados.
        \item \textbf{RF-1.2 Carga de conjunto de datos}: la aplicación debe
        permitir subir un fichero \texttt{ARFF} o \texttt{ACSV} con el conjunto
        de datos. 
        \item \textbf{RF-1.3 Configuración del algoritmo}: la aplicación debe
        permitir configurar la ejecución del algoritmo con los parámetros
        específicos del mismo.
        \item \textbf{RF-1.4 Control visualizaciones}: la aplicación debe
        mostrar visualizaciones interactivas: principal (gráfico de dos
        dimensiones con los puntos del conjunto de datos) y estadísticas
        (gráficos de líneas).
        \begin{itemize}
            \item \textbf{RF-1.3.1 Controlar paso (iteración)}: la aplicación
            debe permitir controlar el paso del proceso de entrenamiento
            (iteración anterior/siguiente).
            \item \textbf{RF-1.3.1 Interacción con el gráfico principal}: el
            gráfico principal debe mostrar información relevante en un
            \texttt{tooltip} cuando el usuario interactúe con cada punto.
        \end{itemize}
    \end{itemize}
    
    \item \textbf{RF-2 Manejo de usuarios}: la aplicación debe manejar cuentas
    de usuario.
    \begin{itemize}
        \item \textbf{RF-2.1 Registro}: la aplicación debe permitir crear
        cuentas de usuario (no administrador).
        \item \textbf{RF-2.2 Inicio de sesión}: la aplicación debe permitir el
        inicio de sesión a aquellos usuarios con cuenta.
    \end{itemize}
    \item \textbf{RF-3 Personalización del perfil}: los usuarios con cuenta (no
    anónimos) deben poder modificar sus datos personales de su perfil de
    usuario.

    \item \textbf{RF-4 Espacio personal}: los usuarios con cuenta (no anónimos)
    deben poseer de un espacio personal.
    \begin{itemize}
        \item \textbf{RF-4.1 Control de conjunto de datos}: los usuarios con
        cuenta deben poder consultar sus ficheros subidos.
        \begin{itemize}
            \item \textbf{RF-4.1.1 Ejecución de un nuevo algoritmo}: los
            usuarios con cuenta deben poder utilizar esos ficheros en un
            algoritmo (nueva ejecución).
            
            \item \textbf{RF-4.1.2 Eliminación}: los usuarios con cuenta deben
            poder eliminar esos ficheros del sistema.
        \end{itemize}
        \item \textbf{RF-4.2 Control de ejecuciones}:  los usuarios con cuenta
        deben poder consultar sus ejecuciones anteriores (con toda su
        información).
        \begin{itemize}
            \item \textbf{RF-4.2.1 Replicar ejecución}: los usuarios con cuenta
            deben poder replicar las ejecuciones.
            
            \item \textbf{RF-4.2.2 Eliminación}: los usuarios con cuenta deben
            poder eliminar las ejecuciones.
        \end{itemize}
    \end{itemize}

    \item \textbf{RF-5 Administración}: la aplicación debe manejar cuentas de
    usuario de tipo administrador.
    \begin{itemize}
        \item \textbf{RF-5.1 Control de usuarios}: el administrador deben poder
        consultar los usuarios registrados.
        \begin{itemize}
            \item \textbf{RF-5.1.1 Edición de usuario}: el administrador debe
            poder modificar los datos de los usuarios.
            \item \textbf{RF-5.1.2 Eliminación}: el administrador debe poder
            eliminar usuarios del sistema.
            \item \textbf{RF-5.1.3 Total}: el administrador debe poder ver el
            número total de usuarios.
        \end{itemize}
        \item \textbf{RF-5.2 Control de conjuntos de datos global}: el
        administrador debe poder consultar todos los ficheros subidos.
        \begin{itemize}          
            \item \textbf{RF-5.2.1 Eliminación}: el administrador debe poder
            eliminar cualquier fichero de los usuarios registrados.
            \item \textbf{RF-5.2.2 Últimos ficheros}: el administrador debe
            poder ver el \textbf{número} de ficheros subidos de los últimos 7
            días.
        \end{itemize}
        \item \textbf{RF-5.3 Control de ejecuciones global}: el administrador
        debe poder consultar todas ejecuciones (con toda su información).
        \begin{itemize}          
            \item \textbf{RF-5.3.1 Eliminación}: el administrador deben poder
            eliminar cualquier ejecución de los usuarios registrados.
            \item \textbf{RF-5.3.2 Últimas ejecuciones}: el administrador debe
            poder ver el \textbf{número} de ficheros subidos de los últimos 7
            días.
        \end{itemize}
    \end{itemize}

    \item \textbf{RF-6 Cambio de idioma}: todos los usuarios deben poder cambiar
    el idioma a placer (inglés o español).

\end{itemize}

\subsection{Requisitos no funcionales}
Los requisitos anteriores definen de alguna manera lo que el sistema debe hacer.
En este caso, los no funcionales son restricciones, por así decirlo, de calidad.
Responden a la pregunta de cómo funciona y no qué es lo que hace. Todas estas
restricciones son aplicadas de forma intrínseca en el desarrollo.

\begin{itemize}
	\item \textbf{RNF-1 Disponibilidad}: el sistema de funcionar con muy alta
	probabilidad ante una petición. Es decir, debe encontrarse en condiciones de
	funcionamiento.
	\item \textbf{RNF-2 Accesibilidad}: el sistema debe poder abarcar el mayor
	público posible, facilitando su acceso y su manejo independientemente de las
	capacidades personales.
    \item \textbf{RNF-3 Soporte}: el sistema debe poder utilizarse en el mayor
    número posible de navegadores (dada su naturaleza Web).
	\item \textbf{RNF-4 Mantenibilidad}: el sistema debe ser fácil de modificar,
	mejorar o adaptar cuando se presenten nuevas necesidades.
	\item \textbf{RNF-5 Seguridad}: el sistema debe asegurar la información
	sensible (mediante cifrado y controles de accesos) y debe ser accesible
	mediante protocolos segurizados (SSL).
	\item \textbf{RNF-6 Privacidad\footnote{La privacidad puede ser definida
	como el ámbito de la vida personal de un individuo, quien se desarrolla en
	un espacio reservado, el cual tiene como propósito principal mantenerse
	confidencial \cite{eswiki:148719517}}}: el sistema debe respetar la
	información privada y, de ninguna forma, compartirla con terceros. Debe ser
	un espacio reservado confidencial entre el usuario y la aplicación.
	\item \textbf{RNF-7 Escalabilidad}: el sistema debe ser capaz de crecer para
	ajustarse a la carga de trabajo.
	\item \textbf{RNF-8 Extensibilidad}: debe ser fácil añadir nueva
	funcionalidad al sistema, concretamente, la adición de nuevos algoritmos
	debe implicar pocas modificaciones.
	\item \textbf{RNF-9 Robustez}: el sistema debe ser altamente capaz de
	manejar los errores durante la ejecución (entradas erróneas, bugs...),
	mostrando información precisa al usuario y recuperándose de ellos a una
	situación estable.
	\item \textbf{RNF-10 Internacionalización}: el sistema debe soportar múltiples idiomas.

\end{itemize}

\section{Especificación de requisitos}

\imagencontamano{DiagramaCasosDeUso}{fd}{1}

% Caso de Uso 1 -> Visualización de un algoritmo.
\begin{table}[p]
	\centering
	\begin{tabularx}{\linewidth}{ p{0.21\columnwidth} p{0.71\columnwidth} }
		\toprule
		\textbf{CU-1}    & \textbf{Visualización de un algoritmo}\\
		\toprule
		\textbf{Versión}              & 1.0    \\
		\textbf{Autor}                & David Martínez Acha \\
		\textbf{Requisitos asociados} & RF-1, RF-1.1, RF-1.2, RF-1.3, RF-1.4 \\
		\textbf{Descripción}          & Visualización del proceso de entrenamiento de un algoritmo semi-supervisado. Con gráfico principal y estadísticos. \\
		\textbf{Precondición}         & No hay precondiciones \\
		\textbf{Acciones}             &
		\begin{enumerate}
			\def\labelenumi{\arabic{enumi}.}
			\tightlist
			\item Ejecución del Caso de Uso 2.
			\item Ejecución del Caso de Uso 3.
            \item Ejecución del Caso de Uso 4.
            \item El usuario verá en la página los distintos gráficos de su visualización. 
            \item [Opcional] Ejecución del Caso de Uso 5.
		\end{enumerate}\\
		\textbf{Postcondición}        & Visualizaciones renderizadas en la Web \\
		\textbf{Excepciones}          &  \\
		\textbf{Importancia}          & Alta\\
		\bottomrule
	\end{tabularx}
	\caption{CU-1 Visualización de un algoritmo.}
\end{table}

% Caso de Uso 2 -> Selección de algoritmo.
\begin{table}[p]
	\centering
	\begin{tabularx}{\linewidth}{ p{0.21\columnwidth} p{0.71\columnwidth} }
		\toprule
		\textbf{CU-1}    & \textbf{Selección de algoritmo}\\
		\toprule
		\textbf{Versión}              & 1.0    \\
		\textbf{Autor}                & David Martínez Acha \\
		\textbf{Requisitos asociados} & RF-1, RF-1.1 \\
		\textbf{Descripción}          & Seleccionar el algoritmo a ejecutar (establecerlo en la sesión del usuario) \\
		\textbf{Precondición}         & Ejecutando el Caso de Uso 1 \\
		\textbf{Acciones}             &
		\begin{enumerate}
			\def\labelenumi{\arabic{enumi}.}
			\tightlist
			\item El usuario selecciona un algoritmo haciendo clic en el nombre del algoritmo en la barra de navegación.
		\end{enumerate}\\
        \textbf{Acciones alternativas}&
		\begin{enumerate}
			\def\labelenumi{\arabic{enumi}.}
			\tightlist
			\item En caso de encontrarse en la página principal, el usuario
		puede selecciona un algoritmo haciendo clic en una de las tarjetas de
		presentación algoritmos. \end{enumerate}\\
		\textbf{Postcondición}        & Algoritmo almacenado en su sesión y redirección a la página de subida \\
		\textbf{Excepciones}          & Sin excepciones \\
		\textbf{Importancia}          & Alta \\
		\bottomrule
	\end{tabularx}
	\caption{CU-3 Selección de algoritmo.}
\end{table}

% Caso de Uso X -> Cambiar de idioma.
\begin{table}[p]
	\centering
	\begin{tabularx}{\linewidth}{ p{0.21\columnwidth} p{0.71\columnwidth} }
		\toprule
		\textbf{CU-1}    & \textbf{Cambiar idioma}\\
		\toprule
		\textbf{Versión}              & 1.0    \\
		\textbf{Autor}                & David Martínez Acha \\
		\textbf{Requisitos asociados} & RF-6 \\
		\textbf{Descripción}          & Cambiar el idioma de la página (entre inglés y español) \\
		\textbf{Precondición}         & No hay precondiciones \\
		\textbf{Acciones}             &
		\begin{enumerate}
			\def\labelenumi{\arabic{enumi}.}
			\tightlist
			\item El usuario hace clic en el símbolo de traducción en la barra de navegación.
            \item Aparece un desplegable de idiomas.
            \item El usuario hace clic en el idioma deseado.
		\end{enumerate}\\
		\textbf{Postcondición}        & La página se ha cambiado al idioma seleccionado \\
		\textbf{Excepciones}          & Sin excepciones \\
		\textbf{Importancia}          & Baja \\
		\bottomrule
	\end{tabularx}
	\caption{CU-1 Cambiar idioma.}
\end{table}
