\apendice{Especificación de Requisitos}

\section{Introducción}

En esta sección se detalla la especificación de requisitos, en ella quedará
registrado qué es lo que la aplicación debe hacer (y cómo).

\section{Objetivos generales}
Los objetivos del proyecto pueden resumirse en los siguientes:
\begin{enumerate}
    \item Implementación de \textit{Self-Training}, \textit{Co-Training},
    \textit{Democratic Co-Learning} y \textit{Tri-Training}.
    \item Creación de una aplicación web e integración de los cuatro algoritmos
    en la aplicación web para su visualización.
    \item Sistema de usuarios que les permita controlar sus ficheros y
    ejecuciones.
    \item Administración de los usuarios y todos los ficheros y ejecuciones
    (mediante un rol administrador).
\end{enumerate}

El núcleo del proyecto, pese a la importancia de las implementaciones los
algoritmos, es la aplicación web, pues representa la parte interactiva del
proyecto. Por este motivo, el presente apartado analizará con precisión qué es
lo que la aplicación debe hacer (con sus limitaciones).

\section{Usuarios de la aplicación}

Como se ha comentado, existirá un sistema de usuarios en el que podrán
diferenciarse: Usuario, Administrador y Anónimo (este último así lo denomina el
sistema de gestión de cuentas utilizado, \textit{Flask-Login}). 

\begin{itemize}
	\item \textbf{Anónimo}: Usuario que no posee una cuenta registrada en el sistema, es
	el usuario base. Puede utilizar toda la aplicación básica: seleccionar
	algoritmo, subir un conjunto de datos, configurar el algoritmo y
	visualizarlo.
	\item \textbf{Usuario}: Usuario que posee una cuenta registrada sin privilegios. Al
	igual que el anónimo base, puede utilizar la aplicación, pero además, los
	ficheros (conjuntos de datos) y ejecuciones son almacenados en el sistema.
	\item \textbf{Administrador}: Usuario que posee una cuenta registrada con
	privilegios. Es exactamente lo mismo que un usuario, con la particularidad
	de que podrá, además, gestionar estos usuarios, todos los ficheros y todas
	las ejecuciones (de todos los usuarios). 
\end{itemize}

\section{Catálogo de requisitos}


A continuación se detallan los requisitos funcionales así como los no
funcionales.

\subsection{Requisitos funcionales}

\begin{itemize}
	\item \textbf{RF-1 Visualización de algoritmos semi-supervisados}: la
	aplicación debe permitir visualizar el proceso de entrenamiento de los
	algoritmos implementados (con sus estadísticas pertinentes).
    \begin{itemize}
        \item \textbf{RF-1.1 Selección de algoritmo}: la aplicación debe
        permitir seleccionar uno de los algoritmos implementados.
        \item \textbf{RF-1.2 Carga de conjunto de datos}: la aplicación debe
        permitir subir un fichero \texttt{ARFF} o \texttt{ACSV} con el conjunto
        de datos. 
        \item \textbf{RF-1.3 Configuración del algoritmo}: la aplicación debe
        permitir configurar la ejecución del algoritmo con los parámetros
        específicos del mismo.
        \item \textbf{RF-1.4 Control visualizaciones}: la aplicación debe
        mostrar visualizaciones interactivas: una principal (gráfico de dos
        dimensiones con los puntos del conjunto de datos) y una o varias
        estadísticas (gráficos de líneas).
        \begin{itemize}
            \item \textbf{RF-1.3.1 Controlar paso (iteración)}: la aplicación
            debe permitir controlar el paso del proceso de entrenamiento
            (iteración anterior/siguiente).
            \item \textbf{RF-1.3.1 Interacción con el gráfico principal}: el
            gráfico principal debe mostrar información relevante en un
            \textit{tooltip} cuando el usuario interactúe con cada punto.
        \end{itemize}
    \end{itemize}
    
    \item \textbf{RF-2 Manejo de usuarios}: la aplicación debe manejar cuentas
    de usuario.
    \begin{itemize}
        \item \textbf{RF-2.1 Registro}: la aplicación debe permitir crear
        cuentas de usuario (no administrador).
        \item \textbf{RF-2.2 Inicio de sesión}: la aplicación debe permitir el
        inicio de sesión a aquellos usuarios con cuenta.
    \end{itemize}
    \item \textbf{RF-3 Personalización del perfil}: los usuarios con cuenta (no
    anónimos) deben poder modificar sus datos personales de su perfil de
    usuario.

    \item \textbf{RF-4 Espacio personal}: los usuarios con cuenta (no anónimos)
    deben poseer de un espacio personal.
    \begin{itemize}
        \item \textbf{RF-4.1 Control de conjunto de datos}: los usuarios con
        cuenta deben poder consultar sus ficheros subidos.
        \begin{itemize}
            \item \textbf{RF-4.1.1 Ejecución de un nuevo algoritmo}: los
            usuarios con cuenta deben poder utilizar esos ficheros en un
            algoritmo (nueva ejecución).
            
            \item \textbf{RF-4.1.2 Eliminación}: los usuarios con cuenta deben
            poder eliminar esos ficheros del sistema.
        \end{itemize}
        \item \textbf{RF-4.2 Control de ejecuciones}:  los usuarios con cuenta
        deben poder consultar sus ejecuciones anteriores (con toda su
        información).
        \begin{itemize}
            \item \textbf{RF-4.2.1 Replicar ejecución}: los usuarios con cuenta
            deben poder replicar las ejecuciones.
            \item \textbf{RF-4.2.2 Eliminación}: los usuarios con cuenta deben
            poder eliminar las ejecuciones.
        \end{itemize}
    \end{itemize}

    \item \textbf{RF-5 Administración}: la aplicación debe manejar cuentas de
    usuario de tipo administrador.
    \begin{itemize}
        \item \textbf{RF-5.1 Control de usuarios}: el administrador debe poder
        consultar los usuarios registrados.
        \begin{itemize}
            \item \textbf{RF-5.1.1 Edición de usuario}: el administrador debe
            poder modificar los datos de los usuarios.
            \item \textbf{RF-5.1.2 Eliminación}: el administrador debe poder
            eliminar usuarios del sistema.
            \item \textbf{RF-5.1.3 Total}: el administrador debe poder ver el
            número total de usuarios.
        \end{itemize}
        \item \textbf{RF-5.2 Control de conjuntos de datos global}: el
        administrador debe poder consultar todos los ficheros subidos.
        \begin{itemize}          
            \item \textbf{RF-5.2.1 Eliminación}: el administrador debe poder
            eliminar cualquier fichero de los usuarios registrados.
            \item \textbf{RF-5.2.2 Últimos ficheros}: el administrador debe
            poder ver el \textbf{número} de ficheros subidos de los últimos 7
            días.
        \end{itemize}
        \item \textbf{RF-5.3 Control de ejecuciones global}: el administrador
        debe poder consultar todas ejecuciones (con toda su información).
        \begin{itemize}          
            \item \textbf{RF-5.3.1 Eliminación}: el administrador debe poder
            eliminar cualquier ejecución de los usuarios registrados.
            \item \textbf{RF-5.3.2 Últimas ejecuciones}: el administrador debe
            poder ver el \textbf{número} de ficheros subidos de los últimos 7
            días.
        \end{itemize}
    \end{itemize}

    \item \textbf{RF-6 Cambio de idioma}: todos los usuarios deben poder cambiar
    el idioma (inglés o español).

	\item \textbf{RF-7 Acceso a la ayuda}: los usuarios deben tener a su
	alcance un manual de ayuda.

\end{itemize}

\subsection{Requisitos no funcionales}
Los requisitos anteriores definen de alguna manera lo que el sistema debe hacer.
En este caso, los no funcionales son restricciones, por así decirlo, de calidad.
Responden a la pregunta de cómo funciona y no qué es lo que hace. Todas estas
restricciones son aplicadas de forma intrínseca en el desarrollo.

\begin{itemize}
	\item \textbf{RNF-1 Disponibilidad}: el sistema de funcionar con muy alta
	probabilidad ante una petición. Es decir, debe encontrarse en condiciones de
	funcionamiento.
	\item \textbf{RNF-2 Accesibilidad}: el sistema debe poder abarcar el mayor
	público posible, facilitando su acceso y su manejo independientemente de las
	capacidades personales.
    \item \textbf{RNF-3 Soporte}: el sistema debe poder utilizarse en el mayor
    número posible de navegadores (dada su naturaleza Web).
	\item \textbf{RNF-4 Mantenibilidad}: el sistema debe ser fácil de modificar,
	mejorar o adaptar cuando se presenten nuevas necesidades.
	\item \textbf{RNF-5 Seguridad}: el sistema debe asegurar la información
	sensible (mediante cifrado y controles de accesos) y debe ser accesible
	mediante protocolos segurizados (SSL).
	\item \textbf{RNF-6 Privacidad\footnote{La privacidad puede ser definida
	como el ámbito de la vida personal de un individuo, quien se desarrolla en
	un espacio reservado, el cual tiene como propósito principal mantenerse
	confidencial~\cite{eswiki:148719517}}}: el sistema debe respetar la
	información privada y, de ninguna forma, compartirla con terceros. Debe ser
	un espacio reservado confidencial con el usuario.
	\item \textbf{RNF-7 Escalabilidad}: el sistema debe ser capaz de crecer para
	ajustarse a la carga de trabajo.
	\item \textbf{RNF-8 Extensibilidad}: debe ser fácil añadir nueva
	funcionalidad al sistema, concretamente, la adición de nuevos algoritmos
	debe implicar pocas modificaciones.
	\item \textbf{RNF-9 Robustez}: el sistema debe ser altamente capaz de
	manejar los errores durante la ejecución (entradas erróneas,
	\textit{bugs}...), mostrando información precisa al usuario y recuperándose
	de ellos a una situación estable.
	\item \textbf{RNF-10 Internacionalización}: el sistema debe estar preparado
	para soportar múltiples idiomas.

\end{itemize}

\section{Especificación de requisitos}

\imagenflotante{anexos/DiagramaCasosDeUso}{Diagrama de casos de uso}

En la figura~\ref{fig:anexos/DiagramaCasosDeUso} se detalla el diagrama de casos
de uso general. Y a continuación, se detalla la descripción de cada uno de los
casos de uso representados en el diagrama anterior.

% Caso de Uso 1 -> Visualizar un algoritmo.
\begin{table}[p]
	\centering
	\begin{tabularx}{\linewidth}{ p{0.21\columnwidth} p{0.71\columnwidth} }
		\toprule
		\textbf{CU-1}    & \textbf{Visualizar un algoritmo}\\
		\toprule
		\textbf{Versión}              & 1.0    \\
		\textbf{Autor}                & David Martínez Acha \\
		\textbf{Requisitos asociados} & RF-1, RF-1.1, RF-1.2, RF-1.3, RF-1.4 \\
		\textbf{Descripción}          & Visualización del proceso de entrenamiento de un algoritmo semi-supervisado. Con gráfico principal y estadísticos. \\
		\textbf{Precondición}         & No hay precondiciones \\
		\textbf{Acciones}             &
		\begin{enumerate}
			\def\labelenumi{\arabic{enumi}.}
			\tightlist
			\item Ejecución del caso de uso 2 (Seleccionar algoritmo).
			\item Ejecución del caso de uso 3 (Cargar conjunto de datos).
            \item Ejecución del caso de uso 4 (Configurar algoritmo).
			\item Se realiza la ejecución interna del algoritmo, obtención de la información y creación de los gráficos.
            \item El usuario verá en la página los distintos gráficos de su visualización. 
            \item [Opcional] Ejecución del caso de uso 5 (Controlar visualizaciones).
		\end{enumerate}\\
		\textbf{Extensiones}          & 4a Si el usuario estuviera registrado, toda la información de la ejecución será almacenada. \\
		\textbf{Postcondición}        & Visualizaciones renderizadas en la Web \\
		\textbf{Excepciones}          & \begin{itemize}
			\item Excepciones de los casos de uso ejecutados controladas por ellos mismos.
			\item El conjunto de datos tenía atributos nominales (paso 4). En este caso se mostrará un mensaje (modal) y al cerrarlo volverá al paso 3.
			\item La ejecución del algoritmo finalizó con excepciones (paso 4), serán capturadas y se volverá al paso 3.
		\end{itemize}	 \\
		\textbf{Importancia}          & Alta\\
		\bottomrule
	\end{tabularx}
	\caption{Caso de uso 1: Visualizar un algoritmo.}
\end{table}



% Caso de Uso 2 -> Seleccionar algoritmo.
\begin{table}[p]
	\centering
	\begin{tabularx}{\linewidth}{ p{0.21\columnwidth} p{0.71\columnwidth} }
		\toprule
		\textbf{CU-2}    & \textbf{Seleccionar algoritmo}\\
		\toprule
		\textbf{Versión}              & 1.0    \\
		\textbf{Autor}                & David Martínez Acha \\
		\textbf{Requisitos asociados} & RF-1, RF-1.1 \\
		\textbf{Descripción}          & Seleccionar el algoritmo a ejecutar (establecerlo en la sesión del usuario). \\
		\textbf{Precondición}         & Ejecutando el caso de uso 1. \\
		\textbf{Acciones}             &
		\begin{enumerate}
			\def\labelenumi{\arabic{enumi}.}
			\tightlist
			\item El usuario selecciona un algoritmo haciendo clic en el nombre del algoritmo en la barra de navegación.
			\item Se redirige al usuario a la página de subida.
		\end{enumerate}\\
        \textbf{Acciones\newline alternativas}&
		\begin{enumerate}
			\def\labelenumi{\arabic{enumi}.}
			\tightlist
			\item En caso de encontrarse en la página principal, el usuario
		puede seleccionar un algoritmo haciendo clic en una de las tarjetas de
		presentación de algoritmos.
			\item Se redirige al usuario a la página de subida. \end{enumerate}\\
		\textbf{Postcondición}        & Algoritmo almacenado en su sesión y redirección a la página de subida. \\
		\textbf{Excepciones}          & Sin excepciones \\
		\textbf{Importancia}          & Alta \\
		\bottomrule
	\end{tabularx}
	\caption{Caso de uso 2: Seleccionar algoritmo.}
\end{table}

% Caso de Uso 3 -> Cargar conjunto de datos.
\begin{table}[p]
	\centering
	\begin{tabularx}{\linewidth}{ p{0.21\columnwidth} p{0.71\columnwidth} }
		\toprule
		\textbf{CU-3}    & \textbf{Cargar conjunto de datos}\\
		\toprule
		\textbf{Versión}              & 1.0    \\
		\textbf{Autor}                & David Martínez Acha \\
		\textbf{Requisitos asociados} & RF-1, RF-1.2 \\
		\textbf{Descripción}          & Carga en el sistema de un fichero ARFF o CSV con el conjunto de datos a utilizar. \\
		\textbf{Precondición}         & Ejecutando el caso de uso 1 y haber ejecutado ya el caso de uso 2.\\
		\textbf{Acciones}             &
		\begin{enumerate}
			\def\labelenumi{\arabic{enumi}.}
			\tightlist
			\item [Opcional] El usuario puede descargar un fichero de prueba si no lo tuviera haciendo clic en el botón de descarga.
			\item El usuario sube un fichero ARFF o CSV en la zona de subida (arrastrando o seleccionando del sistema).
			\item Una vez que la carga se ha completado, se habilita el botón de configuración.
			\item El usuario hace clic en dicho botón.
			\item Se redirige al usuario a la página de configuración.
		\end{enumerate}\\
		\textbf{Extensiones}          & 3a Si el usuario estuviera registrado, el fichero será vinculado a su cuenta (en base de datos). \\
		\textbf{Postcondición}        & Conjunto de datos almacenado en su sesión (y fichero local) y redirección a la página de configuración \\
		\textbf{Excepciones}          & \begin{itemize}
			\item El usuario ha accedido a la página de subida sin seleccionar un algoritmo, será redirigido a la página principal (con un mensaje de aviso de esta situación) y se ejecutará el caso de uso 2.
			\item El fichero no es ARFF ni CSV, al hacer clic en el botón, será redirigido a esta misma página (subida) y deberá realizar el caso de uso de nuevo.
		\end{itemize} \\
		\textbf{Importancia}          & Alta \\
		\bottomrule
	\end{tabularx}
	\caption{Caso de uso 3: Cargar conjunto de datos.}
\end{table}

% Caso de Uso 4 -> Configurar algoritmo.
\begin{table}[p]
	\centering
	\begin{tabularx}{\linewidth}{ p{0.21\columnwidth} p{0.71\columnwidth} }
		\toprule
		\textbf{CU-4}    & \textbf{Configurar algoritmo}\\
		\toprule
		\textbf{Versión}              & 1.0    \\
		\textbf{Autor}                & David Martínez Acha \\
		\textbf{Requisitos asociados} & RF-1, RF-1.3 \\
		\textbf{Descripción}          & Parametrización de la ejecución del algoritmo. Cada algoritmo tiene sus propios parámetros (mismo procedimiento). \\
		\textbf{Precondición}         & Ejecutando el caso de uso 1 y haber ejecutado ya el caso de uso 3. \\
		\textbf{Acciones}             &
		\begin{enumerate}
			\def\labelenumi{\arabic{enumi}.}
			\tightlist
			\item El usuario verá el apartado teórico por un lado y el formulario de parámetros por otro.
			\item El usuario selecciona e introduce los parámetros deseados para la ejecución del algoritmo.
			\item El usuario hace clic en el botón de ejecución.
			\item Se redirige al usuario a la página de visualización.
		\end{enumerate}\\
		\textbf{Postcondición}        & Redirección a la página de visualización \\
		\textbf{Excepciones}          & \begin{itemize}
			\item El usuario ha accedido a la página de subida sin seleccionar un algoritmo, será redirigido a la página principal (con un mensaje de aviso de esta situación) y se ejecutará el caso de uso 2.
			\item El usuario ha accedido a la página de configuración sin cargar un conjunto de datos, será redirigido a la página de subida (con un mensaje de aviso de esta situación) y se ejecutará el caso de uso 3.
			\item El usuario sí que ha cargado un conjunto de datos, pero el sistema no puede acceder al fichero. Redirección a página de error, deberá volver a intentar el caso de uso.
			\item El formulario está incompleto o erróneo. En este caso se bloqueará el envío del formulario indicando qué se debe arreglar.
		\end{itemize} \\
		\textbf{Importancia}          & Alta \\
		\bottomrule
	\end{tabularx}
	\caption{Caso de uso 4: Configurar algoritmo.}
\end{table}

% Caso de Uso 5 -> Controlar visualizaciones.
\begin{table}[p]
	\centering
	\begin{tabularx}{\linewidth}{ p{0.21\columnwidth} p{0.71\columnwidth} }
		\toprule
		\textbf{CU-5}    & \textbf{Controlar visualizaciones}\\
		\toprule
		\textbf{Versión}              & 1.0    \\
		\textbf{Autor}                & David Martínez Acha \\
		\textbf{Requisitos asociados} & RF-1, RF-1.4 \\
		\textbf{Descripción}          & Manipulación de las visualizaciones de forma interactiva. \\
		\textbf{Precondición}         & Ejecutando el caso de uso 1 y haber ejecutado ya el caso de uso 4. \\
		\textbf{Acciones}             &
		\begin{enumerate}
			\def\labelenumi{\arabic{enumi}.}
			\tightlist
			\item El usuario verá la visualización principal así como las estadísticas.
			\item El usuario puede avanzar o retroceder en las iteraciones del algoritmo.
			\item El usuario puede realizar \textit{zoom} sobre el gráfico principal así como reiniciarlo.
			\item El usuario puede ver información particular sobre cada punto en el gráfico principal pasando el ratón por encima de ellos.
			\item El usuario puede interactuar con las estadísticas.
		\end{enumerate}\\
		\textbf{Postcondición}        & El usuario ha podido manejar con libertad los elementos mostrados. \\
		\textbf{Excepciones}          & Sin excepciones \\
		\textbf{Importancia}          & Alta \\
		\bottomrule
	\end{tabularx}
	\caption{Caso de uso 5: Controlar visualizaciones.}
\end{table}

% Caso de Uso 6 -> Registrar.
\begin{table}[p]
	\centering
	\begin{tabularx}{\linewidth}{ p{0.21\columnwidth} p{0.71\columnwidth} }
		\toprule
		\textbf{CU-6}    & \textbf{Registrar}\\
		\toprule
		\textbf{Versión}              & 1.0    \\
		\textbf{Autor}                & David Martínez Acha \\
		\textbf{Requisitos asociados} & RF-2, RF-2.1 \\
		\textbf{Descripción}          & Creación de una cuenta por parte de un usuario anónimo. \\
		\textbf{Precondición}         & Usuario con rol de anónimo (sin cuenta/sin inicio de sesión). \\
		\textbf{Acciones}             &
		\begin{enumerate}
			\def\labelenumi{\arabic{enumi}.}
			\tightlist
			\item El usuario hace clic en el enlace de registro de la barra de navegación.
			\item Se redirige al usuario a la página de registro.
			\item Se mostrará un formulario de registro.
			\item El usuario introduce su nombre.
			\item El usuario introduce su correo electrónico (\textit{email}).
			\item El usuario introduce una contraseña.
			\item El usuario introduce la contraseña del paso anterior repetida (confirmación).
            \item El usuario hace clic en el botón de envío.
			\item Se redirige al usuario a la página principal con sesión iniciada.
		\end{enumerate}\\
		\textbf{Postcondición}        & El usuario ha sido registrado en el sistema (base de datos) y se encuentra con sesión iniciada. \\
		\textbf{Excepciones}          & \begin{itemize}
			\item Un campo del formulario no se ha rellenado. Se impedirá enviar el formulario instando al usuario a completar todos los campos.
			\item El \emph{email} ya tiene una cuenta registrada. Se mostrará un mensaje de error debajo del campo.
			\item La contraseña de confirmación no coincide. Se mostrará un mensaje de error debajo del campo.
		\end{itemize}\\
		\textbf{Importancia}          & Media \\
		\bottomrule
	\end{tabularx}
	\caption{Caso de uso 6: Registrar.}
\end{table}

% Caso de Uso 7 -> Iniciar de sesión.
\begin{table}[p]
	\centering
	\begin{tabularx}{\linewidth}{ p{0.21\columnwidth} p{0.71\columnwidth} }
		\toprule
		\textbf{CU-7}    & \textbf{Iniciar de sesión}\\
		\toprule
		\textbf{Versión}              & 1.0    \\
		\textbf{Autor}                & David Martínez Acha \\
		\textbf{Requisitos asociados} & RF-2, RF-2.2 \\
		\textbf{Descripción}          & Inicio de sesión en la aplicación. \\
		\textbf{Precondición}         & Usuario con rol de anónimo. \\
		\textbf{Acciones}             &
		\begin{enumerate}
			\def\labelenumi{\arabic{enumi}.}
			\tightlist
			\item El usuario hace clic en el enlace de inicio de sesión de la barra de navegación.
			\item Se redirige al usuario a la página de inicio de sesión.
			\item Se mostrará un formulario de inicio de sesión.
			\item El usuario introduce su correo electrónico (\textit{email}).
			\item El usuario introduce su contraseña.
            \item El usuario hace clic en el botón de envío.
			\item Se redirige al usuario a la página principal.
		\end{enumerate}\\
		\textbf{Postcondición}        & El usuario ha iniciado sesión y se encuentra en la página principal. \\
		\textbf{Excepciones}          & \begin{itemize}
			\item Un campo no se ha rellenado. Se impedirá el envío del formulario instando al usuario a completar todos los campos.
			\item El \emph{email} no está en el sistema. Se mostrará un mensaje de error debajo del campo.
			\item La contraseña no coincide con la almacenada. Se mostrará un mensaje de error debajo del campo.
		\end{itemize}\\
		\textbf{Importancia}          & Media \\
		\bottomrule
	\end{tabularx}
	\caption{Caso de uso 7: Iniciar de sesión.}
\end{table}

% Caso de Uso 8 -> Cerrar sesión.
\begin{table}[p]
	\centering
	\begin{tabularx}{\linewidth}{ p{0.21\columnwidth} p{0.71\columnwidth} }
		\toprule
		\textbf{CU-8}    & \textbf{Cerrar sesión}\\
		\toprule
		\textbf{Versión}              & 1.0    \\
		\textbf{Autor}                & David Martínez Acha \\
		\textbf{Requisitos asociados} & RF-2, RF-2.3 \\
		\textbf{Descripción}          & Cerrar sesión en la aplicación \\
		\textbf{Precondición}         & Usuario con sesión iniciada. \\
		\textbf{Acciones}             &
		\begin{enumerate}
			\def\labelenumi{\arabic{enumi}.}
			\tightlist
			\item El usuario hace clic en su nombre usuario en la barra de navegación.
			\item El usuario hace clic en <<Cerrar sesión>> en el desplegable mostrado.
			\item Se redirige al usuario a la página principal y ya no tiene sesión iniciada.
		\end{enumerate}\\
		\textbf{Postcondición}        & El usuario ya no tiene sesión y se encuentra en la página principal. \\
		\textbf{Excepciones}          & Sin excepciones \\
		\textbf{Importancia}          & Media \\
		\bottomrule
	\end{tabularx}
	\caption{Caso de uso 8: Iniciar de sesión.}
\end{table}

% Caso de Uso 9 -> Personalizar perfil.
\begin{table}[p]
	\centering
	\begin{tabularx}{\linewidth}{ p{0.21\columnwidth} p{0.71\columnwidth} }
		\toprule
		\textbf{CU-9}    & \textbf{Personalizar perfil}\\
		\toprule
		\textbf{Versión}              & 1.0    \\
		\textbf{Autor}                & David Martínez Acha \\
		\textbf{Requisitos asociados} & RF-3 \\
		\textbf{Descripción}          & Edición de los datos personales del usuario. \\
		\textbf{Precondición}         & Usuario con sesión iniciada. \\
		\textbf{Acciones}             &
		\begin{enumerate}
			\def\labelenumi{\arabic{enumi}.}
			\tightlist
			\item El usuario hace clic en su nombre usuario en la barra de navegación.
			\item El usuario hace clic en <<Perfil>> en el desplegable mostrado.
			\item Se redirige al usuario a la página de su perfil.
			\item Se mostrará un formulario de edición.
			\item El usuario modifica su nombre, \textit{email} y contraseña (cada uno de manera opcional).
			\item El usuario introduce su contraseña actual.
			\item El usuario hace clic en el botón de envío.
		\end{enumerate}\\
		\textbf{Postcondición}        & Los datos del usuario se han actualizado en la base de datos y mantiene su sesión. \\
		\textbf{Excepciones}          & \begin{itemize}
			\item No se ha introducido contraseña actual. En este caso, se impedirá el envío del formulario instando al usuario a rellenar ese campo.
			\item El \emph{email} ha sido modificado y el nuevo ya está en el sistema. En este caso, se mostrará un mensaje de error debajo del campo (no conllevará cambios).
			\item La contraseña actual no coincide con la almacenada. En este caso, se mostrará un mensaje de error debajo del campo y no se modificarán los datos.
		\end{itemize}\\
		\textbf{Importancia}          & Baja \\
		\bottomrule
	\end{tabularx}
	\caption{Caso de uso 9: Personalizar perfil.}
\end{table}

% Caso de Uso 10 -> Acceder espacio personal.
\begin{table}[p]
	\centering
	\begin{tabularx}{\linewidth}{ p{0.21\columnwidth} p{0.71\columnwidth} }
		\toprule
		\textbf{CU-10}    & \textbf{Acceder espacio personal}\\
		\toprule
		\textbf{Versión}              & 1.0    \\
		\textbf{Autor}                & David Martínez Acha \\
		\textbf{Requisitos asociados} & RF-4, RF-4.1, RF-4.2 \\
		\textbf{Descripción}          & Acceso a al espacio personal del usuario donde visualizar sus conjuntos de datos y ejecuciones realizadas anteriormente. \\
		\textbf{Precondición}         & Usuario con sesión iniciada. \\
		\textbf{Acciones}             &
		\begin{enumerate}
			\def\labelenumi{\arabic{enumi}.}
			\tightlist
			\item El usuario hace clic en su nombre usuario en la barra de navegación.
			\item El usuario hace clic en <<Mi Espacio>> en el desplegable mostrado.
			\item Se redirige al usuario a la página del espacio personal.
			\item Se mostrarán las tablas de sus conjuntos de datos y sus ejecuciones realizadas.
		\end{enumerate}\\
		\textbf{Postcondición}        & El usuario puede visualizar sus conjuntos de datos subidos y sus ejecuciones. \\
		\textbf{Excepciones}          & Los datos no pueden recuperarse de la base de datos, se mostrará el error para después volver a la página principal\\
		\textbf{Importancia}          & Media \\
		\bottomrule
	\end{tabularx}
	\caption{Caso de uso 10: Acceder espacio personal.}
\end{table}


% Caso de Uso 11 -> Ejecutar conjunto de datos.
\begin{table}[p]
	\centering
	\begin{tabularx}{\linewidth}{ p{0.21\columnwidth} p{0.71\columnwidth} }
		\toprule
		\textbf{CU-11}    & \textbf{Ejecutar conjunto de datos}\\
		\toprule
		\textbf{Versión}              & 1.0    \\
		\textbf{Autor}                & David Martínez Acha \\
		\textbf{Requisitos asociados} & RF-4.1.1 \\
		\textbf{Descripción}          & Utilizar un conjunto de datos subido para ejecutar uno de los algoritmos. \\
		\textbf{Precondición}         & Encontrarse en el espacio personal (caso de uso 10). \\
		\textbf{Acciones}             &
		\begin{enumerate}
			\def\labelenumi{\arabic{enumi}.}
			\tightlist
			\item El usuario decide qué conjunto de datos utilizar.
			\item El usuario pulsa en el botón con el símbolo \textit{play}.
			\item Se mostrará un <<modal>> con los algoritmos.
			\item El usuario hace clic en uno de los algoritmos.
			\item Se redirige al usuario a la página de configuración del algoritmo.
			\item Ejecutar caso de uso 1 desde el paso 3.
		\end{enumerate}\\
		\textbf{Postcondición}        & El usuario ha sido redirigido a la configuración del algoritmo seleccionado. El sistema muestra además el fichero seleccionado. \\
		\textbf{Excepciones}          & Sin excepciones (ver excepciones del caso de uso 1)\\
		\textbf{Importancia}          & Media \\
		\bottomrule
	\end{tabularx}
	\caption{Caso de uso 11: Ejecutar conjunto de datos.}
\end{table}

% Caso de Uso 12 -> Replicar ejecución.
\begin{table}[p]
	\centering
	\begin{tabularx}{\linewidth}{ p{0.21\columnwidth} p{0.71\columnwidth} }
		\toprule
		\textbf{CU-12}    & \textbf{Replicar ejecución}\\
		\toprule
		\textbf{Versión}              & 1.0    \\
		\textbf{Autor}                & David Martínez Acha \\
		\textbf{Requisitos asociados} & RF-4.2.1 \\
		\textbf{Descripción}          & Ejecutar de nuevo una ejecución previa exactamente como ocurrió. \\
		\textbf{Precondición}         & Encontrarse en el espacio personal (caso de uso 10). \\
		\textbf{Acciones}             &
		\begin{enumerate}
			\def\labelenumi{\arabic{enumi}.}
			\tightlist
			\item El usuario decide qué ejecución replicar.
			\item El usuario pulsa en el botón con el símbolo de \textit{refresh}.
			\item Se redirige al usuario directamente a la página de visualización del algoritmo.
			\item Ejecutar el caso de uso 1 desde el paso 4.
		\end{enumerate}\\
		\textbf{Postcondición}        & El usuario ha sido redirigido a la visualización de la ejecución (del algoritmo). \\
		\textbf{Excepciones}          & Sin excepciones (ver excepciones del caso de uso 1)\\
		\textbf{Importancia}          & Media \\
		\bottomrule
	\end{tabularx}
	\caption{Caso de uso 12: Replicar ejecución.}
\end{table}

% Caso de Uso 13 -> Acceder panel de administración.
\begin{table}[p]
	\centering
	\begin{tabularx}{\linewidth}{ p{0.21\columnwidth} p{0.71\columnwidth} }
		\toprule
		\textbf{CU-13}    & \textbf{Acceder panel de administración}\\
		\toprule
		\textbf{Versión}              & 1.0    \\
		\textbf{Autor}                & David Martínez Acha \\
		\textbf{Requisitos asociados} & RF-5, RF-5.1, RF-5.1.3, RF-5.2, RF-5.2.2, RF-5.3, RF-5.3.2\\
		\textbf{Descripción}          & Acceso a al panel de administración del usuario con rol de administrador donde visualizar 
		todos los usuarios, los conjuntos de datos y ejecuciones realizadas.\\
		\textbf{Precondición}         & Usuario con sesión iniciada con rol de administrador. \\
		\textbf{Acciones}             &
		\begin{enumerate}
			\def\labelenumi{\arabic{enumi}.}
			\tightlist
			\item El administrador hace clic en su nombre usuario en la barra de navegación.
			\item El administrador hace clic en <<Panel de administración>> en el desplegable mostrado.
			\item Se redirige al administrador a la página del panel de administración (\textit{dashboard}).
			\item Se mostrarán las tablas de los usuarios, conjuntos de datos y ejecuciones realizadas (de todos los usuarios).
			\item Se mostrarán estadísticas básicas:
			\begin{itemize}
				\item Número total de usuarios.
				\item Ficheros (conjuntos de datos) subidos en los últimos 7 días.
				\item Ejecuciones realizadas en los últimos 7 días.
			\end{itemize}
		\end{enumerate}\\
		\textbf{Postcondición}        & El administrador tiene una vista global de todas las tablas con sus estadísticas básicas.\\
		\textbf{Excepciones}          & Los datos no pueden recuperarse de la base de datos, se mostrará el error para después volver a la página principal\\
		\textbf{Importancia}          & Media \\
		\bottomrule
	\end{tabularx}
	\caption{Caso de uso 13: Acceder panel de administración.}
\end{table}

% Caso de Uso 14 -> Eliminar conjunto de datos.
\begin{table}[p]
	\centering
	\begin{tabularx}{\linewidth}{ p{0.21\columnwidth} p{0.71\columnwidth} }
		\toprule
		\textbf{CU-14}    & \textbf{Eliminar conjunto de datos}\\
		\toprule
		\textbf{Versión}              & 1.0    \\
		\textbf{Autor}                & David Martínez Acha \\
		\textbf{Requisitos asociados} & RF-4.1.2, RF-5.1.2 \\
		\textbf{Descripción}          & Eliminar un conjunto de datos (fichero) del sistema. \\
		\textbf{Precondición}         & Encontrarse en el espacio personal (caso de uso 10) o, si tiene rol de administrador, en el panel de administración (caso de uso 13). \\
		\textbf{Acciones}             &
		\begin{enumerate}
			\def\labelenumi{\arabic{enumi}.}
			\tightlist
			\item El usuario decide qué conjunto de datos eliminar sobre la tabla de conjuntos de datos.
			\item El usuario pulsa en el botón rojo con el símbolo de papelera en esa fila.
			\item Se mostrará un <<modal>> para confirmar la operación.
			\item El usuario confirma la operación.
			\item El fichero se ha eliminado y ha desaparecido de la tabla.
		\end{enumerate}\\
		\textbf{Acciones\newline alternativas}&
		\begin{enumerate}
			\def\labelenumi{\arabic{enumi}.}
			\tightlist
			\item El usuario decide qué conjunto de datos eliminar.
			\item El usuario pulsa en el botón rojo con el símbolo de \textit{papelera}.
			\item Se mostrará un <<modal>> para confirmar la operación.
			\item El usuario cancela la operación.
			\item Todo se encuentra en el mismo estado que al iniciar el caso de uso.
		\end{enumerate}\\
		\textbf{Postcondición}        & Si el usuario ha decidido eliminar el fichero, la fila ha sido eliminada y el fichero ha sido borrado del sistema junto con su referencia en la base de datos. 
		En caso contrario, todo se encuentra como al principio.\\
		\textbf{Excepciones}          & No ha sido posible eliminar el fichero (sistema y/o base de datos). En este caso se muestra un mensaje de error para informar al usuario. \\
		\textbf{Importancia}          & Media \\
		\bottomrule
	\end{tabularx}
	\caption{Caso de uso 14: Eliminar conjunto de datos.}
\end{table}

% Caso de Uso 15 -> Eliminar ejecución.
\begin{table}[p]
	\centering
	\begin{tabularx}{\linewidth}{ p{0.21\columnwidth} p{0.71\columnwidth} }
		\toprule
		\textbf{CU-15}    & \textbf{Eliminar ejecución}\\
		\toprule
		\textbf{Versión}              & 1.0    \\
		\textbf{Autor}                & David Martínez Acha \\
		\textbf{Requisitos asociados} & RF-4.2.2, RF-5.2.2 \\
		\textbf{Descripción}          & Eliminar una ejecución previa del sistema. \\
		\textbf{Precondición}         & Encontrarse en el espacio personal (caso de uso 10) o, si tiene rol de administrador, en el panel de administración (caso de uso 13). \\
		\textbf{Acciones}             &
		\begin{enumerate}
			\def\labelenumi{\arabic{enumi}.}
			\tightlist
			\item El usuario decide qué ejecución eliminar sobre la tabla de ejecuciones.
			\item El usuario pulsa en el botón rojo con el símbolo de papelera en esa fila.
			\item Se mostrará un <<modal>> para confirmar la operación.
			\item El usuario confirma la operación.
			\item La ejecución se ha eliminado y ha desaparecido de la tabla.
		\end{enumerate}\\
		\textbf{Acciones\newline alternativas}&
		\begin{enumerate}
			\def\labelenumi{\arabic{enumi}.}
			\tightlist
			\item El usuario decide qué ejecución eliminar sobre la tabla de ejecuciones.
			\item El usuario pulsa en el botón rojo con el símbolo de \textit{papelera} en esa fila.
			\item Se mostrará un <<modal>> para confirmar la operación.
			\item El usuario cancela la operación.
			\item Todo se encuentra en el mismo estado que al iniciar el caso de uso.
		\end{enumerate}\\
		\textbf{Postcondición}        & Si el usuario ha decidido eliminar la ejecución, la fila ha sido eliminada y los datos de la ejecución han sido borrados del sistema junto con su referencia en la base de datos. 
		En caso contrario, todo se encuentra como al principio.\\
		\textbf{Excepciones}          & No ha sido posible eliminar la ejecución (sistema y/o base de datos). En este caso se muestra un mensaje de error para informar al usuario. \\
		\textbf{Importancia}          & Media \\
		\bottomrule
	\end{tabularx}
	\caption{Caso de uso 15: Eliminar ejecución.}
\end{table}

% Caso de Uso 16 -> Editar perfil ajeno.
\begin{table}[p]
	\centering
	\begin{tabularx}{\linewidth}{ p{0.21\columnwidth} p{0.71\columnwidth} }
		\toprule
		\textbf{CU-16}    & \textbf{Editar perfil ajeno}\\
		\toprule
		\textbf{Versión}              & 1.0    \\
		\textbf{Autor}                & David Martínez Acha \\
		\textbf{Requisitos asociados} & RF-5.1.1 \\
		\textbf{Descripción}          & Edición de los datos personales de cualquier usuario del sistema. \\
		\textbf{Precondición}         & Encontrarse en el panel de administración (caso de uso 13). \\
		\textbf{Acciones}             &
		\begin{enumerate}
			\def\labelenumi{\arabic{enumi}.}
			\tightlist
			\item El administrador decide qué usuario editar.
			\item El administrador pulsa en el botón verde con el símbolo de lápiz en esa fila.
			\item Se redirige a la página de personalización del perfil.
			\item Se muestra al administrador el punto de vista del usuario que se está editando con un indicativo.
			\item El administrador modifica los campos que desee del usuario.
			\item El administrador envía el formulario.
			\item Se actualizan los datos del usuario.
			\item Se redirige al administrador al panel de administrador.
		\end{enumerate}\\
		\textbf{Postcondición}        & Los datos del usuario se han actualizado en la base de datos y el administrador se encuentra en el panel de administrador. \\
		\textbf{Excepciones}          & \begin{itemize}
			\item El \texttt{email} ha sido modificado y el nuevo ya está en el sistema. En este caso, se mostrará un mensaje de error debajo del campo.
		\end{itemize}\\
		\textbf{Importancia}          & Baja \\
		\bottomrule
	\end{tabularx}
	\caption{Caso de uso 16: Editar perfil ajeno.}
\end{table}

% Caso de Uso 17 -> Eliminar usuario.
\begin{table}[p]
	\centering
	\begin{tabularx}{\linewidth}{ p{0.21\columnwidth} p{0.71\columnwidth} }
		\toprule
		\textbf{CU-17}    & \textbf{Eliminar usuario}\\
		\toprule
		\textbf{Versión}              & 1.0    \\
		\textbf{Autor}                & David Martínez Acha \\
		\textbf{Requisitos asociados} & RF-5.1.2 \\
		\textbf{Descripción}          & Eliminar un usuario del sistema. \\
		\textbf{Precondición}         & Encontrarse en el panel de administración (caso de uso 13). \\
		\textbf{Acciones}             &
		\begin{enumerate}
			\def\labelenumi{\arabic{enumi}.}
			\tightlist
			\item El usuario decide qué usuario eliminar sobre la tabla de usuarios.
			\item El usuario pulsa en el botón rojo con el símbolo de papelera en esa fila.
			\item Se mostrará un <<modal>> para confirmar la operación.
			\item El usuario confirma la operación.
			\item La ejecución se ha eliminado y ha desaparecido de la tabla.
		\end{enumerate}\\
		\textbf{Acciones\newline alternativas}&
		\begin{enumerate}
			\def\labelenumi{\arabic{enumi}.}
			\tightlist
			\item El usuario decide qué usuario eliminar sobre la tabla de ejecuciones.
			\item El usuario pulsa en el botón rojo con el símbolo de \textit{papelera} en esa fila.
			\item Se mostrará un <<modal>> para confirmar la operación.
			\item El usuario cancela la operación.
			\item Todo se encuentra en el mismo estado que al iniciar el caso de uso.
		\end{enumerate}\\
		\textbf{Postcondición}        & Si el usuario ha decidido eliminar el usuario, la fila ha sido eliminada y los datos del usuario (incluidos ficheros y ejecuciones) han sido borrados del sistema junto con su referencia en la base de datos. 
		En caso contrario, todo se encuentra como al principio.\\
		\textbf{Excepciones}          & No ha sido posible eliminar el usuario (sistema y/o base de datos). En este caso se muestra un mensaje de error para informar al administrador. \\
		\textbf{Importancia}          & Media \\
		\bottomrule
	\end{tabularx}
	\caption{Caso de uso 17: Eliminar usuario.}
\end{table}

% Caso de Uso 18 -> Cambiar idioma.
\begin{table}[p]
	\centering
	\begin{tabularx}{\linewidth}{ p{0.21\columnwidth} p{0.71\columnwidth} }
		\toprule
		\textbf{CU-18}    & \textbf{Cambiar idioma}\\
		\toprule
		\textbf{Versión}              & 1.0    \\
		\textbf{Autor}                & David Martínez Acha \\
		\textbf{Requisitos asociados} & RF-6 \\
		\textbf{Descripción}          & Cambiar el idioma de la página (entre inglés y español) \\
		\textbf{Precondición}         & No hay precondiciones \\
		\textbf{Acciones}             &
		\begin{enumerate}
			\def\labelenumi{\arabic{enumi}.}
			\tightlist
			\item El usuario hace clic en el símbolo de traducción en la barra de navegación.
            \item Aparece un desplegable de idiomas.
            \item El usuario hace clic en el idioma deseado.
            \item Se actualiza la página indicando el idioma a la aplicación.
		\end{enumerate}\\
		\textbf{Postcondición}        & La página se ha cambiado al idioma seleccionado \\
		\textbf{Excepciones}          & Sin excepciones \\
		\textbf{Importancia}          & Baja \\
		\bottomrule
	\end{tabularx}
	\caption{Caso de uso 18: Cambiar idioma.}
\end{table}

% Caso de Uso 19 -> Consultar ayuda.
\begin{table}[p]
	\centering
	\begin{tabularx}{\linewidth}{ p{0.21\columnwidth} p{0.71\columnwidth} }
		\toprule
		\textbf{CU-19}    & \textbf{Consultar ayuda}\\
		\toprule
		\textbf{Versión}              & 1.0    \\
		\textbf{Autor}                & David Martínez Acha \\
		\textbf{Requisitos asociados} & RF-7 \\
		\textbf{Descripción}          & Acceder al manual de usuario de la aplicación \\
		\textbf{Precondición}         & No hay precondiciones \\
		\textbf{Acciones}             &
		\begin{enumerate}
			\def\labelenumi{\arabic{enumi}.}
			\tightlist
			\item El usuario hace clic en el símbolo de la interrogación en la barra de navegación.
            \item El usuario es redireccionado (en una nueva pestaña) al manual de usuario (PDF).
		\end{enumerate}\\
		\textbf{Postcondición}        & El usuario visualiza el manual de usuario \\
		\textbf{Excepciones}          & Sin excepciones \\
		\textbf{Importancia}          & Baja \\
		\bottomrule
	\end{tabularx}
	\caption{Caso de uso 19: Consultar ayuda.}
\end{table}
