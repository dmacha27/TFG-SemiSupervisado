\capitulo{1}{Introducción}

El ámbito del aprendizaje automático es un campo muy interesante y del que cada
vez hay más atención. La realidad es que la mayor parte del conocimiento está
muy centrado en ciertos tipos de aprendizaje automático: el supervisado y el no
supervisado. En cuanto se indaga un poco en <<machine learning>> estos dos
conceptos empiezan a rondar. Pero del que no se oye tanto, y puede ser muy
beneficioso, es el aprendizaje semi-supervisado. 

El aprendizaje supervisado, en pocas palabras, permite aprovechar situaciones en
las que se sabe qué representa un dato (por ejemplo, dada una animal, se sabe si
el animal es un perro o un pato), el no supervisado no tiene esta <<suerte>>, se
utiliza en casos en los que no se tiene ese conocimiento, sino que es él mismo
el que intenta extraer las representaciones (por ejemplo, para un conjunto de
animales, podría distinguir entre los que tiene pico y alas y los que tienen
cuatro patas sin necesidad de saber qué animal concreto es). En la realidad
(obviando los ejemplos tan sencillos comentados), el <<etiquetado>> de los datos
suele ser muy costoso y se tienen muchos más datos de los que no se sabe qué
representan, el aprendizaje semi-supervisado es el ideal en estos casos y es que
permite inferir conocimiento para estos últimos y determinar a qué
corresponden.

Centrando más el objetivo final de este trabajo, no existen muchas aplicaciones
que permitan compaginar la teoría de estos conceptos con visualizaciones
interesantes que permitan comprender su funcionamiento y mucho menos para los
algoritmos semi-supervisados que incluso en muchos casos suelen obviarse. Vista la
carencia en este ámbito, este trabajo pretende resultar en la creación de una
aplicación amigable y atractiva que permita, mediante visualizaciones,
comprender realmente cómo funcionan los principales algoritmos
semi-supervisados.

Las herramientas de fácil acceso, como las páginas Web que no requieren de
instalación por parte del usuario, van de la mano con la globalización de
internet. Es por ello que esta herramienta, categorizada ya como
\textit{docente}, será accesible desde internet. La idea de esto es permitir a
los usuarios la rapidez y facilidad de acceder simplemente a una <<URL>> sin
prácticamente tener que realizar otro trabajo.

Además, los datos de las visualizaciones será proporcionados por una biblioteca
propia donde se implementen estos algoritmos. Estarán adaptados para la
obtención de la información del entrenamiento y estadísticas relevantes, pero su
caso están pensados para que puedan ser utilizados de forma general.


