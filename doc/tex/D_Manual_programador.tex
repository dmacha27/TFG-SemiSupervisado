\apendice{Documentación técnica de programación}

\section{Introducción}
En esta sección se presenta toda la documentación técnica del desarrollo del
proyecto. Trata de ser una guía para entender cómo se ha hecho el proyecto
comenzando por los directorios y su contenido, con un manual introductorio para
un programador iniciado en el proyecto, explicación y ejemplificación de la
instalación del mismo y las pruebas que se han realizado para validarlo.

\clearpage
\section{Estructura de directorios}

\begin{figure}[H]
    \dirtree{%
        .1 /.
        .2 \textbf{algoritmos}: \begin{minipage}[t]{10cm}
            algoritmos Semi-Supervisados implementados{.}\\
        \end{minipage}.
        .3 \textbf{test}: \begin{minipage}[t]{10cm}
            directorios y ficheros mediante los que se valida el
            desarrollo software correcto del proyecto{.}\\
        \end{minipage}.
        .4 \textbf{check\_implementations}: \begin{minipage}[t]{6cm}
            pruebas para la validación de los algoritmos{.}\\
        \end{minipage}.
        .4 \textbf{profile\_results}: \begin{minipage}[t]{10cm}
            resultados de los procesos de <<profiling>>{.}\\
        \end{minipage}.
        .4 \textbf{test\_files}: \begin{minipage}[t]{10cm}
            ficheros de prueba (ARFF y CSV) para las pruebas{.}\\
        \end{minipage}.
        .3 \textbf{utilidades}: \begin{minipage}[t]{10cm}
            utilidades (programas) que realizan ciertos pasos de la
            aplicación y de los algoritmos para centralizar estos procedimientos (comunes){.}\\
        \end{minipage}.
        .2 \textbf{docs}: \begin{minipage}[t]{10cm}
            documentación teórica y técnica del proyecto (hecha en \LaTeX){.}\\
        \end{minipage}.
        .3 \textbf{img}: \begin{minipage}[t]{10cm}
            imágenes utilizadas para la generación de la documentación{.}\\
        \end{minipage}.
        .3 \textbf{tex}: \begin{minipage}[t]{10cm}
            archivos de texto plano con código \LaTeX{.}\\
        \end{minipage}.
        .2 \textbf{web}: \begin{minipage}[t]{10cm}
            estructura o código de la aplicación Web{.}\\
        \end{minipage}.
        .3 \textbf{datasets}: \begin{minipage}[t]{10cm}
            contiene (durante el funcionamiento de la aplicación) todos los conjuntos de
            datos que los usuarios introducen{.}\\
        \end{minipage}.
        .4 \textbf{seleccionar}: \begin{minipage}[t]{10cm}
            conjuntos de datos para seleccionar de prueba, principalmente
            durante el desarrollo de la aplicación{.} También almacena el fichero de
            prueba que los usuarios pueden descargar{.}\\
        \end{minipage}.
    }
\end{figure}

\begin{figure}[H]
    \dirtree{%
        .1 \textbf{web}: \begin{minipage}[t]{10cm}
            estructura o código de la aplicación Web{.}\\
        \end{minipage}.
        .2 \textbf{static}: \begin{minipage}[t]{10cm}
            ficheros estáticos que utiliza la aplicación Web: CSS,
            Javascript o JSON{.}\\
        \end{minipage}.
        .2 \textbf{templates}: \begin{minipage}[t]{10cm}
            plantillas HTML (Jinja2) que renderiza la aplicación Web
            (Flask){.}\\
        \end{minipage}.
        .2 \textbf{translations}: \begin{minipage}[t]{10cm}
            traducciones de los textos de la aplicación (por idiomas){.}\\
        \end{minipage}.
    }
\end{figure}

\section{Manual del programador}
El objetivo de este manual es dar al lector/desarrollador que comience a
trabajar con el proyecto el conocimiento necesario para continuarlo. Se ha de
tener en cuenta que lo descrito a continuación es lo que se ha utilizado para el
entorno desarrollo inicial y será explicado para este.

En primer lugar se listan las herramientas requeridas:



\section{Compilación, instalación y ejecución del proyecto}

\section{Pruebas del sistema}
