\capitulo{7}{Conclusiones y Líneas de trabajo futuras}

En esta sección final se presentan las conclusiones a las que se ha llegado con
la finalización del proyecto. La aplicación desarrollada tiene muchos puntos por
los que puede ser expandida, algunos sencillos y otros más grandes que ofrecen
posibilidades interesantes para los usuarios. Por esto último, también se
incluye una serie de líneas de trabajo para la mejora/expansión del proyecto.

\section{Conclusiones}

Una vez finalizado el proyecto, esta son las conclusiones que pueden extraerse:

\begin{itemize}
    \item Se ha desarrollado la biblioteca de algoritmos semi-supervisados
    personalizados, siendo validados mediante la comparación contra
    \texttt{sslearn} y obteniendo resultados satisfactorios suficientes. Con
    ella se ha podido extraer el proceso de entrenamiento de estos algoritmos
    para poder visualizarlos.
    \item El proyecto se ha culminado con la creación de una aplicación web
    completamente funcional que permite visualizar estos algoritmos. Además,
    también ofrece a los usuarios la posibilidad de registrarse y poder
    gestionar sus conjuntos de datos y ejecuciones.
    \item En líneas generales, la aplicación se ha desarrollado con la intención
    de ser completamente extensible. En este proyecto se han incluido cuatro
    algoritmos semi-supervisados, pero existen muchos más y se ha dejado puntos
    de extensión para ellos.
    \item Con la terminación del proyecto, se ha creado una documentación
    completa de todos los aspectos técnicos considerados, así como una manual
    para el programador y para el usuario. En estos manuales se ha incluido todo
    lo que se ha creído imprescindible para entender perfectamente la
    aplicación.
    \item Durante la realización del proyecto (algoritmos y web) se han
    utilizado muchos conceptos aprendidos en asignaturas este curso y durante
    cursos previos. Entre ellas es de destacar \emph{Bases de datos},
    \emph{Metodología de la programación} (orientación a objetos),
    \emph{Estructuras de datos}, \emph{Sistemas inteligentes},
    \emph{Algoritmia}, \emph{Diseño y Mantenimiento del software} y
    \emph{Minería de datos}.
    \item Se han tenido que aprender otros muchos conceptos nunca antes vistos.
    Principalmente, el aprendizaje se ha centrado en el desarrollo web, con
    aspectos como las peticiones, respuestas, códigos HTTP, seguridad... y
    también sobre minería de datos, particularizando sobre clasificación.
    \item En programación, se ha tenido que aprender el lenguaje de marcas HTML junto
    con JavaScript para generar todo el contenido dinámico. Aunque ha supuesto
    un verdadero reto, ha sido un proceso interesante y satisfactorio.
    \item La documentación se ha realizado en pasos constantes (aumentando el
    ritmo hasta el final). Y pese a que no se ha apurado a realizarlo al final,
    sí que ha conllevado mucho tiempo del proyecto, incluso más que, por
    ejemplo, el desarrollo de los algoritmos.
\end{itemize}

