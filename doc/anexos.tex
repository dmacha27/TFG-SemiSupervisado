\documentclass[a4paper,12pt,twoside]{memoir}

% Castellano
\usepackage[spanish,es-tabla]{babel}
\selectlanguage{spanish}
\usepackage[utf8]{inputenc}
\usepackage[T1]{fontenc}
\usepackage{lmodern} % scalable font
\usepackage{microtype}
\usepackage{placeins}

\RequirePackage{booktabs}
\RequirePackage[table]{xcolor}
\RequirePackage{xtab}
\RequirePackage{multirow}

% Links
\PassOptionsToPackage{hyphens}{url}\usepackage[colorlinks]{hyperref}
\hypersetup{
	allcolors = {red}
}

% Ecuaciones
\usepackage{amsmath}

%Código
\usepackage[most]{tcolorbox}
\usepackage{minted}

% Rutas de fichero / paquete
\newcommand{\ruta}[1]{{\sffamily #1}}
\usepackage{dirtree}

% Párrafos
\nonzeroparskip

% Huérfanas y viudas
\widowpenalty100000
\clubpenalty100000

% Evitar solapes en el header
\nouppercaseheads

% Imagenes
\usepackage{graphicx}
\newcommand{\imagen}[2]{
	\begin{figure}[!h]
		\centering
		\includegraphics[width=0.9\textwidth]{#1}
		\caption{#2}\label{fig:#1}
	\end{figure}
	\FloatBarrier
}

\newcommand{\imagencontamano}[3]{
	\begin{figure}[!h]
		\centering
		\includegraphics[width=#3\textwidth]{#1}
		\caption{#2}\label{fig:#1}
	\end{figure}
	\FloatBarrier
}

\newcommand{\imagenflotante}[2]{
	\begin{figure}%[!h]
		\centering
		\includegraphics[width=0.9\textwidth]{#1}
		\caption{#2}\label{fig:#1}
	\end{figure}
}



% El comando \figura nos permite insertar figuras comodamente, y utilizando
% siempre el mismo formato. Los parametros son:
% 1 -> Porcentaje del ancho de página que ocupará la figura (de 0 a 1)
% 2 --> Fichero de la imagen
% 3 --> Texto a pie de imagen
% 4 --> Etiqueta (label) para referencias
% 5 --> Opciones que queramos pasarle al \includegraphics
% 6 --> Opciones de posicionamiento a pasarle a \begin{figure}
\newcommand{\figuraConPosicion}[6]{%
  \setlength{\anchoFloat}{#1\textwidth}%
  \addtolength{\anchoFloat}{-4\fboxsep}%
  \setlength{\anchoFigura}{\anchoFloat}%
  \begin{figure}[#6]
    \begin{center}%
      \Ovalbox{%
        \begin{minipage}{\anchoFloat}%
          \begin{center}%
            \includegraphics[width=\anchoFigura,#5]{#2}%
            \caption{#3}%
            \label{#4}%
          \end{center}%
        \end{minipage}
      }%
    \end{center}%
  \end{figure}%
}

%
% Comando para incluir imágenes en formato apaisado (sin marco).
\newcommand{\figuraApaisadaSinMarco}[5]{%
  \begin{figure}%
    \begin{center}%
    \includegraphics[angle=90,height=#1\textheight,#5]{#2}%
    \caption{#3}%
    \label{#4}%
    \end{center}%
  \end{figure}%
}
% Para las tablas
\newcommand{\otoprule}{\midrule [\heavyrulewidth]}
%
% Nuevo comando para tablas pequeñas (menos de una página).
\newcommand{\tablaSmall}[5]{%
 \begin{table}
  \begin{center}
   \rowcolors {2}{gray!35}{}
   \begin{tabular}{#2}
    \toprule
    #4
    \otoprule
    #5
    \bottomrule
   \end{tabular}
   \caption{#1}
   \label{tabla:#3}
  \end{center}
 \end{table}
}

%
%Para el float H de tablaSmallSinColores
\usepackage{float}

%
% Nuevo comando para tablas pequeñas (menos de una página).
\newcommand{\tablaSmallSinColores}[5]{%
 \begin{table}[H]
  \begin{center}
   \begin{tabular}{#2}
    \toprule
    #4
    \otoprule
    #5
    \bottomrule
   \end{tabular}
   \caption{#1}
   \label{tabla:#3}
  \end{center}
 \end{table}
}

\newcommand{\tablaApaisadaSmall}[5]{%
\begin{landscape}
  \begin{table}
   \begin{center}
    \rowcolors {2}{gray!35}{}
    \begin{tabular}{#2}
     \toprule
     #4
     \otoprule
     #5
     \bottomrule
    \end{tabular}
    \caption{#1}
    \label{tabla:#3}
   \end{center}
  \end{table}
\end{landscape}
}

%
% Nuevo comando para tablas grandes con cabecera y filas alternas coloreadas en gris.
\newcommand{\tabla}[6]{%
  \begin{center}
    \tablefirsthead{
      \toprule
      #5
      \otoprule
    }
    \tablehead{
      \multicolumn{#3}{l}{\small\sl continúa desde la página anterior}\\
      \toprule
      #5
      \otoprule
    }
    \tabletail{
      \hline
      \multicolumn{#3}{r}{\small\sl continúa en la página siguiente}\\
    }
    \tablelasttail{
      \hline
    }
    \bottomcaption{#1}
    \rowcolors {2}{gray!35}{}
    \begin{xtabular}{#2}
      #6
      \bottomrule
    \end{xtabular}
    \label{tabla:#4}
  \end{center}
}

%
% Nuevo comando para tablas grandes con cabecera.
\newcommand{\tablaSinColores}[6]{%
  \begin{center}
    \tablefirsthead{
      \toprule
      #5
      \otoprule
    }
    \tablehead{
      \multicolumn{#3}{l}{\small\sl continúa desde la página anterior}\\
      \toprule
      #5
      \otoprule
    }
    \tabletail{
      \hline
      \multicolumn{#3}{r}{\small\sl continúa en la página siguiente}\\
    }
    \tablelasttail{
      \hline
    }
    \bottomcaption{#1}
    \begin{xtabular}{#2}
      #6
      \bottomrule
    \end{xtabular}
    \label{tabla:#4}
  \end{center}
}

%
% Nuevo comando para tablas grandes sin cabecera.
\newcommand{\tablaSinCabecera}[5]{%
  \begin{center}
    \tablefirsthead{
      \toprule
    }
    \tablehead{
      \multicolumn{#3}{l}{\small\sl continúa desde la página anterior}\\
      \hline
    }
    \tabletail{
      \hline
      \multicolumn{#3}{r}{\small\sl continúa en la página siguiente}\\
    }
    \tablelasttail{
      \hline
    }
    \bottomcaption{#1}
  \begin{xtabular}{#2}
    #5
   \bottomrule
  \end{xtabular}
  \label{tabla:#4}
  \end{center}
}



\definecolor{cgoLight}{HTML}{EEEEEE}
\definecolor{cgoExtralight}{HTML}{FFFFFF}

%
% Nuevo comando para tablas grandes sin cabecera.
\newcommand{\tablaSinCabeceraConBandas}[5]{%
  \begin{center}
    \tablefirsthead{
      \toprule
    }
    \tablehead{
      \multicolumn{#3}{l}{\small\sl continúa desde la página anterior}\\
      \hline
    }
    \tabletail{
      \hline
      \multicolumn{#3}{r}{\small\sl continúa en la página siguiente}\\
    }
    \tablelasttail{
      \hline
    }
    \bottomcaption{#1}
    \rowcolors[]{1}{cgoExtralight}{cgoLight}

  \begin{xtabular}{#2}
    #5
   \bottomrule
  \end{xtabular}
  \label{tabla:#4}
  \end{center}
}




\graphicspath{ {./img/} }

% Capítulos
\chapterstyle{bianchi}
\newcommand{\capitulo}[2]{
	\setcounter{chapter}{#1}
	\setcounter{section}{0}
	\setcounter{figure}{0}
	\setcounter{table}{0}
	\chapter*{#2}
	\addcontentsline{toc}{chapter}{#2}
	\markboth{#2}{#2}
}

% Apéndices
\renewcommand{\appendixname}{Apéndice}
\renewcommand*\cftappendixname{\appendixname}

\newcommand{\apendice}[1]{
	%\renewcommand{\thechapter}{A}
	\chapter{#1}
}

\renewcommand*\cftappendixname{\appendixname\ }

% Formato de portada
\makeatletter
\usepackage{xcolor}
\newcommand{\tutor}[1]{\def\@tutor{#1}}
\newcommand{\cotutor}[1]{\def\@cotutor{#1}}
\newcommand{\course}[1]{\def\@course{#1}}
\definecolor{cpardoBox}{HTML}{E6E6FF}
\def\maketitle{
  \null
  \thispagestyle{empty}
  % Cabecera ----------------
\noindent\includegraphics[width=\textwidth]{cabecera}\vspace{1cm}%
  \vfill
  % Título proyecto y escudo informática ----------------
  \colorbox{cpardoBox}{%
    \begin{minipage}{.8\textwidth}
      \vspace{.5cm}\Large
      \begin{center}
      \textbf{TFG del Grado en Ingeniería Informática}\vspace{.6cm}\\
      \textbf{\LARGE\@title{}}
      \end{center}
      \vspace{.2cm}
    \end{minipage}

  }%
  \hfill\begin{minipage}{.20\textwidth}
    \includegraphics[width=\textwidth]{escudoInfor}
  \end{minipage}
  \vfill
  % Datos de alumno, curso y tutores ------------------
  \begin{center}%
  {%
    \noindent\LARGE
    Presentado por \@author{}\\ 
    en Universidad de Burgos --- \@date{}\\
    Tutor: \@tutor{}\\
    Cotutor: \@cotutor{}\\
  }%
  \end{center}%
  \null
  \cleardoublepage
  }
\makeatother

\newcommand{\titulo}{Herramienta docente para la visualización en Web de algoritmos de aprendizaje Semi-Supervisado}
\newcommand{\nombre}{David Martínez Acha}
\newcommand{\tut}{Álvar Arnaiz González}
\newcommand{\cotut}{César Ignacio García Osorio}

% Datos de portada
\title{\titulo\\Documentación técnica}
\author{\nombre}
\tutor{\tut}
\cotutor{\cotut}
\date{\today}

\begin{document}

\maketitle



\cleardoublepage



%%%%%%%%%%%%%%%%%%%%%%%%%%%%%%%%%%%%%%%%%%%%%%%%%%%%%%%%%%%%%%%%%%%%%%%%%%%%%%%%%%%%%%%%



\frontmatter


\clearpage

% Indices
\tableofcontents

\clearpage

\listoffigures

\clearpage

\listoftables

\clearpage

\mainmatter

\appendix

\apendice{Plan de Proyecto Software}

\section{Introducción}

En el presente apartado de los anexos se analizará la gestión del proyecto
software desarrollado. Este proyecto será organizado mediante la metodología
Scrum en la que el trabajo estará dividido en Sprints. Por cada Sprint se
realiza una reunión para la revisión del avance y los objetivos para el
siguiente.  Con esta metodología se mantendrá en todo momento lo que se conoce
como \textit{Product Backlog} que es una lista de las tareas a realizar. Esta
lista será actualizada en cada reunión para mantener un desarrollo constante.

Las reuniones, en un principio, se realizan cada dos semanas, intensificando a
cada semana en el momento del inicio del periodo temporal del segundo
cuatrimestre.

Finalmente también se incluirá un análisis de la viabilidad económica y legal
(ámbito de licencias software).

El objetivo de este plan es servir como herramienta para registrar el avance del
proyecto y también para poder cumplir con el objetivo final del desarrollo.

\section{Planificación temporal}
La planificación temporal se comenzó mediante Sprints de dos semanas. En la
presente sección se comentará el desarrollo realizado en cada uno de ellos.

\subsection{Sprint 0}

Desde el punto de vista temporal, corresponde desde el inicio del curso del
primer cuatrimestre académico (septiembre) hasta el Sprint 1. El día 15 de
septiembre se tuvo la primera reunión con los tutores sobre el trabajo presente
donde se establecieron las líneas generales y temática sobre el mismo.

Se creó el repositorio del TFG en Github:
\url{https://github.com/dma1004/TFG-SemiSupervisado} y se añadió la plantilla de
la documentación.

\subsection{Sprint 1}

Corresponde con el periodo temporal del 5 al 19 de octubre de 2022. 

El mismo día 5 tuvo lugar una reunión de seguimiento del trabajo. Durante el sprint se
realizaron unos arreglos de la plantilla y una lectura de conceptos teóricos
para posteriormente añadirlos a la documentación. Concretamente se crearon las
tareas "<Añadir conceptos teóricos aprendizaje"> y "<Trabajos relacionados"> a
día 9 de octubre.

\subsection{Sprint 2}

Corresponde con el periodo temporal del 19 de octubre al 2 de noviembre de 2022. 

Durante el sprint se implementó un prototipo del algoritmo Self-Training en el
que posteriormente se hicieron unas correcciones en el código. También se
comenzó con la redacción de conceptos teóricos, concretamente, sobre el
aprendizaje automático.

\subsection{Sprint 3}

Corresponde con el periodo temporal del 16 al 30 de noviembre de 2022.

Durante el sprint se aumentaron los conceptos teóricos sobre el aprendizaje
supervisado, no supervisado y semi-supervisado. Se refactorizó el prototipo para
su documentación (PEP), evitar datos duplicados y modularizar el código.

La memoria fue parcialmente modificada basándose en las correcciones propuestas
de los tutores.

\subsection{Sprint 4}

Corresponde con el periodo temporal del 25 de enero al 1 de febrero de 2023. En
este momento las duraciones de los Sprints cambiaron a una semana, iniciando así
el periodo temporal real del desarrollo del proyecto (segundo cuatrimestre).

Durante el sprint se retomaron las tareas y el desarrollo general del proyecto.
Se mejoró el algoritmo de \texttt{SelfTraining} que estaba como prototipo y se
avanzó en la tarea de la primera aproximación en la aplicación mediante el
\textit{framework} Flask. Sobre esto último, se creó una visualización del
proceso de entrenamiento muy básica por cada iteración.

Se creó un prototipo del algoritmo \texttt{CoTraining} sin cumplir con todas sus
condiciones que posteriormente se completaron a falta de revisión. 

Sobre estos dos algoritmos se propuso la versión 1.0. 

Continuando con la Web, se realizó la interfaz general funcional. Incluye:
\begin{itemize}
    \item Página de Inicio donde seleccionar el algoritmo.
    \item Página de subida de archivos en formatos <<ARFF>> y <<CSV>> de los
    conjuntos de datos
    \item Páginas correspondientes para SelfTraining y CoTraining: Cada una
    tiene sus parámetros específicos con la posibilidad de seleccionar si
    utilizar PCA (\textit{Principal Component Analysis}) o dos componentes que
    elija el usuario.
    \item Página de visualización del algoritmo (su entrenamiento): Se tiene la
    vista principal que será común a todos los algoritmos (con algunas
    variaciones en caso necesario), con la posibilidad de avanzar en la
    visualización (con controles) y barra de progreso. Desde el punto de vista
    del gráfico los colores están automatizados dependiendo del número de
    clases, leyenda y etiqueta de ejes.
\end{itemize}

En Flask se añadieron los <<endpoints>> correspondientes (subida, configuración,
visualización...) y un control de acceso a las páginas muy básico (por ejemplo,
si no se configuró el algoritmo, no se puede visualizar y le redirecciona a la
configuración con un mensaje de error)

\subsection{Sprint 5}
Corresponde con el periodo temporal del 1 al 8 de febrero de 2023.

En la reunión del 1 de febrero se revisó lo realizado en el anterior y se fijaron
una serie de mejoras/modificaciones y nuevas tareas:
\begin{enumerate}
    \item Modificación de los algoritmos para trabajar con la convención de
    <<-1s>> en el conjunto de datos para los datos no etiquetados. Así el
    usuario podrá subir un archivo ya \textit{Semi-Supervisado}.
    \item Permitir al usuario seleccionar los porcentajes de no etiquetados y de
    test (para las futuras estadísticas).
    \item Sobre la página general de la visualización de los algoritmos: volver
    a la configuración, el <<feedback>> de la iteración actual y el nombre del
    conjunto de datos utilizado.
    \item Del gráfico de la visualización: Diferenciar en el algoritmo
    Co-Training cuál de los dos clasificadores han etiquetado cada punto y los
    puntos <<etiquetados>> en la iteración 0 deben mostrarse de forma diferente.
    \item Avanzar con los trabajos relacionados.
    \item Avanzar con la documentación teórica y anexos.
\end{enumerate}

El punto 1 ha llevado unas 12 horas de compresión y desarrollo. Esto es debido a
que los dos algoritmos implementados hasta ahora debían ser modificados para
trabajar con la nueva convención. Además, el problema principal fue (aunque no
implementado en este Sprint) dejar preparada una forma de carga del conjunto de
datos que permita tratar datos no etiquetados (los <<?>> en el caso de ARFF),
pues además de los algoritmos (su correcto funcionamiento), se han probado con
ficheros. También conllevó la creación de un codificador de etiquetas propio
para ignorar los no etiquetados en clases categóricas (y no realizar la
conversión en esos casos).

El punto 2 volvió a causar bastantes problemas tanto en la ejecución de los
algoritmos como en la Web. Hasta el momento, el usuario no seleccionaba los
porcentajes de las divisiones, pero al incluir esto, los algoritmos ya no se
encargan de esta tarea y había que modificar tanto los algoritmos como aquellas
rutas de la Web que debían encargarse de esto. Aproximadamente 4 horas.

El punto 3 no resultó demasiado difícil más allá de seguir habituándose a
JavaScript/HTML. Unas 3 horas.

El punto 4 requirió unas 10 horas, en un principio se perdió mucho tiempo
intentando solucionarlo de una forma que resultó inútil, pero finalmente ahora
en el algoritmo se diferencian los datos clasificados por cada uno.

Los trabajos relacionados (no terminados) se realizaron en varios días con un
tiempo aproximado de 6 horas.

\subsection{Sprint 6}
Corresponde con el periodo temporal del 8 al 15 de febrero de 2023.

En la reunión del 8 de febrero se revisó lo realizado en el anterior y se
comentaron algunas tareas a realizar:
\begin{enumerate}
    \item En la línea del anterior, los algoritmos deben poder ejecutarse
    directamente con conjuntos de datos semi-supervisados.
    \item Permitir al usuario introducir ese tipo de conjuntos de datos.
    \item Realizar alguna visualización de estadísticas.
    \item Valor por defecto en las configuraciones.
    \item Sobre el gráfico: mejorar la diferenciación de los puntos, información
    útil en los <<tooltips>> y colocación de la leyenda.
    \item Avanzar con los trabajos relacionados.
    \item Avanzar con la memoria y anexos.
\end{enumerate}

Los puntos 1 y 2 estaban muy avanzados gracias al trabajo adicional del sprint
anterior, ya que estaba prácticamente implementada la forma en la que detectar
datos no etiquetados de forma automática. Unas 5 horas para terminar de
implementar, corregir errores sobre la marcha y realizar alguna prueba
confeccionando ficheros semi-supervisados.

Al realizar las pruebas anteriores se encontró un error en la visualización
provocando que los datos que los datos no se habían clasificado ni siquiera eran
retornados a la Web. Entre descubrir cómo corregirlo y sus modificaciones se
tardó unas 3 horas.

El punto 3 fue el más complicado, pese a que era una idea sencilla, se optó por
visualizar la gráfica de la evolución de la precisión. Cada punto del gráfico
está unido por una serie de líneas. Este tipo de gráficos (según la
documentación) se suelen hacer mediante <<paths>> o caminos, que son una única
línea, pero como en este caso era necesario no visualizar todo, sino por cada
iteración, no se encontró una solución rápida. Unas 6 horas para probar muchas
posibilidades hasta encontrar la que funcionó, acoplarla a los controles del
paso de iteración e incluir alguna animación.

Adicionalmente se retocó por completo toda la Web mediante los estilos de
\texttt{Bootstrap} para establecer ya una base vistosa y bonita. Unas 5 horas
(la mayor parte del tiempo para probar y adquirir algo de soltura con estos
estilos).

\subsection{Sprint 7}
Corresponde con el periodo temporal del 15 al 22 de febrero de 2023.

Puntos a desarrollar:
\begin{enumerate}
    \item Implementación Democratic Co-Learning.
    \item Profiling (tiempos de ejecución).
    \item Estadísticas en la aplicación.
    \item Test de las implementaciones.
    \item Avanzar con la memoria y anexos.
\end{enumerate}

La implementación del algoritmo Democratic Co-Learning supuso unas 14 horas
divididas en varios días. Al principio se dedicó un tiempo para leer el artículo
en el que se presentaba su implementación en forma de pseudocódigo junto con sus
explicaciones teóricas. La realidad es que en primera instancia parecía algo
fácil de realizar y entender, pero una vez comenzada la implementación, se
encontraban muchas cuestiones a la hora de resolverlo. 

Además, pese a que en el artículo estaba bien explicado, el formato de
pseudocódigo (en el archivo encontrado) tenía indentaciones incorrectas y se
perdió mucho tiempo comprobando si era una interpretación errónea o si realmente
era un fallo.

Se realizaron algunas pruebas de rendimiento para comprobar si los algoritmos
tardaban demasiado con conjuntos de datos muy grandes (5\,000 instancias). Se
observó que, dada la configuración que se tenía, tardaba alrededor de 40-50
segundos en terminar la ejecución. Es por esto que para este Sprint se añadió la
tarea de hacer un pequeño estudio dedicado a medir los tiempos de ejecución para
ver qué se podía optimizar. 

Este proceso fue de unas 2 horas y el resultado fue que el código implementado
no afectaba mucho, eran los propios algoritmos de entrenamiento de los
clasificadores de Scikit-Learn los que tardaban tanto. Por ejemplo, para un
estimador gaussiano el tiempo se reducía drásticamente.

Para el caso de las estadísticas, se modificaron un poco las plantillas y la
generación de sus gráficas para incluir más y revisarlas en la reunión. Unas 2
horas.

Los tests son una parte importante para validar que el comportamiento que se
espera de la implementación sea el correcto. Se realizaron unos casos de pruebas
sobre las utilidades que se usan a lo largo de todo el proyecto con la intención
de encontrar errores (todo esto sin ver cuál es el resultado y replicarlo en los
casos, sino realizar los casos basándose en lo que se espera de esas
utilidades). Se tardó unas 4 horas en realizar todos los tests.

\subsection{Sprint 8}
Corresponde con el periodo temporal del 22 de febrero al 1 de marzo de 2023.
Además, aprovechando la herramienta Zenhub, se modificó la duración de los
Sprints también en ella para poder extraer los gráficos del trabajo realizado.

Puntos a desarrollar:
\begin{enumerate}
    \item Intervalo de confianza en Democratic Co-Learning.
    \item Control reetiquetado en Democratic Co-Learning.
    \item Correcciones sobre memoria y anexos.
    \item Gráfico de estadísticas unificado.
    \item Internacionalización Web.
    \item Visualización principal de Democratic Co-Learning en la aplicación.
\end{enumerate}

El primer punto fue muy sencillo, se proporcionó la implementación de los
intervalos de confianza tanto de Álvar Arnaiz González como de César Ignacio
García Osorio (tutores) y finalmente se implementó esta segunda.

El control del reetiquetado fue mal estimado (en puntos de historia), al
principio parecía una idea sencilla, pero por un error de pensamiento, la
implementación que se realizó en un principio no funcionaba. Como cada
clasificador tiene su propio conjunto de entrenamiento, se estaba tomando que el
índice de la instancia sumado a la longitud de su conjunto de entrenamiento era
la posición en la que actualizar la etiqueta, obviamente esto no funciona pues
esa suma puede superar la longitud del propio conjunto. Había que almacenar la
posición concreta sin realizar esos cálculos. Se optó por un diccionario de
identificadores para cada clasificador en el que el valor es la posición en las
etiquetas del conjunto del clasificador.

Las correcciones de la documentación se incorporaron según las indicaciones de
Álvar Arnaiz.

El gráfico de estadísticas fue unificado, permitiendo seleccionar al usuario
mediante unos <<checkboxes>>. Aproximadamente unas 4 horas para generar el
formulario correspondiente que controle la aparición de cada línea y controlar
los eventos de las iteraciones (por ejemplo, añadir siguiente punto solo si está
activado su \textit{check}).

En cuanto a la Internacionalización, al principio resultó sencillo, ya que Babel
(Flask-Babel) detecta automáticamente las cadenas de texto dentro de
\texttt{gettext}. Sin embargo, al indicar las traducciones en JavaScript, el
intérprete tomaba \texttt{gettext} como una función que al no estar definida,
lanzaba error. Al final se generó una función en la plantilla principal de tal
forma que actúe como \texttt{gettext}, llamada desde los distintos scripts.

La visualización de Democratic Co-Learning no dio tiempo a implementarse en este
Sprint.

Además, a partir de este Sprint se incorporan todas las tareas en Zenhub para la
generación de los gráficos Burndown (ver
Imagen~\ref{fig:anexos/planproyecto/sprint8}).

\imagen{anexos/planproyecto/sprint8}{Burndown chart del sprint 8.}


\subsection{Sprint 9}
Corresponde con el periodo temporal del 1 al 8 de marzo de 2023.

Puntos a desarrollar:
\begin{enumerate}
    \item Visualización principal de Democratic Co-Learning en la aplicación.
    \item Formulario de los parámetros de los clasificadores.
    \item Estadísticas generales en Democratic Co-Learning.
    \item Estadísticas específicas en Democratic Co-Learning
    \item Condición de parada re-etiquetado
\end{enumerate}

La visualización principal fue relativamente sencilla pues en realidad es muy
similar a Co-Training. Hubo algún ajuste adicional para generar la información
del \textit{tooltip} al pasar por encima de un punto. Como en cada posición
podía haber varios clasificadores había que mostrar esa información.

Para el formulario de los parámetros se consideró el uso de Flask-WTF que al
final se descartó. Se tenía como punto de partida utilizar un JSON del que leer
los parámetros, así que lo primero fue pensar una forma sencilla de codificarlos
con la información necesaria de las entradas (<<type>>,<<step>>...). 

La ventaja de JSON es que aparte de que la lectura es muy sencilla, son
diccionarios, muy fáciles de utilizar (tanto desde Python como desde las propias
plantillas/JavaScript). Para generar el formulario se pensó en hacerlo de la
forma más automática posible para así solo tener que cambiar el JSON en el
futuro. Esto se hizo con un método de JavaScript que para cada clasificador
genera el formulario correspondiente con los parámetros del JSON. El tiempo
total fue unas 6 horas, pues también se perdió tiempo debido a que no era
posible seleccionar elementos del DOM (pues no estaba cargado) y los elementos
debían seleccionarse mediante CSS (algo que no se sabía por desconocimiento de
JavaScript).

Se añadieron las estadísticas generales de Democratic Co-Learning de forma muy
sencilla pues era exactamente igual que Co-Training (y Self-Training) esto se
realizó añadiendo el \textit{DataFrame} en el propio algoritmo con las
estadísticas deseadas (como el resto de los algoritmos).

Había mucho código muy parecido o exactamente igual en las plantillas de los
algoritmos, así que aprovechando la refactorización, se automatizaron por
completo las estadísticas. Se pensó de tal forma que solo con las columnas de
los \textit{DataFrames} que retornan los algoritmos, la Web ya se encargue de
generar el resto. A la vez que esto (e iniciando la tarea de las estadísticas
individuales), se tuvieron en cuenta las futuras estadísticas individuales para
Democratic Co-Learning. Todas las funciones fueron transformadas para trabajar
con elementos del DOM específicos, de esta forma, al indicarle por ejemplo un
<<DIV>>, se generan las estadísticas sobre ese elemento. En total unas 8 horas. 

Para estas estadísticas individuales, aparte de las funciones generalizadas, se
creó una nueva que sobre un elemento (un <<DIV>>) se generase un selector con el
nombre de los clasificadores que intervienen en el algoritmo junto con los
contenedores (<<DIV>>) dentro de él para las estadísticas de cada clasificador.
Finalmente, se utilizan las funciones de la tarea anterior para añadir a esos
contenedores nuevos los gráficos individuales (uno por cada clasificador). Unas
4 horas.

Sobre el último punto, se comentó que una etiqueta que es re-etiquetada al mismo
valor no debe <<contar>> como mejora (o cambio en el algoritmo). Si no se
realiza así, el tiempo de ejecución aumenta demasiado.


El gráfico Burndown se visualiza en la imagen~\ref{fig:anexos/planproyecto/sprint9}.
\imagen{anexos/planproyecto/sprint9}{Burndown chart del sprint 9.}

\subsection{Sprint 10}
Corresponde con el periodo temporal del 8 de febrero al 15 de marzo de 2023.

Puntos a desarrollar:
\begin{enumerate}
    \item Comparación sslearn.
    \item Refactorización de plantillas.
    \item Refactorizacion Javascript.
    \item Refactorización Flask (app).
    \item Añadir zoom a los gráficos.
    \item Conceptos teóricos.
\end{enumerate}

En este Sprint no se contaba con demasiado tiempo así que aunque sí se realizó
alguna tarea para añadir funciones, la idea era dedicarlo a mejorar el código y
mantenimiento.

En la reunión del final del Sprint anterior se comentó el cómo se estaban
validando los algoritmos y hasta ese momento solo se habían probado las
utilidades. Se sugirió compararlo contra \textit{sslearn}, una biblioteca de
José Luis Garrido-Labrador~\cite{jose_luis_garrido_labrador_2023_7781117}. Como
primera aproximación, se realizó una validación cruzada completamente manual en
la que se ejecutaba al mismo tiempo las dos implementaciones de los distintos
algoritmos (de momento sin extraer conclusiones). Se tardó unas 3 horas.

El segundo y tercer punto se iniciaron por separado, pero llegó un momento en el
que las funciones que se había creado en JavaScript, si se modificaban un poco,
podrían simplificarse las plantillas. 

Todos los métodos de los gráficos estaban separados por cada uno de los
algoritmos, esto lo hacía muy engorroso porque incluso algún método tenía el
mismo nombre. Se modificaron las funciones de tal forma que fuesen específicas
para cada algoritmo para así juntarlas en un único Script. Por parte de las
plantillas, como había partes repetidas se crearon macros y se identificó una
parte común a todos los algoritmos en su configuración, esto se añadió a la base
de las configuraciones. Todo esto llevó unas 4 horas.

En \texttt{Flask} se tenían varios problemas. El primero era que las
visualizaciones de cada algoritmo tenían un \textit{endpoint} particular, pero
esto no era necesario si se hacía un método para la obtención de los parámetros
de cada algoritmo por separado (y finalmente un solo <<endpoint>> indicándole el
algoritmo en la propia ruta). 

Cuando se crearon esos nuevos métodos se generó código muy parecido porque todos
ellos tenían dos partes: una en la que se obtenían los parámetros que no eran de
los clasificadores base y otra en la que se incorporaban esos parámetros de los
clasificadores. La primera parte es particular para cada algoritmo, pero la
segunda es común a todos. Se creó otro método para ese paso común. 

Por último, por cada algoritmo se tiene un \textit{endpoint} para la ejecución y
obtención de la información de entrenamiento, pero había una parte que todos
hacían prácticamente igual. Se creó un método que engloba: carga de datos,
separación de los datos para entrenamiento, entrenamiento y obtención de los
algoritmos y la aplicación de PCA (o no).

La visualización principal de los algoritmos tenía el problema de que cuando los
puntos estaban demasiado cerca, no se llegan a apreciar individualmente. Esto se
solucionaría aplicando \textit{zoom} al gráfico. Pese a que en cuanto a código
no fuese un desarrollo largo, fue una tarea compleja. Se probaron unas tres
implementaciones parecidas a ejemplos encontrados en la documentación, pero en
todas ellas se perdía la ayuda contextual del \textit{tooltip}. Al final se optó
por volver a empezar de cero con el conocimiento adquirido y junto con un último
ejemplo\footnote{Ejemplo D3:
\url{https://observablehq.com/@d3/zoom-with-tooltip}} y ciertas modificaciones,
se consiguió. 

Posteriormente se añadió el reinicio del \textit{zoom} para volver a la posición
original. El proceso duró unas 5 horas pues cada intento parecía ser definitivo,
pero al final siempre había ciertos límites.

Al final del Sprint se añadieron los conceptos teóricos de Democratic
Co-Learning (conceptos y pseudocódigos).

Aparte de estas tareas se realizaron pequeñas modificaciones: se completó el
estilo adaptable (<<responsive>>), se arreglaron pequeños bugs de
visualizaciones y ayuda contextual, actualización de traducciones y se añadió un
fichero de prueba que el usuario puede descargar si solo quiere probar la
aplicación.

El gráfico Burndown se visualiza en la imagen~\ref{fig:anexos/planproyecto/sprint10}.
\imagen{anexos/planproyecto/sprint10}{Burndown chart del sprint 10.}

\subsection{Sprint 11}
Corresponde con el periodo temporal del 15 al 22 de marzo de 2023.

Puntos a desarrollar:
\begin{enumerate}
    \item Arreglos en Democratic Co-Learning.
    \item Controlar los límites en los parámetros de los clasificadores.
    \item Modificar las estadísticas generales.
    \item Invertir controles en las estadísticas específicas.
    \item Anexos: Manual del programador.
    \item Modificar validación cruzada.
    \item Comparación exhaustiva contra sslearn.
    \item Añadir validación de los algoritmos a la memoria.
\end{enumerate}

En Democratic Co-Learning se estaba obviando el caso en el que si una instancia
ya estaba etiquetada y esta es reetiquetada, pero además cambiando de etiqueta,
no se consideraba como cambio en el algoritmo. Se ha añadido esta casuística.
Además, Álvar sugirió en el pseudocódigo de la memoria hacer el método de
combinación de hipótesis (predicción) para una sola instancia. Esto se ha
realizado así en la propia implementación (y así el método \texttt{predict} se
encarga de iterar sobre un conjunto de instancias).

Al JSON que codifica los parámetros de los clasificadores base se añadieron los
controles de mínimo y máximo. Esto es porque hay algunos de sus parámetros que
requieren rangos específicos. Ahora la web (JavaScript) genera el formulario de
configuración teniendo en cuenta estos límites.

En la reunión del Sprint anterior se sugirió modificar la visualización de
estadísticas. Para el gráfico general (de estadísticas) no tenía sentido poder
ocultar o no cada una de las estadísticas así que ahora siempre se muestran
todas ellas. Particularmente para Democratic Co-Learning, las estadísticas
individuales estaban manejadas mediante un selector (para seleccionar el
clasificador base) y unos \texttt{checkboxes} para seleccionar las estadísticas
a mostrar. Pero tiene mucho más sentido que el selector sea para las
estadísticas. De esta forma el usuario selecciona una estadística y en el
gráfico puede comparar esa estadística para todos los clasificadores. Además,
los \texttt{checkboxes} se mantienen, pero ahora sirven para elegir qué
clasificadores comparar. Esto llevó unas 6 horas. La organización de las
funciones estaba altamente centrada en la versión anterior, fue un proceso
complicado y lioso.

Pese a que en un principio no se comentó, una vez que se terminó el punto
anterior, se vio necesaria una reestructuración de la página de las
visualizaciones. Se mejoraron las leyendas de tal forma que ya no estaban dentro
de los SVGs, ahora están en su propio cuadro. Esto ha conseguido no preocuparse
por el tamaño de las palabras, centrarla y poder organizar mejor las columnas en
las que se divide esa plantilla.

Sobre el manual del programador, se añadió la estructura de directorios con el
paquete \texttt{dirtree} de \LaTeX{} y se completó el manual del programador,
centrado en qué es lo que debe saber un desarrollador para continuar con el
proyecto. Finalmente se describió el proceso de la compilación, instalación y
ejecución del proyecto. En total fueron unas 6 horas.

El proceso de validación cruzada no estaba bien enfocado, se estaba realizando
de forma manual (y todos los posibles fallos que puede suponer). La librería
scikit-learn incorpora ya utilidades para este proceso. Concretamente se tiene
un método que genera los distintos \textit{Folds} dado un conjunto de datos. El
proceso manual se sustituyó por este nuevo. Se aprovechó para guardar los
resultados como CSV.

Con el proceso de validación cruzada la idea era obtener métricas para
compararlas en alguna gráfica/tabla y comprobar que las implementaciones son
correctas (comparadas con \texttt{sslearn}). Para ello se creó una función que
recogía la información de los CSVs y dibujaba una malla con distintos gráficos.
En las columnas se distribuyen los algoritmos, separadas en dos para la
implementación propia y la de \texttt{sslearn}. En las filas se tiene cada
estadística. Cada uno de los gráficos de esta malla es un gráfico de cajas (para
mostrar mínimos, máximos, medias, medianas...).

Sobre el último punto, aunque la idea era realizarlo en este Sprint, no se
estimó bien la cantidad de trabajo y no pudo ser realizado. Además, en las
ejecuciones del punto anterior (la comparativa) se vio que el algoritmo
Co-Training tenía peor rendimiento que el de \texttt{sslearn}. Esto impedía
justificar en la memoria la validación de los algoritmos.

Durante estas tareas fueron surgiendo pequeños arreglos: Se colorearon las
etiquetas de la ayuda contextual (\texttt{tooltip}) para que quedara claro qué
etiqueta lleva cada punto (no solo el nombre) y se actualizaron las
traducciones.

El gráfico Burndown se visualiza en la imagen~\ref{fig:anexos/planproyecto/sprint11}.
\imagen{anexos/planproyecto/sprint11}{Burndown chart del sprint 11.}


\subsection{Sprint 12}
Corresponde con el periodo temporal del 22 al 29 de marzo de 2023.

Puntos a desarrollar:
\begin{enumerate}
    \item Completar manual del programador.
    \item Introducción memoria.
    \item Añadir bibliotecas Web.
    \item Añadir explicaciones de los algoritmos en la Web.
    \item Añadir los pseudocódigos en la Web.
    \item Algoritmo Tri-Training.
\end{enumerate}

En el Sprint anterior se empezó el manual del programador, pero quedó pendiente
la compilación, instalación y ejecución del proyecto. Lo primero que se hizo en
este Sprint es finalizar este manual describiendo todos los pasos que un
programador debe seguir para poner en funcionamiento la aplicación.

Para continuar con la memoria, se añadió la sección de la \texttt{Introducción}
como presentación de todo este trabajo.

Las dos principales bibliotecas o recursos utilizados en la Web son
\texttt{Bootstrap} y \texttt{D3.js}, el primer caso es un <<framework>> de CSS
que ya viene con una gran cantidad de clases a utilizar. D3 es la biblioteca de
JavaScript que se ha utilizado para la generación de todos los gráficos. Se
añadió una descripción (y justificación) de uso en la memoria (Técnicas y
herramientas).

Teniendo en cuanta la componente docente de este proyecto, se creyó conveniente
la adición de unas pequeñas explicaciones de los algoritmos. Se añadieron en la
fase de configuración de los algoritmos dado que al mismo tiempo el usuario debe
configurar los parámetros y puede que le permita tener una mayor intuición a la
hora de configurarlo.

En esta misma línea se añadieron los pseudocódigos en forma de imágenes, justo
debajo de las explicaciones. También se creyó conveniente que el usuario pudiera
verlo en la fase de visualización. Se creó un desplegable con la imagen del
pseudocódigo.

Finalmente, y como desarrollo más grande de este Sprint, se creó el algoritmo
Tri-Training. La primera versión (el primer \texttt{commit}) no fue validada de
ninguna forma (se confirmó según se terminó). Esta versión no incluía ninguna
forma de obtener el proceso de entrenamiento, simplemente era el esqueleto
funcional del algoritmo. Obviamente, era claro que iba a haber algún que otro
error. Al principio solo se detectó un error en el que el cálculo del error de
clasificación siempre resultaba en 0. Era una cuestión de pura implementación en
Python así que no fue difícil arreglarlo.

Antes de arreglar este error, la rama en la que se ejecutaba el <<subsample>> no
se estaba realizando en ningún caso. En el momento que se arregló el anterior
error se accedió a esta parte y apareció un nuevo error. Se confundió array de
NumPy con lista nativa de Python y se estaban seleccionando ciertas posiciones
de una lista de Python accediendo a ella con unos índices
(\texttt{List{[}indices{]}}). La solución fue sencilla de igual manera, se
recorría la lista en base la los índices y se creaba esta sublista mediante
compresión de listas.

Al final de este desarrollo, se comparó con una sola ejecución con sslearn. El
nuevo algoritmo tardaba mucho más tiempo que el resto (y que sslearn). Esto era
porque al añadir las predicciones (cuando las otras dos hipótesis coincidían) se
hacía de una en una. Esto se sustituyó para predecir todas las instancias no
etiquetadas y con vectorización. El rendimiento se incrementó mucho.

Durante estas tareas fueron surgiendo otras más pequeñas: Nuevas traducciones,
correcciones de memoria y anexos y una pequeña comparación contra sslearn.

La herramienta Zenhub dejó de proporcionar servicios gratuitos. Esta compañía
actualizó sus planes gratuitos, ya no existe una versión de esas
características. Durante este Sprint <<caducó>> la licencia que se tenía y ya no
se podía acceder a ninguna de sus funcionalidades, tampoco a los gráficos
<<Burndown>>.

\subsection{Sprint 13}
Corresponde con el periodo temporal del 29 marzo al 5 de abril de 2023.

Puntos a desarrollar:
\begin{enumerate}
    \item Configuración Tri-Training
    \item Visualizar Tri-Training.
    \item Eliminar todo el código duplicado restante de JavaScript.
    \item Documentar JavaScript.
    \item Reestructurar Flask para aplicaciones grandes.
    \item Añadir media geométrica en la validación/comparación contra sslearn.
\end{enumerate}

Para este Sprint el objetivo principal era dejar lista la visualización de
Tri-Training en la Web. 

El primer paso fue incluir las estructuras de datos necesarias en el esqueleto
del algoritmo para ir almacenando la información de entrenamiento. La idea era
igual a Democratic Co-Learning. La realidad es que la forma en la que se
almacenan los datos es muy distinta a todas las demás. Los datos pueden ser
clasificador varias veces por cada clasificador en diferentes iteraciones. Por
lo tanto, para cada clasificador base se almacena una lista para las etiquetas y
para las iteraciones en las que se clasifican. Las estadísticas sí que se
almacenan exactamente de la misma forma que Democratic Co-Learning.

El siguiente paso fue crear la pantalla de configuración de Tri-Training, para
ello simplemente se copiaron las plantillas ya existentes y la misma estructura
que ya se tenía en Flask para las rutas.

Y finalmente, la pantalla de visualización de Tri-Training. Este fue uno de los
pasos más complejos. Como se tenía una nueva estructura de los datos, se tenían
que crear los métodos que permitiesen interpretarlos, de forma general, los que
se ha realizado es comprobar, en cada iteración, qué clasificador han añadido
datos etiquetados nuevos junto con su etiqueta. Para que el usuario pudiera ver
los cambios que ocurren en cada iteración, cuando pulsa en la siguiente
iteración se descoloran los puntos (a gris) y después se colorean los nuevos,
todo esto mediante animaciones y con un cierto tiempo para que pueda darse
cuenta.

En cuanto a JavaScript, se documentaron por completo todos los métodos que
fueron creados para este proyecto. Además, se eliminó el código duplicado en
cada fichero utilizando métodos comunes. También fue un proceso largo, sobre
todo para comprobar que las modificaciones eran correctas y generales, en
principio, todo parece correcto.

Se reestructuró completamente Flask, esto es porque tal y como se encontraba la
aplicación, era muy básica y general. Los proyectos más grandes con Flask tienen
una estructura más o menos concreta. Esta es la idea que se ha intentado
aplicar, utilizando <<Application Factory>> y <<Blueprints>>. Pero sobre todo,
para hacerla más mantenible y extensible. De hecho, aunque no está siendo usada,
se tiene una pequeña base de datos para posibles adiciones futuras. 

Esta reestructuración causó muchos problemas en cuanto a las rutas de ficheros.
Se tuvieron que especificar manualmente y usar la librería del sistema operativo
de Python para independizar ciertos tratamientos de ficheros del sistema
operativo concreto donde se ejecute la aplicación (se ha probado en Windows y
Linux).

Finalmente, se añadió la media geométrica como métrica de validación a la hora
de comparar contra
\texttt{sslearn}~\cite{jose_luis_garrido_labrador_2023_7781117}. Además, en
Sprints anteriores se comentó la idea de utilizar \textit{Violinplots}, que
resultan más pertinentes para los casos de comparación.

\subsection{Sprint 14}
Corresponde con el periodo temporal del 5 al 12 de abril de 2023.

Puntos a desarrollar:
\begin{enumerate}
    \item Actualizar manual del programador
    \item Rediseño Web completo
    \item Ayuda contextual y errores en Web.
    \item Anexos: Diseño
\end{enumerate}

Debido a las modificaciones realizadas en los recientes Sprints, el manual del
programador quedó desactualizado. El cambio más notable fue la nueva
reestructuración de Flask. Se actualizó el árbol de directorios reflejando esta
nueva estructura. También se actualizaron todas las explicaciones posteriores
del manual.

El diseño de la Web estaba hecho prácticamente desde el principio de la Web y
resultaba muy poco llamativo. El primer paso para este rediseño fue elegir una
paleta de colores, se creyeron convenientes fondos oscuros con alguno más claro
para dar contraste.

Para continuar, se pensó en la disposición y estilo general del contenido. Hasta
el momento la página resultaba muy plana, sin cambios entre las distintas partes
de la página. Consultando páginas Web de ejemplo parecía interesante y bonito
crear efectos de sombra y se estableció como estilo general de la página. A
partir de aquí las distintas secciones de una misma página se encuadran en un
elemento con sombra, para llamar la atención del usuario. Además, para comprobar
que no eran ideas descabelladas, se consultó a compañeros y personas externas
para buscar opinión. Gracias a esto se realizaron retoques (cambio de algún
color, títulos, disposiciones...) que no convencía a la mayoría.


Para proporcionar ayuda básica en las distintas páginas, se creó una macro que
genera una \texttt{tooltip} con un mensaje. Tanto en la subida como en la
configuración se utilizaron estos mensajes para aclarar aspectos confusos. Como
adición, resultaba muy molesto tener que volver a subir un fichero
constantemente al seleccionar otro algoritmo durante la misma sesión. Se añadió
una comprobación en la pestaña de subida en la que si ya se había subido un
fichero, permitiera ir directamente a la configuración.

Durante todo el desarrollo hasta este momento, ocurrían ciertos errores HTTP
(404, por ejemplo) pero no eran controlados. Se creó una plantilla base y se le
indicó a Flask los errores que debía manejar y renderizar en esa plantilla. Se
han añadido los más comunes (y además muchos de ellos no han llegado a ocurrir
nunca).

En reuniones anteriores se comentó que debía explicarse las estructuras de los
ficheros JSON que maneja e interpreta la Web (JavaScript). Se añadieron todas
las explicaciones relativas a la generación de estos ficheros desde la propia
ejecución de cada algoritmo. Se creó también un diagrama de secuencia con la
interacción general con la Web para la visualización de un algoritmo.

\subsection{Sprint 15}
Corresponde con el periodo temporal del 12 al 19 de abril de 2023.

Durante este Sprint tuvo lugar el periodo de exámenes parciales de las
asignaturas cuatrimestrales, y junto con las prácticas en empresa, no se pudo
mantener la prioridad al desarrollo del proyecto. Por lo tanto, no existen
puntos concretos a desarrollar, pero la idea era realizar, en la medida de lo
posible, retoques en el mismo.

Los \texttt{violinplots} que se añadieron para la comparativa de algoritmos
resultaban algo confusos. Aunque este tipo de gráficos se utiliza mucho para
comparar datos, para este ámbito concreto no resultó ser la mejor opción. La
parte superior de estos gráficos, que no son el máximo de los datos, sobrepasa
en algunos casos el valor máximo de las estadísticas. Esto hace parecer que los
cálculos son incorrectos. Se volvió a incluir los gráficos de caja anteriores.

Como no resultaba una tarea grande, se incluyó la integración continua a partir
de este momento. La herramienta SonarCloud proporciona muchas métricas de
calidad del código (bugs, línea duplicadas, seguridad...). Gracias a estos
análisis se realizaron todas las posibles modificaciones para reducir las
alertas que hasta el momento podían abordarse. Por ejemplo, la protección contra
CSRF (\texttt{Cross-Site Request Forgery}) no se había contemplado, pero la
herramienta sí lo considera como un elemento prioritario.

\subsection{Sprint 16}
Corresponde con el periodo temporal del 19 al 26 de abril de 2023.

Puntos a desarrollar:
\begin{enumerate}
    \item Añadir usuarios (con login y registro)
    \item Crear un espacio personal para los usuarios
    \item Continuar con la memoria
    \item Estandarización en la configuración
\end{enumerate}

Durante este Sprint se iniciaron las tareas para mantener usuarios en la
aplicación. El primer paso para realizar esto fue la creación del modelo de
usuario en la base de datos. El usuario es simple, con un identificador, nombre,
email y contraseña (este no es el objetivo principal de la aplicación).

A continuación, se crearon los pares de \texttt{endpoint} y plantilla para
inicio de sesión y registro. Como este tipo de formulario requiere de una
validación exhaustiva y que permita dar al usuario ayuda contextual, se utilizó
Flask-WTForms para ello porque tanto desde el navegador como desde el
\texttt{backend} permite esta validación. Esto trata los formularios como
objetos, que incluso cada uno de ellos mantiene una lista de errores. En ambos
\texttt{endpoints}, se realizan las típicas comprobaciones: al registrar que no
exista ya una cuenta con el mismo email, en el inicio de sesión que la
contraseña coincida con la almacenada, que exista el email... Respecto a las
contraseñas, solo se almacena el Hash (SHA256) por lo que no se podría obtener
la cadena real de vuelta (de forma sencilla). Y como se comentaba, cuando se
detecta un error de este estilo, se muestra un mensaje (en rojo) con lo que ha
ocurrido.

Con todo ello, se modificó la barra de navegación incluyendo el menú de los
usuarios: Registrarse/Iniciar sesión y una vez el usuario ha iniciado sesión, un
menú desplegable que le llevaría a su espacio personal y perfil.

Antes de continuar con el desarrollo, se incluyó un \textit{mockup} con lo que
se tenía pensado hacer.

Para que el usuario pudiera modificar sus datos, se creó otro pequeño formulario
muy parecido al de registro. Por su puesto, con su correspondiente ruta
(\texttt{/perfil}) que valida que los datos introducidos son correctos.

Del mismo modo, era interesante que el usuario pudiera ver algo de su
actividad, la idea era guardar los conjuntos de datos que ha subido para poder
volver a utilizarlos y un historial de ejecuciones con el objetivo de replicar
exactamente esa ejecución. Como se vio que la ruta del perfil y esta nueva eran
parecidas, se unificó el estilo de tal forma que en la columna de la izquierda
apareciesen los principales datos del usuario y a la derecha, o bien la
modificación del perfil (\texttt{/perfil}) o su actividad (\texttt{/miespacio}).

Antes de seguir, para poder almacenar toda esta actividad, se crearon dos nuevas
entidades <<Dataset>> y <<Run>> que almacenarían la información principal como
los nombres de los ficheros, fecha de subida/ejecución...

La parte visual de la actividad se realizó con DataTables, un plug-in de jQuery
que proporciona mucha versatilidad para hacer tablas. Los datos no provenían de
las plantillas, se crearon nuevas rutas a las que hacer peticiones (como una
API). Una vez que se había trasteado con este plug-in e incluso vista la forma
en la que eliminar filas, la idea es realizar una petición y que los datos
recibidos sean incrustados en estas tablas. La obtención de los datos no fue un
problema, se utilizaron los <<fetch>> de JavaScript y cuando se obtenía
respuesta, se creaban las tablas. El verdadero problema fue la incorporación de
acciones (como la de borrar), en la propia documentación de DataTables y gracias
a su foro, se consiguieron generar botones en esta columna (con puro HTML). El
siguiente reto fue crear otras peticiones como respuesta a los clics de estos
botones. La suerte fue que DataTables permite obtener los datos de una fila
directamente, lo que resultó de mucha ayuda para, por ejemplo, extraer el nombre
de un fichero y posteriormente incorporarlo a las peticiones de borrado. Se
añadieron también unos \textit{modales} para las comprobaciones del usuario.

El desarrollo anterior se ha simplificado, primero se creaban versiones
sencillas y luego se iban incorporando retoques hasta el resultado final. Por
ejemplo, al final del todo, se añadió el estilo <<\textit{responsive}>> a estas
tablas. La ventaja fue que DataTables ya tenía extensiones que realizaban esto
automáticamente.

Continuando con la memoria, se añadieron los objetivos de proyecto y el comienzo
de aspectos relevantes.

A la vez que se iban incorporando estas adiciones, no se perdía de vista el
resto de la aplicación, en la fase de configuración de los algoritmos resultaba
conveniente permitir al usuario estandarizar o no el conjunto de datos a mostrar
(hasta ahora siempre se estandarizaba), se incluyó un interruptor para ello.

Además de todo esto, se hicieron algunos cambios pequeños: <<Footer>> al final
de la página, selección de fuentes para la página, arreglar algunos bugs menores
o estilar el <<card>> del perfil.

\subsection{Sprint 17}
Corresponde con el periodo temporal del 26 de abril al 3 de mayo de 2023.

Puntos a desarrollar:
\begin{enumerate}
    \item Botón de idioma.
    \item Panel de administración.
    \item Bootstrap icons.
    \item Mensajes flash (alertas).
    \item Actualizar técnicas y herramientas.
\end{enumerate}

Lo primero que se hizo fue sustituir los SVG de los iconos que se estaban
utilizando, por clases de \texttt{Bootstrap Icons}. Era muy engorroso tener cadenas tan
grandes de texto y esta modificación solo conlleva añadir un enlace.

Lo siguiente que se realizó fue añadir un botón para cambiar el idioma. La idea
es muy sencilla, las traducciones ahora se realizan detectando qué idioma se
ajusta más al navegador del usuario (en las peticiones existe un <<header>> que
incluye esta información), como también se quiere que el usuario decida, todas
las peticiones son procesadas previamente por un \texttt{endpoint} (Flask
permite esto) que detecta si existe un parámetro <<lang>> (\texttt{?lang=X}) y
si lo hay, establece el idioma a este (guardándolo en la sesión). De esta forma
se consigue no modificar lo ya existente.

El panel de administración fue un verdadero reto, como no se quería tener que
hacer todo de cero (mucho código duplicado), se tuvo que pensar primero la forma
de aprovechar las tablas que ya se habían creado. El panel de administración
solo iba a tener una tabla nueva, además de la de los ficheros y ejecuciones que
ya existían. Estas últimas son las que se debían reutilizar. Para ello, la
información que debía ver el administrador además de lo que ya se mostraba era
el usuario al que pertenecen los ficheros y ejecuciones. Se añadió esta columna
a las tablas y las funciones se parametrizaron con un <<flag>> que permitiese
especificar si la tabla debe generarse como administrador o como usuario normal.
En el caso del panel de administración, mostrar esa columna de usuarios. Esto
conllevó a modificar también los modelos pues hasta ahora solo se almacenaba una
clave foránea con el identificador del usuario (había que añadir el email) con
una relación uno a varios.

Finalmente, se creó la tabla de usuarios del mismo modo que el resto, creando
una ruta donde consultar todos los datos y añadir las acciones correspondientes
de editar el usuario o eliminarlo. Se reutilizó mucho de lo que ya se había
codificado. La ventaja es que, aunque se tuvieron que añadir bastantes cosas, ya
se tenía el conocimiento previo fue más laborioso (el desarrollo anterior se
dividió en varios días) pero menos difícil.

Además de lo comentado anteriormente, se quiso incluir unas estadísticas
sencillas en cada una de las tablas. Para los usuarios, el total y para los
ficheros subidos y ejecuciones, el total de los últimos siete días.

Finalmente se tuvo que modificar algún \texttt{endpoint} para aumentar las
comprobaciones de seguridad y sobre todo, para evitar duplicar código. Por
ejemplo, la edición de un usuario por parte de un administrador se <<delega>> al
propio \texttt{endpoint} del perfil. En esta línea, a medida que pasaba el
Sprint, mejoraban las comprobaciones y controles de errores y para que el
usuario tuviera algo de información, se añadió un modal que se activa cuando
ocurre un error.

Los mensajes flash se modificaron como alertas de Bootstrap de tal forma que el
color de las mismas dependiera de las categorías (se añadieron estas a todos los
puntos en los que se lanzaba un mensaje flash).

Se actualizó el apartado de las técnicas y herramientas hasta este punto.

Además, también se realizaron otras modificaciones:

\begin{itemize}
    \item Añadir eliminación de ejecuciones: Al igual que usuarios y ficheros,
    era interesante que pudieran eliminarse las ejecuciones.
    \item Bug de edición perfil: Si la contraseña actual se introducía correcta,
ocurría un error crítico (no se habían pasado todas las variables a la plantilla
porque el código no estaba actualizado).
    \item Bug reducción dimensionalidad: Cuando se desactivaba PCA y se incluía
la misma columna en CX y CY, ocurría una excepción. Pandas parece no permitir
concatenar una columna a un DataFrame que tiene columnas idénticas.
    \item Refactorizaciones generales para evitar código duplicado.
\end{itemize}

\subsection{Sprint 18}
Corresponde con el periodo temporal del 3 al 10 de mayo de 2023.

Puntos a desarrollar:
\begin{enumerate}
    \item <<Toasts>> como alertas.
    \item Arreglar visualizaciones (\texttt{tooltips}).
    \item Cambiar formularios a WTForms.
    \item Mostrar parámetros de las ejecuciones en el historial.
    \item Conceptos teóricos Tri-Training.
    \item Conceptos teóricos HTTP.
\end{enumerate}

En primer lugar, el aspecto que se le había dado a los mensajes <<Flash>> de la
aplicación parecía un poco pobre. Bootstrap tiene los llamados <<Toasts>> que
son avisos más sofisticados y estilados. Partiendo un ejemplo de la
documentación, simplemente se filtra por la categoría del mensaje (error,
advertencia, información...) y se genera el <<Toast>> con unos colores acordes.

Durante el Sprint anterior se estuvieron probando las visualizaciones. El
problema es que los \texttt{tooltips} muestran información confusa.

En primer lugar, estos recuadros que aparecen al pasar el ratón por un punto, no
se posicionaban correctamente. Mediante un ejemplo encontrado en
Internet\footnote{Ejemplo de \texttt{tooltip}:
\url{https://observablehq.com/@clhenrick/tooltip-d3-convention}}, se vio que el
propio D3 tiene una función que determina la posición del ratón correctamente.
Además, el elemento \texttt{tooltip} tenía una posición absoluta respecto de
toda la página para arreglarlo, se le añadió la posición relativa al contenedor.
Esto se aplicó a los cuatro algoritmos.

El mayor problema era arreglar la información confusa. Como es obvio, el
conjunto de datos que introduzca el usuario puede tener puntos duplicados y
además al realizar PCA, puede ocurrir que dos datos aparezcan en la misma
posición. Esto no se estaba contemplando y no se indicaba. Es aún más confuso en
Democratic Co-Learning y Tri-Training pues cada punto puede ser clasificado por
varios clasificadores e incluso en varias iteraciones.

Para ayudar al usuario a tener toda la información, el \texttt{tooltip} muestra
ahora los puntos duplicados. Es decir, cuando solo hay un punto, simplemente
mostrará la información normal, pero cuando haya puntos solapados (quizá por
PCA o porque en el propio conjunto hay datos duplicados), los detectará y
mostrará un identificador de duplicado en orden creciente.

Para determinar esto lo que se ha hecho es comprobar (mediante diccionarios) si
un punto ya había sido visto, en cuyo caso existen puntos solapados.
 
Otra cosa añadida al \texttt{tooltip} es la iteración de clasificación. Cuando
el punto se acaba de clasificar o estamos en una iteración posterior, al lado de
la etiqueta se muestra entre paréntesis la iteración en la que se clasificó.

Al final del Sprint (y justo coincidiendo con una reunión de seguimiento), se
planteó que el \texttt{tooltip} podía ser simplificado. Por una parte, cada vez
que se muestra todos los puntos, cada uno de ellos tiene su posición X e Y.
Ahora solo se muestra como si fuera un título una única vez.

Otra idea era que no se mostrase información futura (para Democratic Co-Learning
y Tri-Training). Por ejemplo, si dos clasificadores habían etiquetado un punto
en iteraciones 4 y 5 respectivamente, hasta ahora, en la iteración 2 (por
ejemplo), el \texttt{tooltip} tenía <<reservado>> el espacio para escribir esos
dos etiquetados. El comportamiento bueno sería que en una iteración menor que 4
simplemente mostrase un único texto para ese punto (ejemplificando que todavía
no ha sido etiquetado). Cuando se llega a la iteración 4, se mostrará la
etiqueta que le dio el primer clasificador, pero sin mostrar todavía el de la
iteración 5. Y así sucesivamente.

En cuanto a los formularios, los de configuración del algoritmo estaban hechos
directamente en HTML. Dado que se había empezado a utilizar WTForms (inicio de
sesión, registro...), era conveniente modificarlos y trabajar con esta librería.
Todos los formularios tienen una parte común, lo primero que se hizo es crear
una clase de formulario base del que el resto extendería. A partir de aquí,
simplemente se trasladó todo lo que se tenía en HTML a los distintos
\texttt{Fields} de WTForms. En el HTML simplemente se hace referencia a estos
campos. Fue un proceso laborioso aunque sencillo.

Esto además permitía añadir el <<token>> CSRF para arreglar ciertas
vulnerabilidades.

En la tabla del historial de ejecuciones faltaba la posibilidad de ver los
parámetros de las ejecuciones. Para verlos, se añadió un campo de texto en la
entidad de la base de datos de ejecuciones. 

El formato JSON se pensó que era suficientemente legible como para mostrar los
parámetros con él. Para ello, se creó una función (en Python) que transforma
todos los parámetros que provienen del formulario de configuración en un
diccionario. Este diccionario luego es transformado en texto para almacenarlo en
ese campo nuevo.

Desde la Web, simplemente se lee ese campo de la ejecución y se formatea como
JSON. Existe un nuevo botón en la tabla que permite mostrar un <<modal>> con los
parámetros.

Se añadieron los conceptos teóricos de HTTP, con una descripción del protocolo,
la estructura de las peticiones, \texttt{cookies} y el ataque CSRF (Cross-Site
Request Forgery) junto a su solución (<<CSRF token>> comentado anteriomente).

En cuanto a Tri-Training, se añadió el pseudocódigo del mismo.

Además, también se realizaron otras modificaciones, las más notorias:
\begin{itemize}
    \item Añadir favicon.
    \item Control de errores.
    \item Arreglar bugs en las tablas del espacio personal.
    \item Añadir algún aspecto relevante.
\end{itemize}


\subsection{Sprint 19}
Corresponde con el periodo temporal del 10 al 17 de mayo de 2023.

Puntos a desarrollar:
\begin{enumerate}
    \item Anexo requisitos.
    \item Anexo documentación de usuario.
    \item Actualizar manual del programador.
    \item Diseño arquitectónico.
    \item Técnicas y herramientas
    \item Conceptos teóricos del tratamiento de datos.
\end{enumerate}

En este Sprint se quería aumentar bastante la documentación. El primer objetivo
era realizar el anexo de los requisitos. Lo primero fue añadir el catálogo de
requisitos que la aplicación debía realizar.

A continuación, se realizó un primer diagrama de casos de uso general. A partir
de este, se realizó la descripción de todos ellos (con sus pasos,
precondiciones, excepciones, requisitos asociados...). Sin embargo, durante la
realización de estos, se vio que el diagrama podía ser simplificado. El
administrador, al igual que los usuarios, pueden eliminar los ficheros y
ejecuciones, y por esta razón realizan el mismo caso de uso en ambos casos. 

Se actualizó el diagrama de casos de uso y se añadieron todas las descripciones
acordes de todos ellos.

El anexo de la documentación de usuario se podría realizar más adelante por si
la aplicación sufre algún cambio final. Sin embargo, estos cambios serán,
probablemente, en cuanto aspecto. Es por esto que las instrucciones útiles para
el usuario serán las mismas. Además, resultaba una tarea apetecible en esos
momentos. En este manual se explicó todo lo que un usuario podía realizar
(anónimo o registrado). Además, como también existe un rol administrador, se
describieron todos sus privilegios.

En el anexo del diseño se añadió el diseño arquitectónico que se ha planteado
para la aplicación, una arquitectura de tres capas. Se explicaron los conceptos
relativos junto a cómo se reflejaba en la aplicación global.

Se añadió PEP8 a las técnicas y \texttt{Draw.io}, \texttt{Putty} y
\texttt{Bootstrap Icons} en las herramientas.

Y por último, como otro gran apartado de este Sprint, se añadieron algunos
conceptos que se han utilizado en la aplicación sobre el tratamiento de datos.
Pese a que la aplicación no realiza un pre-procesado completo, convenía explicar
la codificación de variables categóricas, el análisis de componentes principales
(PCA) y la estandarización que se puede aplicar a los datos al visualizarlos.

Además, también se realizaron numerosos cambios:
\begin{itemize}
    \item Se eliminó el efecto <<hover>> en la barra de navegación que causaba
    muchos problemas en móviles.
    \item Los parámetros mostrados en las ejecuciones se muestran de forma
    legible (nombre apropiados, no los identificadores de los formularios) y con
    traducciones acordes.
    \item Control de errores en la visualización.
    \item Pseudocódigo Tri-Training (actualizado y añadido a Web).
    \item Fijar dependencias del proyecto.
    \item Se mejoró la comprobación de las extensiones de ficheros para que
    obligatoriamente terminen en <<.arff>> o <<.csv>>.
    \item Diagrama de despliegue en el diseño.
\end{itemize}

\subsection{Sprint 19}
Corresponde con el periodo temporal del 17 al 24 de mayo de 2023.

Puntos a desarrollar:
\begin{enumerate}
    \item Análisis de viabilidad: económica y legal.
    \item Añadir Scrum como técnica.
\end{enumerate}

El desarrollo se vio ralentizado en este Sprint debido al periodo de
convocatoria final.

Para ir completando los anexos se añadió tanto la viabilidad legal como
económica. En el caso de la económica se contaba con algún ejemplo de cómo se
debía realizar de alumnos previos (proporcionado por los tutores). Durante su
desarrollo se contemplaron todos los costes derivados (suposiciones) de los
empleados y de activos \textit{hardware} y \textit{software}. Además, se tomó la
decisión lógica de no incluir beneficios, ya que se trata de una aplicación
completamente docente y pensada para que cualquiera pueda usarla.

En cuanto a la legal, se estudiaron todas las licencias del \textit{software}
que se ha empleado para determinar cuál sería la correcta. Finalmente se escogió
la licencia BSD 3-Clause.

Como la metodología Scrum es altamente utilizada (y ya no resulta una clara
novedad), se incluyó en las técnicas una pequeña explicación de todo lo que
interviene en ella.

\subsection{Sprint 20}
Corresponde con el periodo temporal del 24 al 31 de mayo de 2023.

Puntos a desarrollar:
\begin{enumerate}
    \item Añadir la validación de los algoritmos.
    \item Terminar conceptos teóricos.
    \item Añadir la descripción de los formatos de \textit{tooltip}.
    \item Añadir pruebas del sistema.
    \item Revisión general.
\end{enumerate}

En sprint lo que planificó era tener todo el proyecto lo más cerca posible al
objetivo para que una vez se realice la reunión del sprint evaluar cuál es el
estado.

En primer lugar, algo que se estaba retrasando constantemente debido a la
prioridad de nueva funcionalidad era la validación de los algoritmos. En este
sentido sí que se tenían bastante probados, pero faltaba añadirlo a la
documentación.

Como se venía comentado en sprints anteriores, esta validación se trata de
comparar la implementación propia contra la de \emph{sslearn} mediante gráficos
de cajas. La idea es que si se obtienen resultados similares en las estadísticas
comparadas, se puede concluir que la implementación propia es \emph{correcta}.
El resultado final con la ejecución de cada algoritmo con varios conjuntos de
datos fue positivo y se concluyó que eran correctos.

Un punto a destacar de la validación fue la diferencia de implementación de
Co-Training. En líneas generales, lo que se pudo observar de \emph{sslearn} es
que el tamaño de \textit{pool} era más grande que lo que se venía realizando.
Esto se subsanó aumentando los parámetros en la llamada al algoritmo propio para
hacerlo lo más parejo posible. Aunque sigue apareciendo mayor variación que el
resto de los algoritmos, se puede comparar perfectamente y llegar a la misma
conclusión.

En cuanto a los conceptos teóricos, se añadió la descripción de Tri-Training (el
pseudocódigo ya estaba incluido). Para ello se revisó de nuevo el artículo
correspondiente para incluir la idea general más importante.

Algo que podía ser muy confuso para un usuario nuevo es la multitud de formatos
de \textit{tooltip} de la visualización. Se añadió la explicación de cada uno
para cada algoritmo en el manual de usuario.

En cuanto a los anexos (a expensas de revisar y perfeccionar), el apartado que
faltaba era el de las pruebas del sistema en el manual del programador. Por
sugerencia de los tutores, se ha utilizado Selenium como entorno de pruebas.
Además, en la asignatura de Validación y Pruebas se utilizó Katalon Recorder,
que utiliza Selenium. Esto facilitó la parte de las pruebas automáticas.

La idea de las pruebas era tener los suficientes casos de prueba que asegurasen
el funcionamiento de la aplicación. De hecho, durante algunos cambios que se
hacían, se volvían a ejecutar para comprobar que esos cambios no rompían la
aplicación. En total fueron 17 casos de prueba teniendo en cuenta las
visualizaciones, los usuarios (y sus permisos) y las acciones de estos
(incluidas las del administrador).

Durante todo el sprint se intentaba corregir todo lo que se veía como mejora
(revisión general). Entre estas correcciones están:
\begin{itemize}
    \item Arreglar el posicionamiento de los \textit{tooltips} de ayuda
    contextual (no los de la visualización) sustituyéndolos por los de bootstrap.
    \item Arreglar adaptabilidad en el panel de administración ya que las tablas
    de las pestañas dos y tres no se ajustaban (solo la primera).
    \item Las rutas de los usuarios fueron anonimizadas, en el sentido de que
    hasta ahora se incluía por ejemplo el identificador del usuario en la base
    de datos (algo como \texttt{/perfil/2}). Convenía no mostrar estos valores
    (\texttt{/perfil}).
    \item Traducciones.
    \item Justificar el por qué de la licencia BSD 3-Clause.
\end{itemize}

\subsection{Sprint 21}
Corresponde con el último periodo temporal (considerando hasta la entrega).

Puntos a desarrollar:
\begin{enumerate}
    \item Enlazar el manual en web.
    \item Logos de los algoritmos.
    \item Depurar aplicación.
    \item Completar memoria (resumen, conclusiones y líneas futuras).
    \item Añadir videotutoriales.
    \item Completar anexos.
\end{enumerate}

En principio, como sprint final, no se planteaba realizar ninguna
implementación. Quizá alguna modificación/mejora que no altere el
funcionamiento.

En primer lugar, se añadió en la barra de navegación un enlace directo al manual
de usuario de la propia documentación del proyecto. Esto conllevó la
actualización de las capturas del manual para que se apreciara el nuevo icono y
un nuevo caso de uso sencillo.

En la aplicación web faltaban los logos de los algoritmos. Para ello, se generó
un modelo de un robot en Dall-e\footnote{\url{https://openai.com/dall-e-2}}
(consejo de los tutores). A partir de aquí, se fue modificando el logo base
adecuándolo al contexto de los algoritmos.
\begin{itemize}
    \item SelfTraining: Único robot.
    \item Co-Training: Son dos clasificadores (dos robots) pero están combinados
    (funcionamiento interno del algoritmo), ya que es como si actuaran como uno
    solo, pero cada uno ve una parte de los datos (\emph{multi-view}).
    \item Democratic Co-Learning: Tres robots con su papeleta de voto (se basa en la votación).
    \item Tri-Training: Se combina el conocimiento (predicciones) de dos
    clasificadores para que un tercero aprenda.
\end{itemize}

En cuanto a la depuración de la aplicación lo que se intentó fue mejorar la
interacción del usuario. En este sentido, se modificaron los formularios de
configuración de Self-Training y Co-Training.

En Self-Training el límite o \textit{threshold} ya no se realiza con un
\textit{placeholder}, sino con un rango de porcentaje, más sencillo de usar.
Además, se ha incluido un \textit{tooltip} indicando que si introduce 0 en el
número de iteraciones no se limita internamente (0 es como infinito).

En Co-Training se ha añadido el \textit{tooltip} del número de iteraciones como
Self-Training (al poner 0 es como si fuera infinito). También se ha añadido
control del formulario para impedir que el número de positivos y negativos sean
nulos a la vez.

Tanto en Democratic Co-Learning y Tri-Training se ha incluido la comprobación de
diversidad en los clasificadores (deben ser diversos, diferentes). Se muestra un
mensaje en el caso de que no sean distintos. De hecho, al entrar en la
configuración, aparecen seleccionados tres distintos.

Entre otras mejoras, en el gráfico de estadísticas, cuando había muchas
iteraciones, los números se solapaban. Se realiza una operación de división
entre 35 (a partir de este los números se solapaban) y ese resultado actúa de
divisor en el número de <<rayitas>> dibujadas en las iteraciones. De esta forma,
se evitan los solapamientos, pero se mantienen suficientes marcas.

Durante unos días se sustituyó los enlaces <<CDN>> (para descargar los recursos
o bibliotecas) por ficheros locales descargados. Esto fue revertido pues
convenía delegar estas descargas al usuario (no sobrecargar el servidor con
tráfico innecesario).

Como último punto a destacar (además de otros retoques menores), se incorporó la
expiración de las \textit{cookies} de sesión. Cuando el usuario cierra el
navegador, esta \textit{cookie} es eliminada (al volver a la aplicación será
como la primera vez).

Al mismo tiempo que se fue depurando la aplicación se iba finalizando con la
memoria. Se añadieron las conclusiones derivadas, el resumen del proyecto y las
líneas futuras para la expansión de la aplicación. 

Además, tanto para la memoria como los anexos se fueron corrigiendo los errores
vistos por ambos tutores.

Para los anexos, también se fueron incorporando los elementos que faltaban (como
el diagrama de componentes) y completando aquellas partes con alguna explicación
más o aclaración. Por último, se añadieron unos videotutoriales (enlazados a
esta documentación) para ejemplificar las principales acciones de los usuarios
(anónimos, registrados y administradores).

Finalmente, surgió la posibilidad de añadir las estadísticas específicas a
Co-Training, ya que al ser dos clasificadores, podría aportar alguna idea sobre
el comportamiento global. Se añadió sin problemas debido a que la implementación
estaba preparada para funcionar de forma genérica (mismas funciones utilizadas
en Democratic Co-Learning y Tri-Training).

Y en definitiva, aquí concluye el desarrollo proyecto. Ha sido un largo camino e
intenso, pero gratificante y con mucho aprendizaje.

\clearpage
\section{Estudio de viabilidad}

\subsection{Viabilidad económica}

El estudio de la viabilidad económica viene a evaluar cuáles son los costes
derivados del desarrollo del proyecto y beneficios esperados.

Aún con todo esto, el contexto del proyecto, desde el inicio, se ha marcado como
una aplicación de tipo docente. Esto tendrá ciertas implicaciones que se irán
comentando durante este estudio.

Se supondrá un entorno real de una empresa creada con el fin de llevar a cabo
este proyecto. Esta empresa contará con un empleado desarrollador (alumno), dos
supervisores (tutores) y los activos hardware y software. El comienzo oficial
del proyecto fue el 25 de enero de 2023 y su fecha límite estipulada es el 8 de
junio (unos 5 meses). 

\subsubsection{Costes}

\textbf{Empleados}:

El desarrollador ha realizado una media de 4.5 horas diarias (progresivamente
aumentando hasta la fecha límite de entrega): unas 3 horas en días laborables y
en los fines de semana se empleaban alrededor de las 8 horas.

El cómputo del salario bruto del alumno, suponiendo un salario de
11\texteuro/hora, según lo estipulado en las bases mínimas de Ingenieros y
Licenciados en~\cite{cotizacion2023} es:

\begin{center}
$4.5 \frac{horas}{día} \times 7 \frac{días}{semana} \times 11
\frac{\text{\texteuro}}{hora} \times 4 \frac{semanas}{mes} = 1\,386\frac{\text{\texteuro}}{mes} $
\end{center}

Para el caso de los supervisores (tutores), el tiempo empleado de ambos es de 2
horas/semana para la revisión del progreso y reuniones. Por lo que, suponiendo
mismas bases mínimas anteriores, y añadiendo cuatro euros debido al cargo
superior, el salario bruto es:

\begin{center}
$2 \frac{horas}{semana} \times 15 \frac{\text{\texteuro}}{hora} \times 4
\frac{semanas}{mes} = 120\frac{\text{\texteuro}}{mes} $
\end{center}


Constituido el salario bruto que recibe cada empleado, deben añadirse todos los
impuestos de la Seguridad Social estipulados para el año 2023. Estos impuestos
están recogidos en la web oficial de la seguridad social en el <<Régimen General
de la Seguridad Social>> en~\cite{cotizacion2023}.

En la tabla~\ref{tabla:seg-social} se puede ver un resumen de los impuestos que
han de aplicarse al salario bruto de los empleados considerando categorías
generales.

\begin{table}[H]
    \centering
    \begin{tabular}{lr}
        \toprule
    \textbf{Concepto}           & \textbf{Impuesto(\%)} \\ \midrule
    CONTINGENCIAS   (Comunes)   & 23,60 \\
    DESEMPLEO  (Tipo General)   & 5,50 \\
    FOGASA                      & 0,20  \\
    FORMACIÓN PROFESIONAL       & 0,60 \\ \bottomrule
    \end{tabular}%
    \caption{Tabla de impuestos}
    \label{tabla:seg-social}
\end{table}

Por lo que la fórmula del cómputo total que la empresa debe pagar por los
empleados es:

\begin{figure}[H]
\begin{center}
    $Gasto~de~la~empresa = \frac{Salario}{1-(0.236+0.055+0.002+0.006)}$
\end{center}
\caption{Cálculo del gasto de la empresa por empleado}
\end{figure}


\begin{table}[H]
\resizebox{\textwidth}{!}{%
\begin{tabular}{lrr}
\toprule
\textbf{Empleado}                               & \textbf{Salario bruto (€)}       & \textbf{Gasto de la empresa (€)}      \\ \midrule
David Martínez Acha (des)               & 1\,386                    & 1\,977,18                      \\
Álvar Arnaiz González (sup)             & 120                      & 171,18                       \\
César Ignacio García Osorio (sup)       & 120                      & 171,18                       \\ \midrule
\textbf{Total}                          & 1\,626                    & 2\,319,54                      \\ \midrule
\textbf{Total 5 meses}                  & 8\,130                    & 11\,597,7                      \\ \bottomrule
\multicolumn{3}{l}{\begin{tabular}[c]{@{}l@{}}des: desarrollador\\ sup: supervisor\end{tabular}}
\end{tabular}%
}
\caption{Salarios brutos y coste que supone a la empresa}
\label{tabla:salarios}
\end{table}

En la tabla~\ref{tabla:salarios} se presentan todos los costes de los empleados.
El total, para los 5 meses de duración aproximada es 11\,597,70\texteuro.

\textbf{Hardware}:

Para desarrollar el proyecto se ha utilizado un ordenador de sobremesa montado
por piezas. Está compuesto por un AMD Ryzen 7 5800X de 8 núcleos a 3.8 GHz, 16
GB de memoria RAM y 1TB SSD NVMe M.2. Este equipo está valorado en
1\,800\texteuro. La vida útil de este activo es de 3 años.

Además, para hacer la aplicación accesible a todo el mundo, se ha adquirido un
servidor para ello (supuesto considerado). Este servidor está compuesto por dos
AMD EPYC, 8 GB de memoria RAM y 80gb de SSD. El equipo está valorado en
1\,300\texteuro. La vida útil de este activo es de 3 años.

Teniendo en cuenta que la duración del proyecto es de 5 meses, el cálculo de la
amortización se realiza de la siguiente forma:

\begin{figure}[H]
\begin{center}
$Amortización = \frac{Coste}{Vida~útil} \times \frac{5}{12}$
\end{center}
\caption{Cálculo de la amortización para este proyecto}
\end{figure}


En la tabla~\ref{tabla:hardware} se presentan los costes de estos activos y el
coste amortizado de los mismos.

\begin{table}[H]
    \centering
\begin{tabular}{lrr}
\toprule
\textbf{Activo}      & \textbf{Coste (€)}      & \textbf{Coste amortizado (€)}      \\ \midrule
Equipo sobremesa     & 1\,800                    & 250                      \\
Servidor             & 1\,300                    & 180,55                      \\ \midrule
\textbf{Total}       & 3\,100                    & 430,55                      \\ \midrule
\end{tabular}
\caption{Costes del hardware}
\label{tabla:hardware}
\end{table}


\textbf{Software}:

Para el desarrollo del producto se ha hecho uso de \textit{software}. En la
medida de lo posible se han utilizado licencias gratuitas y/o proporcionadas por
pertenecer a la Universidad de Burgos.

Los activos software que han supuesto un coste para la empresa es la adquisición
del sistema operativo Windows 10 Pro del equipo de sobremesa. Está valorado en
250\texteuro. Para el cálculo de la vida útil, se tiene en cuenta la
finalización del soporte de Microsoft el 14 de octubre de 2025, aproximadamente
dos años y medio.

El coste amortizado será 41,66\texteuro~(aplicando el mismo cálculo que antes).

\clearpage
\textbf{Costes indirectos}:

Se consideran unos gastos fijos indirectos iguales al 12\% del resto de gastos
anteriores. A estos se le añade otros gastos considerados.

\begin{table}[H]
    \centering
\begin{tabular}{lr}
\toprule
\textbf{Concepto}      & \textbf{Coste (€)}     \\ \midrule
Dominio <<dmacha.dev>>     & 12,00                    \\
12\% costes                & 1\,448,39                     \\ \midrule
\textbf{Total}       & 1\,460,39                     \\ \midrule
\end{tabular}
\caption{Costes indirectos}
\label{tabla:indirectos}
\end{table}


\textbf{Total}:

En la tabla~\ref{tabla:total} se presenta la suma de todos los costes descritos.

\begin{table}[H]
    \centering
\begin{tabular}{lr}
\toprule
\textbf{Tipo de coste}     & \textbf{Coste (€)}     \\ \midrule
Empleados                  & 11\,597,70               \\
Hardware                   & 430,55                     \\
Software                   & 41,66                     \\
Indirectos                 & 1\,460,39                     \\ \midrule
\textbf{Total}             & 13\,530,30                     \\ \midrule
\end{tabular}
\caption{Coste total}
\label{tabla:total}
\end{table}

El coste total del proyecto, finalmente, es de 13\,530,30\texteuro.

\subsubsection{Beneficios}

La aplicación no está pensada para extraer beneficio con su uso. Se trata de una
aplicación de índole docente que permite el acceso completo a todos los
usuarios. Esto incluye las ventajas que un usuario registrado tiene sobre un
anónimo, este último puede adquirir dichas ventajas de forma gratuita
(registro).

\clearpage
\subsection{Viabilidad legal}

El estudio de la viabilidad legal se basa en las licencias de las
bibliotecas/librerías que se han utilizado en el desarrollo. A partir del
listado de estas, se estudiará la permisividad en conjunto para saber cuál es la
más restrictiva y seleccionar acordemente la de este proyecto.

\begin{table}[h!]
    \centering
    \begin{tabular}{lcl}
    \toprule
    \textbf{Librería}         & \textbf{Versión} & \textbf{Licencia}                        \\ \midrule
    \textit{pandas}           & 1.4.3            & BSD 3                                    \\
    \textit{Babel}            & 2.11.0           & BSD 3                                    \\
    \textit{Flask}            & 2.2.5            & BSD 3                                    \\
    \textit{flask-babel}      & 3.0.1            & BSD 3                                    \\
    \textit{Flask-SQLAlchemy} & 3.0.3            & BSD 3                                    \\
    \textit{Flask-Login}      & 0.6.2            & MIT                                      \\
    \textit{scikit-learn}     & 1.2.0            & BSD 3                                    \\
    \textit{numpy}            & 1.23.3           & BSD 3                                    \\
    \textit{scipy}            & 1.9.3            & BSD 3                                    \\
    \textit{setuptools}       & 65.5.1           & MIT                                      \\
    \textit{pytest}           & 7.2.0            & MIT                                      \\
    \textit{Werkzeug}         & 2.2.3            & BSD 3                                    \\
    \textit{matplotlib}       & 3.7.1            & PSF\tablefootnote{Python
    Software Foundation License. A afectos prácticos solo conlleva realizar un
    pequeño resumen de lo que se ha modificado (y solo en caso de
    modificación/adición)~\cite{psf}.} \\
    \textit{WTForms}          & 3.0.1            & BSD 3                                    \\
    \textit{Flask-WTF}        & 1.1.1            & BSD 3                                    \\
    \textit{SQLAlchemy}       & 2.0.13           & MIT                                      \\
    \textit{sslearn}          & 1.0.3.1          & BSD 3                                    \\
    \textit{imbalanced-learn} & 0.10.1           & MIT                                      \\
    \textit{gunicorn}         & 20.1.0           & MIT                                      \\
    \textit{Bootstrap}        & 5.2.3            & MIT                                      \\
    \textit{D3js}             & 7.8.4            & ISC\tablefootnote{Internet Systems Consortium, equivalente a MIT.}                 \\
    \textit{jQuery}           & 3.6.4            & MIT                                      \\
    \textit{DataTables}       & 1.13.4           & MIT                                      \\ \bottomrule
    \end{tabular}%
    \caption{Bibliotecas usadas, versiones y licencias}
\end{table}

\imagen{anexos/planproyecto/Licencias}{Permisividad de las licencias.}

Para tener cierto contexto, en la figura~\ref{fig:anexos/planproyecto/Licencias}
se representan las licencias organizadas según las restricciones que imponen.
Una anchura mayor representa mayores libertades. Se debe tener en cuenta que
solo es una representación para determinar la licencia más restrictiva, es
decir, no es una representación calculada con exactitud.

La peculiaridad de \textit{BSD 3-Clause} es que incluye la cláusula de <<no
respaldo>>. Esta cláusula impide promocionar un producto con el nombre de la
organización o de sus colaboradores (que han creado el software usado). Es por
ello que su permisividad se ha considerado algo menor.

\paragraph{Selección}
Para seleccionar la licencia apropiada a este proyecto se tiene en cuenta que
incluir una licencia menos restrictiva que \texttt{BSD 3-Clause} podría
incumplir alguna de las <<cláusulas>> que esta impone. Por tanto, para mantener
esas normas y condiciones impuestas, el proyecto tendrá la licencia \textbf{BSD
3-Clause}.

\paragraph{Implicaciones de BSD 3-clause} Esta licencia tiene las siguientes implicaciones
(traducción no oficial)~\cite{bsd}:\\

    Copyright (c) 2023, David Martínez Acha
\begin{enumerate}
    \item Las redistribuciones del código fuente deben conservar el aviso de
    derechos de autor anterior, esta lista de condiciones y el siguiente
    descargo de responsabilidad.
    \begin{verbatim}
        ESTE SOFTWARE ES PROPORCIONADO POR LOS TITULARES 
        DE LOS DERECHOS DE AUTOR Y LOS CONTRIBUYENTES 
        <<TAL CUAL>> Y CUALQUIER GARANTÍA EXPLÍCITA O
        IMPLÍCITA, INCLUYENDO, ENTRE OTRAS, LAS GARANTÍAS
        IMPLÍCITAS DE COMERCIABILIDAD E IDONEIDAD PARA 
        UN PROPÓSITO EN PARTICULAR. EN NINGÚN CASO EL 
        TITULAR DE LOS DERECHOS DE AUTOR O LOS 
        CONTRIBUYENTES SERÁN RESPONSABLES POR CUALQUIER 
        DAÑO DIRECTO, INDIRECTO, INCIDENTAL, ESPECIAL, 
        EJEMPLAR O CONSECUENTE (INCLUYENDO, ENTRE OTROS, 
        LA ADQUISICIÓN DE BIENES O SERVICIOS SUSTITUTOS;
        PÉRDIDA DE USO, DATOS O BENEFICIOS; O 
        INTERRUPCIÓN DEL NEGOCIO) CUALQUIER CAUSA Y 
        SOBRE CUALQUIER TEORÍA DE RESPONSABILIDAD, YA 
        SEA POR CONTRATO, RESPONSABILIDAD ESTRICTA O 
        AGRAVIO (INCLUYENDO NEGLIGENCIA O DE OTRO TIPO)
        QUE SURJA DE CUALQUIER MANERA DEL USO DE ESTE 
        SOFTWARE, INCLUSO SI SE ADVIERTE DE LA 
        POSIBILIDAD DE DICHO DAÑO.
    \end{verbatim}
    \item Las redistribuciones en forma binaria deben reproducir el aviso de
    derechos de autor anterior, esta lista de condiciones y el siguiente
    descargo de responsabilidad en la documentación y/u otros materiales
    provistos con la distribución.
    \item Ni el nombre del titular de los derechos de autor ni los nombres de
    sus colaboradores pueden utilizarse para respaldar o promocionar productos
    derivados de este software sin el permiso previo por escrito.
\end{enumerate}

En general, resulta una licencia adecuada que brinda muchas libertades y dado el
enfoque docente, es lo que se busca.
\apendice{Especificación de Requisitos}

\section{Introducción}

En esta sección se detalla la Especificación de requisitos. Aquí quedará
registrado qué es lo que la aplicación debe hacer (y cómo).

\section{Objetivos generales}
Los objetivos del proyecto pueden resumirse en los siguientes:
\begin{enumerate}
    \item Implementación de Self-Training, Co-Training, Democratic Co-Learning y
    Tri-Training.
    \item Creación de una aplicación Web e integración de los cuatro algoritmos
    en la aplicación Web para su visualización.
    \item Sistema de usuarios que les permita controlar sus ficheros y
    ejecuciones.
    \item Administración de los usuarios y todos los ficheros y ejecuciones
    (mediante un rol administrador).
\end{enumerate}

El núcleo del proyecto, pese a la importancia de las implementaciones los
algoritmos, es la aplicación Web, pues representa la parte interactiva del
proyecto. Por este motivo, el presente apartado analizará con precisión que es
lo que la aplicación debe o no hacer (con sus limitaciones).

\section{Usuarios de la aplicación}

Como se ha comentado, existirá un sistema de usuarios en el que podrán
diferenciarse: Usuario, Administrador y Anónimo (este último así lo denomina el
sistema de gestión de cuentas utilizado, \texttt{Flask-Login}). 

\begin{itemize}
	\item Anónimo: Usuario que no posee una cuenta registrada en el sistema, es
	el usuario base. Puede utilizar toda la aplicación básica: seleccionar
	algoritmo, subir un conjunto de datos, configurar el algoritmo y
	visualizarlo.
	\item Usuario: Usuario que posee una cuenta registrada sin privilegios. Al
	igual que el anónimo base, puede utilizar la aplicación, pero además, los
	ficheros (conjuntos de datos) y ejecuciones son almacenados en el sistema.
	\item Administrador: Usuario que posee una cuenta registrada con
	privilegios. Es exactamente lo mismo que un Usuario, con la particularidad
	de que podrá, además, gestionar estos Usuarios, todos los ficheros y todas
	las ejecuciones (de todos los usuarios). 
\end{itemize}

\section{Catálogo de requisitos}


A continuación se detallan los requisitos funcionales así como los no
funcionales.

\subsection{Requisitos funcionales}

\begin{itemize}
	\item \textbf{RF-1 Visualización de algoritmos semi-supervisados}: la
	aplicación debe permitir visualizar el proceso de entrenamiento de los
	algoritmos implementados (con sus estadísticas pertinentes).
    \begin{itemize}
        \item \textbf{RF-1.1 Selección de algoritmo}: la aplicación debe
        permitir seleccionar uno de los algoritmos implementados.
        \item \textbf{RF-1.2 Carga de conjunto de datos}: la aplicación debe
        permitir subir un fichero \texttt{ARFF} o \texttt{ACSV} con el conjunto
        de datos. 
        \item \textbf{RF-1.3 Configuración del algoritmo}: la aplicación debe
        permitir configurar la ejecución del algoritmo con los parámetros
        específicos del mismo.
        \item \textbf{RF-1.4 Control visualizaciones}: la aplicación debe
        mostrar visualizaciones interactivas: principal (gráfico de dos
        dimensiones con los puntos del conjunto de datos) y estadísticas
        (gráficos de líneas).
        \begin{itemize}
            \item \textbf{RF-1.3.1 Controlar paso (iteración)}: la aplicación
            debe permitir controlar el paso del proceso de entrenamiento
            (iteración anterior/siguiente).
            \item \textbf{RF-1.3.1 Interacción con el gráfico principal}: el
            gráfico principal debe mostrar información relevante en un
            \texttt{tooltip} cuando el usuario interactúe con cada punto.
        \end{itemize}
    \end{itemize}
    
    \item \textbf{RF-2 Manejo de usuarios}: la aplicación debe manejar cuentas
    de usuario.
    \begin{itemize}
        \item \textbf{RF-2.1 Registro}: la aplicación debe permitir crear
        cuentas de usuario (no administrador).
        \item \textbf{RF-2.2 Inicio de sesión}: la aplicación debe permitir el
        inicio de sesión a aquellos usuarios con cuenta.
    \end{itemize}
    \item \textbf{RF-3 Personalización del perfil}: los usuarios con cuenta (no
    anónimos) deben poder modificar sus datos personales de su perfil de
    usuario.

    \item \textbf{RF-4 Espacio personal}: los usuarios con cuenta (no anónimos)
    deben poseer de un espacio personal.
    \begin{itemize}
        \item \textbf{RF-4.1 Control de conjunto de datos}: los usuarios con
        cuenta deben poder consultar sus ficheros subidos.
        \begin{itemize}
            \item \textbf{RF-4.1.1 Ejecución de un nuevo algoritmo}: los
            usuarios con cuenta deben poder utilizar esos ficheros en un
            algoritmo (nueva ejecución).
            
            \item \textbf{RF-4.1.2 Eliminación}: los usuarios con cuenta deben
            poder eliminar esos ficheros del sistema.
        \end{itemize}
        \item \textbf{RF-4.2 Control de ejecuciones}:  los usuarios con cuenta
        deben poder consultar sus ejecuciones anteriores (con toda su
        información).
        \begin{itemize}
            \item \textbf{RF-4.2.1 Replicar ejecución}: los usuarios con cuenta
            deben poder replicar las ejecuciones.
            
            \item \textbf{RF-4.2.2 Eliminación}: los usuarios con cuenta deben
            poder eliminar las ejecuciones.
        \end{itemize}
    \end{itemize}

    \item \textbf{RF-5 Administración}: la aplicación debe manejar cuentas de
    usuario de tipo administrador.
    \begin{itemize}
        \item \textbf{RF-5.1 Control de usuarios}: el administrador deben poder
        consultar los usuarios registrados.
        \begin{itemize}
            \item \textbf{RF-5.1.1 Edición de usuario}: el administrador debe
            poder modificar los datos de los usuarios.
            \item \textbf{RF-5.1.2 Eliminación}: el administrador debe poder
            eliminar usuarios del sistema.
            \item \textbf{RF-5.1.3 Total}: el administrador debe poder ver el
            número total de usuarios.
        \end{itemize}
        \item \textbf{RF-5.2 Control de conjuntos de datos global}: el
        administrador debe poder consultar todos los ficheros subidos.
        \begin{itemize}          
            \item \textbf{RF-5.2.1 Eliminación}: el administrador debe poder
            eliminar cualquier fichero de los usuarios registrados.
            \item \textbf{RF-5.2.2 Últimos ficheros}: el administrador debe
            poder ver el \textbf{número} de ficheros subidos de los últimos 7
            días.
        \end{itemize}
        \item \textbf{RF-5.3 Control de ejecuciones global}: el administrador
        debe poder consultar todas ejecuciones (con toda su información).
        \begin{itemize}          
            \item \textbf{RF-5.3.1 Eliminación}: el administrador deben poder
            eliminar cualquier ejecución de los usuarios registrados.
            \item \textbf{RF-5.3.2 Últimas ejecuciones}: el administrador debe
            poder ver el \textbf{número} de ficheros subidos de los últimos 7
            días.
        \end{itemize}
    \end{itemize}

    \item \textbf{RF-6 Cambio de idioma}: todos los usuarios deben poder cambiar
    el idioma a placer (inglés o español).

\end{itemize}

\subsection{Requisitos no funcionales}
Los requisitos anteriores definen de alguna manera lo que el sistema debe hacer.
En este caso, los no funcionales son restricciones, por así decirlo, de calidad.
Responden a la pregunta de cómo funciona y no qué es lo que hace. Todas estas
restricciones son aplicadas de forma intrínseca en el desarrollo.

\begin{itemize}
	\item \textbf{RNF-1 Disponibilidad}: el sistema de funcionar con muy alta
	probabilidad ante una petición. Es decir, debe encontrarse en condiciones de
	funcionamiento.
	\item \textbf{RNF-2 Accesibilidad}: el sistema debe poder abarcar el mayor
	público posible, facilitando su acceso y su manejo independientemente de las
	capacidades personales.
    \item \textbf{RNF-3 Soporte}: el sistema debe poder utilizarse en el mayor
    número posible de navegadores (dada su naturaleza Web).
	\item \textbf{RNF-4 Mantenibilidad}: el sistema debe ser fácil de modificar,
	mejorar o adaptar cuando se presenten nuevas necesidades.
	\item \textbf{RNF-5 Seguridad}: el sistema debe asegurar la información
	sensible (mediante cifrado y controles de accesos) y debe ser accesible
	mediante protocolos segurizados (SSL).
	\item \textbf{RNF-6 Privacidad\footnote{La privacidad puede ser definida
	como el ámbito de la vida personal de un individuo, quien se desarrolla en
	un espacio reservado, el cual tiene como propósito principal mantenerse
	confidencial \cite{eswiki:148719517}}}: el sistema debe respetar la
	información privada y, de ninguna forma, compartirla con terceros. Debe ser
	un espacio reservado confidencial entre el usuario y la aplicación.
	\item \textbf{RNF-7 Escalabilidad}: el sistema debe ser capaz de crecer para
	ajustarse a la carga de trabajo.
	\item \textbf{RNF-8 Extensibilidad}: debe ser fácil añadir nueva
	funcionalidad al sistema, concretamente, la adición de nuevos algoritmos
	debe implicar pocas modificaciones.
	\item \textbf{RNF-9 Robustez}: el sistema debe ser altamente capaz de
	manejar los errores durante la ejecución (entradas erróneas, bugs...),
	mostrando información precisa al usuario y recuperándose de ellos a una
	situación estable.
	\item \textbf{RNF-10 Internacionalización}: el sistema debe soportar múltiples idiomas.

\end{itemize}

\section{Especificación de requisitos}

\imagencontamano{DiagramaCasosDeUso}{fd}{1}

% Caso de Uso 1 -> Visualización de un algoritmo.
\begin{table}[p]
	\centering
	\begin{tabularx}{\linewidth}{ p{0.21\columnwidth} p{0.71\columnwidth} }
		\toprule
		\textbf{CU-1}    & \textbf{Visualización de un algoritmo}\\
		\toprule
		\textbf{Versión}              & 1.0    \\
		\textbf{Autor}                & David Martínez Acha \\
		\textbf{Requisitos asociados} & RF-1, RF-1.1, RF-1.2, RF-1.3, RF-1.4 \\
		\textbf{Descripción}          & Visualización del proceso de entrenamiento de un algoritmo semi-supervisado. Con gráfico principal y estadísticos. \\
		\textbf{Precondición}         & No hay precondiciones \\
		\textbf{Acciones}             &
		\begin{enumerate}
			\def\labelenumi{\arabic{enumi}.}
			\tightlist
			\item Ejecución del Caso de Uso 2.
			\item Ejecución del Caso de Uso 3.
            \item Ejecución del Caso de Uso 4.
            \item El usuario verá en la página los distintos gráficos de su visualización. 
            \item [Opcional] Ejecución del Caso de Uso 5.
		\end{enumerate}\\
		\textbf{Postcondición}        & Visualizaciones renderizadas en la Web \\
		\textbf{Excepciones}          &  \\
		\textbf{Importancia}          & Alta\\
		\bottomrule
	\end{tabularx}
	\caption{CU-1 Visualización de un algoritmo.}
\end{table}

% Caso de Uso 2 -> Selección de algoritmo.
\begin{table}[p]
	\centering
	\begin{tabularx}{\linewidth}{ p{0.21\columnwidth} p{0.71\columnwidth} }
		\toprule
		\textbf{CU-1}    & \textbf{Selección de algoritmo}\\
		\toprule
		\textbf{Versión}              & 1.0    \\
		\textbf{Autor}                & David Martínez Acha \\
		\textbf{Requisitos asociados} & RF-1, RF-1.1 \\
		\textbf{Descripción}          & Seleccionar el algoritmo a ejecutar (establecerlo en la sesión del usuario) \\
		\textbf{Precondición}         & Ejecutando el Caso de Uso 1 \\
		\textbf{Acciones}             &
		\begin{enumerate}
			\def\labelenumi{\arabic{enumi}.}
			\tightlist
			\item El usuario selecciona un algoritmo haciendo clic en el nombre del algoritmo en la barra de navegación.
		\end{enumerate}\\
        \textbf{Acciones alternativas}&
		\begin{enumerate}
			\def\labelenumi{\arabic{enumi}.}
			\tightlist
			\item En caso de encontrarse en la página principal, el usuario
		puede selecciona un algoritmo haciendo clic en una de las tarjetas de
		presentación algoritmos. \end{enumerate}\\
		\textbf{Postcondición}        & Algoritmo almacenado en su sesión y redirección a la página de subida \\
		\textbf{Excepciones}          & Sin excepciones \\
		\textbf{Importancia}          & Alta \\
		\bottomrule
	\end{tabularx}
	\caption{CU-3 Selección de algoritmo.}
\end{table}

% Caso de Uso X -> Cambiar de idioma.
\begin{table}[p]
	\centering
	\begin{tabularx}{\linewidth}{ p{0.21\columnwidth} p{0.71\columnwidth} }
		\toprule
		\textbf{CU-1}    & \textbf{Cambiar idioma}\\
		\toprule
		\textbf{Versión}              & 1.0    \\
		\textbf{Autor}                & David Martínez Acha \\
		\textbf{Requisitos asociados} & RF-6 \\
		\textbf{Descripción}          & Cambiar el idioma de la página (entre inglés y español) \\
		\textbf{Precondición}         & No hay precondiciones \\
		\textbf{Acciones}             &
		\begin{enumerate}
			\def\labelenumi{\arabic{enumi}.}
			\tightlist
			\item El usuario hace clic en el símbolo de traducción en la barra de navegación.
            \item Aparece un desplegable de idiomas.
            \item El usuario hace clic en el idioma deseado.
		\end{enumerate}\\
		\textbf{Postcondición}        & La página se ha cambiado al idioma seleccionado \\
		\textbf{Excepciones}          & Sin excepciones \\
		\textbf{Importancia}          & Baja \\
		\bottomrule
	\end{tabularx}
	\caption{CU-1 Cambiar idioma.}
\end{table}

\apendice{Especificación de diseño}

\section{Introducción}

En esta sección se van a describir las decisiones tomadas para llevar a cabo
todos los objetivos y requisitos iniciales establecidos. Se presentará el
formato que se utiliza para tratar los datos, cuál es el procedimiento interno
que realiza la Web para ofrecer al usuario las visualizaciones y cómo se traduce
todo ello en la arquitectura subyacente.

\section{Diseño de datos}

\subsection{Información de entrenamiento de los algoritmos}
\label{datos:entrenamiento}
Todas las visualizaciones de los algoritmos se nutren del proceso de
entrenamiento. Es decir, se tuvo que diseñar una forma de aglutinar la
información que ocurre durante este proceso.

Para adelantar un poco el funcionamiento de la Web, todos estos datos son
transformados a JSON, el formato de texto con la sintaxis de JavaScript que
resulta muy sencillo para el intercambio de datos. Además, a la hora de trabajar
con ello en dicho lenguaje, se hace de forma directa (como diccionarios en otros
lenguajes de programación).

Pero antes de esa transformación, todos los datos son generados en Python. La
estructura de datos por excelencia para almacenar muchos datos (gracias a su
multitud de opciones) son los <<DataFrames>>. Toda la información que generan
los algoritmos son de este tipo.

Cada algoritmo retorna varios de estos <<DataFrame>>, al menos uno del proceso
de etiquetado y otro con las estadísticas generales. En el caso de Democratic
Co-Learning y Tri-Training, estos además retornan los de las estadísticas
específicas de cada clasificador base.

\subsubsection{Etiquetas}
Este <<DataFrame>> contiene todas las operaciones en las que los clasificadores
han añadido nuevas etiquetas a instancias no etiquetadas. Por lo general
indicarán los momentos en los que fueron etiquetadas (iteración) y el valor de
las mismas.

\paragraph{Self-Training}
Este formato sirve como base para todos los siguientes. Es una idea muy sencilla
(pues Self-Training también lo es). 

La estructura de datos contendrá múltiples filas, cada una representa cada
instancia de todo el conjunto de datos de entrenamiento. Lo que se entiende por
conjunto de entrenamiento es, todos los datos etiquetados de los que aprenderá
el clasificador base junto con los no etiquetados.

Ahora bien, la información útil que permitirá saber los momentos de
clasificación y/o etiquetas estarán en las columnas:

\begin{itemize}
    \item Columnas con los nombres de los atributos de las instancias. Por
    ejemplo, para el famoso conjunto de datos \texttt{Iris} se tendrán 4
    columnas (una por cada atributo, \textbf{no se incluye la clase}): <<sepal
    length>>, <<sepal width>>, <<petal length>> y <<petal width>>.
    \item Columna <<iter>>: Por cada fila, representa el número de la iteración
    en la que esa instancia fue clasificada. Si por el criterio de parada no
    llega a ser clasificada corresponderá con el número de iteraciones final +
    1. Es decir, si en la iteración 9 el entrenamiento finalizó, corresponderá
    con 10. Obviamente, si la instancia es un dato inicial tendrá como iteración
    0.
    \item Columna <<target>>: Representa la etiqueta de la instancia. Es
    importante destacar que esta siempre será de tipo entero. Esto es porque en
    pasos previos se eliminan los valores nominales (no están permitidos). Si
    tampoco llega a ser clasificado, corresponderá con <<-1>>.
\end{itemize}

Ejemplo (entrenamiento finalizado en la iteración 9):
\begin{table}[H]
\begin{tabular}{rrrrrrr}
    & petal.length & petal.width & sepal.length & sepal.width & iter & target \\ \toprule
0   & 5.6          & 2.5         & 3.9          & 1.1         & 0    & 1      \\
1   & 6.1          & 2.8         & 4.0          & 1.3         & 0    & 2      \\
78  & 6.7          & 3.1         & 5.6          & 2.4         & 6    & 2      \\
79  & 6.8          & 3.2         & 5.7          & 2.5         & 7    & 1      \\
142 & 6.8          & 2.8         & 4.8          & 1.4         & 10   & -1    
\end{tabular}
\caption{Ejemplo de DataFrame de Self-Training}
\end{table}

En la tabla anterior se puede ver un extracto de lo que podría ser una
ejecución. Las columnas de los atributos, la iteración (con iteración 10 para
denotar que no fue etiquetado) y la etiqueta o \textit{target} (con -1 para los
que no fueron etiquetados).

\paragraph{Co-Training}
Es muy similar a Self-Training salvo que este algoritmo concreto envuelve dos
clasificadores base. La consecuencia de esto a nivel del formato es que se
necesita almacenar cuál de los dos clasificadores clasifica cada instancia. Pero
más allá de esto, es el mismo formato.

Recopilando, las columnas para el <<DataFrame>> de Co-Training son:
\begin{itemize}
    \item Columnas con los nombres de los atributos de las instancias.
    \item Columna <<iter>>.
    \item Columna <<target>>.
    \item Columna <<clf>>: Esta nueva columna indica el clasificador que le dio
    el valor  a la etiqueta. Por convención (necesaria para otros procesos en
    Web) es que si el dato es inicial, <<clf>> valdrá <<inicio>>, si es un dato
    clasificado durante el proceso valdrá <<CLF(\textit{nombre\_clasificador})>>
    donde \textit{nombre\_clasificador} será el extraído de \texttt{scikit-learn}
    y si no llega a ser clasificado valdrá -1.
\end{itemize}

Ejemplo (entrenamiento finalizado en la iteración 9):
\begin{table}[H]
    \resizebox{\textwidth}{!}{%
\begin{tabular}{rrrrrrrl}
    & petal.length & petal.width & sepal.length & sepal.width & iter & target & clf \\ \toprule
0   & 5.6          & 2.5         & 3.9          & 1.1         & 0    & 1  & inicio    \\
1   & 6.1          & 2.8         & 4.0          & 1.3         & 0    & 2  & inicio     \\
78  & 6.7          & 3.1         & 5.6          & 2.4         & 6    & 2  & CLF(SVC)    \\
79  & 6.8          & 3.2         & 5.7          & 2.5         & 7    & 1  & CLF(GaussianNB)     \\
142 & 6.8          & 2.8         & 4.8          & 1.4         & 10   & -1 & -1    
\end{tabular}
    }
    \caption{Ejemplo de DataFrame de Co-Training}
\end{table}


\paragraph{Democratic Co-Learning}
Tomando como base Co-Training este algoritmo no añade ninguna columna más (ni
elimina), solo modifica las existentes. Concretamente, tres última columnas
(<<iter>>,<<target>> y <<clf>>) que estaban en singular ahora pasan a plural:
<<iters>>,<<targets>> y <<clfs>>. 

La razón es simplemente para mantener una lógica y semántica interna, este
algoritmo <<comparte>> las instancias entre tres clasificadores base. Cada uno
de ellos puede clasificar cada instancia, incluso con distintas etiquetas y en
la misma iteración que los otros clasificadores. Por lo tanto, para cada una de
estas columnas y para cada instancia se tendrá ahora una lista.

Recopilando, las columnas para el <<DataFrame>> de Democratic Co-Learning son:

\begin{itemize}
    \item Columnas con los nombres de los atributos de las instancias.
    \item Columna <<iters>>: Lista con la iteración en la que cada clasificador
    etiqueta la instancia. En el caso de un dato inicial, esta lista sol o
    tendrá una posición y contendrá 0 (\texttt{[0]}), para el resto siempre
    tendrá tres posiciones (por los tres clasificadores base). Para este último
    caso cada posición es independiente, es decir, si un clasificador base no ha
    etiquetado, contendrá -1 (\textbf{diferencia con los anteriores}\footnote{La
    razón de que se indique -1 en vez de número de iteraciones + 1, es porque
    toda esta estructura se genera antes del entrenamiento. En los algoritmos
    anteriores, si quedaba alguna sin etiquetar, se añadía al final y se sabía
    el número de iteraciones final.}), pero para esa misma instancia otro
    clasificador base sí que puede haber etiquetado y contendrá dicha iteración
    (por ejemplo: $[-1,4,-1]$)
    \item Columna <<targets>>: Exactamente igual a <<iters>> salvo que no indica
    la iteración, sino la etiqueta que asigna el clasificador base. De nuevo, si
    no la etiqueta, su posición contendrá -1.
    \item Columna <<clfs>>: Hasta ahora se ha hablado de posiciones en las dos
    columnas anteriores. Para saber a qué clasificador se refiere esta columna
    contiene otra lista con los tres nombres de los clasificadores. Si es un
    dato inicial, contendrá [inicio], en otro caso contendrá algo como
    \texttt{[CLF1(KNeighborsClassifier), CLF2(DecisionTreeClassifier),
    CLF3(GaussianNB)]}.
\end{itemize}

Ejemplo (entrenamiento finalizado en la iteración 9)\footnote{Se han acortado los nombres de los clasificadores [CLF1, CLF2, CLF3]
debería ser [CLF1(KNeighborsClassifier), CLF2(DecisionTreeClassifier),
CLF3(GaussianNB)]}:
\begin{table}[H]
    \resizebox{\textwidth}{!}{%
\begin{tabular}{rrrrrlll}
    & petal.length & petal.width & sepal.length & sepal.width & iter & target & clf \\ \hline
0   & 5.6          & 2.5         & 3.9          & 1.1         & [0]    & [1]  & [inicio]    \\
1   & 6.1          & 2.8         & 4.0          & 1.3         & [0]   & [2]  & [inicio]     \\
78  & 6.7          & 3.1         & 5.6          & 2.4         & [1, 4, -1]  & [2, 2, -1]  & [CLF1, CLF2, CLF3]    \\
79  & 6.8          & 3.2         & 5.7          & 2.5         & [-1, 2, -1]    & [-1, 1, -1]  & [CLF1, CLF2, CLF3]     \\
142 & 6.8          & 2.8         & 4.8          & 1.4         & [-1, -1, -1]   & [-1, -1, -1] & [CLF1, CLF2, CLF3]    
\end{tabular}
    }
    \caption{Ejemplo de DataFrame de Democratic Co-Learning}
\end{table}


\paragraph{Tri-Training}
Tomando la idea de Democratic Co-Learning, en este algoritmo también se tienen
tres clasificadores que pueden clasificar individualmente cada instancia. Sin
embargo, cada iteración, el conjunto de datos \textbf{nuevos} etiquetados se
vacía. Es decir, en una iteración se acaban etiquetando algunos, pero al
principio de la siguiente se vacía ese conjunto y se vuelven a etiquetar de
nuevo.

Entonces, el mecanismo de las listas de Democratic Co-Learning no funcionaría.
Se debe añadir un nivel más de registro. Simplemente con añadir una lista en
cada posición de las listas de <<iters>> y <<targets>> ya se puede registrar
todos los momentos en los que una instancia se etiqueta (de nuevo, cada
instancia podría ser etiquetada dos o más veces por un mismo clasificador base,
al contrario del anterior).


Recopilando, las columnas para el <<DataFrame>> de Tri-Training son:

\begin{itemize}
    \item Columnas con los nombres de los atributos de las instancias.
    \item Columna <<iters>>: Lista de listas con la iteración en la que cada
    clasificador etiqueta la instancia. En el caso de un dato inicial, esta
    lista sol o tendrá una posición y contendrá 0 (\texttt{[0]}), para el resto
    siempre tendrá una lista en las tres posiciones (por los tres clasificadores
    base). Para esto último caso las posiciones siguen siendo independientes
    solo que ahora si no se etiqueta la lista estará vacía (por ejemplo, las
    iteraciones de un dato podrían ser: \texttt{[[2], [], []]}, los dos últimos
    clasificadores no etiquetaron esa instancia en ningún momento). Para
    clarificar lo comentado anteriormente, los clasificadores podrían etiquetar
    una misma instancia dos o más veces (por ejemplo: \texttt{[[2], [2], [1,
    2]]}, el último clasificador etiquetó la instancia en dos ocasiones).
    \item Columna <<targets>>: Exactamente igual a <<iters>> salvo que no indica
    la iteración, sino la etiqueta que asigna el clasificador base. Y de nuevo,
    si un clasificador no etiqueta en ningún momento esa instancia su lista
    interna estará vacía (por ejemplo: \texttt{[[2], [], [2]]}).
    \item Columna <<clfs>>: No se modifica respecto a Co-Training, contiene la
    lista de los nombres de los tres clasificadores.
\end{itemize}

Ejemplo (entrenamiento finalizado en la iteración 9)\footnote{Se han acortado los nombres de los clasificadores [CLF1, CLF2, CLF3]
debería ser [CLF1(KNeighborsClassifier), CLF2(DecisionTreeClassifier),
CLF3(GaussianNB)]}:
\begin{table}[H]
    \resizebox{\textwidth}{!}{%
\begin{tabular}{rrrrrlll}
    & petal.length & petal.width & sepal.length & sepal.width & iter & target & clf \\ \hline
0   & 5.6          & 2.5         & 3.9          & 1.1         & [0]    & [1]  & [inicio]    \\
1   & 6.1          & 2.8         & 4.0          & 1.3         & [0]   & [2]  & [inicio]     \\
78  & 6.7          & 3.1         & 5.6          & 2.4         & [[1], [1,2], [~]]  & [[2], [2, 2], [~]]  & [CLF1, CLF2, CLF3]    \\
79  & 6.8          & 3.2         & 5.7          & 2.5         & [[~], [2], [~]]    & [[~], [1], [~]]  & [CLF1, CLF2, CLF3]     \\
142 & 6.8          & 2.8         & 4.8          & 1.4         & [[~], [~], [~]]   & [[~], [~], [~]]  & [CLF1, CLF2, CLF3]    
\end{tabular}
    }
    \caption{Ejemplo de DataFrame de Tri-Training}
\end{table}

\paragraph{Interpretación sencilla de estos formatos} Cuando se realiza su
visualización, el primer paso es extraer los datos. Lo que se hace es crear un
punto en el gráfico por cada instancia, con la particularidad de que cuando se
entra en los algoritmos Democratic Co-Learning o Tri-Training, se añade más de
un punto por cada instancia, representando la individualidad de cada
clasificador base (cada uno de ellos puede haber clasificado esa instancia por
separado e incluso más de una vez).

Cuando se genera el gráfico con sus puntos, cada punto lleva guardada la
información de las iteraciones, etiquetas y clasificadores. Así se controla
cuándo ocultar/colorear/mostrar un punto. Por ejemplo, cuando se navega a la
siguiente iteración se filtran todos los puntos que tenga esa iteración en la
columna <<iter>> (o <<iters>>). Obviamente, cada algoritmo tiene sus
particularidades, pero esta es la idea general.

\subsubsection{Estadísticas generales}
Las estadísticas generales son comunes a todos los algoritmos, no hay ninguna
modificación entre ellos.

Se tiene un <<DataFrame>> con tantas filas como iteraciones se han ejecutado. En
cuanto a las columnas, simplemente son los nombres de las estadísticas que se
desean mostrar. Los nombres son los que se mostrarán en la Web.

Ejemplo:
\begin{table}[H]
\begin{tabular}{llllll}
  & Accuracy  & Precision & Error    & F1\_score & Recall    \\ \hline
0 & 0.833333  & 0.888889  & 0.166667 & 0.822222  & 0.833333  \\
1 & 1.000000 & 1.000000 & 0.000000 & 1.000000 & 1.000000 \\
2 & 1.000000 & 1.000000 & 0.000000 & 1.000000 & 1.000000 \\
3 & 1.000000 & 1.000000 & 0.000000 & 1.000000 & 1.000000 \\
4 & 1.000000 & 1.000000 & 0.000000 & 1.000000 & 1.000000
\end{tabular}
\end{table}

\subsubsection{Estadísticas específicas}
Este tipo de estadísticas se refiere a la necesidad de mostrar también las
estadísticas particulares de los clasificadores base. Esto aplica para
Democratic Co-Learning y Tri-Training.

El núcleo de esta estructura de datos siguen siendo los <<DataFrames>>, sin
embargo, se han envuelto en un diccionario. El diccionario tiene como claves,
cada uno de los nombres de los clasificadores base de la ejecución (por ejemplo:
\texttt{CLF3(GaussianNB)}). Los valores serán esos <<DataFrames>> que son
exactamente iguales que para las estadísticas generales. Guardarán por cada
iteración (fila), las estadísticas (columnas) para el clasificador base
concreto.

\section{Diseño procedimental}

\imagencontamano{ProcedimientoWeb}{Diagrama de secuencia de la interacción Web}{1.1}

\section{Diseño arquitectónico}

\section{Diseño Web}

\subsection{Primer mockup o maqueta}
Se presenta el primer Mockup o maqueta que se comentó de la página Web. Todas
las páginas tendrán una base común en la que aparecerá información general como
la Universidad de Burgos (barra superior).

\imagen{MUP_Inicio}{Página inicial de la Web.}

En esta página inicial el usuario podrá seleccionar el algoritmo que desea
visualizar. En los cuadrados existirá un logo o imagen representativa del
algoritmo junto con su nombre.

\imagen{MUP_ConfAlgoritmo}{Página de configuración del algoritmo.}

En esta ventana el usuario podrá subir el conjunto de datos que desee o incluso
seleccionar alguno de los almacenados localmente. Además, como los algoritmos
tienen parámetros personalizables también habrá elemento para configurarlos.

Antes de iniciar, se muestra una explicación del algoritmo general y su pseudocódigo.


\imagen{MUP_Algoritmo}{Página de ejecución del algoritmo.}

Mostrará la evolución del entrenamiento de los algoritmos con una vista
principal (izquierda) de la clasificación y un compendio de métricas como la
precisión o el error en su caso (derecha). Esto último principalmente
planteado para ocultar/ver lo que el usuario desee en cada momento.

\subsection{Diseño de Usuarios}

\imagen{MUP_NAVBAR}{Barra de navegación principal}
\imagen{MUP_NAVBAR_LOGGED}{Barra de navegación con usuario}
\imagen{MUP_NAVBAR_HOVER}{Animación <<hover>> en barra de navegación}
\imagen{MUP_MIESPACIO}{Espacio personal del usuario}
\imagen{MUP_PERFIL}{visualización del perfil y modificación}
\apendice{Documentación técnica de programación}

\section{Introducción}
En esta sección se presenta toda la documentación técnica del desarrollo del
proyecto. Trata de ser una guía para entender cómo se ha hecho el proyecto
comenzando por los directorios y su contenido, con un manual introductorio para
un programador que es iniciado en el proyecto, una explicación y ejemplificación de la
instalación del mismo y las pruebas que se han realizado para validarlo.

\tcbset{colback=red!5!white,fonttitle=\bfseries}
\begin{tcolorbox}[enhanced,title=Repositorio Github,
frame style={left color=blue!75!black,
right color=cyan!75!black}]
\url{https://github.com/dma1004/TFG-SemiSupervisado}
\end{tcolorbox}

\section{Estructura de directorios}
Estos son los directorios en los que se organiza el proyecto:

\begin{figure}[H]
    \dirtree{% 
        .1 /.
        .2 \textbf{algoritmos}: \begin{minipage}[t]{8cm}
            algoritmos Semi-Supervisados implementados{.}\\
        \end{minipage}.
        .3 \textbf{test}: \begin{minipage}[t]{8cm}
            directorios y ficheros mediante los que se valida el
            desarrollo software correcto del proyecto{.}\\
        \end{minipage}.
        .4 \textbf{check\_implementations}: \begin{minipage}[t]{6cm}
            pruebas para la validación de los algoritmos{.}\\
        \end{minipage}.
        .5 \textbf{results}: \begin{minipage}[t]{8cm}
            ficheros CSV con resultados de validación cruzada{.}\\
        \end{minipage}.
        .4 \textbf{check\_utils}: \begin{minipage}[t]{8cm}
            pruebas para la validación de las utilidades{.}\\
        \end{minipage}.
        .5 \textbf{test\_files}: \begin{minipage}[t]{8cm}
            ficheros de prueba (ARFF y CSV) para las pruebas{.}\\
        \end{minipage}.
        .4 \textbf{profiling}: \begin{minipage}[t]{8cm}
            pruebas para medir el tiempo de ejecución{.}\\
        \end{minipage}.
        .5 \textbf{profile\_results}: \begin{minipage}[t]{8cm}
            resultados de los procesos de <<profiling>>{.}\\
        \end{minipage}.
        .3 \textbf{utilidades}: \begin{minipage}[t]{8cm}
            utilidades (programas) que realizan ciertos pasos de la
            aplicación y de los algoritmos para centralizar estos procedimientos (comunes){.}\\
        \end{minipage}.
        .2 \textbf{docs}: \begin{minipage}[t]{8cm}
            documentación teórica y técnica del proyecto (hecha en \LaTeX){.}\\
        \end{minipage}.
        .3 \textbf{img}: \begin{minipage}[t]{8cm}
            imágenes utilizadas para la generación de la documentación{.}\\
        \end{minipage}.
        .3 \textbf{tex}: \begin{minipage}[t]{8cm}
            archivos de texto plano con código \LaTeX{.}\\
        \end{minipage}.
        .2 \textbf{web}: \begin{minipage}[t]{8cm}
            estructura o código de la aplicación Web{.}\\
        \end{minipage}.
    }
\end{figure}

\begin{figure}[H]
    \dirtree{%
        .1 \textbf{web}: \begin{minipage}[t]{10cm}
            estructura o código de la aplicación Web{.}\\
        \end{minipage}. 
        .2 \textbf{app}: \begin{minipage}[t]{10cm} contiene la
        definición de las rutas, modelos de la base de datos y la
        \texttt{Application Factory} que instancia la aplicación{.}\\
        \end{minipage}.
        .3 \textbf{datasets}: \begin{minipage}[t]{8cm}
            contiene (durante el funcionamiento de la aplicación) todos los conjuntos de
            datos que los usuarios introducen{.}\\
        \end{minipage}.
        .4 \textbf{anonimos}: \begin{minipage}[t]{8cm}
            conjuntos de datos subidos por los usuarios anónimos{.}\\
        \end{minipage}.
        .4 \textbf{registrados}: \begin{minipage}[t]{8cm}
            conjuntos de datos subidos por los usuarios registrados{.}\\
        \end{minipage}.
        .4 \textbf{seleccionar}: \begin{minipage}[t]{8cm}
            conjuntos de datos para seleccionar de prueba, principalmente
            durante el desarrollo de la aplicación{.} También almacena el fichero de
            prueba que los usuarios pueden descargar{.}\\
        \end{minipage}.
        .3 \textbf{runs}: \begin{minipage}[t]{8cm}
            contiene las ejecuciones de los algoritmos en formato JSON, 
            solo de los usuarios registrados{.}\\
        \end{minipage}.
        .3 \textbf{static}: \begin{minipage}[t]{10cm}
            ficheros estáticos que utiliza la aplicación Web: CSS,
            Javascript, JSON o imágenes{.}\\
        \end{minipage}.
        .4 \textbf{css}: \begin{minipage}[t]{10cm}
            ficheros CSS{.}\\
        \end{minipage}.
        .4 \textbf{js}: \begin{minipage}[t]{10cm}
            ficheros JavaScript{.}\\
        \end{minipage}.
        .5 \textbf{configuracion}: \begin{minipage}[t]{10cm}
            ficheros JavaScript encargados de la configuración de la ejecución de los algoritmos{.}\\
        \end{minipage}.
        .5 \textbf{usuarios}: \begin{minipage}[t]{10cm}
            ficheros JavaScript encargados de la generación de tablas y control del espacio personal de los usuarios y el panel de administración{.}\\
        \end{minipage}.
        .5 \textbf{visualizacion}: \begin{minipage}[t]{10cm}
            ficheros JavaScript encargados de la inicialización y generación de los gráficos en la visualización (gráfico principal y estadísticas){.}\\
        \end{minipage}.
    }
\end{figure}


\begin{figure}[H]
    \dirtree{%
        .1 \textbf{web}: \begin{minipage}[t]{10cm}
            estructura o código de la aplicación Web{.}\\
        \end{minipage}. 
        .2 \textbf{app}: \begin{minipage}[t]{10cm} contiene la
        definición de las rutas, modelos de la base de datos y la
        \texttt{Application Factory} que instancia la aplicación{.}\\
        \end{minipage}.
        .3 \textbf{static}: \begin{minipage}[t]{10cm}
            ficheros estáticos que utiliza la aplicación Web: CSS,
            Javascript, JSON o imágenes{.}\\
        \end{minipage}.
        .4 \textbf{css}: \begin{minipage}[t]{10cm}
            ficheros CSS{.}\\
        \end{minipage}.
        .4 \textbf{js}: \begin{minipage}[t]{10cm}
            ficheros JavaScript{.}\\
        \end{minipage}.
        .4 \textbf{json}: \begin{minipage}[t]{10cm}
            ficheros JSON{.}\\
        \end{minipage}.
        .4 \textbf{pseudocodigos}: \begin{minipage}[t]{10cm}
            imágenes de los pseudocódigos{.}\\
        \end{minipage}.
        .5 \textbf{en}: \begin{minipage}[t]{10cm}
            imágenes de los pseudocódigos en inglés{.}\\
        \end{minipage}.
        .5 \textbf{es}: \begin{minipage}[t]{10cm}
            imágenes de los pseudocódigos en español{.}\\
        \end{minipage}.
        .3 \textbf{templates}: \begin{minipage}[t]{10cm}
            plantillas HTML (Jinja2) que renderiza la aplicación Web
            (Flask){.}\\
        \end{minipage}.
        .4 \textbf{configuracion}: \begin{minipage}[t]{10cm}
            plantillas concretas de la ventana de configuración de la ejecución de los algoritmos{.}\\
        \end{minipage}.
        .4 \textbf{usuarios}: \begin{minipage}[t]{10cm}
            plantillas concretas de las distintas ventanas de los usuarios: login, registro, perfil, espacio personal y panel de administración{.}\\
        \end{minipage}.
        .4 \textbf{visualizacion}: \begin{minipage}[t]{10cm}
            plantillas concretas de las distintas ventanas de visualización de los algoritmos{.}\\
        \end{minipage}.
        .3 \textbf{translations}: \begin{minipage}[t]{10cm}
            traducciones de los textos de la aplicación (por idiomas){.}\\
        \end{minipage}. 
        .2 \textbf{instance}: \begin{minipage}[t]{10cm} contiene
        la base de datos de la aplicación (SQLite){.}\\
            \end{minipage}.
    }
\end{figure}

\section{Manual del programador}
El objetivo de este manual es dar el conocimiento necesario al 
lector/desarrollador que comience a trabajar con este proyecto para continuarlo . Se ha de
tener en cuenta que lo descrito a continuación es lo que se ha estado utilizando 
para el entorno desarrollo inicial y será explicado para este (sin limitación, por ejemplo,
a usar otras herramientas).

En primer lugar se listan las herramientas consideradas para el desarrollo:
\begin{itemize}
    \item Python 3.10: Todo el proyecto, desde su inicio, ha sido desarrollado en la versión 3.10.
    \item Git: Necesario para continuar con el control de versiones del proyecto.
    \item Pycharm: Editor de código utilizado, podría utilizarse otro si así se considerase.
\end{itemize}

Python puede descargarse desde su página principal\footnote{Descargas de
Python: \url{https://www.python.org/downloads/}}. En las herramientas no se ha
mencionado <<pip>> (el administrador de paquetes) pues desde la versión 3.4 de
Python este está instalado con él, sin embargo, sería conveniente asegurarse de
ello comprobado la versión pues en el futuro será necesario. 

\begin{tcolorbox}[colback=cyan!5!white,colframe=cyan!75!black,title=Comprobar pip]
\begin{minted}{shell}
pip --version 
python -m ensurepip --upgrade
\end{minted}
\end{tcolorbox}

Y de forma general, comprobar que los ficheros binarios pueden ser utilizados por lo
menos, en el entorno del programador (solo en la cuenta del usuario o en el equipo completo).

En el caso de Git, desde Linux simplemente se puede instalar con su gestor de
paquetes:

\begin{tcolorbox}[colback=cyan!5!white,colframe=cyan!75!black,title=Instalar Git en Linux]
\begin{minted}{shell}
sudo apt install git-all
\end{minted}
\end{tcolorbox}

Si el entorno es Windows, existe un instalador directo que puede descargarse
desde la página de Git SCM (Source code management)\footnote{Git para Windows:
\url{https://git-scm.com/download/win}}.

Pycharm se puede descargar tanto para Windows como Linux en la página oficial de
JetBrains\footnote{Pycharm: \url{https://www.jetbrains.com/pycharm/download/}}.

\subsection{Comprensión de la estructura}

Se recomienda leer la sección anterior donde se pueden consultar todos los
directorios del proyecto con una breve descripción. La preparación del entorno
virtual con el que trabajar se explicará en la próxima sección.

El proyecto se desarrolla en dos ramas comunicadas (de forma unidireccional):
los algoritmos implementados y la aplicación Web. La aplicación es la que
utiliza los algoritmos para obtener la información presentada en la Web.

\paragraph{Algoritmos semi-supervisados} (contenidos en el directorio
\texttt{algoritmos}): Los algoritmos desarrollados y nuevos han de situarse en
este directorio como raíz. La idea es que cada fichero <<.py>>, homónimo al
algoritmo, contenga la definición de un objeto que encapsule el desarrollo del
mismo, para que no haya confusión.

Estructura de los objetos (mínima):

\textbf{Constructor}: Donde se configuran los parámetros que necesita el algoritmo. Es
recomendable realizar una validación de los mismos por si fueran utilizados de manera individual.

\textbf{Método de entrenamiento (Fit)}: Dado que estos algoritmos están pensados
no solo para entrenar, sino para almacenar el proceso de entrenamiento y
estadísticas, siempre ha de recibir el conjunto de entrenamiento ($\mathbf{x}$,
$y$), el conjunto de test (\texttt{x\_test}, \texttt{y\_test}) y el nombre de
las características de cada instancia (el nombre de las columnas de
$\mathbf{x}$).

En principio, el método de desarrollo seguido es el de primero implementar el
algoritmo para después añadir, con la librería Pandas, un registro completo de
los momentos de etiquetado y de las estadísticas de cada iteración.

Este método deberá retornar el registro de etiquetado, el registro estadístico y el
número de iteraciones realizadas.

\begin{tcolorbox}[colback=cyan!5!white,colframe=cyan!75!black,title=Cabecera fit]
\begin{minted}{python}
def fit(x, y, x_test, y_test, features)
\end{minted}
\end{tcolorbox}

Es importante tener en cuenta que esta estructura puede variar en la medida de
cómo sea el algoritmo. Por ejemplo, en el caso de Democratic Co-Learning se
hacía imprescindible añadir estadísticas específicas para cada clasificador que
encapsula y por tanto, retornaba más elementos.

\textbf{Método de predicción}: En el caso de algoritmos que trabajan con un único
clasificador podría ser opcional, pero en el caso de varios, es necesario
considerar cómo se predicen las etiquetas en combinación.

\begin{tcolorbox}[colback=cyan!5!white,colframe=cyan!75!black,title=Cabecera predict]
\begin{minted}{python}
def predict(self, instances)
\end{minted}
\end{tcolorbox}
    

\textbf{Métodos estadísticos}: Dependiendo de lo que se desee mostrar en la
aplicación, se incluirán ciertas estadísticas.

La convención utilizada hasta ahora es crear un método que comience por
<<\texttt{get\_}>> seguido del nombre de la estadística, por ejemplo
\texttt{get\_accuracy\_score}.

\begin{tcolorbox}[colback=cyan!5!white,colframe=cyan!75!black,title=Cabecera ejemplo estadística]
\begin{minted}{python}
def get_accuracy_score(self, x_test, y_test):
\end{minted}
\end{tcolorbox}

\paragraph{Utilidades} En el directorio de las utilidades se han de alojar
aquellos métodos que se reutilizan en el proyecto (y que intervenga en algún paso
del algoritmo, como por ejemplo, la carga de datos). 

\paragraph{Test} El desarrollo del software debe ser validado para asegurar su
correcto funcionamiento ante las distintas casuísticas. Cuando se desarrolla
código en esta sección, sus casos de prueba codificados deben incluirse en el
directorio correspondiente. Se utiliza <<pytest>> como \textit{framework} de
pruebas.

\paragraph{Aplicación Web} (contenida en el directorio \texttt{web}):  

La aplicación está desarrollada con el <<micro-framework>> Flask, que permite la
creación de aplicaciones Web en Python. A lo largo del desarrollo la estructura
de la aplicación ha variado bastante. Finalmente, se ha modularizado
completamente con el uso de \texttt{Blueprints} y una \texttt{Application
Factory} que se encarga de instancia la aplicación, base de datos y pone en
funcionamiento las distintas rutas accesibles.

\textbf{\texttt{run.py}}: Centraliza la ejecución de la aplicación (mediante la
\texttt{Application Factory}) 

\textbf{\texttt{instance}}: Donde se almacena la instancia de la base de datos.

\textbf{\texttt{app}}: Este directorio contiene toda la definición de la
aplicación, el resto de los apartados comentados siguientes se encuentran dentro
de este.

El fichero \texttt{\_\_init\_\_.py} contiene la creación de la aplicación con un
método \texttt{create\_app} (\texttt{Application Factory}). Aquí se específica
toda la configuración, los elementos comunes (como los filtros de Jinja), se
instancia la base de datos y se registran las rutas (organizadas con
\texttt{Blueprints} como paquetes o extensiones de la aplicación básica).

\textbf{Blueprints}: Las rutas que tiene la aplicación están organizadas en
función de su categoría mediante los Blueprints. Y es en los ficheros que
terminan por <<\_routes.py>> donde se definen. Si se añade una categoría nueva
se debe crear un Blueprint que la identifique dentro de su fichero y después
debe ser registrado en la aplicación (en \texttt{\_\_init\_\_.py}).

\begin{tcolorbox}[colback=cyan!5!white,colframe=cyan!75!black,title=Crear Blueprint]
\begin{minted}{python}
# El nombre es a elección propia
nuevo_bp = Blueprint('nuevo_bp', __name__)
\end{minted}
\end{tcolorbox}

\begin{tcolorbox}[colback=cyan!5!white,colframe=cyan!75!black,title=Registrar Blueprint en la aplicación (en \texttt{\_\_init\_\_.py})] 
\begin{minted}{python}
# Seleccionando el prefijo deseado
app.register_blueprint(nuevo_bp, url_prefix='/') 
\end{minted}
\end{tcolorbox}

\textbf{Modelos de la base de datos}: En <<models.py>> se tienen definidas todas
las entidades que maneja la aplicación: \texttt{Users}, \texttt{Datasets} y
\texttt{Runs}. Si se desea añadir alguna, primero se debe crear aquí la
definición de la misma. Para que la aplicación sepa las entidades que debe
manejar, también es necesario importarla en la creación de la aplicación.

\begin{tcolorbox}[colback=cyan!5!white,colframe=cyan!75!black,title=Importar modelo (en <<create\_app()>> de \texttt{\_\_init\_\_.py})]
    \begin{minted}{python}
    from .models import Nueva
    \end{minted}
\end{tcolorbox}

La \textbf{visualización de un algoritmo} se centra en dos pasos: la
configuración del mismo (en las rutas \texttt{/configuracion/<algoritmo>}) donde
se debe renderizar una página con el formulario de configuración (gestionado por
\texttt{configuration\_routes.py}), y la visualización
(\texttt{/visualizacion/<algoritmo>}) donde se renderiza la página donde se
encuentran todos los gráficos de los algoritmos (gestionado por
\texttt{visualization\_routes.py}). Además, existe un paso intermedio en el que
se obtienen los datos de la ejecución de los algoritmos, este paso no es una
ruta que pueda visualizarse (gestionado por \texttt{data\_routes.py}), es un
método auxiliar que accede el propio usuario (su navegador) para obtener la
información. Por lo general, la convención utilizada es que todos los métodos y
rutas lleven el nombre del algoritmo.

Las primeras tareas para añadir un algoritmo nuevo serán incorporarlos al flujo
de las sesiones y actualizar la plantilla base (por la barra de navegación) y el
inicio. Para incorporarlos al flujo, esto se hace incorporándolos a la selección
(\texttt{/seleccionar/<algoritmo>}).

\textbf{Formularios}: A la hora de crear un nuevo algoritmo, y como se ha
comentado, el usuario tiene un paso de configuración del mismo. Esto se realiza
mediante un formulario. Para crear uno (WTForm), se debe crear su objeto
correspondiente en <<forms.py>> con los campos necesarios. Hasta el momento, los
algoritmos que hay implementados tienen una parte en común que está definida en
la clase <<\texttt{FormConfiguracionBase}>> y en prinicpio, si se añade uno
nuevo, deberá extender dicha clase.

\textbf{Plantillas (templates)}: Flask utiliza el motor de plantillas Jinja2,
que añaden instrucciones en HTML. Obviando que cada algoritmo tiene sus
particularidades, están organizadas mediante la extensión de plantillas. De
forma general, añadir un algoritmo podría solo intervenir la creación de una
plantilla de configuración (incluyendo la creación del formulario) y otra de
visualización. En ambos casos se tiene una plantilla base de la que se debe
extender. También se deben nombrar como resto de algoritmos (están nombrados de
forma intuitiva).

De forma general, las plantillas se encuentran repartidas en varios
subdirectorios, cada uno de ellos engloban una parte de la aplicación.

\begin{figure}[H]
    \dirtree{%
        .1 \textbf{templates}: \begin{minipage}[t]{10cm}
            plantillas HTML (Jinja2) que renderiza la aplicación Web
            (Flask){.}\\
        \end{minipage}.
        .2 \textbf{configuracion}: \begin{minipage}[t]{10cm}
            plantillas concretas de la ventana de configuración de la ejecución de los algoritmos{.}\\
        \end{minipage}.
        .2 \textbf{usuarios}: \begin{minipage}[t]{10cm}
            plantillas concretas de las distintas ventanas de los usuarios: login, registro, perfil, espacio personal y panel de administración{.}\\
        \end{minipage}.
        .2 \textbf{visualizacion}: \begin{minipage}[t]{10cm}
            plantillas concretas de las distintas ventanas de visualización de los algoritmos{.}\\
        \end{minipage}.
    }
\end{figure}

Debe evaluarse la adición de nuevas sub-carpetas si así fuera. Además, aquellas
plantillas que sean generales y no vinculadas a una parte concreta, se incluirán
directamente en \texttt{templates}.

\textbf{Ficheros estáticos (static)}: 

El contenido dinámico debe ser creado mediante Javascript, de forma <<vanilla>> 
(sin utilizar frameworks). 

Al igual que con las plantillas, es completamente necesario mantener la
organización actual de las carpetas, las funciones que se añadan deberán
incluirse en la carpeta de su contexto.

\begin{figure}[H]
    \dirtree{%
        .1 \textbf{js}: \begin{minipage}[t]{10cm}
            ficheros JavaScript{.}\\
        \end{minipage}.
        .2 \textbf{configuracion}: \begin{minipage}[t]{10cm}
            ficheros JavaScript encargados de la configuración de la ejecución de los algoritmos{.}\\
        \end{minipage}.
        .2 \textbf{usuarios}: \begin{minipage}[t]{10cm}
            ficheros JavaScript encargados de la generación de tablas y control del espacio personal de los usuarios y el panel de administración{.}\\
        \end{minipage}.
        .2 \textbf{visualizacion}: \begin{minipage}[t]{10cm}
            ficheros JavaScript encargados de la inicialización y generación de los gráficos en la visualización (gráfico principal y estadísticas){.}\\
        \end{minipage}.
    }
\end{figure}

En menor medida, los estilos deben ser retocados en estos ficheros
(\texttt{css/style.css}) aunque en principio la aplicación está estilada con
Bootstrap~5.

En estos ficheros estáticos también se tienen unas imágenes con los
pseudocódigos de los algoritmos. Estas imágenes deben verse tanto en la
configuración como la visualización.

Otra parte muy importante es si se quieren añadir nuevos clasificadores base con
sus parámetros, la especificación de los mismos está almacenada en el fichero
JSON <<parametros.json>>. 

Hasta el momento del desarrollo solo existen dos tipos de entradas, los
selectores y los numéricos.

\begin{tcolorbox}[colback=cyan!5!white,colframe=cyan!75!black,title=Estructura para añadir clasificadores y sus parámetros]
\begin{minted}{json}
{
    "ClasificadorBase": {
        "parametro_numerico": {
        "label": "Nombre a mostrar",
        "type": "number",
        "step": 1,
        "min": 1,
        "max": "Infinity",
        "default": 5
        },
        "parametro_selector": {
        "label": "Nombre a mostrar",
        "type": "select",
        "options": ["lista", "de", "elementos"],
        "default": "elementos"
        }
    }
}
\end{minted}
\end{tcolorbox}

Con <<step>> se debe tener en cuenta la posibilidad de permitir números enteros
(al introducir 1) o números decimales (al incluir un flotante, por ejemplo,
0.01).

Según se ha codificado, la aparición de este nuevo clasificador será automática,
ya que una vez leído este fichero, JavaScript lo recorre creando tantas opciones
en los selectores como clasificadores haya y a su vez, todos los parámetros de
estos clasificadores.

Además de esto, hay tres zonas en la lógica de los \texttt{endpoints} que deben
ser modificadas si se añade un algoritmo. En <<\texttt{data\_routes.py}>>,
<<\texttt{configuration\_routes.py}>> y <<\texttt{visualization\_routes.py}>> se
tiene código específico para cada algoritmo. La <<activación>> de cada código se
realiza mediante un \textit{if} de tal forma que cuando coincide con el nombre
del algoritmo, se realizan ciertos pasos concretos para él. Se deberá ajustar la
lógica para el nuevo algoritmo.

En <<\texttt{data\_routes.py}>> se deberá crear la ruta de datos del algoritmo
(que se encarga de instanciar el algoritmo con los parámetros de la
configuración, y de ejecutarlo), en <<\texttt{configuration\_routes.py}>> es
donde se instancian los formularios de los algoritmos y en
<<\texttt{visualization\_routes.py}>> se obtienen los parámetros introducidos en
el formulario mediante una función creada para ello.

\textbf{Conjunto de datos (\textit{datasets})}: Los conjuntos de datos de los
usuarios (vinculados a sus sesiones) se almacenan localmente en la aplicación
con el <<Timestamp>> concatenado el nombre del fichero introducido.

Pero para agilizar el proceso del mantenimiento (principalmente del
almacenamiento del servidor), los ficheros subidos por usuarios anónimos se
guardan en una sub-carpeta distinta a la de los usuarios registrados.

\textbf{Internacionalización}: La aplicación está pensada para ser
internacionalizada en cualquier idioma, aunque solo se ha incluido inglés y
español. Una de las herramientas utilizadas es Babel, para incluir nuevo texto
en la aplicación se debe utilizar la función \texttt{gettext} o
\texttt{lazy\_gettext} (tanto en Jinja2 como Python si fuera necesario). El
proceso de extracción y compilación de las traducciones está explicado en la
siguiente sección.

Estas funciones envolventes no funcionan en JavaScript. Por si esto ocurre
durante el posterior desarrollo, se han considerado dos alternativas. 

Para el caso de las visualizaciones, en la plantilla
<<\texttt{base\_visualizacion}>> se ha incluido una función de traducción
embebida directamente en <<\texttt{</script>}>>. Al incluirse de esta forma,
Babel si puede capturar esas funciones. 

La otra posibilidad es mediante diccionarios y la declaración de una variable
que indique el idioma (\textit{locale}). Por ejemplo, en el espacio personal
(plantilla <<\texttt{miespacio}>>) y en el panel de administración (plantilla
<<\texttt{admin}>>), se ha definido una constante que consulta el idioma actual.
A partir de aquí, cuando existe un texto generado en JavaScript que necesita ser
traducido, se define un diccionario que contiene en los valores, las palabras
deseadas. Este diccionario realmente será un diccionario de diccionarios, el
primer nivel tendrá el idioma.

Ejemplo:

\begin{minted}{javascript}
    let diccionario = {
        "en": {"clave": "palabra en inglés"},
        "es": {"clave": "palabra en español"}
    };
\end{minted}

Ahora, cuando un texto ha de ser traducido, primer se accede al primer nivel y luego a la palabra concreta.


\begin{minted}{javascript}
    // locale es la constante declarada en la propia plantilla
    let palabra_localizada = diccionario[locale]["clave"];
\end{minted}

\section{Compilación, instalación y ejecución del proyecto}

Para poner en funcionamiento la aplicación primero se debe preparar el entorno
de ejecución. Se recuerda la necesidad de tener instalado Python (y el
administrador de paquetes <<pip>>).

\paragraph{Código fuente de la aplicación} Para obtener el código fuente de la
aplicación, este debe ser descargado del repositorio Github utilizado\footnote{
Repositorio: \url{https://github.com/dma1004/TFG-SemiSupervisado}}.

Tanto en Windows como en Linux esto se puede realizar descargando el fichero
comprimido de todo el repositorio desde el navegador.

Pero si se quiere realizar mediante consola de comandos, y suponiendo que se ha
instalado Git:

\begin{tcolorbox}[colback=cyan!5!white,colframe=cyan!75!black,fontupper=\footnotesize,title=Clonación de repositorio desde consola]
\begin{minted}{shell}
$ git clone https://github.com/dma1004/TFG-SemiSupervisado.git
\end{minted}
\end{tcolorbox}
La localización del proyecto queda a discreción del programador.
\paragraph{Entorno virtual Python} El entorno virtual no es estrictamente
necesario aunque es \textbf{altamente} recomendable, pues en los próximos pasos
se instalarán múltiples librerías que sin entorno virtual quedarían instaladas
globalmente. El proceso descrito a continuación puede ser sustituido en caso de
que solo se esté trabajando con Pycharm. Este editor permite crear entornos
virtuales e incluso realizarlo automáticamente al detectar los distintos
requisitos del proyecto. Pero de forma general, suponiendo que se desea preparar
un entorno de <<pruebas>> o <<producción>> y que la aplicación esté en
funcionamiento, se realizan los siguientes pasos:
\begin{tcolorbox}[colback=cyan!5!white,colframe=orange!75!black,title=Creación del entorno virtual (dentro de la carpeta deseado)]
\begin{minted}{shell}
$ python -m venv ./venv
\end{minted}
\end{tcolorbox}


\begin{tcolorbox}[colback=cyan!5!white,colframe=cyan!75!black,fontupper=\footnotesize,fontlower=\footnotesize,title=Activación del entorno virtual]
\begin{minted}{matlab}
$ venv\activate.bat //Windows
\end{minted}
\tcblower
\begin{minted}{matlab}
$ source ruta/al/entorno/virtual/bin/activate //Linux
\end{minted}
\end{tcolorbox}

\begin{tcolorbox}[colback=cyan!5!white,colframe=cyan!75!black,title=Para desactivar el entorno (una vez en él)]
\begin{minted}{shell}
$ deactivate
\end{minted}
\end{tcolorbox}
\paragraph{Instalación de paquetes} En este paso se van a instalar todas las
librerías necesarias, el proyecto trae un fichero <<requirements.txt>> en el que
vienen especificados estos paquetes y sus versiones. Además, los algoritmos
implementados también están configurados como un paquete que puede ser
instalado.
\begin{tcolorbox}[colback=cyan!5!white,colframe=orange!75!black,title=Instalar librerías externas]
\begin{minted}{shell}
$ python -m pip install -r requierements.txt
\end{minted}
\end{tcolorbox}

\begin{tcolorbox}[colback=cyan!5!white,colframe=orange!75!black,title=Instalar paquete de algoritmos]
\begin{minted}{shell}
$ python -m pip install .
\end{minted}
\end{tcolorbox}
A partir de aquí ya se tiene configurado todo el entorno y los requisitos
necesarios para poder ejecutar la aplicación.

\paragraph{Ejecución} Referida a la ejecución de la aplicación Web (Flask), se
debe estar situado en el directorio \textbf{/web} del proyecto.
\begin{tcolorbox}[colback=cyan!5!white,colframe=cyan!75!black,fontupper=\footnotesize,fontlower=\footnotesize,title=Ejecución]
\begin{minted}{shell}
$ set FLASK_APP=app.py //Windows
$ flask run
$ flask run --debug
\end{minted}
\tcblower
\begin{minted}{shell}
$ flask run //Linux
$ flask run --debug
\end{minted}
\end{tcolorbox}

Existe una última cuestión que no es estrictamente necesaria para funcionar,
pero que añade la internacionalización a la aplicación automáticamente.

\paragraph{Internacionalización} En el caso de que se haya incluido más texto
traducido, este debe ser compilado con Babel para que la aplicación pueda
detectarlo.

\begin{tcolorbox}[colback=cyan!5!white,colframe=cyan!75!black,fontupper=\footnotesize,title=Proceso de internacionalización (desde \texttt{/web/app})]
\begin{minted}{shell}
Extraer los textos encapsulados por gettext
$ pybabel extract -F babel.cfg -o messages.pot .

Extraer los textos encapsulados por lazy_gettext
$ pybabel extract -F babel.cfg -k lazy_gettext -o messages.pot .

Actualizar los textos extraidos en el fichero de compilación
$ pybabel update -i messages.pot -d translations
\end{minted}
\end{tcolorbox}

Todos los pasos anteriores son \textbf{estrictamente} necesarios, si se realiza
una extracción sin incluir <<\texttt{lazy\_gettext}>> (por ejemplo) y a
continuación se actualizan las traducciones, se perderán aquellas que tenían
<<\texttt{lazy\_gettext}>>.

Una vez que se han extraido los textos, se han de indicar las traducciones.
Todos los textos envueltos por las funciones antes comentadas actúan como
identificadores. En \texttt{translations/es/LC\_MESSAGES/messages.po} se tienen
esos identificadores junto con la traducción (estará vacía si no se ha
introducido). En ese fichero se incluirán las traducciones.

El siguiente paso es la compilación de estas traducciones para que \texttt{Babel} pueda sustituir
los textos dinámicamente.

\begin{tcolorbox}[colback=cyan!5!white,colframe=cyan!75!black,fontupper=\footnotesize,title=Compilación de traducciones (desde \texttt{/web/app})]
\begin{minted}{shell}
Compilación de las traducciones
$ pybabel compile -d translations 
\end{minted}
\end{tcolorbox}
\apendice{Documentación de usuario}

\section{Introducción}

Es esta sección se presenta los requisitos que el usuario debe satisfacer para
utilizar la aplicación junto con la instalación y manual de la misma.

\section{Requisitos de usuarios}

Los requisitos que debe cumplir el usuario son:
\begin{itemize}
    \item Periféricos básicos: pantalla, teclado y ratón.
    \item Navegador Web (Firefox, Chrome, Edge...).
    \item Conexión a internet.
    \item JavaScript habilitado en el navegador.
\end{itemize}

\section{Instalación}

El usuario no necesita instalar el software para utilizarlo. Puede acceder a él
directamente desde su navegador.

\section{Manual del usuario}

Con la ayuda de imágenes capturadas directamente de la Web, esta sección
describe cómo se realizan todas las acciones de la aplicación.

Dado que la documentación presente se encuentra en español, todas las interfaces
se mostrarán en español. Aun así, la aplicación está preparada para el idioma
inglés de igual forma.

Este manual está pensado para los usuarios anónimos y los usuarios con cuenta
registrada.

\subsection{Visualizar un algoritmo}

El flujo para visualizar un algoritmo en el siguiente:

\imagen{anexos/manual-usuario/FlujoVisualizarAlgoritmo}{Flujo de la visualización de un algoritmo}

\subsubsection{Seleccionar algoritmo}
Para seleccionar un algoritmo, se puede hacer de dos formas. La primera es
haciendo clic a los enlaces que aparecen en la barra de navegación (que además
está siempre presente en todas las pestañas) y también haciendo clic en las
tarjetas de presentación de cada algoritmo en la página principal.

IMAGEN DE LA PÁGINA PRINCIPAL INDICANDO LOS ENLACES DE LA NAV Y LAS TARJETAS.

Una vez que se haya seleccionado, será redirigido a la página de subida del
fichero. 

\subsubsection{Carga del conjunto de datos}

El fichero contendrá el conjunto de datos. Para subir un fichero, simplemente se
puede arrastrar desde el propio sistema hasta la zona marcada con rayas o
abriendo el explorador de archivos con el botón de <<Selecciona fichero>>. Podrá
ver durante la carga el progreso de la misma.

Si el usuario no dispone de un fichero, la aplicación incluye un enlace para
descargar uno de prueba, pulsando en el botón <<Descargar fichero de prueba>>. A
partir de aquí, es el mismo procedimiento comentado anteriormente.

\imagen{anexos/manual-usuario/Carga del conjunto de datos - VASS}{Carga del conjunto de datos}

\paragraph{Consideraciones del conjunto de datos} En primer lugar, los ficheros
subidos solo podrán tener extensiones ARFF o CSV, en caso contrario, al intentar
pasar al siguiente paso, el usuario será devuelto a esta misma página con un
mensaje de error.

El contenido del fichero de datos tiene que cumplir un requisito fundamental
derivado de la ausencia de un pre-procesado completo: 

\begin{tcolorbox}[colback=red!5!white,colframe=red!75!black,fontupper=\footnotesize,title=Requisito fundamental]
Todos los atributos del conjunto de datos deben ser numéricos (internamente los
algoritmos requieren de este tipo de datos), esto \textbf{no} incluye al
atributo de la clase, que sí puede ser categórico/nominal (esa parte del
pre-procesado sí que es realizada).
\end{tcolorbox}

Una vez que el fichero ha sido cargado (porcentaje completado), se habrá
habilitado el botón de configuración. Pulsando en él, se redirigirá a la
siguiente página del flujo (configuración).

\subsubsection{Configuración del algoritmo}
\label{mu:configuracion}
Se encontrará en la página de configuración del algoritmo, donde podrá observar
un apartado teórico con sus conceptos generales y su pseudocódigo. Por otro
lado, tendrá un formulario con todos los parámetros que se pueden configurar
para ese algoritmo.

\imagen{anexos/manual-usuario/Configuración del algoritmo - VASS}{Configuración del algoritmo}

Todos los parámetros tienen establecido un valor por defecto (con la
configuración <<estable>>), pero se tiene libertad completa para modificar cada
uno de ellos. En principio, si solamente se quiere ejecutar sin modificar ningún
parámetro, sí es necesario seleccionar correctamente el atributo de la clase.
Este atributo no puede ser establecido automáticamente por la aplicación, ya que
depende del conjunto de datos subido. 

Una vez configurado, se puede visualizar el algoritmo pulsando en el botón de
<<Ejecutar>>.

\subsubsection{Visualización}
\label{mu:visualizacion}
Ya en la página de visualización, se mostrará durante unos momentos una
animación de carga. Cuando el sistema haya finalizado la ejecución, se podrán
ver los distintos gráficos.

En principio, los errores que ocurran en el sistema serán mostrados. Sin
embargo, en el caso de que la animación dure un periodo de tiempo excesivo, se
recomienda reintentar la configuración (podrían ser problema de red
simplemente).

\imagen{anexos/manual-usuario/Visualización - VASS}{Vista general de una visualización}

\paragraph{Visualización principal} En el gráfico principal se mostrará el
conjunto de datos en dos dimensiones. Aquí podrá verse qué es lo que ocurre
durante el proceso de entrenamiento del algoritmo.

\imagen{anexos/manual-usuario/Visualización principal}{Visualización principal}

Este gráfico es interactivo:
\begin{itemize}
    \item Permite realizar \texttt{zoom} sobre una zona deseada mediante el
    doble clic o moviendo la ruleta del ratón.
    \item Al pasar el ratón por uno de los puntos, se mostrará toda la
    información relativa a esa posición en un recuadro informativo (aparecerá en
    las proximidades del ratón). Esto se está ejemplificando en la imagen
    anterior.
\end{itemize}

Para controlar la evolución, en la parte inferior se encuentra un panel de
control que permite:
\begin{itemize}
    \item Reiniciar \texttt{zoom}. Pese a que es posible reducir/aumentar el
    \texttt{zoom} manualmente, sirve para volver a la posición original.
    \item Visualizar la iteración actual: mediante el número y una barra de
    progreso.
    \item Reproducir automáticamente pulsando en el botón con el símbolo
    \textit{play}.
    \item Avanzar iteración manualmente.
    \item Reducir iteración.
\end{itemize}

Todas estas acciones modificarán el estado de los gráficos.

\paragraph{Tooltip} El \textit{tooltip} que se muestra al pasar el ratón por
encima de un punto tiene varios formatos dependiendo del algoritmo mostrado y de
los datos introducidos.

\paragraph{Tooltip: Casos comunes}

Existe un formato común a todos los algoritmos para los casos de los datos
iniciales. Este tipo de puntos se representa mediante un círculo. Como dato
inicial, tendrá la etiqueta correspondiente. Además, cada punto tiene en la
parte superior la posición que ocupa en el gráfico. Esta posición coincide con
los atributos seleccionados para representar los datos. En la siguiente imagen
se presenta un ejemplo de todo lo anterior donde se seleccionó PCA y por eso
aparece <<C1>> y <<C2>>.

\imagencontamano{anexos/manual-usuario/tooltips/Inicial}{Tooltip con dato inicial}{0.25}

Otro formato común es en el caso de puntos solapados (derivados de ejemplos
duplicados en el conjunto de datos o por PCA). En este caso, donde en el anterior
aparecía la información del punto, ahora aparecerá un listado con todos los
puntos solapados. Ejemplo:

\imagencontamano{anexos/manual-usuario/tooltips/Solapados}{Tooltip con datos solapados}{0.25}

Como se puede ver en este ejemplo anterior, los puntos no han sido clasificados
(captura tomada en iteración cero), esto no es importante, la diferencia con una
iteración posterior es que aparecerá la etiqueta asignada (se verá en cada
algoritmo). Además, aparece un indicativo de <<Clasificador>>, se ha querido
incluir en este ejemplo porque en todos los algoritmos excepto Self-Training se
indica el clasificador que se ha encargado de etiquetar cada punto.

Es interesante comentar que puede darse el caso de puntos solapados en el que
alguno o todos sean datos iniciales. Por ejemplo:

\imagencontamano{anexos/manual-usuario/tooltips/Solapados+Inicial}{Tooltip con datos solapados con uno inicial}{0.25}

\paragraph{Tooltip: Self-Training}

El formato de tooltip para Self-Training no tiene grandes complicaciones y
comprendiendo los anteriores ejemplos resulta sencillo de interpretar.

Existen dos formatos, el primero es en el que el dato no ha sido etiquetado
(porque se etiqueta en una iteración posterior o simplemente porque nunca es
etiquetado). Por ejemplo, una captura tomada en la iteración dos de una dato no
etiquetado es\footnote{En el ejemplo puede ser extraño que no aparezca un título
indicando el punto (algo como <<Punto 1>>), ese formato solo se <<activa>>
cuando existen datos solapados.}:

\imagencontamano{anexos/manual-usuario/tooltips/STNoClasificado}{Tooltip con un dato no clasificado}{0.25}

El otro formato es en el que el dato ya ha sido clasificado (en la iteración
actual o en una previa). Por ejemplo, una captura tomada en la iteración ocho de
una dato etiquetado en la iteración seis es:

\imagencontamano{anexos/manual-usuario/tooltips/STClasificado}{Tooltip con un dato clasificado}{0.3}

En este formato anterior se muestra la etiqueta asignada así como la iteración
en la que se clasificó entre paréntesis.

\paragraph{Tooltip: Co-Training}

Realmente, el tooltip para Co-Training es muy similar a Self-Training. No existe
ningún caso extraño o adicional.

Vuelven a existir dos formatos, cuando el dato no ha sido etiquetado todavía (o
nunca) y cuando sí está etiquetado. 

\imagencontamano{anexos/manual-usuario/tooltips/CTNoClasificado}{Tooltip con un dato no clasificado}{0.3}

Para el caso del dato clasificado sí que es necesario puntualizar alguna
cuestión (se verá en el ejemplo siguiente). Co-Training considera dos
clasificadores base y cada uno de ellos puede clasificar a un punto. Esto se ha
representado de dos maneras. La primera señal es que el símbolo del punto será
uno concreto para cada clasificador. La segunda señal es que en el tooltip
aparecerá una línea de <<Clasificador:>> que estará seguida de dicho símbolo y
el nombre del clasificador. Se mantiene también el número de la iteración en la
que se ha clasificado.

\imagencontamano{anexos/manual-usuario/tooltips/CTClasificado}{Tooltip con un dato clasificado}{0.45}

Esta idea también es la misma si existen datos solapados. En ese caso, en cada
uno de los puntos que se indiquen en el listado de solapados aparecerá el
clasificador (el símbolo y el nombre) junto con la etiqueta.

\imagencontamano{anexos/manual-usuario/tooltips/CTSolapadosClasificados}{Tooltip con datos solapados y clasificados}{0.45}

\paragraph{Tooltip: Democratic Co-Learning y Tri-Training}

En el caso de que un dato no haya sido clasificado es exactamente lo mismo que
para Co-Training, simplemente mostrará la posición y que no ha sido clasificado
(<<No clasificado>>).

La particularidad que tiene Democratic Co-Learning y Tri-Training es que cada
dato puede ser etiquetado por varios clasificadores. Para ejemplificar esto
simplemente se realiza un listado de los clasificadores que lo han etiquetado.

\imagencontamano{anexos/manual-usuario/tooltips/DCLClasificados1}{Tooltip con un dato clasificado por dos clasificadores}{0.45}

Es de destacar que solo aparecen aquellos que en la iteración actual o una
previa han etiquetado el dato.

El formato de ambos algoritmos es el mismo. Sin embargo, se cree conveniente
explicar el funcionamiento básico.

En \textbf{Democratic Co-Learning} cada punto solo puede ser clasificado
\textbf{una vez} por cada clasificador durante toda la ejecución. Esto, a
efectos de visualización, significa que el listado de clasificadores del tooltip
solo puede aumentar (si es que el punto es clasificado alguna vez).

En \textbf{Democratic Co-Learning} cada punto puede ser clasificado
\textbf{varias veces} por cada clasificador. Esto es así porque cada
clasificador mantiene su propio conjunto de entrenamiento y este es vaciado al
comienzo de cada iteración (y rellenado durante la misma). Todo esto significa
que durante una visualización, un ejemplo puede ser clasificado por una lista de
clasificadores y en la siguiente iteración por otra (igual o distinta).

Esto no afecta al formato comentado, pero puede llegar a ser extraño sin
especificarlo.

Por último, como en todos los anteriores formatos, pueden existir datos
solapados. El ejemplo mostrado a continuación es de Democratic Co-Learning, pero
como se ha comentado es el mismo formato que el de Tri-Training.

\imagencontamano{anexos/manual-usuario/tooltips/DCLSolapadosClasificados}{Tooltip con datos solapados y clasificados}{0.45}

En este ejemplo solo un clasificador ha clasificado cada dato solapado. Se ha
elegido este ejemplo porque, como es de esperar, si los tres clasificadores
etiquetan, el tooltip crecerá de igual manera.

\paragraph{Gráficos estadísticos} Pasando a otra parte de la ventana completa de
visualización, en la zona de la derecha se incluirá el resto de gráficos
adicionales, que serán principalmente estadísticas.

\imagen{anexos/manual-usuario/ZonaEstadisticas}{Gráficos estadísticos}

En la parte superior de esta zona se tiene un desplegable (contraído por
defecto) que contiene el pseudocódigo (el mismo que en la fase de
configuración). Si se desea consultar, simplemente se pulsa en cualquier parte
del desplegable:

\imagen{anexos/manual-usuario/Pseudocodigo}{Desplegable pseudocódigo}

El gráfico de estadísticas generales simplemente será a interpretación del
usuario (no puede realizar ninguna opción). 

En el caso de las estadísticas específicas puede seleccionar qué estadística
mostrar mediante el selector superior, y de qué clasificadores mostrarla
mediante las casillas en la parte inferior.

Ambos gráficos anteriores contienen información similar, en el eje X se indican
las iteraciones y en el eje Y el valor de la estadística(s).

\subsection{Cambiar de idioma}

Aunque la propia aplicación detecta el idioma más adecuado (entre español e
inglés) que mostrar, se puede seleccionar el idioma de forma manual.

En la barra de navegación hay un símbolo de traducción característico que al
pulsar aparece un desplegable.

\imagencontamano{anexos/manual-usuario/Cambiar de idioma}{Cambiar de idioma}{0.4}

Pulsando en el idioma se refrescará la página acorde al idioma seleccionado.
Esta característica \textbf{no} se incluye en la página de visualización.

\subsection{Registrarse}

Para aquellos usuarios anónimos que deseen crear una cuenta en la aplicación,
será necesario realizar el proceso de registro.

Para acceder a él, en la barra de navegación se dispone de un enlace (en la
parte derecha) que redirecciona al formulario de creación. Una vez accedido, se
deben rellenar los siguientes campos (todos obligatorios):
\begin{itemize}
    \item Nombre: entre 2 y 10 caracteres.
    \item Correo electrónico: será el identificador del usuario en el sistema y
    por lo tanto solo podrá haber uno.
    \item Contraseña: con al menos ocho caracteres.
    \item Confirmar contraseña: misma contraseña que el campo anterior.
\end{itemize}

\imagen{anexos/manual-usuario/Registrarse - VASS}{Formulario de registro}

Una vez enviado el formulario (y después de la comprobación de todos los
campos), la cuenta quedará registrada y se habrá iniciado sesión
automáticamente.

\subsection{Iniciar sesión}

Al igual que para el registro, para iniciar sesión existe un enlace en la barra
de navegación (en la parte derecha) que redirecciona al formulario de inicio de
sesión.

\imagen{anexos/manual-usuario/Iniciar sesión - VASS}{Formulario de inicio de sesión}

Este formulario es más sencillo y solo requiere el correo electrónico y
contraseña introducidos en el registro, o los nuevos si se han modificado (la
modificación de un perfil se verá más adelante).

\subsection{Cerrar sesión}

Si ya tiene sesión iniciada, debe hacer clic en su nombre en la barra de
navegación (parte derecha). Esto abrirá un desplegable en el que, aparte de
otras opciones, podrá cerrar la sesión.

\imagencontamano{anexos/manual-usuario/Cerrar sesión}{Cierre de sesión}{0.4}

\subsection{Personalizar perfil}

Es posible ver el perfil propio y modificar los datos con los que se creó la
cuenta.

En primer lugar, y similar a otros casos, en el desplegable de la barra de
navegación del usuario se pulsa en <<Perfil>>.

\imagencontamano{anexos/manual-usuario/DesplegablePerfil}{Acceso al perfil}{0.4}

Una vez dentro, en el lateral izquierdo aparece la información general del
perfil (ficheros subidos, ejecuciones, correo electrónico...).

La parte de edición (zona derecha) contiene un formulario similar al del
registro. Se pueden modificar todos los datos mostrados, pero para que las
modificaciones puedan realizarse, se debe introducir la contraseña actual. Si
fuera errónea o no se introduce, el formulario no se enviará.

\imagen{anexos/manual-usuario/Mi Perfil - VASS}{Perfil personal y edición}
\label{mu:perfil}

\subsection{Espacio personal}

Todos los usuarios poseen de un espacio personal en el que visualizar y
controlar sus ficheros subidos y las ejecuciones realizadas hasta el momento.

En primer lugar, y similar a otros casos, en el desplegable de la barra de
navegación del usuario se pulsa en <<Mi Espacio>>.

\imagencontamano{anexos/manual-usuario/DesplegableMiEspacio}{Acceso al espacio personal}{0.4}

Una vez dentro, en el lateral izquierdo aparece la información general del
perfil (ficheros subidos, ejecuciones, correo electrónico...).

En la parte derecha se encontrarán dos tablas en las que se reflejan los
ficheros subidos y las ejecuciones.

\imagen{anexos/manual-usuario/Mi Espacio - VASS}{Espacio personal}

Ambas tablas tienen un buscador donde se puede filtrar por cualquier palabra, en
todas las columnas de todas las filas. Además, puede elegir cuantas entradas
mostrar (selector en la esquina superior izquierda) y en su caso, pasar las
páginas para seguir mostrando más entradas (paginado en la esquina inferior
derecha).

\paragraph{Control de los conjuntos de datos subidos} Particularmente, los
conjuntos de datos (ficheros) pueden ser ejecutados o eliminados.

En el caso de querer utilizar el fichero para \textbf{ejecutar un algoritmo},
simplemente se ha de pulsar en el botón con el símbolo \texttt{play}. Al
hacerlo, se mostrará una ventana emergente (\texttt{modal}) para seleccionar el
algoritmo deseado.

\imagen{anexos/manual-usuario/EjecuciónDataset - VASS}{Selección de algoritmo}

Cuando se pulse en uno de los botones se redirige a la pestaña de configuración
(\ref{mu:configuracion}).

\label{mu:eliminardataset}
Por otro lado, si se quiere \textbf{eliminar un fichero} de la cuenta (y del
sistema), se pulsa en el botón con el símbolo de la papelera. De igual manera,
se mostrará una ventana emergente de confirmación.

\imagen{anexos/manual-usuario/EliminarDataset - VASS}{Eliminar fichero}

Si todo ha ido correctamente, habrá desaparecido la fila correspondiente del
fichero. En caso contrario se mostrará otra ventana emergente con el error.

\paragraph{Control de las ejecuciones} En el historial de ejecuciones también
pueden realizarse varias acciones.

\label{mu:parametrosrun}
En primer lugar, los \textbf{parámetros de configuración} que se introdujeron en
una ejecución se pueden consultar pulsando en el botón de la columna
<<Parámetros>>. Esto mostrará una ventana emergente con un JSON legible y
formateado.

\imagen{anexos/manual-usuario/ParametrosRun - VASS}{Parámetros de una ejecución}

De forma similar a los conjuntos de datos, las ejecuciones pueden ser
\textbf{re-ejecutadas}, repitiendo exactamente lo mismo que ocurrió en su
momento. Para ello simplemente se debe pulsar en el botón amarillo con el
símbolo típico de recargar página.

Este caso no se mostrará ninguna ventana emergente, redirigirá directamente a la
visualización del algoritmo (\ref{mu:visualizacion}).

\label{mu:eliminarrun}
Exactamente igual a los conjuntos de datos, las ejecuciones pueden
\textbf{eliminarse} pulsando en el botón con el símbolo de papelera mostrando
una ventana emergente de confirmación similar a la anterior.

\imagen{anexos/manual-usuario/EliminarRun - VASS}{Eliminar ejecución}


\section{Manual del administrador}

Conviene dividir el manual general para usuarios anónimos y registrados de los
administradores. Aun con ello, un usuario administrador puede realizar las
mismas acciones que el resto de roles.

Para acceder a esta sección se ha creado un administrador público con las
siguientes credenciales:
\begin{itemize}
    \item Email: admin@admin.es
    \item Contraseña: 12345678
\end{itemize}

\subsection{Panel de administración}

El administrador puede controlar a los usuarios, todos los ficheros subidos y
todas las ejecuciones. Para ello, posee de un panel de administración en el que
visualizar tablas con toda esa información.

En primer lugar, para acceder al panel, similar a otros casos, en el desplegable
de la barra de navegación del usuario (con rol administrador) se pulsa en <<Mi
Espacio>>.

\imagencontamano{anexos/manual-usuario/DesplegablePanel}{Acceso al panel de administración}{0.4}

Una vez dentro, se tiene un menú organizado en pestañas donde ver las tres tablas.

\imagen{anexos/manual-usuario/PanelUsuarios - VASS}{Administración de usuarios}

\imagen{anexos/manual-usuario/PanelDatasets - VASS}{Conjunto de datos subidos}

\imagen{anexos/manual-usuario/PanelRuns - VASS}{Historial de ejecuciones}

Para el caso de <<Conjunto de datos subidos>> e <<Historial de ejecuciones>>, el
proceso es el mismo que en el manual del usuario. La diferencia es que en las
acciones solo puede eliminar (no ejecutar o re-ejecutar respectivamente).
Consultar \ref{mu:eliminardataset} (eliminar fichero), \ref{mu:parametrosrun}
(mostrar parámetros de ejecución) y \ref{mu:eliminarrun} (eliminar ejecución).

En el caso de los usuarios, el administrador tiene dos posibilidades, editar sus
datos o eliminar el usuario.

Si se quiere \textbf{editar} un usuario, se debe pulsar el botón con el símbolo
del lápiz. Esto redirigirá a una pestaña similar a \ref{mu:perfil} (perfil del
usuario) y de hecho, será como adoptar la vista del usuario que se está editando
salvo por la inclusión de un indicativo como recordatorio al administrador.

\imagen{anexos/manual-usuario/PerfilAjeno - VASS}{Edición de un perfil ajeno}

Es de destacar también que en este caso, el formulario no incluye la contraseña
actual de ese usuario como confirmación. Esto es porque el administrador tiene
todos los privilegios para realizar esta acción. Cuando el administrador
modifique algún dato y envíe el formulario, los datos serán actualizados en el
sistema. 

Obviamente, los campos tienen ciertas limitaciones (similares a las del
registro):
\begin{itemize}
    \item Nombre: entre 2 y 10 caracteres.
    \item Correo electrónico: será el identificador del usuario en el sistema y
    por lo tanto solo podrá haber uno en el sistema.
    \item Nueva contraseña: con al menos ocho caracteres.
\end{itemize}

Si lo que se quiere es \textbf{eliminar} un usuario, el proceso es el mismo que
se ha visto para el resto de eliminaciones. Se debe pulsar en el botón que
incluye el símbolo de la papelera y se pedirá confirmación del proceso.

\imagen{anexos/manual-usuario/EliminarUsuario - VASS}{Eliminación de usuario}


\bibliographystyle{plain}
\bibliography{bibliografiaAnexos}

\end{document}
